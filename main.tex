\documentclass[12pt,english]{smfart}

\usepackage[T1]{fontenc}
\usepackage[english]{babel}

\usepackage{amssymb,url,xspace,smfthm}
\usepackage{graphicx, caption, subcaption}
\usepackage{amsmath}
\usepackage[utf8]{inputenc}
\usepackage[colorlinks,bookmarks=false,linkcolor=black,urlcolor=blue, citecolor=black]{hyperref}
\usepackage{units}
\usepackage{array}
\usepackage{sidecap}

\usepackage{float}

\usepackage{makecell}
\usepackage{multirow}

\usepackage{mathtools}
\usepackage{gensymb}
\usepackage{wrapfig}

\usepackage{chemist, xymtex}

% boundaries, use default when deactivated
\usepackage[top=2.5cm, bottom=2.5cm, left=2cm, right=2cm]{geometry}

%package setup for graphs
\usepackage{tikz}
\usepackage{pgfplots}
\usepackage{pgfplotstable}
\usetikzlibrary{patterns}

\usepgfplotslibrary{external}
\pgfplotsset{compat=newest}

\IfFileExists{./tikzext.perm}{\tikzexternalize[prefix=tikzext/]}{\tikzexternalize}

% units of mesure
\usepackage{siunitx}

% comments
\usepackage{comment}

% compute to result
\usepackage{xfp}

% definition of absolute value
\DeclarePairedDelimiter\abs{\lvert}{\rvert}

% copyright command
\def\meta#1{$\langle${\it #1}$\rangle$}
\newcommand{\myast}{($\star$)\ }
\makeatletter
    \def\ps@copyright{\ps@empty
    \def\@oddfoot{\hfil\small\copyright 2019, \EPFL}}
\makeatother

%pgfplot setup
\makeatletter
\pgfplotsset{
    /pgfplots/flexible xticklabels from table/.code n args={3}{%
        \pgfplotstableread[#3]{#1}\coordinate@table
        \pgfplotstablegetcolumn{#2}\of{\coordinate@table}\to\pgfplots@xticklabels
        \let\pgfplots@xticklabel=\pgfplots@user@ticklabel@list@x
    },
    % layer definition
    layers/my layer set/.define layer set={
        background,
        main,
        foreground
    }{
        % you could state styles here which should be moved to
        % corresponding layers, but that is not necessary here.
        % That is why we don't state anything here
    },
    % activate the newly created layer set
    set layers=my layer set
}
\makeatother

% something else
\newcommand{\EPFL}{\'Ecole Polytechnique F\'ed\'erale de Lausanne}

\def \be {\begin{equation}}
\def \ee {\end{equation}}

\def \bf {\begin{figure}}
\def \ef {\end{figure}}

\tolerance 400
\pretolerance 200

% define equations and values
\newcommand{\I}{3.4601865870615996e-07}
\newcommand{\wzero}{152.053084433746}
\newcommand{\kboltz}{1.380649e-23}

% #1 = alpha, #2 = shift, #3 = gain
\newcommand{\chimod}[3]{#3 / sqrt( ((2*pi*(x - #2))^2 - \wzero^2)^2 * \I^2 + (#1 * 2 * pi * (x - #2))^2 )}

% #1 = T, #2 = alpha, #3 = shift
\newcommand{\psd}[3]{4 * #1 * #2 * \kboltz / ( ((2*pi*(x - #3))^2 - \wzero^2)^2 * \I^2 + (#2 * 2 * pi * (x - #3))^2 ) }

\newcommand{\chimodvalue}[4]{#4 / sqrt( ((2*pi*(#1 - #3))^2 - \wzero^2)^2 * \I^2 + (#2 * 2 * pi * (#1 - #3))^2 )}

% #1 = deep/change, #2 = subname (5mm, plomb, ecc..), #3 = step number
\newcommand{\stepfile}[3]{"output/#1/#2_timosophraffa/#2_timosophraffa_step#3.out"}


\title{Task 2: molecular dynamics in a three-dimensional cubic lattice}
\date {\today}
\author{Raffaele Ancarola}
\dedicatory{Alfredo Pasquarello}

\begin{document}
\def\smfbyname{}

\maketitle

\tableofcontents

%%%%%%%%%%%%%%%%%%%%%%
%%%%%%%%%%%%%%%%%%%%%%
\section{Introduction}
\label{intro}
\noindent 
Numerical simulations are powerful tools allowing to obtain predictions in physical systems. 
In the particular case of molecular dynamics of liquid Argon, they give access to the estimates of thermodynamical quantities 
that otherwise wouldn't be possible to compute analytically. 
Specifically, the liquid Argon is modelled by $N$ classical particles interacting each other through the Lennard-Jones potential \cite{Rahman}. The simulation concern the evolution in time of those particles in a classical perspective using Runge-Kutta integrators (specifically \textit{second order Euler} and \textit{Verlet} \cite{course031}) and find the equilibrium states of the system.
At first, in this report the velocity of the particles will be manipulated in order to fix the temperature $T$ and reach the closest equilibrium energy state (see section \ref{sec:energy}). 
Secondly, the pair correlation function $g(r)$ and the structure factor $S(k)$ (see section \ref{sec:structural}) will be estimated, deducing the average bond properties of the Argon particles.
At last, the diffusion coefficient will be estimated using two different methods (see section \ref{sec:diffusion}).
A study of Rahman \cite{Rahman} has already been performed on this purpose and all numerical results of this study will be compared with that article.

\subsection{System description}
\noindent Our system is composed by a cubic periodic lattice initially containing 4 atoms per cell, where each cell is distanced of a factor $a = 1.7048$. Atoms interact with each other by the Lennard-Jones potential:

\begin{equation} \label{lennardjones}
V(r) = 4 \varepsilon \left[\left(\frac{\sigma}{r}\right)^{12} - \left(\frac{\sigma}{r}\right)^{6}\right] 
\end{equation}

where $r$ is the distance between each atom.
For instance, $\sigma = 3.4$ \si{\angstrom} and $\varepsilon/k_B = 120$ \si{K} for Argon.  For the sake of simplification during the simulation, $r$ and $V(r)$ are normalized in \si{LJ} units \cite{ljunits}.

\noindent
For simplicity, only $N_{cells}^{(x)} := 6$ cells per dimension are considered, thus having three dimensions, the total number of lattice cells is $N_{cells}^3 = 216$.
Here periodic boundary conditions for positions are set with the purpose to approximately describe an infinitely long lattice over all dimensions. Finally, atoms are considered as classical particles and having 4 atoms per cell implies that the total number of particles is $N := 864$.

\subsection{Numerical integration}
\noindent Each particle is described by its position $x^i$ and its velocity $v^i$, both in three dimensions. The time evolution of the system is performed using a \textit{second order Euler scheme}, which at the $n^{th}$ time step is given in vector form by:

\begin{align}\label{scheme}
    x_{n+1} &= x_n + v_{n+1} \Delta t  \\
    v_{n+1} &= v_n + \frac{1}{2} f_n \Delta t \nonumber
\end{align}

and $f_n = f(x_n, v_n)$ is the acceleration deduced by the Lennard-Jones potential (eq. \ref{lennardjones}).
A second and more performing integration method is implemented using verlet 

\subsection{Conservation of energy} \label{sec:energy}

As the simulation runs in the \textit{NVE} regime, the energy is supposed to be constant for a stable behaviour. In figure \ref{task1_total_energy} is shown that the energy oscillates around a stable value $E_{tot} = -5.927 \pm 0.005$. In a first moment, the energy of the system raises drastically until the value $E_{tot}$ is attended, then it tends to slowly decrease and stabilise itself.
\\
The amplitude of the oscillations depend on the time step, i.e. that a larger time step induces the numerical simulation to be more unstable. The second order Euler method (\ref{scheme}) is explicit, thus it exists a limit time step $dt_{lim}$ such that $dt > dt_{lim} \implies $ "the method in unstable".
In fact, we observe that the simulation involving the largest time step $dt = 0.03$ presents the largest amplitude in the error oscillations. 

\begin{figure}
    \begin{subfigure}{0.5\textwidth}
        \resizebox{\textwidth}{!}{
            %% Creator: Matplotlib, PGF backend
%%
%% To include the figure in your LaTeX document, write
%%   \input{<filename>.pgf}
%%
%% Make sure the required packages are loaded in your preamble
%%   \usepackage{pgf}
%%
%% Figures using additional raster images can only be included by \input if
%% they are in the same directory as the main LaTeX file. For loading figures
%% from other directories you can use the `import` package
%%   \usepackage{import}
%%
%% and then include the figures with
%%   \import{<path to file>}{<filename>.pgf}
%%
%% Matplotlib used the following preamble
%%
\begingroup%
\makeatletter%
\begin{pgfpicture}%
\pgfpathrectangle{\pgfpointorigin}{\pgfqpoint{6.400000in}{4.800000in}}%
\pgfusepath{use as bounding box, clip}%
\begin{pgfscope}%
\pgfsetbuttcap%
\pgfsetmiterjoin%
\definecolor{currentfill}{rgb}{1.000000,1.000000,1.000000}%
\pgfsetfillcolor{currentfill}%
\pgfsetlinewidth{0.000000pt}%
\definecolor{currentstroke}{rgb}{1.000000,1.000000,1.000000}%
\pgfsetstrokecolor{currentstroke}%
\pgfsetdash{}{0pt}%
\pgfpathmoveto{\pgfqpoint{0.000000in}{0.000000in}}%
\pgfpathlineto{\pgfqpoint{6.400000in}{0.000000in}}%
\pgfpathlineto{\pgfqpoint{6.400000in}{4.800000in}}%
\pgfpathlineto{\pgfqpoint{0.000000in}{4.800000in}}%
\pgfpathclose%
\pgfusepath{fill}%
\end{pgfscope}%
\begin{pgfscope}%
\pgfsetbuttcap%
\pgfsetmiterjoin%
\definecolor{currentfill}{rgb}{1.000000,1.000000,1.000000}%
\pgfsetfillcolor{currentfill}%
\pgfsetlinewidth{0.000000pt}%
\definecolor{currentstroke}{rgb}{0.000000,0.000000,0.000000}%
\pgfsetstrokecolor{currentstroke}%
\pgfsetstrokeopacity{0.000000}%
\pgfsetdash{}{0pt}%
\pgfpathmoveto{\pgfqpoint{0.800000in}{0.528000in}}%
\pgfpathlineto{\pgfqpoint{5.760000in}{0.528000in}}%
\pgfpathlineto{\pgfqpoint{5.760000in}{4.224000in}}%
\pgfpathlineto{\pgfqpoint{0.800000in}{4.224000in}}%
\pgfpathclose%
\pgfusepath{fill}%
\end{pgfscope}%
\begin{pgfscope}%
\pgfsetbuttcap%
\pgfsetroundjoin%
\definecolor{currentfill}{rgb}{0.000000,0.000000,0.000000}%
\pgfsetfillcolor{currentfill}%
\pgfsetlinewidth{0.803000pt}%
\definecolor{currentstroke}{rgb}{0.000000,0.000000,0.000000}%
\pgfsetstrokecolor{currentstroke}%
\pgfsetdash{}{0pt}%
\pgfsys@defobject{currentmarker}{\pgfqpoint{0.000000in}{-0.048611in}}{\pgfqpoint{0.000000in}{0.000000in}}{%
\pgfpathmoveto{\pgfqpoint{0.000000in}{0.000000in}}%
\pgfpathlineto{\pgfqpoint{0.000000in}{-0.048611in}}%
\pgfusepath{stroke,fill}%
}%
\begin{pgfscope}%
\pgfsys@transformshift{1.025455in}{0.528000in}%
\pgfsys@useobject{currentmarker}{}%
\end{pgfscope}%
\end{pgfscope}%
\begin{pgfscope}%
\definecolor{textcolor}{rgb}{0.000000,0.000000,0.000000}%
\pgfsetstrokecolor{textcolor}%
\pgfsetfillcolor{textcolor}%
\pgftext[x=1.025455in,y=0.430778in,,top]{\color{textcolor}\rmfamily\fontsize{10.000000}{12.000000}\selectfont \(\displaystyle {0}\)}%
\end{pgfscope}%
\begin{pgfscope}%
\pgfsetbuttcap%
\pgfsetroundjoin%
\definecolor{currentfill}{rgb}{0.000000,0.000000,0.000000}%
\pgfsetfillcolor{currentfill}%
\pgfsetlinewidth{0.803000pt}%
\definecolor{currentstroke}{rgb}{0.000000,0.000000,0.000000}%
\pgfsetstrokecolor{currentstroke}%
\pgfsetdash{}{0pt}%
\pgfsys@defobject{currentmarker}{\pgfqpoint{0.000000in}{-0.048611in}}{\pgfqpoint{0.000000in}{0.000000in}}{%
\pgfpathmoveto{\pgfqpoint{0.000000in}{0.000000in}}%
\pgfpathlineto{\pgfqpoint{0.000000in}{-0.048611in}}%
\pgfusepath{stroke,fill}%
}%
\begin{pgfscope}%
\pgfsys@transformshift{1.591923in}{0.528000in}%
\pgfsys@useobject{currentmarker}{}%
\end{pgfscope}%
\end{pgfscope}%
\begin{pgfscope}%
\definecolor{textcolor}{rgb}{0.000000,0.000000,0.000000}%
\pgfsetstrokecolor{textcolor}%
\pgfsetfillcolor{textcolor}%
\pgftext[x=1.591923in,y=0.430778in,,top]{\color{textcolor}\rmfamily\fontsize{10.000000}{12.000000}\selectfont \(\displaystyle {25}\)}%
\end{pgfscope}%
\begin{pgfscope}%
\pgfsetbuttcap%
\pgfsetroundjoin%
\definecolor{currentfill}{rgb}{0.000000,0.000000,0.000000}%
\pgfsetfillcolor{currentfill}%
\pgfsetlinewidth{0.803000pt}%
\definecolor{currentstroke}{rgb}{0.000000,0.000000,0.000000}%
\pgfsetstrokecolor{currentstroke}%
\pgfsetdash{}{0pt}%
\pgfsys@defobject{currentmarker}{\pgfqpoint{0.000000in}{-0.048611in}}{\pgfqpoint{0.000000in}{0.000000in}}{%
\pgfpathmoveto{\pgfqpoint{0.000000in}{0.000000in}}%
\pgfpathlineto{\pgfqpoint{0.000000in}{-0.048611in}}%
\pgfusepath{stroke,fill}%
}%
\begin{pgfscope}%
\pgfsys@transformshift{2.158392in}{0.528000in}%
\pgfsys@useobject{currentmarker}{}%
\end{pgfscope}%
\end{pgfscope}%
\begin{pgfscope}%
\definecolor{textcolor}{rgb}{0.000000,0.000000,0.000000}%
\pgfsetstrokecolor{textcolor}%
\pgfsetfillcolor{textcolor}%
\pgftext[x=2.158392in,y=0.430778in,,top]{\color{textcolor}\rmfamily\fontsize{10.000000}{12.000000}\selectfont \(\displaystyle {50}\)}%
\end{pgfscope}%
\begin{pgfscope}%
\pgfsetbuttcap%
\pgfsetroundjoin%
\definecolor{currentfill}{rgb}{0.000000,0.000000,0.000000}%
\pgfsetfillcolor{currentfill}%
\pgfsetlinewidth{0.803000pt}%
\definecolor{currentstroke}{rgb}{0.000000,0.000000,0.000000}%
\pgfsetstrokecolor{currentstroke}%
\pgfsetdash{}{0pt}%
\pgfsys@defobject{currentmarker}{\pgfqpoint{0.000000in}{-0.048611in}}{\pgfqpoint{0.000000in}{0.000000in}}{%
\pgfpathmoveto{\pgfqpoint{0.000000in}{0.000000in}}%
\pgfpathlineto{\pgfqpoint{0.000000in}{-0.048611in}}%
\pgfusepath{stroke,fill}%
}%
\begin{pgfscope}%
\pgfsys@transformshift{2.724861in}{0.528000in}%
\pgfsys@useobject{currentmarker}{}%
\end{pgfscope}%
\end{pgfscope}%
\begin{pgfscope}%
\definecolor{textcolor}{rgb}{0.000000,0.000000,0.000000}%
\pgfsetstrokecolor{textcolor}%
\pgfsetfillcolor{textcolor}%
\pgftext[x=2.724861in,y=0.430778in,,top]{\color{textcolor}\rmfamily\fontsize{10.000000}{12.000000}\selectfont \(\displaystyle {75}\)}%
\end{pgfscope}%
\begin{pgfscope}%
\pgfsetbuttcap%
\pgfsetroundjoin%
\definecolor{currentfill}{rgb}{0.000000,0.000000,0.000000}%
\pgfsetfillcolor{currentfill}%
\pgfsetlinewidth{0.803000pt}%
\definecolor{currentstroke}{rgb}{0.000000,0.000000,0.000000}%
\pgfsetstrokecolor{currentstroke}%
\pgfsetdash{}{0pt}%
\pgfsys@defobject{currentmarker}{\pgfqpoint{0.000000in}{-0.048611in}}{\pgfqpoint{0.000000in}{0.000000in}}{%
\pgfpathmoveto{\pgfqpoint{0.000000in}{0.000000in}}%
\pgfpathlineto{\pgfqpoint{0.000000in}{-0.048611in}}%
\pgfusepath{stroke,fill}%
}%
\begin{pgfscope}%
\pgfsys@transformshift{3.291329in}{0.528000in}%
\pgfsys@useobject{currentmarker}{}%
\end{pgfscope}%
\end{pgfscope}%
\begin{pgfscope}%
\definecolor{textcolor}{rgb}{0.000000,0.000000,0.000000}%
\pgfsetstrokecolor{textcolor}%
\pgfsetfillcolor{textcolor}%
\pgftext[x=3.291329in,y=0.430778in,,top]{\color{textcolor}\rmfamily\fontsize{10.000000}{12.000000}\selectfont \(\displaystyle {100}\)}%
\end{pgfscope}%
\begin{pgfscope}%
\pgfsetbuttcap%
\pgfsetroundjoin%
\definecolor{currentfill}{rgb}{0.000000,0.000000,0.000000}%
\pgfsetfillcolor{currentfill}%
\pgfsetlinewidth{0.803000pt}%
\definecolor{currentstroke}{rgb}{0.000000,0.000000,0.000000}%
\pgfsetstrokecolor{currentstroke}%
\pgfsetdash{}{0pt}%
\pgfsys@defobject{currentmarker}{\pgfqpoint{0.000000in}{-0.048611in}}{\pgfqpoint{0.000000in}{0.000000in}}{%
\pgfpathmoveto{\pgfqpoint{0.000000in}{0.000000in}}%
\pgfpathlineto{\pgfqpoint{0.000000in}{-0.048611in}}%
\pgfusepath{stroke,fill}%
}%
\begin{pgfscope}%
\pgfsys@transformshift{3.857798in}{0.528000in}%
\pgfsys@useobject{currentmarker}{}%
\end{pgfscope}%
\end{pgfscope}%
\begin{pgfscope}%
\definecolor{textcolor}{rgb}{0.000000,0.000000,0.000000}%
\pgfsetstrokecolor{textcolor}%
\pgfsetfillcolor{textcolor}%
\pgftext[x=3.857798in,y=0.430778in,,top]{\color{textcolor}\rmfamily\fontsize{10.000000}{12.000000}\selectfont \(\displaystyle {125}\)}%
\end{pgfscope}%
\begin{pgfscope}%
\pgfsetbuttcap%
\pgfsetroundjoin%
\definecolor{currentfill}{rgb}{0.000000,0.000000,0.000000}%
\pgfsetfillcolor{currentfill}%
\pgfsetlinewidth{0.803000pt}%
\definecolor{currentstroke}{rgb}{0.000000,0.000000,0.000000}%
\pgfsetstrokecolor{currentstroke}%
\pgfsetdash{}{0pt}%
\pgfsys@defobject{currentmarker}{\pgfqpoint{0.000000in}{-0.048611in}}{\pgfqpoint{0.000000in}{0.000000in}}{%
\pgfpathmoveto{\pgfqpoint{0.000000in}{0.000000in}}%
\pgfpathlineto{\pgfqpoint{0.000000in}{-0.048611in}}%
\pgfusepath{stroke,fill}%
}%
\begin{pgfscope}%
\pgfsys@transformshift{4.424267in}{0.528000in}%
\pgfsys@useobject{currentmarker}{}%
\end{pgfscope}%
\end{pgfscope}%
\begin{pgfscope}%
\definecolor{textcolor}{rgb}{0.000000,0.000000,0.000000}%
\pgfsetstrokecolor{textcolor}%
\pgfsetfillcolor{textcolor}%
\pgftext[x=4.424267in,y=0.430778in,,top]{\color{textcolor}\rmfamily\fontsize{10.000000}{12.000000}\selectfont \(\displaystyle {150}\)}%
\end{pgfscope}%
\begin{pgfscope}%
\pgfsetbuttcap%
\pgfsetroundjoin%
\definecolor{currentfill}{rgb}{0.000000,0.000000,0.000000}%
\pgfsetfillcolor{currentfill}%
\pgfsetlinewidth{0.803000pt}%
\definecolor{currentstroke}{rgb}{0.000000,0.000000,0.000000}%
\pgfsetstrokecolor{currentstroke}%
\pgfsetdash{}{0pt}%
\pgfsys@defobject{currentmarker}{\pgfqpoint{0.000000in}{-0.048611in}}{\pgfqpoint{0.000000in}{0.000000in}}{%
\pgfpathmoveto{\pgfqpoint{0.000000in}{0.000000in}}%
\pgfpathlineto{\pgfqpoint{0.000000in}{-0.048611in}}%
\pgfusepath{stroke,fill}%
}%
\begin{pgfscope}%
\pgfsys@transformshift{4.990735in}{0.528000in}%
\pgfsys@useobject{currentmarker}{}%
\end{pgfscope}%
\end{pgfscope}%
\begin{pgfscope}%
\definecolor{textcolor}{rgb}{0.000000,0.000000,0.000000}%
\pgfsetstrokecolor{textcolor}%
\pgfsetfillcolor{textcolor}%
\pgftext[x=4.990735in,y=0.430778in,,top]{\color{textcolor}\rmfamily\fontsize{10.000000}{12.000000}\selectfont \(\displaystyle {175}\)}%
\end{pgfscope}%
\begin{pgfscope}%
\pgfsetbuttcap%
\pgfsetroundjoin%
\definecolor{currentfill}{rgb}{0.000000,0.000000,0.000000}%
\pgfsetfillcolor{currentfill}%
\pgfsetlinewidth{0.803000pt}%
\definecolor{currentstroke}{rgb}{0.000000,0.000000,0.000000}%
\pgfsetstrokecolor{currentstroke}%
\pgfsetdash{}{0pt}%
\pgfsys@defobject{currentmarker}{\pgfqpoint{0.000000in}{-0.048611in}}{\pgfqpoint{0.000000in}{0.000000in}}{%
\pgfpathmoveto{\pgfqpoint{0.000000in}{0.000000in}}%
\pgfpathlineto{\pgfqpoint{0.000000in}{-0.048611in}}%
\pgfusepath{stroke,fill}%
}%
\begin{pgfscope}%
\pgfsys@transformshift{5.557204in}{0.528000in}%
\pgfsys@useobject{currentmarker}{}%
\end{pgfscope}%
\end{pgfscope}%
\begin{pgfscope}%
\definecolor{textcolor}{rgb}{0.000000,0.000000,0.000000}%
\pgfsetstrokecolor{textcolor}%
\pgfsetfillcolor{textcolor}%
\pgftext[x=5.557204in,y=0.430778in,,top]{\color{textcolor}\rmfamily\fontsize{10.000000}{12.000000}\selectfont \(\displaystyle {200}\)}%
\end{pgfscope}%
\begin{pgfscope}%
\definecolor{textcolor}{rgb}{0.000000,0.000000,0.000000}%
\pgfsetstrokecolor{textcolor}%
\pgfsetfillcolor{textcolor}%
\pgftext[x=3.280000in,y=0.251766in,,top]{\color{textcolor}\rmfamily\fontsize{10.000000}{12.000000}\selectfont \(\displaystyle t\)}%
\end{pgfscope}%
\begin{pgfscope}%
\pgfsetbuttcap%
\pgfsetroundjoin%
\definecolor{currentfill}{rgb}{0.000000,0.000000,0.000000}%
\pgfsetfillcolor{currentfill}%
\pgfsetlinewidth{0.803000pt}%
\definecolor{currentstroke}{rgb}{0.000000,0.000000,0.000000}%
\pgfsetstrokecolor{currentstroke}%
\pgfsetdash{}{0pt}%
\pgfsys@defobject{currentmarker}{\pgfqpoint{-0.048611in}{0.000000in}}{\pgfqpoint{-0.000000in}{0.000000in}}{%
\pgfpathmoveto{\pgfqpoint{-0.000000in}{0.000000in}}%
\pgfpathlineto{\pgfqpoint{-0.048611in}{0.000000in}}%
\pgfusepath{stroke,fill}%
}%
\begin{pgfscope}%
\pgfsys@transformshift{0.800000in}{0.534679in}%
\pgfsys@useobject{currentmarker}{}%
\end{pgfscope}%
\end{pgfscope}%
\begin{pgfscope}%
\definecolor{textcolor}{rgb}{0.000000,0.000000,0.000000}%
\pgfsetstrokecolor{textcolor}%
\pgfsetfillcolor{textcolor}%
\pgftext[x=0.278394in, y=0.486453in, left, base]{\color{textcolor}\rmfamily\fontsize{10.000000}{12.000000}\selectfont \(\displaystyle {\ensuremath{-}5.315}\)}%
\end{pgfscope}%
\begin{pgfscope}%
\pgfsetbuttcap%
\pgfsetroundjoin%
\definecolor{currentfill}{rgb}{0.000000,0.000000,0.000000}%
\pgfsetfillcolor{currentfill}%
\pgfsetlinewidth{0.803000pt}%
\definecolor{currentstroke}{rgb}{0.000000,0.000000,0.000000}%
\pgfsetstrokecolor{currentstroke}%
\pgfsetdash{}{0pt}%
\pgfsys@defobject{currentmarker}{\pgfqpoint{-0.048611in}{0.000000in}}{\pgfqpoint{-0.000000in}{0.000000in}}{%
\pgfpathmoveto{\pgfqpoint{-0.000000in}{0.000000in}}%
\pgfpathlineto{\pgfqpoint{-0.048611in}{0.000000in}}%
\pgfusepath{stroke,fill}%
}%
\begin{pgfscope}%
\pgfsys@transformshift{0.800000in}{1.036160in}%
\pgfsys@useobject{currentmarker}{}%
\end{pgfscope}%
\end{pgfscope}%
\begin{pgfscope}%
\definecolor{textcolor}{rgb}{0.000000,0.000000,0.000000}%
\pgfsetstrokecolor{textcolor}%
\pgfsetfillcolor{textcolor}%
\pgftext[x=0.278394in, y=0.987935in, left, base]{\color{textcolor}\rmfamily\fontsize{10.000000}{12.000000}\selectfont \(\displaystyle {\ensuremath{-}5.310}\)}%
\end{pgfscope}%
\begin{pgfscope}%
\pgfsetbuttcap%
\pgfsetroundjoin%
\definecolor{currentfill}{rgb}{0.000000,0.000000,0.000000}%
\pgfsetfillcolor{currentfill}%
\pgfsetlinewidth{0.803000pt}%
\definecolor{currentstroke}{rgb}{0.000000,0.000000,0.000000}%
\pgfsetstrokecolor{currentstroke}%
\pgfsetdash{}{0pt}%
\pgfsys@defobject{currentmarker}{\pgfqpoint{-0.048611in}{0.000000in}}{\pgfqpoint{-0.000000in}{0.000000in}}{%
\pgfpathmoveto{\pgfqpoint{-0.000000in}{0.000000in}}%
\pgfpathlineto{\pgfqpoint{-0.048611in}{0.000000in}}%
\pgfusepath{stroke,fill}%
}%
\begin{pgfscope}%
\pgfsys@transformshift{0.800000in}{1.537641in}%
\pgfsys@useobject{currentmarker}{}%
\end{pgfscope}%
\end{pgfscope}%
\begin{pgfscope}%
\definecolor{textcolor}{rgb}{0.000000,0.000000,0.000000}%
\pgfsetstrokecolor{textcolor}%
\pgfsetfillcolor{textcolor}%
\pgftext[x=0.278394in, y=1.489416in, left, base]{\color{textcolor}\rmfamily\fontsize{10.000000}{12.000000}\selectfont \(\displaystyle {\ensuremath{-}5.305}\)}%
\end{pgfscope}%
\begin{pgfscope}%
\pgfsetbuttcap%
\pgfsetroundjoin%
\definecolor{currentfill}{rgb}{0.000000,0.000000,0.000000}%
\pgfsetfillcolor{currentfill}%
\pgfsetlinewidth{0.803000pt}%
\definecolor{currentstroke}{rgb}{0.000000,0.000000,0.000000}%
\pgfsetstrokecolor{currentstroke}%
\pgfsetdash{}{0pt}%
\pgfsys@defobject{currentmarker}{\pgfqpoint{-0.048611in}{0.000000in}}{\pgfqpoint{-0.000000in}{0.000000in}}{%
\pgfpathmoveto{\pgfqpoint{-0.000000in}{0.000000in}}%
\pgfpathlineto{\pgfqpoint{-0.048611in}{0.000000in}}%
\pgfusepath{stroke,fill}%
}%
\begin{pgfscope}%
\pgfsys@transformshift{0.800000in}{2.039123in}%
\pgfsys@useobject{currentmarker}{}%
\end{pgfscope}%
\end{pgfscope}%
\begin{pgfscope}%
\definecolor{textcolor}{rgb}{0.000000,0.000000,0.000000}%
\pgfsetstrokecolor{textcolor}%
\pgfsetfillcolor{textcolor}%
\pgftext[x=0.278394in, y=1.990897in, left, base]{\color{textcolor}\rmfamily\fontsize{10.000000}{12.000000}\selectfont \(\displaystyle {\ensuremath{-}5.300}\)}%
\end{pgfscope}%
\begin{pgfscope}%
\pgfsetbuttcap%
\pgfsetroundjoin%
\definecolor{currentfill}{rgb}{0.000000,0.000000,0.000000}%
\pgfsetfillcolor{currentfill}%
\pgfsetlinewidth{0.803000pt}%
\definecolor{currentstroke}{rgb}{0.000000,0.000000,0.000000}%
\pgfsetstrokecolor{currentstroke}%
\pgfsetdash{}{0pt}%
\pgfsys@defobject{currentmarker}{\pgfqpoint{-0.048611in}{0.000000in}}{\pgfqpoint{-0.000000in}{0.000000in}}{%
\pgfpathmoveto{\pgfqpoint{-0.000000in}{0.000000in}}%
\pgfpathlineto{\pgfqpoint{-0.048611in}{0.000000in}}%
\pgfusepath{stroke,fill}%
}%
\begin{pgfscope}%
\pgfsys@transformshift{0.800000in}{2.540604in}%
\pgfsys@useobject{currentmarker}{}%
\end{pgfscope}%
\end{pgfscope}%
\begin{pgfscope}%
\definecolor{textcolor}{rgb}{0.000000,0.000000,0.000000}%
\pgfsetstrokecolor{textcolor}%
\pgfsetfillcolor{textcolor}%
\pgftext[x=0.278394in, y=2.492379in, left, base]{\color{textcolor}\rmfamily\fontsize{10.000000}{12.000000}\selectfont \(\displaystyle {\ensuremath{-}5.295}\)}%
\end{pgfscope}%
\begin{pgfscope}%
\pgfsetbuttcap%
\pgfsetroundjoin%
\definecolor{currentfill}{rgb}{0.000000,0.000000,0.000000}%
\pgfsetfillcolor{currentfill}%
\pgfsetlinewidth{0.803000pt}%
\definecolor{currentstroke}{rgb}{0.000000,0.000000,0.000000}%
\pgfsetstrokecolor{currentstroke}%
\pgfsetdash{}{0pt}%
\pgfsys@defobject{currentmarker}{\pgfqpoint{-0.048611in}{0.000000in}}{\pgfqpoint{-0.000000in}{0.000000in}}{%
\pgfpathmoveto{\pgfqpoint{-0.000000in}{0.000000in}}%
\pgfpathlineto{\pgfqpoint{-0.048611in}{0.000000in}}%
\pgfusepath{stroke,fill}%
}%
\begin{pgfscope}%
\pgfsys@transformshift{0.800000in}{3.042085in}%
\pgfsys@useobject{currentmarker}{}%
\end{pgfscope}%
\end{pgfscope}%
\begin{pgfscope}%
\definecolor{textcolor}{rgb}{0.000000,0.000000,0.000000}%
\pgfsetstrokecolor{textcolor}%
\pgfsetfillcolor{textcolor}%
\pgftext[x=0.278394in, y=2.993860in, left, base]{\color{textcolor}\rmfamily\fontsize{10.000000}{12.000000}\selectfont \(\displaystyle {\ensuremath{-}5.290}\)}%
\end{pgfscope}%
\begin{pgfscope}%
\pgfsetbuttcap%
\pgfsetroundjoin%
\definecolor{currentfill}{rgb}{0.000000,0.000000,0.000000}%
\pgfsetfillcolor{currentfill}%
\pgfsetlinewidth{0.803000pt}%
\definecolor{currentstroke}{rgb}{0.000000,0.000000,0.000000}%
\pgfsetstrokecolor{currentstroke}%
\pgfsetdash{}{0pt}%
\pgfsys@defobject{currentmarker}{\pgfqpoint{-0.048611in}{0.000000in}}{\pgfqpoint{-0.000000in}{0.000000in}}{%
\pgfpathmoveto{\pgfqpoint{-0.000000in}{0.000000in}}%
\pgfpathlineto{\pgfqpoint{-0.048611in}{0.000000in}}%
\pgfusepath{stroke,fill}%
}%
\begin{pgfscope}%
\pgfsys@transformshift{0.800000in}{3.543567in}%
\pgfsys@useobject{currentmarker}{}%
\end{pgfscope}%
\end{pgfscope}%
\begin{pgfscope}%
\definecolor{textcolor}{rgb}{0.000000,0.000000,0.000000}%
\pgfsetstrokecolor{textcolor}%
\pgfsetfillcolor{textcolor}%
\pgftext[x=0.278394in, y=3.495341in, left, base]{\color{textcolor}\rmfamily\fontsize{10.000000}{12.000000}\selectfont \(\displaystyle {\ensuremath{-}5.285}\)}%
\end{pgfscope}%
\begin{pgfscope}%
\pgfsetbuttcap%
\pgfsetroundjoin%
\definecolor{currentfill}{rgb}{0.000000,0.000000,0.000000}%
\pgfsetfillcolor{currentfill}%
\pgfsetlinewidth{0.803000pt}%
\definecolor{currentstroke}{rgb}{0.000000,0.000000,0.000000}%
\pgfsetstrokecolor{currentstroke}%
\pgfsetdash{}{0pt}%
\pgfsys@defobject{currentmarker}{\pgfqpoint{-0.048611in}{0.000000in}}{\pgfqpoint{-0.000000in}{0.000000in}}{%
\pgfpathmoveto{\pgfqpoint{-0.000000in}{0.000000in}}%
\pgfpathlineto{\pgfqpoint{-0.048611in}{0.000000in}}%
\pgfusepath{stroke,fill}%
}%
\begin{pgfscope}%
\pgfsys@transformshift{0.800000in}{4.045048in}%
\pgfsys@useobject{currentmarker}{}%
\end{pgfscope}%
\end{pgfscope}%
\begin{pgfscope}%
\definecolor{textcolor}{rgb}{0.000000,0.000000,0.000000}%
\pgfsetstrokecolor{textcolor}%
\pgfsetfillcolor{textcolor}%
\pgftext[x=0.278394in, y=3.996823in, left, base]{\color{textcolor}\rmfamily\fontsize{10.000000}{12.000000}\selectfont \(\displaystyle {\ensuremath{-}5.280}\)}%
\end{pgfscope}%
\begin{pgfscope}%
\definecolor{textcolor}{rgb}{0.000000,0.000000,0.000000}%
\pgfsetstrokecolor{textcolor}%
\pgfsetfillcolor{textcolor}%
\pgftext[x=0.222838in,y=2.376000in,,bottom,rotate=90.000000]{\color{textcolor}\rmfamily\fontsize{10.000000}{12.000000}\selectfont Total energy [L.J.]}%
\end{pgfscope}%
\begin{pgfscope}%
\pgfpathrectangle{\pgfqpoint{0.800000in}{0.528000in}}{\pgfqpoint{4.960000in}{3.696000in}}%
\pgfusepath{clip}%
\pgfsetrectcap%
\pgfsetroundjoin%
\pgfsetlinewidth{1.505625pt}%
\definecolor{currentstroke}{rgb}{0.121569,0.466667,0.705882}%
\pgfsetstrokecolor{currentstroke}%
\pgfsetdash{}{0pt}%
\pgfpathmoveto{\pgfqpoint{1.025455in}{1.068534in}}%
\pgfpathlineto{\pgfqpoint{1.138748in}{1.066849in}}%
\pgfpathlineto{\pgfqpoint{1.161407in}{1.066227in}}%
\pgfpathlineto{\pgfqpoint{1.184066in}{1.074978in}}%
\pgfpathlineto{\pgfqpoint{1.206725in}{1.096956in}}%
\pgfpathlineto{\pgfqpoint{1.229383in}{1.166156in}}%
\pgfpathlineto{\pgfqpoint{1.252042in}{1.212579in}}%
\pgfpathlineto{\pgfqpoint{1.274701in}{1.289328in}}%
\pgfpathlineto{\pgfqpoint{1.387995in}{1.817475in}}%
\pgfpathlineto{\pgfqpoint{1.410653in}{1.906261in}}%
\pgfpathlineto{\pgfqpoint{1.433312in}{2.025653in}}%
\pgfpathlineto{\pgfqpoint{1.478630in}{2.178591in}}%
\pgfpathlineto{\pgfqpoint{1.523947in}{2.310801in}}%
\pgfpathlineto{\pgfqpoint{1.546606in}{2.349203in}}%
\pgfpathlineto{\pgfqpoint{1.569265in}{2.399257in}}%
\pgfpathlineto{\pgfqpoint{1.591923in}{2.455138in}}%
\pgfpathlineto{\pgfqpoint{1.614582in}{2.492112in}}%
\pgfpathlineto{\pgfqpoint{1.637241in}{2.511774in}}%
\pgfpathlineto{\pgfqpoint{1.659899in}{2.523496in}}%
\pgfpathlineto{\pgfqpoint{1.682558in}{2.531029in}}%
\pgfpathlineto{\pgfqpoint{1.705217in}{2.551350in}}%
\pgfpathlineto{\pgfqpoint{1.727876in}{2.550350in}}%
\pgfpathlineto{\pgfqpoint{1.750534in}{2.543084in}}%
\pgfpathlineto{\pgfqpoint{1.773193in}{2.565704in}}%
\pgfpathlineto{\pgfqpoint{1.795852in}{2.563431in}}%
\pgfpathlineto{\pgfqpoint{1.818511in}{2.555120in}}%
\pgfpathlineto{\pgfqpoint{1.841169in}{2.567422in}}%
\pgfpathlineto{\pgfqpoint{1.863828in}{2.560661in}}%
\pgfpathlineto{\pgfqpoint{1.886487in}{2.567036in}}%
\pgfpathlineto{\pgfqpoint{1.909146in}{2.546827in}}%
\pgfpathlineto{\pgfqpoint{1.931804in}{2.545485in}}%
\pgfpathlineto{\pgfqpoint{1.954463in}{2.517628in}}%
\pgfpathlineto{\pgfqpoint{1.977122in}{2.520103in}}%
\pgfpathlineto{\pgfqpoint{1.999781in}{2.511271in}}%
\pgfpathlineto{\pgfqpoint{2.022439in}{2.496818in}}%
\pgfpathlineto{\pgfqpoint{2.045098in}{2.472922in}}%
\pgfpathlineto{\pgfqpoint{2.067757in}{2.471861in}}%
\pgfpathlineto{\pgfqpoint{2.090416in}{2.457729in}}%
\pgfpathlineto{\pgfqpoint{2.113074in}{2.455035in}}%
\pgfpathlineto{\pgfqpoint{2.135733in}{2.458177in}}%
\pgfpathlineto{\pgfqpoint{2.158392in}{2.459578in}}%
\pgfpathlineto{\pgfqpoint{2.181051in}{2.453471in}}%
\pgfpathlineto{\pgfqpoint{2.203709in}{2.445604in}}%
\pgfpathlineto{\pgfqpoint{2.226368in}{2.441607in}}%
\pgfpathlineto{\pgfqpoint{2.271686in}{2.479413in}}%
\pgfpathlineto{\pgfqpoint{2.294344in}{2.453035in}}%
\pgfpathlineto{\pgfqpoint{2.317003in}{2.458894in}}%
\pgfpathlineto{\pgfqpoint{2.339662in}{2.451519in}}%
\pgfpathlineto{\pgfqpoint{2.362321in}{2.427196in}}%
\pgfpathlineto{\pgfqpoint{2.384979in}{2.419954in}}%
\pgfpathlineto{\pgfqpoint{2.407638in}{2.433558in}}%
\pgfpathlineto{\pgfqpoint{2.430297in}{2.435833in}}%
\pgfpathlineto{\pgfqpoint{2.452956in}{2.441979in}}%
\pgfpathlineto{\pgfqpoint{2.475614in}{2.459401in}}%
\pgfpathlineto{\pgfqpoint{2.498273in}{2.471165in}}%
\pgfpathlineto{\pgfqpoint{2.520932in}{2.454566in}}%
\pgfpathlineto{\pgfqpoint{2.543591in}{2.436007in}}%
\pgfpathlineto{\pgfqpoint{2.566249in}{2.430708in}}%
\pgfpathlineto{\pgfqpoint{2.588908in}{2.429096in}}%
\pgfpathlineto{\pgfqpoint{2.611567in}{2.417973in}}%
\pgfpathlineto{\pgfqpoint{2.634226in}{2.401062in}}%
\pgfpathlineto{\pgfqpoint{2.656884in}{2.387830in}}%
\pgfpathlineto{\pgfqpoint{2.679543in}{2.395344in}}%
\pgfpathlineto{\pgfqpoint{2.702202in}{2.355553in}}%
\pgfpathlineto{\pgfqpoint{2.724861in}{2.370601in}}%
\pgfpathlineto{\pgfqpoint{2.747519in}{2.370477in}}%
\pgfpathlineto{\pgfqpoint{2.770178in}{2.351377in}}%
\pgfpathlineto{\pgfqpoint{2.792837in}{2.349423in}}%
\pgfpathlineto{\pgfqpoint{2.815496in}{2.334193in}}%
\pgfpathlineto{\pgfqpoint{2.838154in}{2.343650in}}%
\pgfpathlineto{\pgfqpoint{2.860813in}{2.356915in}}%
\pgfpathlineto{\pgfqpoint{2.883472in}{2.355047in}}%
\pgfpathlineto{\pgfqpoint{2.906131in}{2.343718in}}%
\pgfpathlineto{\pgfqpoint{2.928789in}{2.319172in}}%
\pgfpathlineto{\pgfqpoint{2.951448in}{2.317276in}}%
\pgfpathlineto{\pgfqpoint{2.974107in}{2.317310in}}%
\pgfpathlineto{\pgfqpoint{2.996766in}{2.309711in}}%
\pgfpathlineto{\pgfqpoint{3.019424in}{2.309672in}}%
\pgfpathlineto{\pgfqpoint{3.042083in}{2.307680in}}%
\pgfpathlineto{\pgfqpoint{3.064742in}{2.311324in}}%
\pgfpathlineto{\pgfqpoint{3.087401in}{2.318766in}}%
\pgfpathlineto{\pgfqpoint{3.110059in}{2.324304in}}%
\pgfpathlineto{\pgfqpoint{3.132718in}{2.318437in}}%
\pgfpathlineto{\pgfqpoint{3.155377in}{2.303097in}}%
\pgfpathlineto{\pgfqpoint{3.178036in}{2.291567in}}%
\pgfpathlineto{\pgfqpoint{3.200694in}{2.293331in}}%
\pgfpathlineto{\pgfqpoint{3.223353in}{2.281904in}}%
\pgfpathlineto{\pgfqpoint{3.268671in}{2.289295in}}%
\pgfpathlineto{\pgfqpoint{3.291329in}{2.272206in}}%
\pgfpathlineto{\pgfqpoint{3.313988in}{2.289240in}}%
\pgfpathlineto{\pgfqpoint{3.336647in}{2.302475in}}%
\pgfpathlineto{\pgfqpoint{3.359306in}{2.287371in}}%
\pgfpathlineto{\pgfqpoint{3.381964in}{2.264734in}}%
\pgfpathlineto{\pgfqpoint{3.404623in}{2.259158in}}%
\pgfpathlineto{\pgfqpoint{3.427282in}{2.259324in}}%
\pgfpathlineto{\pgfqpoint{3.449941in}{2.251868in}}%
\pgfpathlineto{\pgfqpoint{3.472599in}{2.238747in}}%
\pgfpathlineto{\pgfqpoint{3.495258in}{2.246543in}}%
\pgfpathlineto{\pgfqpoint{3.517917in}{2.239234in}}%
\pgfpathlineto{\pgfqpoint{3.540576in}{2.226205in}}%
\pgfpathlineto{\pgfqpoint{3.563234in}{2.211187in}}%
\pgfpathlineto{\pgfqpoint{3.585893in}{2.199967in}}%
\pgfpathlineto{\pgfqpoint{3.608552in}{2.217155in}}%
\pgfpathlineto{\pgfqpoint{3.631211in}{2.190686in}}%
\pgfpathlineto{\pgfqpoint{3.653869in}{2.196417in}}%
\pgfpathlineto{\pgfqpoint{3.676528in}{2.203958in}}%
\pgfpathlineto{\pgfqpoint{3.699187in}{2.198227in}}%
\pgfpathlineto{\pgfqpoint{3.721846in}{2.173527in}}%
\pgfpathlineto{\pgfqpoint{3.744504in}{2.175343in}}%
\pgfpathlineto{\pgfqpoint{3.767163in}{2.148811in}}%
\pgfpathlineto{\pgfqpoint{3.789822in}{2.143105in}}%
\pgfpathlineto{\pgfqpoint{3.812481in}{2.148774in}}%
\pgfpathlineto{\pgfqpoint{3.835139in}{2.162060in}}%
\pgfpathlineto{\pgfqpoint{3.857798in}{2.154584in}}%
\pgfpathlineto{\pgfqpoint{3.880457in}{2.164208in}}%
\pgfpathlineto{\pgfqpoint{3.903116in}{2.160645in}}%
\pgfpathlineto{\pgfqpoint{3.925774in}{2.166613in}}%
\pgfpathlineto{\pgfqpoint{3.948433in}{2.151700in}}%
\pgfpathlineto{\pgfqpoint{3.971092in}{2.140483in}}%
\pgfpathlineto{\pgfqpoint{3.993751in}{2.169055in}}%
\pgfpathlineto{\pgfqpoint{4.016409in}{2.155849in}}%
\pgfpathlineto{\pgfqpoint{4.039068in}{2.174753in}}%
\pgfpathlineto{\pgfqpoint{4.061727in}{2.182095in}}%
\pgfpathlineto{\pgfqpoint{4.084386in}{2.178000in}}%
\pgfpathlineto{\pgfqpoint{4.107044in}{2.160653in}}%
\pgfpathlineto{\pgfqpoint{4.129703in}{2.150806in}}%
\pgfpathlineto{\pgfqpoint{4.152362in}{2.159914in}}%
\pgfpathlineto{\pgfqpoint{4.220338in}{2.136252in}}%
\pgfpathlineto{\pgfqpoint{4.242997in}{2.151267in}}%
\pgfpathlineto{\pgfqpoint{4.265656in}{2.172001in}}%
\pgfpathlineto{\pgfqpoint{4.288314in}{2.154977in}}%
\pgfpathlineto{\pgfqpoint{4.310973in}{2.139910in}}%
\pgfpathlineto{\pgfqpoint{4.333632in}{2.153325in}}%
\pgfpathlineto{\pgfqpoint{4.356291in}{2.147898in}}%
\pgfpathlineto{\pgfqpoint{4.378949in}{2.140549in}}%
\pgfpathlineto{\pgfqpoint{4.401608in}{2.121883in}}%
\pgfpathlineto{\pgfqpoint{4.424267in}{2.118414in}}%
\pgfpathlineto{\pgfqpoint{4.446926in}{2.118725in}}%
\pgfpathlineto{\pgfqpoint{4.469584in}{2.138009in}}%
\pgfpathlineto{\pgfqpoint{4.492243in}{2.149671in}}%
\pgfpathlineto{\pgfqpoint{4.514902in}{2.151743in}}%
\pgfpathlineto{\pgfqpoint{4.537561in}{2.157556in}}%
\pgfpathlineto{\pgfqpoint{4.560219in}{2.151863in}}%
\pgfpathlineto{\pgfqpoint{4.582878in}{2.148055in}}%
\pgfpathlineto{\pgfqpoint{4.605537in}{2.149884in}}%
\pgfpathlineto{\pgfqpoint{4.628196in}{2.159210in}}%
\pgfpathlineto{\pgfqpoint{4.650854in}{2.149676in}}%
\pgfpathlineto{\pgfqpoint{4.673513in}{2.132487in}}%
\pgfpathlineto{\pgfqpoint{4.696172in}{2.143758in}}%
\pgfpathlineto{\pgfqpoint{4.718831in}{2.130367in}}%
\pgfpathlineto{\pgfqpoint{4.741489in}{2.137800in}}%
\pgfpathlineto{\pgfqpoint{4.764148in}{2.147166in}}%
\pgfpathlineto{\pgfqpoint{4.786807in}{2.149015in}}%
\pgfpathlineto{\pgfqpoint{4.809466in}{2.128142in}}%
\pgfpathlineto{\pgfqpoint{4.832124in}{2.132046in}}%
\pgfpathlineto{\pgfqpoint{4.854783in}{2.126376in}}%
\pgfpathlineto{\pgfqpoint{4.877442in}{2.143509in}}%
\pgfpathlineto{\pgfqpoint{4.900101in}{2.154941in}}%
\pgfpathlineto{\pgfqpoint{4.922759in}{2.143679in}}%
\pgfpathlineto{\pgfqpoint{4.968077in}{2.155188in}}%
\pgfpathlineto{\pgfqpoint{4.990735in}{2.134428in}}%
\pgfpathlineto{\pgfqpoint{5.013394in}{2.147738in}}%
\pgfpathlineto{\pgfqpoint{5.036053in}{2.155419in}}%
\pgfpathlineto{\pgfqpoint{5.058712in}{2.153516in}}%
\pgfpathlineto{\pgfqpoint{5.081370in}{2.147897in}}%
\pgfpathlineto{\pgfqpoint{5.104029in}{2.136569in}}%
\pgfpathlineto{\pgfqpoint{5.126688in}{2.155450in}}%
\pgfpathlineto{\pgfqpoint{5.149347in}{2.140218in}}%
\pgfpathlineto{\pgfqpoint{5.172005in}{2.121209in}}%
\pgfpathlineto{\pgfqpoint{5.194664in}{2.134361in}}%
\pgfpathlineto{\pgfqpoint{5.217323in}{2.151335in}}%
\pgfpathlineto{\pgfqpoint{5.239982in}{2.149378in}}%
\pgfpathlineto{\pgfqpoint{5.262640in}{2.160643in}}%
\pgfpathlineto{\pgfqpoint{5.285299in}{2.154869in}}%
\pgfpathlineto{\pgfqpoint{5.307958in}{2.162400in}}%
\pgfpathlineto{\pgfqpoint{5.330617in}{2.164197in}}%
\pgfpathlineto{\pgfqpoint{5.353275in}{2.177433in}}%
\pgfpathlineto{\pgfqpoint{5.375934in}{2.148966in}}%
\pgfpathlineto{\pgfqpoint{5.398593in}{2.141324in}}%
\pgfpathlineto{\pgfqpoint{5.421252in}{2.129837in}}%
\pgfpathlineto{\pgfqpoint{5.443910in}{2.110854in}}%
\pgfpathlineto{\pgfqpoint{5.466569in}{2.095646in}}%
\pgfpathlineto{\pgfqpoint{5.489228in}{2.099322in}}%
\pgfpathlineto{\pgfqpoint{5.511887in}{2.104908in}}%
\pgfpathlineto{\pgfqpoint{5.534545in}{2.085846in}}%
\pgfpathlineto{\pgfqpoint{5.534545in}{2.085846in}}%
\pgfusepath{stroke}%
\end{pgfscope}%
\begin{pgfscope}%
\pgfpathrectangle{\pgfqpoint{0.800000in}{0.528000in}}{\pgfqpoint{4.960000in}{3.696000in}}%
\pgfusepath{clip}%
\pgfsetrectcap%
\pgfsetroundjoin%
\pgfsetlinewidth{1.505625pt}%
\definecolor{currentstroke}{rgb}{1.000000,0.498039,0.054902}%
\pgfsetstrokecolor{currentstroke}%
\pgfsetdash{}{0pt}%
\pgfpathmoveto{\pgfqpoint{1.025455in}{1.067539in}}%
\pgfpathlineto{\pgfqpoint{1.048113in}{1.064422in}}%
\pgfpathlineto{\pgfqpoint{1.070772in}{1.059241in}}%
\pgfpathlineto{\pgfqpoint{1.093431in}{1.071188in}}%
\pgfpathlineto{\pgfqpoint{1.116090in}{1.179595in}}%
\pgfpathlineto{\pgfqpoint{1.138748in}{1.357585in}}%
\pgfpathlineto{\pgfqpoint{1.206725in}{2.011833in}}%
\pgfpathlineto{\pgfqpoint{1.229383in}{2.172844in}}%
\pgfpathlineto{\pgfqpoint{1.252042in}{2.321866in}}%
\pgfpathlineto{\pgfqpoint{1.274701in}{2.450115in}}%
\pgfpathlineto{\pgfqpoint{1.297360in}{2.533060in}}%
\pgfpathlineto{\pgfqpoint{1.320018in}{2.576229in}}%
\pgfpathlineto{\pgfqpoint{1.342677in}{2.601048in}}%
\pgfpathlineto{\pgfqpoint{1.365336in}{2.621627in}}%
\pgfpathlineto{\pgfqpoint{1.387995in}{2.605002in}}%
\pgfpathlineto{\pgfqpoint{1.410653in}{2.599360in}}%
\pgfpathlineto{\pgfqpoint{1.433312in}{2.567190in}}%
\pgfpathlineto{\pgfqpoint{1.455971in}{2.532819in}}%
\pgfpathlineto{\pgfqpoint{1.478630in}{2.492014in}}%
\pgfpathlineto{\pgfqpoint{1.523947in}{2.441791in}}%
\pgfpathlineto{\pgfqpoint{1.546606in}{2.440673in}}%
\pgfpathlineto{\pgfqpoint{1.569265in}{2.427125in}}%
\pgfpathlineto{\pgfqpoint{1.591923in}{2.460883in}}%
\pgfpathlineto{\pgfqpoint{1.614582in}{2.439578in}}%
\pgfpathlineto{\pgfqpoint{1.637241in}{2.400366in}}%
\pgfpathlineto{\pgfqpoint{1.659899in}{2.430182in}}%
\pgfpathlineto{\pgfqpoint{1.682558in}{2.438127in}}%
\pgfpathlineto{\pgfqpoint{1.705217in}{2.450023in}}%
\pgfpathlineto{\pgfqpoint{1.727876in}{2.429090in}}%
\pgfpathlineto{\pgfqpoint{1.750534in}{2.420017in}}%
\pgfpathlineto{\pgfqpoint{1.773193in}{2.397808in}}%
\pgfpathlineto{\pgfqpoint{1.795852in}{2.357158in}}%
\pgfpathlineto{\pgfqpoint{1.818511in}{2.346637in}}%
\pgfpathlineto{\pgfqpoint{1.841169in}{2.340380in}}%
\pgfpathlineto{\pgfqpoint{1.863828in}{2.359287in}}%
\pgfpathlineto{\pgfqpoint{1.886487in}{2.344558in}}%
\pgfpathlineto{\pgfqpoint{1.909146in}{2.314662in}}%
\pgfpathlineto{\pgfqpoint{1.931804in}{2.312783in}}%
\pgfpathlineto{\pgfqpoint{1.954463in}{2.312139in}}%
\pgfpathlineto{\pgfqpoint{1.977122in}{2.327665in}}%
\pgfpathlineto{\pgfqpoint{1.999781in}{2.299392in}}%
\pgfpathlineto{\pgfqpoint{2.022439in}{2.288404in}}%
\pgfpathlineto{\pgfqpoint{2.045098in}{2.279672in}}%
\pgfpathlineto{\pgfqpoint{2.090416in}{2.275220in}}%
\pgfpathlineto{\pgfqpoint{2.113074in}{2.253233in}}%
\pgfpathlineto{\pgfqpoint{2.135733in}{2.246925in}}%
\pgfpathlineto{\pgfqpoint{2.158392in}{2.246949in}}%
\pgfpathlineto{\pgfqpoint{2.181051in}{2.224329in}}%
\pgfpathlineto{\pgfqpoint{2.203709in}{2.199286in}}%
\pgfpathlineto{\pgfqpoint{2.226368in}{2.188824in}}%
\pgfpathlineto{\pgfqpoint{2.249027in}{2.198095in}}%
\pgfpathlineto{\pgfqpoint{2.271686in}{2.172875in}}%
\pgfpathlineto{\pgfqpoint{2.294344in}{2.143882in}}%
\pgfpathlineto{\pgfqpoint{2.317003in}{2.164766in}}%
\pgfpathlineto{\pgfqpoint{2.362321in}{2.162786in}}%
\pgfpathlineto{\pgfqpoint{2.384979in}{2.151502in}}%
\pgfpathlineto{\pgfqpoint{2.407638in}{2.170808in}}%
\pgfpathlineto{\pgfqpoint{2.430297in}{2.183848in}}%
\pgfpathlineto{\pgfqpoint{2.452956in}{2.151953in}}%
\pgfpathlineto{\pgfqpoint{2.475614in}{2.152163in}}%
\pgfpathlineto{\pgfqpoint{2.498273in}{2.139777in}}%
\pgfpathlineto{\pgfqpoint{2.520932in}{2.153616in}}%
\pgfpathlineto{\pgfqpoint{2.543591in}{2.148591in}}%
\pgfpathlineto{\pgfqpoint{2.566249in}{2.135508in}}%
\pgfpathlineto{\pgfqpoint{2.588908in}{2.113765in}}%
\pgfpathlineto{\pgfqpoint{2.634226in}{2.160608in}}%
\pgfpathlineto{\pgfqpoint{2.656884in}{2.161031in}}%
\pgfpathlineto{\pgfqpoint{2.679543in}{2.158579in}}%
\pgfpathlineto{\pgfqpoint{2.702202in}{2.138447in}}%
\pgfpathlineto{\pgfqpoint{2.724861in}{2.135135in}}%
\pgfpathlineto{\pgfqpoint{2.747519in}{2.147325in}}%
\pgfpathlineto{\pgfqpoint{2.770178in}{2.126035in}}%
\pgfpathlineto{\pgfqpoint{2.792837in}{2.126007in}}%
\pgfpathlineto{\pgfqpoint{2.815496in}{2.152977in}}%
\pgfpathlineto{\pgfqpoint{2.838154in}{2.153840in}}%
\pgfpathlineto{\pgfqpoint{2.860813in}{2.149050in}}%
\pgfpathlineto{\pgfqpoint{2.883472in}{2.149868in}}%
\pgfpathlineto{\pgfqpoint{2.906131in}{2.142663in}}%
\pgfpathlineto{\pgfqpoint{2.928789in}{2.148249in}}%
\pgfpathlineto{\pgfqpoint{2.951448in}{2.138259in}}%
\pgfpathlineto{\pgfqpoint{2.974107in}{2.162423in}}%
\pgfpathlineto{\pgfqpoint{2.996766in}{2.148377in}}%
\pgfpathlineto{\pgfqpoint{3.019424in}{2.174199in}}%
\pgfpathlineto{\pgfqpoint{3.042083in}{2.143203in}}%
\pgfpathlineto{\pgfqpoint{3.064742in}{2.108275in}}%
\pgfpathlineto{\pgfqpoint{3.087401in}{2.103666in}}%
\pgfpathlineto{\pgfqpoint{3.110059in}{2.087988in}}%
\pgfpathlineto{\pgfqpoint{3.132718in}{2.110621in}}%
\pgfpathlineto{\pgfqpoint{3.155377in}{2.085120in}}%
\pgfpathlineto{\pgfqpoint{3.178036in}{2.085013in}}%
\pgfpathlineto{\pgfqpoint{3.200694in}{2.060437in}}%
\pgfpathlineto{\pgfqpoint{3.223353in}{2.067730in}}%
\pgfpathlineto{\pgfqpoint{3.246012in}{2.063071in}}%
\pgfpathlineto{\pgfqpoint{3.268671in}{2.050429in}}%
\pgfpathlineto{\pgfqpoint{3.291329in}{2.031922in}}%
\pgfpathlineto{\pgfqpoint{3.313988in}{2.045520in}}%
\pgfpathlineto{\pgfqpoint{3.336647in}{2.073910in}}%
\pgfpathlineto{\pgfqpoint{3.359306in}{2.079145in}}%
\pgfpathlineto{\pgfqpoint{3.381964in}{2.094999in}}%
\pgfpathlineto{\pgfqpoint{3.404623in}{2.059049in}}%
\pgfpathlineto{\pgfqpoint{3.427282in}{2.038226in}}%
\pgfpathlineto{\pgfqpoint{3.449941in}{2.019627in}}%
\pgfpathlineto{\pgfqpoint{3.472599in}{2.030265in}}%
\pgfpathlineto{\pgfqpoint{3.517917in}{2.008578in}}%
\pgfpathlineto{\pgfqpoint{3.540576in}{2.015228in}}%
\pgfpathlineto{\pgfqpoint{3.563234in}{2.002213in}}%
\pgfpathlineto{\pgfqpoint{3.585893in}{1.961554in}}%
\pgfpathlineto{\pgfqpoint{3.608552in}{1.969901in}}%
\pgfpathlineto{\pgfqpoint{3.631211in}{1.963865in}}%
\pgfpathlineto{\pgfqpoint{3.653869in}{2.004265in}}%
\pgfpathlineto{\pgfqpoint{3.676528in}{1.984487in}}%
\pgfpathlineto{\pgfqpoint{3.699187in}{1.970208in}}%
\pgfpathlineto{\pgfqpoint{3.721846in}{1.966469in}}%
\pgfpathlineto{\pgfqpoint{3.744504in}{1.956363in}}%
\pgfpathlineto{\pgfqpoint{3.767163in}{1.923384in}}%
\pgfpathlineto{\pgfqpoint{3.789822in}{1.932320in}}%
\pgfpathlineto{\pgfqpoint{3.812481in}{1.960259in}}%
\pgfpathlineto{\pgfqpoint{3.835139in}{1.942507in}}%
\pgfpathlineto{\pgfqpoint{3.857798in}{1.951284in}}%
\pgfpathlineto{\pgfqpoint{3.880457in}{1.981824in}}%
\pgfpathlineto{\pgfqpoint{3.903116in}{2.001977in}}%
\pgfpathlineto{\pgfqpoint{3.925774in}{1.964176in}}%
\pgfpathlineto{\pgfqpoint{3.948433in}{1.947116in}}%
\pgfpathlineto{\pgfqpoint{3.971092in}{1.952392in}}%
\pgfpathlineto{\pgfqpoint{3.993751in}{1.994440in}}%
\pgfpathlineto{\pgfqpoint{4.016409in}{1.973372in}}%
\pgfpathlineto{\pgfqpoint{4.039068in}{1.949854in}}%
\pgfpathlineto{\pgfqpoint{4.061727in}{1.941322in}}%
\pgfpathlineto{\pgfqpoint{4.084386in}{1.944278in}}%
\pgfpathlineto{\pgfqpoint{4.107044in}{1.943193in}}%
\pgfpathlineto{\pgfqpoint{4.129703in}{1.909721in}}%
\pgfpathlineto{\pgfqpoint{4.152362in}{1.889616in}}%
\pgfpathlineto{\pgfqpoint{4.175021in}{1.892736in}}%
\pgfpathlineto{\pgfqpoint{4.197679in}{1.890906in}}%
\pgfpathlineto{\pgfqpoint{4.220338in}{1.923966in}}%
\pgfpathlineto{\pgfqpoint{4.242997in}{1.889636in}}%
\pgfpathlineto{\pgfqpoint{4.265656in}{1.876646in}}%
\pgfpathlineto{\pgfqpoint{4.288314in}{1.864941in}}%
\pgfpathlineto{\pgfqpoint{4.310973in}{1.871108in}}%
\pgfpathlineto{\pgfqpoint{4.333632in}{1.856589in}}%
\pgfpathlineto{\pgfqpoint{4.356291in}{1.879335in}}%
\pgfpathlineto{\pgfqpoint{4.378949in}{1.892368in}}%
\pgfpathlineto{\pgfqpoint{4.401608in}{1.866098in}}%
\pgfpathlineto{\pgfqpoint{4.424267in}{1.850034in}}%
\pgfpathlineto{\pgfqpoint{4.446926in}{1.827117in}}%
\pgfpathlineto{\pgfqpoint{4.469584in}{1.852319in}}%
\pgfpathlineto{\pgfqpoint{4.492243in}{1.873601in}}%
\pgfpathlineto{\pgfqpoint{4.514902in}{1.887351in}}%
\pgfpathlineto{\pgfqpoint{4.537561in}{1.850268in}}%
\pgfpathlineto{\pgfqpoint{4.560219in}{1.821175in}}%
\pgfpathlineto{\pgfqpoint{4.582878in}{1.855158in}}%
\pgfpathlineto{\pgfqpoint{4.605537in}{1.858481in}}%
\pgfpathlineto{\pgfqpoint{4.628196in}{1.858892in}}%
\pgfpathlineto{\pgfqpoint{4.650854in}{1.874741in}}%
\pgfpathlineto{\pgfqpoint{4.673513in}{1.862002in}}%
\pgfpathlineto{\pgfqpoint{4.696172in}{1.887319in}}%
\pgfpathlineto{\pgfqpoint{4.718831in}{1.909325in}}%
\pgfpathlineto{\pgfqpoint{4.741489in}{1.903762in}}%
\pgfpathlineto{\pgfqpoint{4.764148in}{1.904511in}}%
\pgfpathlineto{\pgfqpoint{4.786807in}{1.888323in}}%
\pgfpathlineto{\pgfqpoint{4.809466in}{1.894985in}}%
\pgfpathlineto{\pgfqpoint{4.832124in}{1.924097in}}%
\pgfpathlineto{\pgfqpoint{4.854783in}{1.948637in}}%
\pgfpathlineto{\pgfqpoint{4.900101in}{1.966325in}}%
\pgfpathlineto{\pgfqpoint{4.922759in}{1.985942in}}%
\pgfpathlineto{\pgfqpoint{4.945418in}{2.003392in}}%
\pgfpathlineto{\pgfqpoint{4.968077in}{2.028928in}}%
\pgfpathlineto{\pgfqpoint{4.990735in}{2.022788in}}%
\pgfpathlineto{\pgfqpoint{5.013394in}{2.040259in}}%
\pgfpathlineto{\pgfqpoint{5.036053in}{2.045488in}}%
\pgfpathlineto{\pgfqpoint{5.058712in}{2.068397in}}%
\pgfpathlineto{\pgfqpoint{5.081370in}{2.079902in}}%
\pgfpathlineto{\pgfqpoint{5.104029in}{2.088873in}}%
\pgfpathlineto{\pgfqpoint{5.126688in}{2.113902in}}%
\pgfpathlineto{\pgfqpoint{5.149347in}{2.092508in}}%
\pgfpathlineto{\pgfqpoint{5.172005in}{2.103277in}}%
\pgfpathlineto{\pgfqpoint{5.194664in}{2.082049in}}%
\pgfpathlineto{\pgfqpoint{5.217323in}{2.117583in}}%
\pgfpathlineto{\pgfqpoint{5.239982in}{2.115200in}}%
\pgfpathlineto{\pgfqpoint{5.262640in}{2.093627in}}%
\pgfpathlineto{\pgfqpoint{5.285299in}{2.092499in}}%
\pgfpathlineto{\pgfqpoint{5.330617in}{2.085320in}}%
\pgfpathlineto{\pgfqpoint{5.353275in}{2.053737in}}%
\pgfpathlineto{\pgfqpoint{5.375934in}{2.052898in}}%
\pgfpathlineto{\pgfqpoint{5.398593in}{2.077488in}}%
\pgfpathlineto{\pgfqpoint{5.421252in}{2.095627in}}%
\pgfpathlineto{\pgfqpoint{5.443910in}{2.099472in}}%
\pgfpathlineto{\pgfqpoint{5.466569in}{2.085060in}}%
\pgfpathlineto{\pgfqpoint{5.489228in}{2.075892in}}%
\pgfpathlineto{\pgfqpoint{5.511887in}{2.079138in}}%
\pgfpathlineto{\pgfqpoint{5.534545in}{2.062523in}}%
\pgfpathlineto{\pgfqpoint{5.534545in}{2.062523in}}%
\pgfusepath{stroke}%
\end{pgfscope}%
\begin{pgfscope}%
\pgfpathrectangle{\pgfqpoint{0.800000in}{0.528000in}}{\pgfqpoint{4.960000in}{3.696000in}}%
\pgfusepath{clip}%
\pgfsetrectcap%
\pgfsetroundjoin%
\pgfsetlinewidth{1.505625pt}%
\definecolor{currentstroke}{rgb}{0.172549,0.627451,0.172549}%
\pgfsetstrokecolor{currentstroke}%
\pgfsetdash{}{0pt}%
\pgfpathmoveto{\pgfqpoint{1.025455in}{1.046096in}}%
\pgfpathlineto{\pgfqpoint{1.048113in}{1.034642in}}%
\pgfpathlineto{\pgfqpoint{1.070772in}{1.325185in}}%
\pgfpathlineto{\pgfqpoint{1.093431in}{1.742631in}}%
\pgfpathlineto{\pgfqpoint{1.138748in}{2.667052in}}%
\pgfpathlineto{\pgfqpoint{1.161407in}{2.849828in}}%
\pgfpathlineto{\pgfqpoint{1.184066in}{2.807582in}}%
\pgfpathlineto{\pgfqpoint{1.206725in}{2.648279in}}%
\pgfpathlineto{\pgfqpoint{1.229383in}{2.470446in}}%
\pgfpathlineto{\pgfqpoint{1.252042in}{2.395121in}}%
\pgfpathlineto{\pgfqpoint{1.274701in}{2.403843in}}%
\pgfpathlineto{\pgfqpoint{1.297360in}{2.355634in}}%
\pgfpathlineto{\pgfqpoint{1.320018in}{2.440481in}}%
\pgfpathlineto{\pgfqpoint{1.342677in}{2.448404in}}%
\pgfpathlineto{\pgfqpoint{1.365336in}{2.402042in}}%
\pgfpathlineto{\pgfqpoint{1.387995in}{2.350957in}}%
\pgfpathlineto{\pgfqpoint{1.410653in}{2.354977in}}%
\pgfpathlineto{\pgfqpoint{1.433312in}{2.322413in}}%
\pgfpathlineto{\pgfqpoint{1.455971in}{2.321860in}}%
\pgfpathlineto{\pgfqpoint{1.478630in}{2.281335in}}%
\pgfpathlineto{\pgfqpoint{1.501288in}{2.260067in}}%
\pgfpathlineto{\pgfqpoint{1.523947in}{2.243797in}}%
\pgfpathlineto{\pgfqpoint{1.546606in}{2.229130in}}%
\pgfpathlineto{\pgfqpoint{1.591923in}{2.178087in}}%
\pgfpathlineto{\pgfqpoint{1.614582in}{2.170361in}}%
\pgfpathlineto{\pgfqpoint{1.637241in}{2.183302in}}%
\pgfpathlineto{\pgfqpoint{1.659899in}{2.214267in}}%
\pgfpathlineto{\pgfqpoint{1.682558in}{2.131346in}}%
\pgfpathlineto{\pgfqpoint{1.705217in}{2.130429in}}%
\pgfpathlineto{\pgfqpoint{1.727876in}{2.128321in}}%
\pgfpathlineto{\pgfqpoint{1.750534in}{2.152371in}}%
\pgfpathlineto{\pgfqpoint{1.773193in}{2.185869in}}%
\pgfpathlineto{\pgfqpoint{1.795852in}{2.154457in}}%
\pgfpathlineto{\pgfqpoint{1.818511in}{2.132448in}}%
\pgfpathlineto{\pgfqpoint{1.841169in}{2.159217in}}%
\pgfpathlineto{\pgfqpoint{1.863828in}{2.166713in}}%
\pgfpathlineto{\pgfqpoint{1.886487in}{2.162896in}}%
\pgfpathlineto{\pgfqpoint{1.909146in}{2.150178in}}%
\pgfpathlineto{\pgfqpoint{1.931804in}{2.171108in}}%
\pgfpathlineto{\pgfqpoint{1.954463in}{2.115354in}}%
\pgfpathlineto{\pgfqpoint{1.977122in}{2.077108in}}%
\pgfpathlineto{\pgfqpoint{1.999781in}{2.086168in}}%
\pgfpathlineto{\pgfqpoint{2.022439in}{2.062761in}}%
\pgfpathlineto{\pgfqpoint{2.045098in}{2.076928in}}%
\pgfpathlineto{\pgfqpoint{2.067757in}{2.080995in}}%
\pgfpathlineto{\pgfqpoint{2.090416in}{2.104821in}}%
\pgfpathlineto{\pgfqpoint{2.113074in}{2.095028in}}%
\pgfpathlineto{\pgfqpoint{2.135733in}{2.025933in}}%
\pgfpathlineto{\pgfqpoint{2.158392in}{2.016518in}}%
\pgfpathlineto{\pgfqpoint{2.181051in}{2.011953in}}%
\pgfpathlineto{\pgfqpoint{2.203709in}{1.973213in}}%
\pgfpathlineto{\pgfqpoint{2.226368in}{1.980535in}}%
\pgfpathlineto{\pgfqpoint{2.249027in}{1.981925in}}%
\pgfpathlineto{\pgfqpoint{2.271686in}{1.957768in}}%
\pgfpathlineto{\pgfqpoint{2.294344in}{1.935870in}}%
\pgfpathlineto{\pgfqpoint{2.317003in}{1.932923in}}%
\pgfpathlineto{\pgfqpoint{2.339662in}{1.973050in}}%
\pgfpathlineto{\pgfqpoint{2.362321in}{1.959992in}}%
\pgfpathlineto{\pgfqpoint{2.384979in}{1.997638in}}%
\pgfpathlineto{\pgfqpoint{2.407638in}{1.987173in}}%
\pgfpathlineto{\pgfqpoint{2.430297in}{1.935390in}}%
\pgfpathlineto{\pgfqpoint{2.452956in}{1.902779in}}%
\pgfpathlineto{\pgfqpoint{2.475614in}{1.877148in}}%
\pgfpathlineto{\pgfqpoint{2.498273in}{1.900137in}}%
\pgfpathlineto{\pgfqpoint{2.520932in}{1.887603in}}%
\pgfpathlineto{\pgfqpoint{2.543591in}{1.887490in}}%
\pgfpathlineto{\pgfqpoint{2.566249in}{1.897415in}}%
\pgfpathlineto{\pgfqpoint{2.588908in}{1.847487in}}%
\pgfpathlineto{\pgfqpoint{2.611567in}{1.854069in}}%
\pgfpathlineto{\pgfqpoint{2.634226in}{1.865151in}}%
\pgfpathlineto{\pgfqpoint{2.656884in}{1.850984in}}%
\pgfpathlineto{\pgfqpoint{2.679543in}{1.864421in}}%
\pgfpathlineto{\pgfqpoint{2.702202in}{1.864395in}}%
\pgfpathlineto{\pgfqpoint{2.724861in}{1.902288in}}%
\pgfpathlineto{\pgfqpoint{2.747519in}{1.909962in}}%
\pgfpathlineto{\pgfqpoint{2.770178in}{1.920390in}}%
\pgfpathlineto{\pgfqpoint{2.792837in}{1.966135in}}%
\pgfpathlineto{\pgfqpoint{2.815496in}{1.969901in}}%
\pgfpathlineto{\pgfqpoint{2.838154in}{2.014279in}}%
\pgfpathlineto{\pgfqpoint{2.860813in}{2.002874in}}%
\pgfpathlineto{\pgfqpoint{2.883472in}{2.073638in}}%
\pgfpathlineto{\pgfqpoint{2.906131in}{2.103999in}}%
\pgfpathlineto{\pgfqpoint{2.928789in}{2.094034in}}%
\pgfpathlineto{\pgfqpoint{2.951448in}{2.116193in}}%
\pgfpathlineto{\pgfqpoint{2.974107in}{2.146246in}}%
\pgfpathlineto{\pgfqpoint{2.996766in}{2.079339in}}%
\pgfpathlineto{\pgfqpoint{3.019424in}{2.025916in}}%
\pgfpathlineto{\pgfqpoint{3.042083in}{2.058910in}}%
\pgfpathlineto{\pgfqpoint{3.064742in}{2.077919in}}%
\pgfpathlineto{\pgfqpoint{3.087401in}{2.074017in}}%
\pgfpathlineto{\pgfqpoint{3.110059in}{2.068942in}}%
\pgfpathlineto{\pgfqpoint{3.132718in}{2.102712in}}%
\pgfpathlineto{\pgfqpoint{3.155377in}{2.082704in}}%
\pgfpathlineto{\pgfqpoint{3.178036in}{2.055331in}}%
\pgfpathlineto{\pgfqpoint{3.200694in}{2.021799in}}%
\pgfpathlineto{\pgfqpoint{3.223353in}{2.005033in}}%
\pgfpathlineto{\pgfqpoint{3.246012in}{2.046556in}}%
\pgfpathlineto{\pgfqpoint{3.268671in}{2.044060in}}%
\pgfpathlineto{\pgfqpoint{3.291329in}{2.069019in}}%
\pgfpathlineto{\pgfqpoint{3.313988in}{2.038196in}}%
\pgfpathlineto{\pgfqpoint{3.336647in}{2.024051in}}%
\pgfpathlineto{\pgfqpoint{3.359306in}{2.058937in}}%
\pgfpathlineto{\pgfqpoint{3.381964in}{2.046831in}}%
\pgfpathlineto{\pgfqpoint{3.404623in}{2.050416in}}%
\pgfpathlineto{\pgfqpoint{3.427282in}{2.014053in}}%
\pgfpathlineto{\pgfqpoint{3.449941in}{2.013614in}}%
\pgfpathlineto{\pgfqpoint{3.472599in}{2.026713in}}%
\pgfpathlineto{\pgfqpoint{3.495258in}{2.003539in}}%
\pgfpathlineto{\pgfqpoint{3.517917in}{1.986343in}}%
\pgfpathlineto{\pgfqpoint{3.540576in}{1.949081in}}%
\pgfpathlineto{\pgfqpoint{3.563234in}{1.933274in}}%
\pgfpathlineto{\pgfqpoint{3.585893in}{1.945356in}}%
\pgfpathlineto{\pgfqpoint{3.608552in}{1.989667in}}%
\pgfpathlineto{\pgfqpoint{3.631211in}{1.978959in}}%
\pgfpathlineto{\pgfqpoint{3.653869in}{1.907386in}}%
\pgfpathlineto{\pgfqpoint{3.676528in}{1.881529in}}%
\pgfpathlineto{\pgfqpoint{3.699187in}{1.891093in}}%
\pgfpathlineto{\pgfqpoint{3.721846in}{1.884676in}}%
\pgfpathlineto{\pgfqpoint{3.744504in}{1.895281in}}%
\pgfpathlineto{\pgfqpoint{3.767163in}{1.915394in}}%
\pgfpathlineto{\pgfqpoint{3.789822in}{1.877542in}}%
\pgfpathlineto{\pgfqpoint{3.812481in}{1.892672in}}%
\pgfpathlineto{\pgfqpoint{3.835139in}{1.881624in}}%
\pgfpathlineto{\pgfqpoint{3.857798in}{1.928786in}}%
\pgfpathlineto{\pgfqpoint{3.880457in}{1.947668in}}%
\pgfpathlineto{\pgfqpoint{3.903116in}{1.951922in}}%
\pgfpathlineto{\pgfqpoint{3.925774in}{2.008005in}}%
\pgfpathlineto{\pgfqpoint{3.948433in}{1.995390in}}%
\pgfpathlineto{\pgfqpoint{3.971092in}{2.011872in}}%
\pgfpathlineto{\pgfqpoint{3.993751in}{1.989885in}}%
\pgfpathlineto{\pgfqpoint{4.016409in}{1.980315in}}%
\pgfpathlineto{\pgfqpoint{4.039068in}{1.976511in}}%
\pgfpathlineto{\pgfqpoint{4.061727in}{1.959388in}}%
\pgfpathlineto{\pgfqpoint{4.084386in}{1.985485in}}%
\pgfpathlineto{\pgfqpoint{4.107044in}{2.041442in}}%
\pgfpathlineto{\pgfqpoint{4.129703in}{2.027256in}}%
\pgfpathlineto{\pgfqpoint{4.152362in}{2.022383in}}%
\pgfpathlineto{\pgfqpoint{4.175021in}{2.071258in}}%
\pgfpathlineto{\pgfqpoint{4.197679in}{2.040646in}}%
\pgfpathlineto{\pgfqpoint{4.220338in}{2.079099in}}%
\pgfpathlineto{\pgfqpoint{4.242997in}{2.039779in}}%
\pgfpathlineto{\pgfqpoint{4.265656in}{2.063113in}}%
\pgfpathlineto{\pgfqpoint{4.288314in}{2.165705in}}%
\pgfpathlineto{\pgfqpoint{4.310973in}{2.189106in}}%
\pgfpathlineto{\pgfqpoint{4.333632in}{2.259446in}}%
\pgfpathlineto{\pgfqpoint{4.356291in}{2.254561in}}%
\pgfpathlineto{\pgfqpoint{4.378949in}{2.246076in}}%
\pgfpathlineto{\pgfqpoint{4.401608in}{2.240392in}}%
\pgfpathlineto{\pgfqpoint{4.424267in}{2.192920in}}%
\pgfpathlineto{\pgfqpoint{4.446926in}{2.248356in}}%
\pgfpathlineto{\pgfqpoint{4.469584in}{2.247790in}}%
\pgfpathlineto{\pgfqpoint{4.492243in}{2.218380in}}%
\pgfpathlineto{\pgfqpoint{4.514902in}{2.183734in}}%
\pgfpathlineto{\pgfqpoint{4.537561in}{2.171841in}}%
\pgfpathlineto{\pgfqpoint{4.582878in}{2.177224in}}%
\pgfpathlineto{\pgfqpoint{4.605537in}{2.181573in}}%
\pgfpathlineto{\pgfqpoint{4.628196in}{2.191387in}}%
\pgfpathlineto{\pgfqpoint{4.650854in}{2.139797in}}%
\pgfpathlineto{\pgfqpoint{4.673513in}{2.145297in}}%
\pgfpathlineto{\pgfqpoint{4.696172in}{2.160724in}}%
\pgfpathlineto{\pgfqpoint{4.718831in}{2.137276in}}%
\pgfpathlineto{\pgfqpoint{4.764148in}{2.164995in}}%
\pgfpathlineto{\pgfqpoint{4.786807in}{2.167130in}}%
\pgfpathlineto{\pgfqpoint{4.809466in}{2.138752in}}%
\pgfpathlineto{\pgfqpoint{4.832124in}{2.148200in}}%
\pgfpathlineto{\pgfqpoint{4.854783in}{2.167557in}}%
\pgfpathlineto{\pgfqpoint{4.877442in}{2.144582in}}%
\pgfpathlineto{\pgfqpoint{4.900101in}{2.095863in}}%
\pgfpathlineto{\pgfqpoint{4.922759in}{2.100899in}}%
\pgfpathlineto{\pgfqpoint{4.945418in}{2.149722in}}%
\pgfpathlineto{\pgfqpoint{4.968077in}{2.094099in}}%
\pgfpathlineto{\pgfqpoint{4.990735in}{2.028271in}}%
\pgfpathlineto{\pgfqpoint{5.013394in}{2.008737in}}%
\pgfpathlineto{\pgfqpoint{5.036053in}{2.007341in}}%
\pgfpathlineto{\pgfqpoint{5.058712in}{1.957705in}}%
\pgfpathlineto{\pgfqpoint{5.081370in}{1.976987in}}%
\pgfpathlineto{\pgfqpoint{5.104029in}{2.014404in}}%
\pgfpathlineto{\pgfqpoint{5.126688in}{2.009576in}}%
\pgfpathlineto{\pgfqpoint{5.149347in}{2.043100in}}%
\pgfpathlineto{\pgfqpoint{5.172005in}{2.019498in}}%
\pgfpathlineto{\pgfqpoint{5.194664in}{2.033917in}}%
\pgfpathlineto{\pgfqpoint{5.217323in}{2.054109in}}%
\pgfpathlineto{\pgfqpoint{5.239982in}{2.095128in}}%
\pgfpathlineto{\pgfqpoint{5.262640in}{2.095643in}}%
\pgfpathlineto{\pgfqpoint{5.285299in}{2.119310in}}%
\pgfpathlineto{\pgfqpoint{5.307958in}{2.039179in}}%
\pgfpathlineto{\pgfqpoint{5.330617in}{2.028382in}}%
\pgfpathlineto{\pgfqpoint{5.353275in}{1.969776in}}%
\pgfpathlineto{\pgfqpoint{5.375934in}{2.013055in}}%
\pgfpathlineto{\pgfqpoint{5.398593in}{1.983126in}}%
\pgfpathlineto{\pgfqpoint{5.421252in}{2.010011in}}%
\pgfpathlineto{\pgfqpoint{5.443910in}{1.969309in}}%
\pgfpathlineto{\pgfqpoint{5.466569in}{1.974768in}}%
\pgfpathlineto{\pgfqpoint{5.489228in}{2.053307in}}%
\pgfpathlineto{\pgfqpoint{5.511887in}{2.062752in}}%
\pgfpathlineto{\pgfqpoint{5.534545in}{2.045616in}}%
\pgfpathlineto{\pgfqpoint{5.534545in}{2.045616in}}%
\pgfusepath{stroke}%
\end{pgfscope}%
\begin{pgfscope}%
\pgfpathrectangle{\pgfqpoint{0.800000in}{0.528000in}}{\pgfqpoint{4.960000in}{3.696000in}}%
\pgfusepath{clip}%
\pgfsetrectcap%
\pgfsetroundjoin%
\pgfsetlinewidth{1.505625pt}%
\definecolor{currentstroke}{rgb}{0.839216,0.152941,0.156863}%
\pgfsetstrokecolor{currentstroke}%
\pgfsetdash{}{0pt}%
\pgfpathmoveto{\pgfqpoint{1.025455in}{0.696000in}}%
\pgfpathlineto{\pgfqpoint{1.048113in}{0.959260in}}%
\pgfpathlineto{\pgfqpoint{1.070772in}{4.056000in}}%
\pgfpathlineto{\pgfqpoint{1.093431in}{3.440369in}}%
\pgfpathlineto{\pgfqpoint{1.116090in}{2.317351in}}%
\pgfpathlineto{\pgfqpoint{1.138748in}{2.308027in}}%
\pgfpathlineto{\pgfqpoint{1.161407in}{2.675908in}}%
\pgfpathlineto{\pgfqpoint{1.184066in}{2.521507in}}%
\pgfpathlineto{\pgfqpoint{1.206725in}{2.525871in}}%
\pgfpathlineto{\pgfqpoint{1.229383in}{2.342558in}}%
\pgfpathlineto{\pgfqpoint{1.252042in}{2.441767in}}%
\pgfpathlineto{\pgfqpoint{1.274701in}{2.352822in}}%
\pgfpathlineto{\pgfqpoint{1.297360in}{2.545897in}}%
\pgfpathlineto{\pgfqpoint{1.320018in}{2.175347in}}%
\pgfpathlineto{\pgfqpoint{1.342677in}{2.341299in}}%
\pgfpathlineto{\pgfqpoint{1.365336in}{2.381069in}}%
\pgfpathlineto{\pgfqpoint{1.387995in}{2.319663in}}%
\pgfpathlineto{\pgfqpoint{1.410653in}{2.386280in}}%
\pgfpathlineto{\pgfqpoint{1.433312in}{2.301247in}}%
\pgfpathlineto{\pgfqpoint{1.455971in}{2.240812in}}%
\pgfpathlineto{\pgfqpoint{1.478630in}{2.269395in}}%
\pgfpathlineto{\pgfqpoint{1.501288in}{2.381115in}}%
\pgfpathlineto{\pgfqpoint{1.523947in}{2.200137in}}%
\pgfpathlineto{\pgfqpoint{1.546606in}{2.198985in}}%
\pgfpathlineto{\pgfqpoint{1.569265in}{2.161690in}}%
\pgfpathlineto{\pgfqpoint{1.591923in}{2.084792in}}%
\pgfpathlineto{\pgfqpoint{1.614582in}{2.057487in}}%
\pgfpathlineto{\pgfqpoint{1.637241in}{2.214957in}}%
\pgfpathlineto{\pgfqpoint{1.659899in}{2.135746in}}%
\pgfpathlineto{\pgfqpoint{1.682558in}{1.943000in}}%
\pgfpathlineto{\pgfqpoint{1.705217in}{2.193278in}}%
\pgfpathlineto{\pgfqpoint{1.727876in}{2.020325in}}%
\pgfpathlineto{\pgfqpoint{1.750534in}{2.047262in}}%
\pgfpathlineto{\pgfqpoint{1.773193in}{2.126415in}}%
\pgfpathlineto{\pgfqpoint{1.795852in}{2.085817in}}%
\pgfpathlineto{\pgfqpoint{1.818511in}{2.199494in}}%
\pgfpathlineto{\pgfqpoint{1.841169in}{2.167617in}}%
\pgfpathlineto{\pgfqpoint{1.863828in}{2.272370in}}%
\pgfpathlineto{\pgfqpoint{1.886487in}{2.392087in}}%
\pgfpathlineto{\pgfqpoint{1.909146in}{2.290082in}}%
\pgfpathlineto{\pgfqpoint{1.931804in}{2.080943in}}%
\pgfpathlineto{\pgfqpoint{1.954463in}{2.296929in}}%
\pgfpathlineto{\pgfqpoint{1.977122in}{2.253950in}}%
\pgfpathlineto{\pgfqpoint{1.999781in}{2.232294in}}%
\pgfpathlineto{\pgfqpoint{2.022439in}{2.164293in}}%
\pgfpathlineto{\pgfqpoint{2.045098in}{2.263254in}}%
\pgfpathlineto{\pgfqpoint{2.067757in}{2.307431in}}%
\pgfpathlineto{\pgfqpoint{2.090416in}{2.432679in}}%
\pgfpathlineto{\pgfqpoint{2.113074in}{2.111326in}}%
\pgfpathlineto{\pgfqpoint{2.135733in}{2.212443in}}%
\pgfpathlineto{\pgfqpoint{2.158392in}{2.060590in}}%
\pgfpathlineto{\pgfqpoint{2.181051in}{2.044323in}}%
\pgfpathlineto{\pgfqpoint{2.203709in}{2.112050in}}%
\pgfpathlineto{\pgfqpoint{2.226368in}{2.031718in}}%
\pgfpathlineto{\pgfqpoint{2.249027in}{2.031088in}}%
\pgfpathlineto{\pgfqpoint{2.271686in}{2.227082in}}%
\pgfpathlineto{\pgfqpoint{2.294344in}{2.263790in}}%
\pgfpathlineto{\pgfqpoint{2.317003in}{2.057190in}}%
\pgfpathlineto{\pgfqpoint{2.339662in}{2.211040in}}%
\pgfpathlineto{\pgfqpoint{2.362321in}{2.374245in}}%
\pgfpathlineto{\pgfqpoint{2.384979in}{2.151266in}}%
\pgfpathlineto{\pgfqpoint{2.407638in}{2.207375in}}%
\pgfpathlineto{\pgfqpoint{2.430297in}{2.240123in}}%
\pgfpathlineto{\pgfqpoint{2.452956in}{2.277424in}}%
\pgfpathlineto{\pgfqpoint{2.475614in}{2.100156in}}%
\pgfpathlineto{\pgfqpoint{2.498273in}{2.090780in}}%
\pgfpathlineto{\pgfqpoint{2.520932in}{2.157380in}}%
\pgfpathlineto{\pgfqpoint{2.543591in}{2.244817in}}%
\pgfpathlineto{\pgfqpoint{2.566249in}{2.266063in}}%
\pgfpathlineto{\pgfqpoint{2.588908in}{2.176811in}}%
\pgfpathlineto{\pgfqpoint{2.611567in}{2.274392in}}%
\pgfpathlineto{\pgfqpoint{2.634226in}{2.354496in}}%
\pgfpathlineto{\pgfqpoint{2.656884in}{2.204683in}}%
\pgfpathlineto{\pgfqpoint{2.679543in}{2.438954in}}%
\pgfpathlineto{\pgfqpoint{2.702202in}{2.379749in}}%
\pgfpathlineto{\pgfqpoint{2.724861in}{2.264954in}}%
\pgfpathlineto{\pgfqpoint{2.747519in}{2.621431in}}%
\pgfpathlineto{\pgfqpoint{2.770178in}{2.262907in}}%
\pgfpathlineto{\pgfqpoint{2.792837in}{2.450007in}}%
\pgfpathlineto{\pgfqpoint{2.815496in}{2.503890in}}%
\pgfpathlineto{\pgfqpoint{2.838154in}{2.479596in}}%
\pgfpathlineto{\pgfqpoint{2.860813in}{2.519799in}}%
\pgfpathlineto{\pgfqpoint{2.883472in}{2.425423in}}%
\pgfpathlineto{\pgfqpoint{2.906131in}{2.452324in}}%
\pgfpathlineto{\pgfqpoint{2.928789in}{2.566342in}}%
\pgfpathlineto{\pgfqpoint{2.951448in}{2.434306in}}%
\pgfpathlineto{\pgfqpoint{2.974107in}{2.271740in}}%
\pgfpathlineto{\pgfqpoint{2.996766in}{2.518758in}}%
\pgfpathlineto{\pgfqpoint{3.019424in}{2.335173in}}%
\pgfpathlineto{\pgfqpoint{3.042083in}{2.384802in}}%
\pgfpathlineto{\pgfqpoint{3.064742in}{2.393002in}}%
\pgfpathlineto{\pgfqpoint{3.087401in}{2.491313in}}%
\pgfpathlineto{\pgfqpoint{3.110059in}{2.387628in}}%
\pgfpathlineto{\pgfqpoint{3.132718in}{2.295185in}}%
\pgfpathlineto{\pgfqpoint{3.155377in}{2.265096in}}%
\pgfpathlineto{\pgfqpoint{3.178036in}{2.481998in}}%
\pgfpathlineto{\pgfqpoint{3.200694in}{2.391828in}}%
\pgfpathlineto{\pgfqpoint{3.223353in}{2.192798in}}%
\pgfpathlineto{\pgfqpoint{3.246012in}{2.384275in}}%
\pgfpathlineto{\pgfqpoint{3.268671in}{2.421021in}}%
\pgfpathlineto{\pgfqpoint{3.291329in}{2.261289in}}%
\pgfpathlineto{\pgfqpoint{3.313988in}{2.156667in}}%
\pgfpathlineto{\pgfqpoint{3.336647in}{2.353526in}}%
\pgfpathlineto{\pgfqpoint{3.359306in}{2.255938in}}%
\pgfpathlineto{\pgfqpoint{3.381964in}{2.216418in}}%
\pgfpathlineto{\pgfqpoint{3.404623in}{2.218162in}}%
\pgfpathlineto{\pgfqpoint{3.427282in}{2.115801in}}%
\pgfpathlineto{\pgfqpoint{3.449941in}{2.226541in}}%
\pgfpathlineto{\pgfqpoint{3.472599in}{2.279496in}}%
\pgfpathlineto{\pgfqpoint{3.495258in}{2.465679in}}%
\pgfpathlineto{\pgfqpoint{3.517917in}{2.184129in}}%
\pgfpathlineto{\pgfqpoint{3.540576in}{2.138964in}}%
\pgfpathlineto{\pgfqpoint{3.585893in}{2.230360in}}%
\pgfpathlineto{\pgfqpoint{3.608552in}{2.143861in}}%
\pgfpathlineto{\pgfqpoint{3.631211in}{2.355340in}}%
\pgfpathlineto{\pgfqpoint{3.653869in}{2.150437in}}%
\pgfpathlineto{\pgfqpoint{3.676528in}{2.271422in}}%
\pgfpathlineto{\pgfqpoint{3.699187in}{2.171967in}}%
\pgfpathlineto{\pgfqpoint{3.721846in}{2.164750in}}%
\pgfpathlineto{\pgfqpoint{3.744504in}{2.340842in}}%
\pgfpathlineto{\pgfqpoint{3.767163in}{2.216367in}}%
\pgfpathlineto{\pgfqpoint{3.789822in}{2.170996in}}%
\pgfpathlineto{\pgfqpoint{3.812481in}{2.191483in}}%
\pgfpathlineto{\pgfqpoint{3.835139in}{2.289543in}}%
\pgfpathlineto{\pgfqpoint{3.857798in}{2.323991in}}%
\pgfpathlineto{\pgfqpoint{3.880457in}{2.214487in}}%
\pgfpathlineto{\pgfqpoint{3.903116in}{2.166796in}}%
\pgfpathlineto{\pgfqpoint{3.925774in}{2.039938in}}%
\pgfpathlineto{\pgfqpoint{3.948433in}{2.341903in}}%
\pgfpathlineto{\pgfqpoint{3.971092in}{2.417410in}}%
\pgfpathlineto{\pgfqpoint{3.993751in}{2.134548in}}%
\pgfpathlineto{\pgfqpoint{4.016409in}{1.966833in}}%
\pgfpathlineto{\pgfqpoint{4.039068in}{2.319226in}}%
\pgfpathlineto{\pgfqpoint{4.061727in}{2.345199in}}%
\pgfpathlineto{\pgfqpoint{4.084386in}{2.264508in}}%
\pgfpathlineto{\pgfqpoint{4.107044in}{2.222433in}}%
\pgfpathlineto{\pgfqpoint{4.129703in}{2.324828in}}%
\pgfpathlineto{\pgfqpoint{4.152362in}{2.320096in}}%
\pgfpathlineto{\pgfqpoint{4.175021in}{2.459049in}}%
\pgfpathlineto{\pgfqpoint{4.197679in}{2.410844in}}%
\pgfpathlineto{\pgfqpoint{4.220338in}{2.380361in}}%
\pgfpathlineto{\pgfqpoint{4.242997in}{2.248443in}}%
\pgfpathlineto{\pgfqpoint{4.265656in}{2.381902in}}%
\pgfpathlineto{\pgfqpoint{4.288314in}{2.553052in}}%
\pgfpathlineto{\pgfqpoint{4.310973in}{2.445677in}}%
\pgfpathlineto{\pgfqpoint{4.333632in}{2.250376in}}%
\pgfpathlineto{\pgfqpoint{4.356291in}{2.192687in}}%
\pgfpathlineto{\pgfqpoint{4.378949in}{2.411036in}}%
\pgfpathlineto{\pgfqpoint{4.401608in}{2.333202in}}%
\pgfpathlineto{\pgfqpoint{4.424267in}{2.378988in}}%
\pgfpathlineto{\pgfqpoint{4.446926in}{2.380073in}}%
\pgfpathlineto{\pgfqpoint{4.469584in}{2.335322in}}%
\pgfpathlineto{\pgfqpoint{4.492243in}{2.364338in}}%
\pgfpathlineto{\pgfqpoint{4.514902in}{2.537235in}}%
\pgfpathlineto{\pgfqpoint{4.537561in}{2.250453in}}%
\pgfpathlineto{\pgfqpoint{4.560219in}{2.375107in}}%
\pgfpathlineto{\pgfqpoint{4.582878in}{2.384699in}}%
\pgfpathlineto{\pgfqpoint{4.605537in}{2.368375in}}%
\pgfpathlineto{\pgfqpoint{4.628196in}{2.440660in}}%
\pgfpathlineto{\pgfqpoint{4.650854in}{2.473844in}}%
\pgfpathlineto{\pgfqpoint{4.673513in}{2.465693in}}%
\pgfpathlineto{\pgfqpoint{4.696172in}{2.358510in}}%
\pgfpathlineto{\pgfqpoint{4.718831in}{2.391753in}}%
\pgfpathlineto{\pgfqpoint{4.741489in}{2.339342in}}%
\pgfpathlineto{\pgfqpoint{4.764148in}{2.272516in}}%
\pgfpathlineto{\pgfqpoint{4.786807in}{2.382350in}}%
\pgfpathlineto{\pgfqpoint{4.809466in}{2.341372in}}%
\pgfpathlineto{\pgfqpoint{4.832124in}{2.372538in}}%
\pgfpathlineto{\pgfqpoint{4.854783in}{2.355503in}}%
\pgfpathlineto{\pgfqpoint{4.877442in}{2.563300in}}%
\pgfpathlineto{\pgfqpoint{4.900101in}{2.410321in}}%
\pgfpathlineto{\pgfqpoint{4.922759in}{2.334652in}}%
\pgfpathlineto{\pgfqpoint{4.945418in}{2.381565in}}%
\pgfpathlineto{\pgfqpoint{4.968077in}{2.242533in}}%
\pgfpathlineto{\pgfqpoint{4.990735in}{2.421118in}}%
\pgfpathlineto{\pgfqpoint{5.013394in}{2.471401in}}%
\pgfpathlineto{\pgfqpoint{5.036053in}{2.372986in}}%
\pgfpathlineto{\pgfqpoint{5.058712in}{2.206208in}}%
\pgfpathlineto{\pgfqpoint{5.081370in}{2.525785in}}%
\pgfpathlineto{\pgfqpoint{5.104029in}{2.266939in}}%
\pgfpathlineto{\pgfqpoint{5.126688in}{2.500518in}}%
\pgfpathlineto{\pgfqpoint{5.149347in}{2.337350in}}%
\pgfpathlineto{\pgfqpoint{5.172005in}{2.110982in}}%
\pgfpathlineto{\pgfqpoint{5.194664in}{2.355697in}}%
\pgfpathlineto{\pgfqpoint{5.217323in}{2.375892in}}%
\pgfpathlineto{\pgfqpoint{5.239982in}{2.329863in}}%
\pgfpathlineto{\pgfqpoint{5.262640in}{2.213289in}}%
\pgfpathlineto{\pgfqpoint{5.285299in}{2.502655in}}%
\pgfpathlineto{\pgfqpoint{5.307958in}{2.302508in}}%
\pgfpathlineto{\pgfqpoint{5.330617in}{2.388824in}}%
\pgfpathlineto{\pgfqpoint{5.353275in}{2.356506in}}%
\pgfpathlineto{\pgfqpoint{5.375934in}{2.375375in}}%
\pgfpathlineto{\pgfqpoint{5.398593in}{2.323721in}}%
\pgfpathlineto{\pgfqpoint{5.421252in}{2.317362in}}%
\pgfpathlineto{\pgfqpoint{5.466569in}{2.390213in}}%
\pgfpathlineto{\pgfqpoint{5.489228in}{2.342651in}}%
\pgfpathlineto{\pgfqpoint{5.511887in}{2.404395in}}%
\pgfpathlineto{\pgfqpoint{5.534545in}{2.173048in}}%
\pgfpathlineto{\pgfqpoint{5.534545in}{2.173048in}}%
\pgfusepath{stroke}%
\end{pgfscope}%
\begin{pgfscope}%
\pgfsetrectcap%
\pgfsetmiterjoin%
\pgfsetlinewidth{0.803000pt}%
\definecolor{currentstroke}{rgb}{0.000000,0.000000,0.000000}%
\pgfsetstrokecolor{currentstroke}%
\pgfsetdash{}{0pt}%
\pgfpathmoveto{\pgfqpoint{0.800000in}{0.528000in}}%
\pgfpathlineto{\pgfqpoint{0.800000in}{4.224000in}}%
\pgfusepath{stroke}%
\end{pgfscope}%
\begin{pgfscope}%
\pgfsetrectcap%
\pgfsetmiterjoin%
\pgfsetlinewidth{0.803000pt}%
\definecolor{currentstroke}{rgb}{0.000000,0.000000,0.000000}%
\pgfsetstrokecolor{currentstroke}%
\pgfsetdash{}{0pt}%
\pgfpathmoveto{\pgfqpoint{5.760000in}{0.528000in}}%
\pgfpathlineto{\pgfqpoint{5.760000in}{4.224000in}}%
\pgfusepath{stroke}%
\end{pgfscope}%
\begin{pgfscope}%
\pgfsetrectcap%
\pgfsetmiterjoin%
\pgfsetlinewidth{0.803000pt}%
\definecolor{currentstroke}{rgb}{0.000000,0.000000,0.000000}%
\pgfsetstrokecolor{currentstroke}%
\pgfsetdash{}{0pt}%
\pgfpathmoveto{\pgfqpoint{0.800000in}{0.528000in}}%
\pgfpathlineto{\pgfqpoint{5.760000in}{0.528000in}}%
\pgfusepath{stroke}%
\end{pgfscope}%
\begin{pgfscope}%
\pgfsetrectcap%
\pgfsetmiterjoin%
\pgfsetlinewidth{0.803000pt}%
\definecolor{currentstroke}{rgb}{0.000000,0.000000,0.000000}%
\pgfsetstrokecolor{currentstroke}%
\pgfsetdash{}{0pt}%
\pgfpathmoveto{\pgfqpoint{0.800000in}{4.224000in}}%
\pgfpathlineto{\pgfqpoint{5.760000in}{4.224000in}}%
\pgfusepath{stroke}%
\end{pgfscope}%
\begin{pgfscope}%
\pgfsetbuttcap%
\pgfsetmiterjoin%
\definecolor{currentfill}{rgb}{1.000000,1.000000,1.000000}%
\pgfsetfillcolor{currentfill}%
\pgfsetfillopacity{0.800000}%
\pgfsetlinewidth{1.003750pt}%
\definecolor{currentstroke}{rgb}{0.800000,0.800000,0.800000}%
\pgfsetstrokecolor{currentstroke}%
\pgfsetstrokeopacity{0.800000}%
\pgfsetdash{}{0pt}%
\pgfpathmoveto{\pgfqpoint{4.662776in}{3.338198in}}%
\pgfpathlineto{\pgfqpoint{5.662778in}{3.338198in}}%
\pgfpathquadraticcurveto{\pgfqpoint{5.690556in}{3.338198in}}{\pgfqpoint{5.690556in}{3.365976in}}%
\pgfpathlineto{\pgfqpoint{5.690556in}{4.126778in}}%
\pgfpathquadraticcurveto{\pgfqpoint{5.690556in}{4.154556in}}{\pgfqpoint{5.662778in}{4.154556in}}%
\pgfpathlineto{\pgfqpoint{4.662776in}{4.154556in}}%
\pgfpathquadraticcurveto{\pgfqpoint{4.634998in}{4.154556in}}{\pgfqpoint{4.634998in}{4.126778in}}%
\pgfpathlineto{\pgfqpoint{4.634998in}{3.365976in}}%
\pgfpathquadraticcurveto{\pgfqpoint{4.634998in}{3.338198in}}{\pgfqpoint{4.662776in}{3.338198in}}%
\pgfpathclose%
\pgfusepath{stroke,fill}%
\end{pgfscope}%
\begin{pgfscope}%
\pgfsetrectcap%
\pgfsetroundjoin%
\pgfsetlinewidth{1.505625pt}%
\definecolor{currentstroke}{rgb}{0.121569,0.466667,0.705882}%
\pgfsetstrokecolor{currentstroke}%
\pgfsetdash{}{0pt}%
\pgfpathmoveto{\pgfqpoint{4.690554in}{4.050389in}}%
\pgfpathlineto{\pgfqpoint{4.968332in}{4.050389in}}%
\pgfusepath{stroke}%
\end{pgfscope}%
\begin{pgfscope}%
\definecolor{textcolor}{rgb}{0.000000,0.000000,0.000000}%
\pgfsetstrokecolor{textcolor}%
\pgfsetfillcolor{textcolor}%
\pgftext[x=5.079443in,y=4.001778in,left,base]{\color{textcolor}\rmfamily\fontsize{10.000000}{12.000000}\selectfont dt=0.003}%
\end{pgfscope}%
\begin{pgfscope}%
\pgfsetrectcap%
\pgfsetroundjoin%
\pgfsetlinewidth{1.505625pt}%
\definecolor{currentstroke}{rgb}{1.000000,0.498039,0.054902}%
\pgfsetstrokecolor{currentstroke}%
\pgfsetdash{}{0pt}%
\pgfpathmoveto{\pgfqpoint{4.690554in}{3.856716in}}%
\pgfpathlineto{\pgfqpoint{4.968332in}{3.856716in}}%
\pgfusepath{stroke}%
\end{pgfscope}%
\begin{pgfscope}%
\definecolor{textcolor}{rgb}{0.000000,0.000000,0.000000}%
\pgfsetstrokecolor{textcolor}%
\pgfsetfillcolor{textcolor}%
\pgftext[x=5.079443in,y=3.808105in,left,base]{\color{textcolor}\rmfamily\fontsize{10.000000}{12.000000}\selectfont dt=0.006}%
\end{pgfscope}%
\begin{pgfscope}%
\pgfsetrectcap%
\pgfsetroundjoin%
\pgfsetlinewidth{1.505625pt}%
\definecolor{currentstroke}{rgb}{0.172549,0.627451,0.172549}%
\pgfsetstrokecolor{currentstroke}%
\pgfsetdash{}{0pt}%
\pgfpathmoveto{\pgfqpoint{4.690554in}{3.663043in}}%
\pgfpathlineto{\pgfqpoint{4.968332in}{3.663043in}}%
\pgfusepath{stroke}%
\end{pgfscope}%
\begin{pgfscope}%
\definecolor{textcolor}{rgb}{0.000000,0.000000,0.000000}%
\pgfsetstrokecolor{textcolor}%
\pgfsetfillcolor{textcolor}%
\pgftext[x=5.079443in,y=3.614432in,left,base]{\color{textcolor}\rmfamily\fontsize{10.000000}{12.000000}\selectfont dt=0.01}%
\end{pgfscope}%
\begin{pgfscope}%
\pgfsetrectcap%
\pgfsetroundjoin%
\pgfsetlinewidth{1.505625pt}%
\definecolor{currentstroke}{rgb}{0.839216,0.152941,0.156863}%
\pgfsetstrokecolor{currentstroke}%
\pgfsetdash{}{0pt}%
\pgfpathmoveto{\pgfqpoint{4.690554in}{3.469371in}}%
\pgfpathlineto{\pgfqpoint{4.968332in}{3.469371in}}%
\pgfusepath{stroke}%
\end{pgfscope}%
\begin{pgfscope}%
\definecolor{textcolor}{rgb}{0.000000,0.000000,0.000000}%
\pgfsetstrokecolor{textcolor}%
\pgfsetfillcolor{textcolor}%
\pgftext[x=5.079443in,y=3.420759in,left,base]{\color{textcolor}\rmfamily\fontsize{10.000000}{12.000000}\selectfont dt=0.03}%
\end{pgfscope}%
\end{pgfpicture}%
\makeatother%
\endgroup%

        }
        \caption{NVE for different time steps.}
        \label{task1_total_energy}
    \end{subfigure}
    \begin{subfigure}{0.5\textwidth}
        \resizebox{\textwidth}{!}{
            %% Creator: Matplotlib, PGF backend
%%
%% To include the figure in your LaTeX document, write
%%   \input{<filename>.pgf}
%%
%% Make sure the required packages are loaded in your preamble
%%   \usepackage{pgf}
%%
%% Figures using additional raster images can only be included by \input if
%% they are in the same directory as the main LaTeX file. For loading figures
%% from other directories you can use the `import` package
%%   \usepackage{import}
%%
%% and then include the figures with
%%   \import{<path to file>}{<filename>.pgf}
%%
%% Matplotlib used the following preamble
%%
\begingroup%
\makeatletter%
\begin{pgfpicture}%
\pgfpathrectangle{\pgfpointorigin}{\pgfqpoint{6.400000in}{4.800000in}}%
\pgfusepath{use as bounding box, clip}%
\begin{pgfscope}%
\pgfsetbuttcap%
\pgfsetmiterjoin%
\definecolor{currentfill}{rgb}{1.000000,1.000000,1.000000}%
\pgfsetfillcolor{currentfill}%
\pgfsetlinewidth{0.000000pt}%
\definecolor{currentstroke}{rgb}{1.000000,1.000000,1.000000}%
\pgfsetstrokecolor{currentstroke}%
\pgfsetdash{}{0pt}%
\pgfpathmoveto{\pgfqpoint{0.000000in}{0.000000in}}%
\pgfpathlineto{\pgfqpoint{6.400000in}{0.000000in}}%
\pgfpathlineto{\pgfqpoint{6.400000in}{4.800000in}}%
\pgfpathlineto{\pgfqpoint{0.000000in}{4.800000in}}%
\pgfpathclose%
\pgfusepath{fill}%
\end{pgfscope}%
\begin{pgfscope}%
\pgfsetbuttcap%
\pgfsetmiterjoin%
\definecolor{currentfill}{rgb}{1.000000,1.000000,1.000000}%
\pgfsetfillcolor{currentfill}%
\pgfsetlinewidth{0.000000pt}%
\definecolor{currentstroke}{rgb}{0.000000,0.000000,0.000000}%
\pgfsetstrokecolor{currentstroke}%
\pgfsetstrokeopacity{0.000000}%
\pgfsetdash{}{0pt}%
\pgfpathmoveto{\pgfqpoint{0.800000in}{0.528000in}}%
\pgfpathlineto{\pgfqpoint{5.760000in}{0.528000in}}%
\pgfpathlineto{\pgfqpoint{5.760000in}{4.224000in}}%
\pgfpathlineto{\pgfqpoint{0.800000in}{4.224000in}}%
\pgfpathclose%
\pgfusepath{fill}%
\end{pgfscope}%
\begin{pgfscope}%
\pgfsetbuttcap%
\pgfsetroundjoin%
\definecolor{currentfill}{rgb}{0.000000,0.000000,0.000000}%
\pgfsetfillcolor{currentfill}%
\pgfsetlinewidth{0.803000pt}%
\definecolor{currentstroke}{rgb}{0.000000,0.000000,0.000000}%
\pgfsetstrokecolor{currentstroke}%
\pgfsetdash{}{0pt}%
\pgfsys@defobject{currentmarker}{\pgfqpoint{0.000000in}{-0.048611in}}{\pgfqpoint{0.000000in}{0.000000in}}{%
\pgfpathmoveto{\pgfqpoint{0.000000in}{0.000000in}}%
\pgfpathlineto{\pgfqpoint{0.000000in}{-0.048611in}}%
\pgfusepath{stroke,fill}%
}%
\begin{pgfscope}%
\pgfsys@transformshift{1.025455in}{0.528000in}%
\pgfsys@useobject{currentmarker}{}%
\end{pgfscope}%
\end{pgfscope}%
\begin{pgfscope}%
\definecolor{textcolor}{rgb}{0.000000,0.000000,0.000000}%
\pgfsetstrokecolor{textcolor}%
\pgfsetfillcolor{textcolor}%
\pgftext[x=1.025455in,y=0.430778in,,top]{\color{textcolor}\rmfamily\fontsize{10.000000}{12.000000}\selectfont \(\displaystyle {0}\)}%
\end{pgfscope}%
\begin{pgfscope}%
\pgfsetbuttcap%
\pgfsetroundjoin%
\definecolor{currentfill}{rgb}{0.000000,0.000000,0.000000}%
\pgfsetfillcolor{currentfill}%
\pgfsetlinewidth{0.803000pt}%
\definecolor{currentstroke}{rgb}{0.000000,0.000000,0.000000}%
\pgfsetstrokecolor{currentstroke}%
\pgfsetdash{}{0pt}%
\pgfsys@defobject{currentmarker}{\pgfqpoint{0.000000in}{-0.048611in}}{\pgfqpoint{0.000000in}{0.000000in}}{%
\pgfpathmoveto{\pgfqpoint{0.000000in}{0.000000in}}%
\pgfpathlineto{\pgfqpoint{0.000000in}{-0.048611in}}%
\pgfusepath{stroke,fill}%
}%
\begin{pgfscope}%
\pgfsys@transformshift{1.591923in}{0.528000in}%
\pgfsys@useobject{currentmarker}{}%
\end{pgfscope}%
\end{pgfscope}%
\begin{pgfscope}%
\definecolor{textcolor}{rgb}{0.000000,0.000000,0.000000}%
\pgfsetstrokecolor{textcolor}%
\pgfsetfillcolor{textcolor}%
\pgftext[x=1.591923in,y=0.430778in,,top]{\color{textcolor}\rmfamily\fontsize{10.000000}{12.000000}\selectfont \(\displaystyle {25}\)}%
\end{pgfscope}%
\begin{pgfscope}%
\pgfsetbuttcap%
\pgfsetroundjoin%
\definecolor{currentfill}{rgb}{0.000000,0.000000,0.000000}%
\pgfsetfillcolor{currentfill}%
\pgfsetlinewidth{0.803000pt}%
\definecolor{currentstroke}{rgb}{0.000000,0.000000,0.000000}%
\pgfsetstrokecolor{currentstroke}%
\pgfsetdash{}{0pt}%
\pgfsys@defobject{currentmarker}{\pgfqpoint{0.000000in}{-0.048611in}}{\pgfqpoint{0.000000in}{0.000000in}}{%
\pgfpathmoveto{\pgfqpoint{0.000000in}{0.000000in}}%
\pgfpathlineto{\pgfqpoint{0.000000in}{-0.048611in}}%
\pgfusepath{stroke,fill}%
}%
\begin{pgfscope}%
\pgfsys@transformshift{2.158392in}{0.528000in}%
\pgfsys@useobject{currentmarker}{}%
\end{pgfscope}%
\end{pgfscope}%
\begin{pgfscope}%
\definecolor{textcolor}{rgb}{0.000000,0.000000,0.000000}%
\pgfsetstrokecolor{textcolor}%
\pgfsetfillcolor{textcolor}%
\pgftext[x=2.158392in,y=0.430778in,,top]{\color{textcolor}\rmfamily\fontsize{10.000000}{12.000000}\selectfont \(\displaystyle {50}\)}%
\end{pgfscope}%
\begin{pgfscope}%
\pgfsetbuttcap%
\pgfsetroundjoin%
\definecolor{currentfill}{rgb}{0.000000,0.000000,0.000000}%
\pgfsetfillcolor{currentfill}%
\pgfsetlinewidth{0.803000pt}%
\definecolor{currentstroke}{rgb}{0.000000,0.000000,0.000000}%
\pgfsetstrokecolor{currentstroke}%
\pgfsetdash{}{0pt}%
\pgfsys@defobject{currentmarker}{\pgfqpoint{0.000000in}{-0.048611in}}{\pgfqpoint{0.000000in}{0.000000in}}{%
\pgfpathmoveto{\pgfqpoint{0.000000in}{0.000000in}}%
\pgfpathlineto{\pgfqpoint{0.000000in}{-0.048611in}}%
\pgfusepath{stroke,fill}%
}%
\begin{pgfscope}%
\pgfsys@transformshift{2.724861in}{0.528000in}%
\pgfsys@useobject{currentmarker}{}%
\end{pgfscope}%
\end{pgfscope}%
\begin{pgfscope}%
\definecolor{textcolor}{rgb}{0.000000,0.000000,0.000000}%
\pgfsetstrokecolor{textcolor}%
\pgfsetfillcolor{textcolor}%
\pgftext[x=2.724861in,y=0.430778in,,top]{\color{textcolor}\rmfamily\fontsize{10.000000}{12.000000}\selectfont \(\displaystyle {75}\)}%
\end{pgfscope}%
\begin{pgfscope}%
\pgfsetbuttcap%
\pgfsetroundjoin%
\definecolor{currentfill}{rgb}{0.000000,0.000000,0.000000}%
\pgfsetfillcolor{currentfill}%
\pgfsetlinewidth{0.803000pt}%
\definecolor{currentstroke}{rgb}{0.000000,0.000000,0.000000}%
\pgfsetstrokecolor{currentstroke}%
\pgfsetdash{}{0pt}%
\pgfsys@defobject{currentmarker}{\pgfqpoint{0.000000in}{-0.048611in}}{\pgfqpoint{0.000000in}{0.000000in}}{%
\pgfpathmoveto{\pgfqpoint{0.000000in}{0.000000in}}%
\pgfpathlineto{\pgfqpoint{0.000000in}{-0.048611in}}%
\pgfusepath{stroke,fill}%
}%
\begin{pgfscope}%
\pgfsys@transformshift{3.291329in}{0.528000in}%
\pgfsys@useobject{currentmarker}{}%
\end{pgfscope}%
\end{pgfscope}%
\begin{pgfscope}%
\definecolor{textcolor}{rgb}{0.000000,0.000000,0.000000}%
\pgfsetstrokecolor{textcolor}%
\pgfsetfillcolor{textcolor}%
\pgftext[x=3.291329in,y=0.430778in,,top]{\color{textcolor}\rmfamily\fontsize{10.000000}{12.000000}\selectfont \(\displaystyle {100}\)}%
\end{pgfscope}%
\begin{pgfscope}%
\pgfsetbuttcap%
\pgfsetroundjoin%
\definecolor{currentfill}{rgb}{0.000000,0.000000,0.000000}%
\pgfsetfillcolor{currentfill}%
\pgfsetlinewidth{0.803000pt}%
\definecolor{currentstroke}{rgb}{0.000000,0.000000,0.000000}%
\pgfsetstrokecolor{currentstroke}%
\pgfsetdash{}{0pt}%
\pgfsys@defobject{currentmarker}{\pgfqpoint{0.000000in}{-0.048611in}}{\pgfqpoint{0.000000in}{0.000000in}}{%
\pgfpathmoveto{\pgfqpoint{0.000000in}{0.000000in}}%
\pgfpathlineto{\pgfqpoint{0.000000in}{-0.048611in}}%
\pgfusepath{stroke,fill}%
}%
\begin{pgfscope}%
\pgfsys@transformshift{3.857798in}{0.528000in}%
\pgfsys@useobject{currentmarker}{}%
\end{pgfscope}%
\end{pgfscope}%
\begin{pgfscope}%
\definecolor{textcolor}{rgb}{0.000000,0.000000,0.000000}%
\pgfsetstrokecolor{textcolor}%
\pgfsetfillcolor{textcolor}%
\pgftext[x=3.857798in,y=0.430778in,,top]{\color{textcolor}\rmfamily\fontsize{10.000000}{12.000000}\selectfont \(\displaystyle {125}\)}%
\end{pgfscope}%
\begin{pgfscope}%
\pgfsetbuttcap%
\pgfsetroundjoin%
\definecolor{currentfill}{rgb}{0.000000,0.000000,0.000000}%
\pgfsetfillcolor{currentfill}%
\pgfsetlinewidth{0.803000pt}%
\definecolor{currentstroke}{rgb}{0.000000,0.000000,0.000000}%
\pgfsetstrokecolor{currentstroke}%
\pgfsetdash{}{0pt}%
\pgfsys@defobject{currentmarker}{\pgfqpoint{0.000000in}{-0.048611in}}{\pgfqpoint{0.000000in}{0.000000in}}{%
\pgfpathmoveto{\pgfqpoint{0.000000in}{0.000000in}}%
\pgfpathlineto{\pgfqpoint{0.000000in}{-0.048611in}}%
\pgfusepath{stroke,fill}%
}%
\begin{pgfscope}%
\pgfsys@transformshift{4.424267in}{0.528000in}%
\pgfsys@useobject{currentmarker}{}%
\end{pgfscope}%
\end{pgfscope}%
\begin{pgfscope}%
\definecolor{textcolor}{rgb}{0.000000,0.000000,0.000000}%
\pgfsetstrokecolor{textcolor}%
\pgfsetfillcolor{textcolor}%
\pgftext[x=4.424267in,y=0.430778in,,top]{\color{textcolor}\rmfamily\fontsize{10.000000}{12.000000}\selectfont \(\displaystyle {150}\)}%
\end{pgfscope}%
\begin{pgfscope}%
\pgfsetbuttcap%
\pgfsetroundjoin%
\definecolor{currentfill}{rgb}{0.000000,0.000000,0.000000}%
\pgfsetfillcolor{currentfill}%
\pgfsetlinewidth{0.803000pt}%
\definecolor{currentstroke}{rgb}{0.000000,0.000000,0.000000}%
\pgfsetstrokecolor{currentstroke}%
\pgfsetdash{}{0pt}%
\pgfsys@defobject{currentmarker}{\pgfqpoint{0.000000in}{-0.048611in}}{\pgfqpoint{0.000000in}{0.000000in}}{%
\pgfpathmoveto{\pgfqpoint{0.000000in}{0.000000in}}%
\pgfpathlineto{\pgfqpoint{0.000000in}{-0.048611in}}%
\pgfusepath{stroke,fill}%
}%
\begin{pgfscope}%
\pgfsys@transformshift{4.990735in}{0.528000in}%
\pgfsys@useobject{currentmarker}{}%
\end{pgfscope}%
\end{pgfscope}%
\begin{pgfscope}%
\definecolor{textcolor}{rgb}{0.000000,0.000000,0.000000}%
\pgfsetstrokecolor{textcolor}%
\pgfsetfillcolor{textcolor}%
\pgftext[x=4.990735in,y=0.430778in,,top]{\color{textcolor}\rmfamily\fontsize{10.000000}{12.000000}\selectfont \(\displaystyle {175}\)}%
\end{pgfscope}%
\begin{pgfscope}%
\pgfsetbuttcap%
\pgfsetroundjoin%
\definecolor{currentfill}{rgb}{0.000000,0.000000,0.000000}%
\pgfsetfillcolor{currentfill}%
\pgfsetlinewidth{0.803000pt}%
\definecolor{currentstroke}{rgb}{0.000000,0.000000,0.000000}%
\pgfsetstrokecolor{currentstroke}%
\pgfsetdash{}{0pt}%
\pgfsys@defobject{currentmarker}{\pgfqpoint{0.000000in}{-0.048611in}}{\pgfqpoint{0.000000in}{0.000000in}}{%
\pgfpathmoveto{\pgfqpoint{0.000000in}{0.000000in}}%
\pgfpathlineto{\pgfqpoint{0.000000in}{-0.048611in}}%
\pgfusepath{stroke,fill}%
}%
\begin{pgfscope}%
\pgfsys@transformshift{5.557204in}{0.528000in}%
\pgfsys@useobject{currentmarker}{}%
\end{pgfscope}%
\end{pgfscope}%
\begin{pgfscope}%
\definecolor{textcolor}{rgb}{0.000000,0.000000,0.000000}%
\pgfsetstrokecolor{textcolor}%
\pgfsetfillcolor{textcolor}%
\pgftext[x=5.557204in,y=0.430778in,,top]{\color{textcolor}\rmfamily\fontsize{10.000000}{12.000000}\selectfont \(\displaystyle {200}\)}%
\end{pgfscope}%
\begin{pgfscope}%
\definecolor{textcolor}{rgb}{0.000000,0.000000,0.000000}%
\pgfsetstrokecolor{textcolor}%
\pgfsetfillcolor{textcolor}%
\pgftext[x=3.280000in,y=0.251766in,,top]{\color{textcolor}\rmfamily\fontsize{10.000000}{12.000000}\selectfont \(\displaystyle t\)}%
\end{pgfscope}%
\begin{pgfscope}%
\pgfsetbuttcap%
\pgfsetroundjoin%
\definecolor{currentfill}{rgb}{0.000000,0.000000,0.000000}%
\pgfsetfillcolor{currentfill}%
\pgfsetlinewidth{0.803000pt}%
\definecolor{currentstroke}{rgb}{0.000000,0.000000,0.000000}%
\pgfsetstrokecolor{currentstroke}%
\pgfsetdash{}{0pt}%
\pgfsys@defobject{currentmarker}{\pgfqpoint{-0.048611in}{0.000000in}}{\pgfqpoint{-0.000000in}{0.000000in}}{%
\pgfpathmoveto{\pgfqpoint{-0.000000in}{0.000000in}}%
\pgfpathlineto{\pgfqpoint{-0.048611in}{0.000000in}}%
\pgfusepath{stroke,fill}%
}%
\begin{pgfscope}%
\pgfsys@transformshift{0.800000in}{1.001968in}%
\pgfsys@useobject{currentmarker}{}%
\end{pgfscope}%
\end{pgfscope}%
\begin{pgfscope}%
\definecolor{textcolor}{rgb}{0.000000,0.000000,0.000000}%
\pgfsetstrokecolor{textcolor}%
\pgfsetfillcolor{textcolor}%
\pgftext[x=0.417283in, y=0.953743in, left, base]{\color{textcolor}\rmfamily\fontsize{10.000000}{12.000000}\selectfont \(\displaystyle {\ensuremath{-}5.2}\)}%
\end{pgfscope}%
\begin{pgfscope}%
\pgfsetbuttcap%
\pgfsetroundjoin%
\definecolor{currentfill}{rgb}{0.000000,0.000000,0.000000}%
\pgfsetfillcolor{currentfill}%
\pgfsetlinewidth{0.803000pt}%
\definecolor{currentstroke}{rgb}{0.000000,0.000000,0.000000}%
\pgfsetstrokecolor{currentstroke}%
\pgfsetdash{}{0pt}%
\pgfsys@defobject{currentmarker}{\pgfqpoint{-0.048611in}{0.000000in}}{\pgfqpoint{-0.000000in}{0.000000in}}{%
\pgfpathmoveto{\pgfqpoint{-0.000000in}{0.000000in}}%
\pgfpathlineto{\pgfqpoint{-0.048611in}{0.000000in}}%
\pgfusepath{stroke,fill}%
}%
\begin{pgfscope}%
\pgfsys@transformshift{0.800000in}{1.656178in}%
\pgfsys@useobject{currentmarker}{}%
\end{pgfscope}%
\end{pgfscope}%
\begin{pgfscope}%
\definecolor{textcolor}{rgb}{0.000000,0.000000,0.000000}%
\pgfsetstrokecolor{textcolor}%
\pgfsetfillcolor{textcolor}%
\pgftext[x=0.417283in, y=1.607952in, left, base]{\color{textcolor}\rmfamily\fontsize{10.000000}{12.000000}\selectfont \(\displaystyle {\ensuremath{-}5.0}\)}%
\end{pgfscope}%
\begin{pgfscope}%
\pgfsetbuttcap%
\pgfsetroundjoin%
\definecolor{currentfill}{rgb}{0.000000,0.000000,0.000000}%
\pgfsetfillcolor{currentfill}%
\pgfsetlinewidth{0.803000pt}%
\definecolor{currentstroke}{rgb}{0.000000,0.000000,0.000000}%
\pgfsetstrokecolor{currentstroke}%
\pgfsetdash{}{0pt}%
\pgfsys@defobject{currentmarker}{\pgfqpoint{-0.048611in}{0.000000in}}{\pgfqpoint{-0.000000in}{0.000000in}}{%
\pgfpathmoveto{\pgfqpoint{-0.000000in}{0.000000in}}%
\pgfpathlineto{\pgfqpoint{-0.048611in}{0.000000in}}%
\pgfusepath{stroke,fill}%
}%
\begin{pgfscope}%
\pgfsys@transformshift{0.800000in}{2.310387in}%
\pgfsys@useobject{currentmarker}{}%
\end{pgfscope}%
\end{pgfscope}%
\begin{pgfscope}%
\definecolor{textcolor}{rgb}{0.000000,0.000000,0.000000}%
\pgfsetstrokecolor{textcolor}%
\pgfsetfillcolor{textcolor}%
\pgftext[x=0.417283in, y=2.262162in, left, base]{\color{textcolor}\rmfamily\fontsize{10.000000}{12.000000}\selectfont \(\displaystyle {\ensuremath{-}4.8}\)}%
\end{pgfscope}%
\begin{pgfscope}%
\pgfsetbuttcap%
\pgfsetroundjoin%
\definecolor{currentfill}{rgb}{0.000000,0.000000,0.000000}%
\pgfsetfillcolor{currentfill}%
\pgfsetlinewidth{0.803000pt}%
\definecolor{currentstroke}{rgb}{0.000000,0.000000,0.000000}%
\pgfsetstrokecolor{currentstroke}%
\pgfsetdash{}{0pt}%
\pgfsys@defobject{currentmarker}{\pgfqpoint{-0.048611in}{0.000000in}}{\pgfqpoint{-0.000000in}{0.000000in}}{%
\pgfpathmoveto{\pgfqpoint{-0.000000in}{0.000000in}}%
\pgfpathlineto{\pgfqpoint{-0.048611in}{0.000000in}}%
\pgfusepath{stroke,fill}%
}%
\begin{pgfscope}%
\pgfsys@transformshift{0.800000in}{2.964596in}%
\pgfsys@useobject{currentmarker}{}%
\end{pgfscope}%
\end{pgfscope}%
\begin{pgfscope}%
\definecolor{textcolor}{rgb}{0.000000,0.000000,0.000000}%
\pgfsetstrokecolor{textcolor}%
\pgfsetfillcolor{textcolor}%
\pgftext[x=0.417283in, y=2.916371in, left, base]{\color{textcolor}\rmfamily\fontsize{10.000000}{12.000000}\selectfont \(\displaystyle {\ensuremath{-}4.6}\)}%
\end{pgfscope}%
\begin{pgfscope}%
\pgfsetbuttcap%
\pgfsetroundjoin%
\definecolor{currentfill}{rgb}{0.000000,0.000000,0.000000}%
\pgfsetfillcolor{currentfill}%
\pgfsetlinewidth{0.803000pt}%
\definecolor{currentstroke}{rgb}{0.000000,0.000000,0.000000}%
\pgfsetstrokecolor{currentstroke}%
\pgfsetdash{}{0pt}%
\pgfsys@defobject{currentmarker}{\pgfqpoint{-0.048611in}{0.000000in}}{\pgfqpoint{-0.000000in}{0.000000in}}{%
\pgfpathmoveto{\pgfqpoint{-0.000000in}{0.000000in}}%
\pgfpathlineto{\pgfqpoint{-0.048611in}{0.000000in}}%
\pgfusepath{stroke,fill}%
}%
\begin{pgfscope}%
\pgfsys@transformshift{0.800000in}{3.618806in}%
\pgfsys@useobject{currentmarker}{}%
\end{pgfscope}%
\end{pgfscope}%
\begin{pgfscope}%
\definecolor{textcolor}{rgb}{0.000000,0.000000,0.000000}%
\pgfsetstrokecolor{textcolor}%
\pgfsetfillcolor{textcolor}%
\pgftext[x=0.417283in, y=3.570580in, left, base]{\color{textcolor}\rmfamily\fontsize{10.000000}{12.000000}\selectfont \(\displaystyle {\ensuremath{-}4.4}\)}%
\end{pgfscope}%
\begin{pgfscope}%
\definecolor{textcolor}{rgb}{0.000000,0.000000,0.000000}%
\pgfsetstrokecolor{textcolor}%
\pgfsetfillcolor{textcolor}%
\pgftext[x=0.361727in,y=2.376000in,,bottom,rotate=90.000000]{\color{textcolor}\rmfamily\fontsize{10.000000}{12.000000}\selectfont Total energy [L.J.]}%
\end{pgfscope}%
\begin{pgfscope}%
\pgfpathrectangle{\pgfqpoint{0.800000in}{0.528000in}}{\pgfqpoint{4.960000in}{3.696000in}}%
\pgfusepath{clip}%
\pgfsetrectcap%
\pgfsetroundjoin%
\pgfsetlinewidth{1.505625pt}%
\definecolor{currentstroke}{rgb}{0.121569,0.466667,0.705882}%
\pgfsetstrokecolor{currentstroke}%
\pgfsetdash{}{0pt}%
\pgfpathmoveto{\pgfqpoint{1.025455in}{0.696000in}}%
\pgfpathlineto{\pgfqpoint{1.048113in}{2.469338in}}%
\pgfpathlineto{\pgfqpoint{1.070772in}{2.470155in}}%
\pgfpathlineto{\pgfqpoint{1.093431in}{2.484247in}}%
\pgfpathlineto{\pgfqpoint{1.116090in}{2.511471in}}%
\pgfpathlineto{\pgfqpoint{1.138748in}{2.551197in}}%
\pgfpathlineto{\pgfqpoint{1.161407in}{2.603092in}}%
\pgfpathlineto{\pgfqpoint{1.184066in}{2.668001in}}%
\pgfpathlineto{\pgfqpoint{1.206725in}{2.740622in}}%
\pgfpathlineto{\pgfqpoint{1.252042in}{2.902987in}}%
\pgfpathlineto{\pgfqpoint{1.297360in}{3.071257in}}%
\pgfpathlineto{\pgfqpoint{1.320018in}{3.150477in}}%
\pgfpathlineto{\pgfqpoint{1.342677in}{3.223631in}}%
\pgfpathlineto{\pgfqpoint{1.365336in}{3.290438in}}%
\pgfpathlineto{\pgfqpoint{1.387995in}{3.351914in}}%
\pgfpathlineto{\pgfqpoint{1.410653in}{3.408302in}}%
\pgfpathlineto{\pgfqpoint{1.433312in}{3.461227in}}%
\pgfpathlineto{\pgfqpoint{1.455971in}{3.510742in}}%
\pgfpathlineto{\pgfqpoint{1.501288in}{3.599032in}}%
\pgfpathlineto{\pgfqpoint{1.523947in}{3.639163in}}%
\pgfpathlineto{\pgfqpoint{1.546606in}{3.675601in}}%
\pgfpathlineto{\pgfqpoint{1.569265in}{3.706867in}}%
\pgfpathlineto{\pgfqpoint{1.591923in}{3.734512in}}%
\pgfpathlineto{\pgfqpoint{1.614582in}{3.758580in}}%
\pgfpathlineto{\pgfqpoint{1.637241in}{3.778656in}}%
\pgfpathlineto{\pgfqpoint{1.659899in}{3.795190in}}%
\pgfpathlineto{\pgfqpoint{1.682558in}{3.806241in}}%
\pgfpathlineto{\pgfqpoint{1.705217in}{3.812080in}}%
\pgfpathlineto{\pgfqpoint{1.727876in}{3.812191in}}%
\pgfpathlineto{\pgfqpoint{1.750534in}{3.808376in}}%
\pgfpathlineto{\pgfqpoint{1.773193in}{3.799030in}}%
\pgfpathlineto{\pgfqpoint{1.795852in}{3.787069in}}%
\pgfpathlineto{\pgfqpoint{1.818511in}{3.772262in}}%
\pgfpathlineto{\pgfqpoint{1.841169in}{3.755764in}}%
\pgfpathlineto{\pgfqpoint{1.909146in}{3.701189in}}%
\pgfpathlineto{\pgfqpoint{1.931804in}{3.685588in}}%
\pgfpathlineto{\pgfqpoint{1.954463in}{3.675077in}}%
\pgfpathlineto{\pgfqpoint{1.977122in}{3.666693in}}%
\pgfpathlineto{\pgfqpoint{1.999781in}{3.666007in}}%
\pgfpathlineto{\pgfqpoint{2.022439in}{3.670868in}}%
\pgfpathlineto{\pgfqpoint{2.045098in}{3.681456in}}%
\pgfpathlineto{\pgfqpoint{2.067757in}{3.697764in}}%
\pgfpathlineto{\pgfqpoint{2.090416in}{3.718861in}}%
\pgfpathlineto{\pgfqpoint{2.181051in}{3.816252in}}%
\pgfpathlineto{\pgfqpoint{2.203709in}{3.837554in}}%
\pgfpathlineto{\pgfqpoint{2.249027in}{3.874317in}}%
\pgfpathlineto{\pgfqpoint{2.294344in}{3.907951in}}%
\pgfpathlineto{\pgfqpoint{2.317003in}{3.927273in}}%
\pgfpathlineto{\pgfqpoint{2.384979in}{3.990105in}}%
\pgfpathlineto{\pgfqpoint{2.407638in}{4.008896in}}%
\pgfpathlineto{\pgfqpoint{2.430297in}{4.023799in}}%
\pgfpathlineto{\pgfqpoint{2.452956in}{4.032267in}}%
\pgfpathlineto{\pgfqpoint{2.475614in}{4.033370in}}%
\pgfpathlineto{\pgfqpoint{2.498273in}{4.027677in}}%
\pgfpathlineto{\pgfqpoint{2.520932in}{4.015922in}}%
\pgfpathlineto{\pgfqpoint{2.543591in}{3.999996in}}%
\pgfpathlineto{\pgfqpoint{2.588908in}{3.963241in}}%
\pgfpathlineto{\pgfqpoint{2.611567in}{3.947710in}}%
\pgfpathlineto{\pgfqpoint{2.634226in}{3.937505in}}%
\pgfpathlineto{\pgfqpoint{2.656884in}{3.931945in}}%
\pgfpathlineto{\pgfqpoint{2.679543in}{3.930021in}}%
\pgfpathlineto{\pgfqpoint{2.702202in}{3.933235in}}%
\pgfpathlineto{\pgfqpoint{2.724861in}{3.941059in}}%
\pgfpathlineto{\pgfqpoint{2.747519in}{3.951360in}}%
\pgfpathlineto{\pgfqpoint{2.770178in}{3.965939in}}%
\pgfpathlineto{\pgfqpoint{2.838154in}{4.018103in}}%
\pgfpathlineto{\pgfqpoint{2.860813in}{4.032086in}}%
\pgfpathlineto{\pgfqpoint{2.883472in}{4.042907in}}%
\pgfpathlineto{\pgfqpoint{2.906131in}{4.051596in}}%
\pgfpathlineto{\pgfqpoint{2.928789in}{4.054782in}}%
\pgfpathlineto{\pgfqpoint{2.951448in}{4.055278in}}%
\pgfpathlineto{\pgfqpoint{2.974107in}{4.052097in}}%
\pgfpathlineto{\pgfqpoint{2.996766in}{4.044997in}}%
\pgfpathlineto{\pgfqpoint{3.019424in}{4.036650in}}%
\pgfpathlineto{\pgfqpoint{3.064742in}{4.015157in}}%
\pgfpathlineto{\pgfqpoint{3.132718in}{3.978249in}}%
\pgfpathlineto{\pgfqpoint{3.178036in}{3.950098in}}%
\pgfpathlineto{\pgfqpoint{3.223353in}{3.919384in}}%
\pgfpathlineto{\pgfqpoint{3.291329in}{3.868660in}}%
\pgfpathlineto{\pgfqpoint{3.313988in}{3.852316in}}%
\pgfpathlineto{\pgfqpoint{3.336647in}{3.839180in}}%
\pgfpathlineto{\pgfqpoint{3.359306in}{3.827563in}}%
\pgfpathlineto{\pgfqpoint{3.381964in}{3.819025in}}%
\pgfpathlineto{\pgfqpoint{3.404623in}{3.813553in}}%
\pgfpathlineto{\pgfqpoint{3.427282in}{3.813191in}}%
\pgfpathlineto{\pgfqpoint{3.449941in}{3.817046in}}%
\pgfpathlineto{\pgfqpoint{3.472599in}{3.824035in}}%
\pgfpathlineto{\pgfqpoint{3.495258in}{3.833000in}}%
\pgfpathlineto{\pgfqpoint{3.517917in}{3.843454in}}%
\pgfpathlineto{\pgfqpoint{3.540576in}{3.852307in}}%
\pgfpathlineto{\pgfqpoint{3.563234in}{3.858800in}}%
\pgfpathlineto{\pgfqpoint{3.585893in}{3.862352in}}%
\pgfpathlineto{\pgfqpoint{3.608552in}{3.863305in}}%
\pgfpathlineto{\pgfqpoint{3.676528in}{3.855793in}}%
\pgfpathlineto{\pgfqpoint{3.721846in}{3.848501in}}%
\pgfpathlineto{\pgfqpoint{3.744504in}{3.844873in}}%
\pgfpathlineto{\pgfqpoint{3.767163in}{3.844423in}}%
\pgfpathlineto{\pgfqpoint{3.789822in}{3.845914in}}%
\pgfpathlineto{\pgfqpoint{3.812481in}{3.852907in}}%
\pgfpathlineto{\pgfqpoint{3.835139in}{3.862891in}}%
\pgfpathlineto{\pgfqpoint{3.880457in}{3.893100in}}%
\pgfpathlineto{\pgfqpoint{3.925774in}{3.928494in}}%
\pgfpathlineto{\pgfqpoint{3.948433in}{3.944009in}}%
\pgfpathlineto{\pgfqpoint{3.971092in}{3.956444in}}%
\pgfpathlineto{\pgfqpoint{3.993751in}{3.966262in}}%
\pgfpathlineto{\pgfqpoint{4.016409in}{3.974129in}}%
\pgfpathlineto{\pgfqpoint{4.039068in}{3.978737in}}%
\pgfpathlineto{\pgfqpoint{4.084386in}{3.983151in}}%
\pgfpathlineto{\pgfqpoint{4.107044in}{3.982124in}}%
\pgfpathlineto{\pgfqpoint{4.129703in}{3.979407in}}%
\pgfpathlineto{\pgfqpoint{4.175021in}{3.967667in}}%
\pgfpathlineto{\pgfqpoint{4.288314in}{3.929846in}}%
\pgfpathlineto{\pgfqpoint{4.333632in}{3.916224in}}%
\pgfpathlineto{\pgfqpoint{4.446926in}{3.875058in}}%
\pgfpathlineto{\pgfqpoint{4.469584in}{3.868807in}}%
\pgfpathlineto{\pgfqpoint{4.492243in}{3.864467in}}%
\pgfpathlineto{\pgfqpoint{4.514902in}{3.862332in}}%
\pgfpathlineto{\pgfqpoint{4.537561in}{3.863119in}}%
\pgfpathlineto{\pgfqpoint{4.560219in}{3.866839in}}%
\pgfpathlineto{\pgfqpoint{4.582878in}{3.871839in}}%
\pgfpathlineto{\pgfqpoint{4.605537in}{3.878814in}}%
\pgfpathlineto{\pgfqpoint{4.650854in}{3.894454in}}%
\pgfpathlineto{\pgfqpoint{4.786807in}{3.930116in}}%
\pgfpathlineto{\pgfqpoint{4.832124in}{3.946396in}}%
\pgfpathlineto{\pgfqpoint{4.854783in}{3.956778in}}%
\pgfpathlineto{\pgfqpoint{4.922759in}{3.982136in}}%
\pgfpathlineto{\pgfqpoint{4.945418in}{3.988724in}}%
\pgfpathlineto{\pgfqpoint{4.968077in}{3.993115in}}%
\pgfpathlineto{\pgfqpoint{5.013394in}{3.997766in}}%
\pgfpathlineto{\pgfqpoint{5.036053in}{3.997380in}}%
\pgfpathlineto{\pgfqpoint{5.081370in}{3.990995in}}%
\pgfpathlineto{\pgfqpoint{5.104029in}{3.986063in}}%
\pgfpathlineto{\pgfqpoint{5.126688in}{3.979151in}}%
\pgfpathlineto{\pgfqpoint{5.172005in}{3.969135in}}%
\pgfpathlineto{\pgfqpoint{5.194664in}{3.963127in}}%
\pgfpathlineto{\pgfqpoint{5.239982in}{3.954847in}}%
\pgfpathlineto{\pgfqpoint{5.262640in}{3.954196in}}%
\pgfpathlineto{\pgfqpoint{5.285299in}{3.956218in}}%
\pgfpathlineto{\pgfqpoint{5.307958in}{3.959589in}}%
\pgfpathlineto{\pgfqpoint{5.330617in}{3.965070in}}%
\pgfpathlineto{\pgfqpoint{5.398593in}{3.986946in}}%
\pgfpathlineto{\pgfqpoint{5.443910in}{4.007089in}}%
\pgfpathlineto{\pgfqpoint{5.466569in}{4.018435in}}%
\pgfpathlineto{\pgfqpoint{5.534545in}{4.056000in}}%
\pgfpathlineto{\pgfqpoint{5.534545in}{4.056000in}}%
\pgfusepath{stroke}%
\end{pgfscope}%
\begin{pgfscope}%
\pgfsetrectcap%
\pgfsetmiterjoin%
\pgfsetlinewidth{0.803000pt}%
\definecolor{currentstroke}{rgb}{0.000000,0.000000,0.000000}%
\pgfsetstrokecolor{currentstroke}%
\pgfsetdash{}{0pt}%
\pgfpathmoveto{\pgfqpoint{0.800000in}{0.528000in}}%
\pgfpathlineto{\pgfqpoint{0.800000in}{4.224000in}}%
\pgfusepath{stroke}%
\end{pgfscope}%
\begin{pgfscope}%
\pgfsetrectcap%
\pgfsetmiterjoin%
\pgfsetlinewidth{0.803000pt}%
\definecolor{currentstroke}{rgb}{0.000000,0.000000,0.000000}%
\pgfsetstrokecolor{currentstroke}%
\pgfsetdash{}{0pt}%
\pgfpathmoveto{\pgfqpoint{5.760000in}{0.528000in}}%
\pgfpathlineto{\pgfqpoint{5.760000in}{4.224000in}}%
\pgfusepath{stroke}%
\end{pgfscope}%
\begin{pgfscope}%
\pgfsetrectcap%
\pgfsetmiterjoin%
\pgfsetlinewidth{0.803000pt}%
\definecolor{currentstroke}{rgb}{0.000000,0.000000,0.000000}%
\pgfsetstrokecolor{currentstroke}%
\pgfsetdash{}{0pt}%
\pgfpathmoveto{\pgfqpoint{0.800000in}{0.528000in}}%
\pgfpathlineto{\pgfqpoint{5.760000in}{0.528000in}}%
\pgfusepath{stroke}%
\end{pgfscope}%
\begin{pgfscope}%
\pgfsetrectcap%
\pgfsetmiterjoin%
\pgfsetlinewidth{0.803000pt}%
\definecolor{currentstroke}{rgb}{0.000000,0.000000,0.000000}%
\pgfsetstrokecolor{currentstroke}%
\pgfsetdash{}{0pt}%
\pgfpathmoveto{\pgfqpoint{0.800000in}{4.224000in}}%
\pgfpathlineto{\pgfqpoint{5.760000in}{4.224000in}}%
\pgfusepath{stroke}%
\end{pgfscope}%
\begin{pgfscope}%
\pgfsetbuttcap%
\pgfsetmiterjoin%
\definecolor{currentfill}{rgb}{1.000000,1.000000,1.000000}%
\pgfsetfillcolor{currentfill}%
\pgfsetfillopacity{0.800000}%
\pgfsetlinewidth{1.003750pt}%
\definecolor{currentstroke}{rgb}{0.800000,0.800000,0.800000}%
\pgfsetstrokecolor{currentstroke}%
\pgfsetstrokeopacity{0.800000}%
\pgfsetdash{}{0pt}%
\pgfpathmoveto{\pgfqpoint{5.607222in}{4.043444in}}%
\pgfpathlineto{\pgfqpoint{5.662778in}{4.043444in}}%
\pgfpathquadraticcurveto{\pgfqpoint{5.690556in}{4.043444in}}{\pgfqpoint{5.690556in}{4.071222in}}%
\pgfpathlineto{\pgfqpoint{5.690556in}{4.126778in}}%
\pgfpathquadraticcurveto{\pgfqpoint{5.690556in}{4.154556in}}{\pgfqpoint{5.662778in}{4.154556in}}%
\pgfpathlineto{\pgfqpoint{5.607222in}{4.154556in}}%
\pgfpathquadraticcurveto{\pgfqpoint{5.579444in}{4.154556in}}{\pgfqpoint{5.579444in}{4.126778in}}%
\pgfpathlineto{\pgfqpoint{5.579444in}{4.071222in}}%
\pgfpathquadraticcurveto{\pgfqpoint{5.579444in}{4.043444in}}{\pgfqpoint{5.607222in}{4.043444in}}%
\pgfpathclose%
\pgfusepath{stroke,fill}%
\end{pgfscope}%
\end{pgfpicture}%
\makeatother%
\endgroup%

        }
        \caption{Constant velocity rescaling at temperature $T = 94.4$ \si{K} ($dt =0.0046$).}
        \label{task1_fixT}
    \end{subfigure}
    \caption{Plot of total energy as function of the simulation time. Time and energy are presented in Lennard-Jones units \cite{ljunits}.}
\end{figure}

\paragraph{Constant velocity rescaling}
This procedure aims to attend the most stable \textit{NVE} configuration imposing a fixed temperature. The system is indeed evolved using the \textit{NVE} conditions, but each time step is corrected rescaling the velocities in order to attend the chosen temperature. The resulting energy of the system is still a statistically conserved value in average over time, but it's way higher than the one obtained by a free \textit{NVE} regime simulation. This result is show in figure (\ref{task1_fixT}), where we see that the system attends in few steps the most stable configuration under the condition to be at the forced temperature of $94.4$ \si{K}.

\section{Structural properties} \label{sec:structural}

\subsection{Correlation function $g(r)$ and $S(k)$} 
% plot g(r), S(k), show Rahman results (image)

Those two functions give important informations on the structural properties of the system and they are respectectively defined as \cite{rahman}:

\begin{align}\label{correlation}
g(r) &= \frac{\varrho(r)}{\varrho_0} \\
S(k) &= 1 + \varrho_0 \int_0^\infty \frac{\sin(kr)}{kr} [g(r) - 1] 4\pi r^2 dr
\end{align}

where $\varrho_0 = N/V$ is the particle density at the infinity limit. 

\begin{figure}
    \begin{subfigure}{0.5\textwidth}
        \resizebox{\textwidth}{!}{
            %% Creator: Matplotlib, PGF backend
%%
%% To include the figure in your LaTeX document, write
%%   \input{<filename>.pgf}
%%
%% Make sure the required packages are loaded in your preamble
%%   \usepackage{pgf}
%%
%% Also ensure that all the required font packages are loaded; for instance,
%% the lmodern package is sometimes necessary when using math font.
%%   \usepackage{lmodern}
%%
%% Figures using additional raster images can only be included by \input if
%% they are in the same directory as the main LaTeX file. For loading figures
%% from other directories you can use the `import` package
%%   \usepackage{import}
%%
%% and then include the figures with
%%   \import{<path to file>}{<filename>.pgf}
%%
%% Matplotlib used the following preamble
%%   \usepackage[utf8]{inputenc}
%%   \usepackage[T1]{fontenc}
%%   \usepackage{siunitx}
%%
\begingroup%
\makeatletter%
\begin{pgfpicture}%
\pgfpathrectangle{\pgfpointorigin}{\pgfqpoint{5.200000in}{4.000000in}}%
\pgfusepath{use as bounding box, clip}%
\begin{pgfscope}%
\pgfsetbuttcap%
\pgfsetmiterjoin%
\definecolor{currentfill}{rgb}{1.000000,1.000000,1.000000}%
\pgfsetfillcolor{currentfill}%
\pgfsetlinewidth{0.000000pt}%
\definecolor{currentstroke}{rgb}{1.000000,1.000000,1.000000}%
\pgfsetstrokecolor{currentstroke}%
\pgfsetdash{}{0pt}%
\pgfpathmoveto{\pgfqpoint{0.000000in}{0.000000in}}%
\pgfpathlineto{\pgfqpoint{5.200000in}{0.000000in}}%
\pgfpathlineto{\pgfqpoint{5.200000in}{4.000000in}}%
\pgfpathlineto{\pgfqpoint{0.000000in}{4.000000in}}%
\pgfpathlineto{\pgfqpoint{0.000000in}{0.000000in}}%
\pgfpathclose%
\pgfusepath{fill}%
\end{pgfscope}%
\begin{pgfscope}%
\pgfsetbuttcap%
\pgfsetmiterjoin%
\definecolor{currentfill}{rgb}{1.000000,1.000000,1.000000}%
\pgfsetfillcolor{currentfill}%
\pgfsetlinewidth{0.000000pt}%
\definecolor{currentstroke}{rgb}{0.000000,0.000000,0.000000}%
\pgfsetstrokecolor{currentstroke}%
\pgfsetstrokeopacity{0.000000}%
\pgfsetdash{}{0pt}%
\pgfpathmoveto{\pgfqpoint{0.650000in}{0.440000in}}%
\pgfpathlineto{\pgfqpoint{4.680000in}{0.440000in}}%
\pgfpathlineto{\pgfqpoint{4.680000in}{3.520000in}}%
\pgfpathlineto{\pgfqpoint{0.650000in}{3.520000in}}%
\pgfpathlineto{\pgfqpoint{0.650000in}{0.440000in}}%
\pgfpathclose%
\pgfusepath{fill}%
\end{pgfscope}%
\begin{pgfscope}%
\pgfsetbuttcap%
\pgfsetroundjoin%
\definecolor{currentfill}{rgb}{0.000000,0.000000,0.000000}%
\pgfsetfillcolor{currentfill}%
\pgfsetlinewidth{0.803000pt}%
\definecolor{currentstroke}{rgb}{0.000000,0.000000,0.000000}%
\pgfsetstrokecolor{currentstroke}%
\pgfsetdash{}{0pt}%
\pgfsys@defobject{currentmarker}{\pgfqpoint{0.000000in}{-0.048611in}}{\pgfqpoint{0.000000in}{0.000000in}}{%
\pgfpathmoveto{\pgfqpoint{0.000000in}{0.000000in}}%
\pgfpathlineto{\pgfqpoint{0.000000in}{-0.048611in}}%
\pgfusepath{stroke,fill}%
}%
\begin{pgfscope}%
\pgfsys@transformshift{0.827055in}{0.440000in}%
\pgfsys@useobject{currentmarker}{}%
\end{pgfscope}%
\end{pgfscope}%
\begin{pgfscope}%
\definecolor{textcolor}{rgb}{0.000000,0.000000,0.000000}%
\pgfsetstrokecolor{textcolor}%
\pgfsetfillcolor{textcolor}%
\pgftext[x=0.827055in,y=0.342778in,,top]{\color{textcolor}\rmfamily\fontsize{10.000000}{12.000000}\selectfont \(\displaystyle {0.0}\)}%
\end{pgfscope}%
\begin{pgfscope}%
\pgfsetbuttcap%
\pgfsetroundjoin%
\definecolor{currentfill}{rgb}{0.000000,0.000000,0.000000}%
\pgfsetfillcolor{currentfill}%
\pgfsetlinewidth{0.803000pt}%
\definecolor{currentstroke}{rgb}{0.000000,0.000000,0.000000}%
\pgfsetstrokecolor{currentstroke}%
\pgfsetdash{}{0pt}%
\pgfsys@defobject{currentmarker}{\pgfqpoint{0.000000in}{-0.048611in}}{\pgfqpoint{0.000000in}{0.000000in}}{%
\pgfpathmoveto{\pgfqpoint{0.000000in}{0.000000in}}%
\pgfpathlineto{\pgfqpoint{0.000000in}{-0.048611in}}%
\pgfusepath{stroke,fill}%
}%
\begin{pgfscope}%
\pgfsys@transformshift{1.355536in}{0.440000in}%
\pgfsys@useobject{currentmarker}{}%
\end{pgfscope}%
\end{pgfscope}%
\begin{pgfscope}%
\definecolor{textcolor}{rgb}{0.000000,0.000000,0.000000}%
\pgfsetstrokecolor{textcolor}%
\pgfsetfillcolor{textcolor}%
\pgftext[x=1.355536in,y=0.342778in,,top]{\color{textcolor}\rmfamily\fontsize{10.000000}{12.000000}\selectfont \(\displaystyle {2.5}\)}%
\end{pgfscope}%
\begin{pgfscope}%
\pgfsetbuttcap%
\pgfsetroundjoin%
\definecolor{currentfill}{rgb}{0.000000,0.000000,0.000000}%
\pgfsetfillcolor{currentfill}%
\pgfsetlinewidth{0.803000pt}%
\definecolor{currentstroke}{rgb}{0.000000,0.000000,0.000000}%
\pgfsetstrokecolor{currentstroke}%
\pgfsetdash{}{0pt}%
\pgfsys@defobject{currentmarker}{\pgfqpoint{0.000000in}{-0.048611in}}{\pgfqpoint{0.000000in}{0.000000in}}{%
\pgfpathmoveto{\pgfqpoint{0.000000in}{0.000000in}}%
\pgfpathlineto{\pgfqpoint{0.000000in}{-0.048611in}}%
\pgfusepath{stroke,fill}%
}%
\begin{pgfscope}%
\pgfsys@transformshift{1.884016in}{0.440000in}%
\pgfsys@useobject{currentmarker}{}%
\end{pgfscope}%
\end{pgfscope}%
\begin{pgfscope}%
\definecolor{textcolor}{rgb}{0.000000,0.000000,0.000000}%
\pgfsetstrokecolor{textcolor}%
\pgfsetfillcolor{textcolor}%
\pgftext[x=1.884016in,y=0.342778in,,top]{\color{textcolor}\rmfamily\fontsize{10.000000}{12.000000}\selectfont \(\displaystyle {5.0}\)}%
\end{pgfscope}%
\begin{pgfscope}%
\pgfsetbuttcap%
\pgfsetroundjoin%
\definecolor{currentfill}{rgb}{0.000000,0.000000,0.000000}%
\pgfsetfillcolor{currentfill}%
\pgfsetlinewidth{0.803000pt}%
\definecolor{currentstroke}{rgb}{0.000000,0.000000,0.000000}%
\pgfsetstrokecolor{currentstroke}%
\pgfsetdash{}{0pt}%
\pgfsys@defobject{currentmarker}{\pgfqpoint{0.000000in}{-0.048611in}}{\pgfqpoint{0.000000in}{0.000000in}}{%
\pgfpathmoveto{\pgfqpoint{0.000000in}{0.000000in}}%
\pgfpathlineto{\pgfqpoint{0.000000in}{-0.048611in}}%
\pgfusepath{stroke,fill}%
}%
\begin{pgfscope}%
\pgfsys@transformshift{2.412496in}{0.440000in}%
\pgfsys@useobject{currentmarker}{}%
\end{pgfscope}%
\end{pgfscope}%
\begin{pgfscope}%
\definecolor{textcolor}{rgb}{0.000000,0.000000,0.000000}%
\pgfsetstrokecolor{textcolor}%
\pgfsetfillcolor{textcolor}%
\pgftext[x=2.412496in,y=0.342778in,,top]{\color{textcolor}\rmfamily\fontsize{10.000000}{12.000000}\selectfont \(\displaystyle {7.5}\)}%
\end{pgfscope}%
\begin{pgfscope}%
\pgfsetbuttcap%
\pgfsetroundjoin%
\definecolor{currentfill}{rgb}{0.000000,0.000000,0.000000}%
\pgfsetfillcolor{currentfill}%
\pgfsetlinewidth{0.803000pt}%
\definecolor{currentstroke}{rgb}{0.000000,0.000000,0.000000}%
\pgfsetstrokecolor{currentstroke}%
\pgfsetdash{}{0pt}%
\pgfsys@defobject{currentmarker}{\pgfqpoint{0.000000in}{-0.048611in}}{\pgfqpoint{0.000000in}{0.000000in}}{%
\pgfpathmoveto{\pgfqpoint{0.000000in}{0.000000in}}%
\pgfpathlineto{\pgfqpoint{0.000000in}{-0.048611in}}%
\pgfusepath{stroke,fill}%
}%
\begin{pgfscope}%
\pgfsys@transformshift{2.940977in}{0.440000in}%
\pgfsys@useobject{currentmarker}{}%
\end{pgfscope}%
\end{pgfscope}%
\begin{pgfscope}%
\definecolor{textcolor}{rgb}{0.000000,0.000000,0.000000}%
\pgfsetstrokecolor{textcolor}%
\pgfsetfillcolor{textcolor}%
\pgftext[x=2.940977in,y=0.342778in,,top]{\color{textcolor}\rmfamily\fontsize{10.000000}{12.000000}\selectfont \(\displaystyle {10.0}\)}%
\end{pgfscope}%
\begin{pgfscope}%
\pgfsetbuttcap%
\pgfsetroundjoin%
\definecolor{currentfill}{rgb}{0.000000,0.000000,0.000000}%
\pgfsetfillcolor{currentfill}%
\pgfsetlinewidth{0.803000pt}%
\definecolor{currentstroke}{rgb}{0.000000,0.000000,0.000000}%
\pgfsetstrokecolor{currentstroke}%
\pgfsetdash{}{0pt}%
\pgfsys@defobject{currentmarker}{\pgfqpoint{0.000000in}{-0.048611in}}{\pgfqpoint{0.000000in}{0.000000in}}{%
\pgfpathmoveto{\pgfqpoint{0.000000in}{0.000000in}}%
\pgfpathlineto{\pgfqpoint{0.000000in}{-0.048611in}}%
\pgfusepath{stroke,fill}%
}%
\begin{pgfscope}%
\pgfsys@transformshift{3.469457in}{0.440000in}%
\pgfsys@useobject{currentmarker}{}%
\end{pgfscope}%
\end{pgfscope}%
\begin{pgfscope}%
\definecolor{textcolor}{rgb}{0.000000,0.000000,0.000000}%
\pgfsetstrokecolor{textcolor}%
\pgfsetfillcolor{textcolor}%
\pgftext[x=3.469457in,y=0.342778in,,top]{\color{textcolor}\rmfamily\fontsize{10.000000}{12.000000}\selectfont \(\displaystyle {12.5}\)}%
\end{pgfscope}%
\begin{pgfscope}%
\pgfsetbuttcap%
\pgfsetroundjoin%
\definecolor{currentfill}{rgb}{0.000000,0.000000,0.000000}%
\pgfsetfillcolor{currentfill}%
\pgfsetlinewidth{0.803000pt}%
\definecolor{currentstroke}{rgb}{0.000000,0.000000,0.000000}%
\pgfsetstrokecolor{currentstroke}%
\pgfsetdash{}{0pt}%
\pgfsys@defobject{currentmarker}{\pgfqpoint{0.000000in}{-0.048611in}}{\pgfqpoint{0.000000in}{0.000000in}}{%
\pgfpathmoveto{\pgfqpoint{0.000000in}{0.000000in}}%
\pgfpathlineto{\pgfqpoint{0.000000in}{-0.048611in}}%
\pgfusepath{stroke,fill}%
}%
\begin{pgfscope}%
\pgfsys@transformshift{3.997937in}{0.440000in}%
\pgfsys@useobject{currentmarker}{}%
\end{pgfscope}%
\end{pgfscope}%
\begin{pgfscope}%
\definecolor{textcolor}{rgb}{0.000000,0.000000,0.000000}%
\pgfsetstrokecolor{textcolor}%
\pgfsetfillcolor{textcolor}%
\pgftext[x=3.997937in,y=0.342778in,,top]{\color{textcolor}\rmfamily\fontsize{10.000000}{12.000000}\selectfont \(\displaystyle {15.0}\)}%
\end{pgfscope}%
\begin{pgfscope}%
\pgfsetbuttcap%
\pgfsetroundjoin%
\definecolor{currentfill}{rgb}{0.000000,0.000000,0.000000}%
\pgfsetfillcolor{currentfill}%
\pgfsetlinewidth{0.803000pt}%
\definecolor{currentstroke}{rgb}{0.000000,0.000000,0.000000}%
\pgfsetstrokecolor{currentstroke}%
\pgfsetdash{}{0pt}%
\pgfsys@defobject{currentmarker}{\pgfqpoint{0.000000in}{-0.048611in}}{\pgfqpoint{0.000000in}{0.000000in}}{%
\pgfpathmoveto{\pgfqpoint{0.000000in}{0.000000in}}%
\pgfpathlineto{\pgfqpoint{0.000000in}{-0.048611in}}%
\pgfusepath{stroke,fill}%
}%
\begin{pgfscope}%
\pgfsys@transformshift{4.526418in}{0.440000in}%
\pgfsys@useobject{currentmarker}{}%
\end{pgfscope}%
\end{pgfscope}%
\begin{pgfscope}%
\definecolor{textcolor}{rgb}{0.000000,0.000000,0.000000}%
\pgfsetstrokecolor{textcolor}%
\pgfsetfillcolor{textcolor}%
\pgftext[x=4.526418in,y=0.342778in,,top]{\color{textcolor}\rmfamily\fontsize{10.000000}{12.000000}\selectfont \(\displaystyle {17.5}\)}%
\end{pgfscope}%
\begin{pgfscope}%
\definecolor{textcolor}{rgb}{0.000000,0.000000,0.000000}%
\pgfsetstrokecolor{textcolor}%
\pgfsetfillcolor{textcolor}%
\pgftext[x=2.665000in,y=0.164567in,,top]{\color{textcolor}\rmfamily\fontsize{10.000000}{12.000000}\selectfont \(\displaystyle r\) [\si{\angstrom}]}%
\end{pgfscope}%
\begin{pgfscope}%
\pgfsetbuttcap%
\pgfsetroundjoin%
\definecolor{currentfill}{rgb}{0.000000,0.000000,0.000000}%
\pgfsetfillcolor{currentfill}%
\pgfsetlinewidth{0.803000pt}%
\definecolor{currentstroke}{rgb}{0.000000,0.000000,0.000000}%
\pgfsetstrokecolor{currentstroke}%
\pgfsetdash{}{0pt}%
\pgfsys@defobject{currentmarker}{\pgfqpoint{-0.048611in}{0.000000in}}{\pgfqpoint{-0.000000in}{0.000000in}}{%
\pgfpathmoveto{\pgfqpoint{-0.000000in}{0.000000in}}%
\pgfpathlineto{\pgfqpoint{-0.048611in}{0.000000in}}%
\pgfusepath{stroke,fill}%
}%
\begin{pgfscope}%
\pgfsys@transformshift{0.650000in}{0.621176in}%
\pgfsys@useobject{currentmarker}{}%
\end{pgfscope}%
\end{pgfscope}%
\begin{pgfscope}%
\definecolor{textcolor}{rgb}{0.000000,0.000000,0.000000}%
\pgfsetstrokecolor{textcolor}%
\pgfsetfillcolor{textcolor}%
\pgftext[x=0.375308in, y=0.573349in, left, base]{\color{textcolor}\rmfamily\fontsize{10.000000}{12.000000}\selectfont \(\displaystyle {0.0}\)}%
\end{pgfscope}%
\begin{pgfscope}%
\pgfsetbuttcap%
\pgfsetroundjoin%
\definecolor{currentfill}{rgb}{0.000000,0.000000,0.000000}%
\pgfsetfillcolor{currentfill}%
\pgfsetlinewidth{0.803000pt}%
\definecolor{currentstroke}{rgb}{0.000000,0.000000,0.000000}%
\pgfsetstrokecolor{currentstroke}%
\pgfsetdash{}{0pt}%
\pgfsys@defobject{currentmarker}{\pgfqpoint{-0.048611in}{0.000000in}}{\pgfqpoint{-0.000000in}{0.000000in}}{%
\pgfpathmoveto{\pgfqpoint{-0.000000in}{0.000000in}}%
\pgfpathlineto{\pgfqpoint{-0.048611in}{0.000000in}}%
\pgfusepath{stroke,fill}%
}%
\begin{pgfscope}%
\pgfsys@transformshift{0.650000in}{1.074118in}%
\pgfsys@useobject{currentmarker}{}%
\end{pgfscope}%
\end{pgfscope}%
\begin{pgfscope}%
\definecolor{textcolor}{rgb}{0.000000,0.000000,0.000000}%
\pgfsetstrokecolor{textcolor}%
\pgfsetfillcolor{textcolor}%
\pgftext[x=0.375308in, y=1.026290in, left, base]{\color{textcolor}\rmfamily\fontsize{10.000000}{12.000000}\selectfont \(\displaystyle {0.5}\)}%
\end{pgfscope}%
\begin{pgfscope}%
\pgfsetbuttcap%
\pgfsetroundjoin%
\definecolor{currentfill}{rgb}{0.000000,0.000000,0.000000}%
\pgfsetfillcolor{currentfill}%
\pgfsetlinewidth{0.803000pt}%
\definecolor{currentstroke}{rgb}{0.000000,0.000000,0.000000}%
\pgfsetstrokecolor{currentstroke}%
\pgfsetdash{}{0pt}%
\pgfsys@defobject{currentmarker}{\pgfqpoint{-0.048611in}{0.000000in}}{\pgfqpoint{-0.000000in}{0.000000in}}{%
\pgfpathmoveto{\pgfqpoint{-0.000000in}{0.000000in}}%
\pgfpathlineto{\pgfqpoint{-0.048611in}{0.000000in}}%
\pgfusepath{stroke,fill}%
}%
\begin{pgfscope}%
\pgfsys@transformshift{0.650000in}{1.527059in}%
\pgfsys@useobject{currentmarker}{}%
\end{pgfscope}%
\end{pgfscope}%
\begin{pgfscope}%
\definecolor{textcolor}{rgb}{0.000000,0.000000,0.000000}%
\pgfsetstrokecolor{textcolor}%
\pgfsetfillcolor{textcolor}%
\pgftext[x=0.375308in, y=1.479231in, left, base]{\color{textcolor}\rmfamily\fontsize{10.000000}{12.000000}\selectfont \(\displaystyle {1.0}\)}%
\end{pgfscope}%
\begin{pgfscope}%
\pgfsetbuttcap%
\pgfsetroundjoin%
\definecolor{currentfill}{rgb}{0.000000,0.000000,0.000000}%
\pgfsetfillcolor{currentfill}%
\pgfsetlinewidth{0.803000pt}%
\definecolor{currentstroke}{rgb}{0.000000,0.000000,0.000000}%
\pgfsetstrokecolor{currentstroke}%
\pgfsetdash{}{0pt}%
\pgfsys@defobject{currentmarker}{\pgfqpoint{-0.048611in}{0.000000in}}{\pgfqpoint{-0.000000in}{0.000000in}}{%
\pgfpathmoveto{\pgfqpoint{-0.000000in}{0.000000in}}%
\pgfpathlineto{\pgfqpoint{-0.048611in}{0.000000in}}%
\pgfusepath{stroke,fill}%
}%
\begin{pgfscope}%
\pgfsys@transformshift{0.650000in}{1.980000in}%
\pgfsys@useobject{currentmarker}{}%
\end{pgfscope}%
\end{pgfscope}%
\begin{pgfscope}%
\definecolor{textcolor}{rgb}{0.000000,0.000000,0.000000}%
\pgfsetstrokecolor{textcolor}%
\pgfsetfillcolor{textcolor}%
\pgftext[x=0.375308in, y=1.932172in, left, base]{\color{textcolor}\rmfamily\fontsize{10.000000}{12.000000}\selectfont \(\displaystyle {1.5}\)}%
\end{pgfscope}%
\begin{pgfscope}%
\pgfsetbuttcap%
\pgfsetroundjoin%
\definecolor{currentfill}{rgb}{0.000000,0.000000,0.000000}%
\pgfsetfillcolor{currentfill}%
\pgfsetlinewidth{0.803000pt}%
\definecolor{currentstroke}{rgb}{0.000000,0.000000,0.000000}%
\pgfsetstrokecolor{currentstroke}%
\pgfsetdash{}{0pt}%
\pgfsys@defobject{currentmarker}{\pgfqpoint{-0.048611in}{0.000000in}}{\pgfqpoint{-0.000000in}{0.000000in}}{%
\pgfpathmoveto{\pgfqpoint{-0.000000in}{0.000000in}}%
\pgfpathlineto{\pgfqpoint{-0.048611in}{0.000000in}}%
\pgfusepath{stroke,fill}%
}%
\begin{pgfscope}%
\pgfsys@transformshift{0.650000in}{2.432941in}%
\pgfsys@useobject{currentmarker}{}%
\end{pgfscope}%
\end{pgfscope}%
\begin{pgfscope}%
\definecolor{textcolor}{rgb}{0.000000,0.000000,0.000000}%
\pgfsetstrokecolor{textcolor}%
\pgfsetfillcolor{textcolor}%
\pgftext[x=0.375308in, y=2.385113in, left, base]{\color{textcolor}\rmfamily\fontsize{10.000000}{12.000000}\selectfont \(\displaystyle {2.0}\)}%
\end{pgfscope}%
\begin{pgfscope}%
\pgfsetbuttcap%
\pgfsetroundjoin%
\definecolor{currentfill}{rgb}{0.000000,0.000000,0.000000}%
\pgfsetfillcolor{currentfill}%
\pgfsetlinewidth{0.803000pt}%
\definecolor{currentstroke}{rgb}{0.000000,0.000000,0.000000}%
\pgfsetstrokecolor{currentstroke}%
\pgfsetdash{}{0pt}%
\pgfsys@defobject{currentmarker}{\pgfqpoint{-0.048611in}{0.000000in}}{\pgfqpoint{-0.000000in}{0.000000in}}{%
\pgfpathmoveto{\pgfqpoint{-0.000000in}{0.000000in}}%
\pgfpathlineto{\pgfqpoint{-0.048611in}{0.000000in}}%
\pgfusepath{stroke,fill}%
}%
\begin{pgfscope}%
\pgfsys@transformshift{0.650000in}{2.885882in}%
\pgfsys@useobject{currentmarker}{}%
\end{pgfscope}%
\end{pgfscope}%
\begin{pgfscope}%
\definecolor{textcolor}{rgb}{0.000000,0.000000,0.000000}%
\pgfsetstrokecolor{textcolor}%
\pgfsetfillcolor{textcolor}%
\pgftext[x=0.375308in, y=2.838055in, left, base]{\color{textcolor}\rmfamily\fontsize{10.000000}{12.000000}\selectfont \(\displaystyle {2.5}\)}%
\end{pgfscope}%
\begin{pgfscope}%
\pgfsetbuttcap%
\pgfsetroundjoin%
\definecolor{currentfill}{rgb}{0.000000,0.000000,0.000000}%
\pgfsetfillcolor{currentfill}%
\pgfsetlinewidth{0.803000pt}%
\definecolor{currentstroke}{rgb}{0.000000,0.000000,0.000000}%
\pgfsetstrokecolor{currentstroke}%
\pgfsetdash{}{0pt}%
\pgfsys@defobject{currentmarker}{\pgfqpoint{-0.048611in}{0.000000in}}{\pgfqpoint{-0.000000in}{0.000000in}}{%
\pgfpathmoveto{\pgfqpoint{-0.000000in}{0.000000in}}%
\pgfpathlineto{\pgfqpoint{-0.048611in}{0.000000in}}%
\pgfusepath{stroke,fill}%
}%
\begin{pgfscope}%
\pgfsys@transformshift{0.650000in}{3.338824in}%
\pgfsys@useobject{currentmarker}{}%
\end{pgfscope}%
\end{pgfscope}%
\begin{pgfscope}%
\definecolor{textcolor}{rgb}{0.000000,0.000000,0.000000}%
\pgfsetstrokecolor{textcolor}%
\pgfsetfillcolor{textcolor}%
\pgftext[x=0.375308in, y=3.290996in, left, base]{\color{textcolor}\rmfamily\fontsize{10.000000}{12.000000}\selectfont \(\displaystyle {3.0}\)}%
\end{pgfscope}%
\begin{pgfscope}%
\definecolor{textcolor}{rgb}{0.000000,0.000000,0.000000}%
\pgfsetstrokecolor{textcolor}%
\pgfsetfillcolor{textcolor}%
\pgftext[x=0.319753in,y=1.980000in,,bottom,rotate=90.000000]{\color{textcolor}\rmfamily\fontsize{10.000000}{12.000000}\selectfont \(\displaystyle g(r)\)}%
\end{pgfscope}%
\begin{pgfscope}%
\pgfpathrectangle{\pgfqpoint{0.650000in}{0.440000in}}{\pgfqpoint{4.030000in}{3.080000in}}%
\pgfusepath{clip}%
\pgfsetrectcap%
\pgfsetroundjoin%
\pgfsetlinewidth{1.505625pt}%
\definecolor{currentstroke}{rgb}{0.121569,0.466667,0.705882}%
\pgfsetstrokecolor{currentstroke}%
\pgfsetdash{}{0pt}%
\pgfpathmoveto{\pgfqpoint{0.833182in}{0.621176in}}%
\pgfpathlineto{\pgfqpoint{1.470336in}{0.621404in}}%
\pgfpathlineto{\pgfqpoint{1.482589in}{0.623526in}}%
\pgfpathlineto{\pgfqpoint{1.494842in}{0.638261in}}%
\pgfpathlineto{\pgfqpoint{1.507095in}{0.700973in}}%
\pgfpathlineto{\pgfqpoint{1.519348in}{0.866924in}}%
\pgfpathlineto{\pgfqpoint{1.531601in}{1.174955in}}%
\pgfpathlineto{\pgfqpoint{1.543854in}{1.603773in}}%
\pgfpathlineto{\pgfqpoint{1.568360in}{2.541167in}}%
\pgfpathlineto{\pgfqpoint{1.580613in}{2.908406in}}%
\pgfpathlineto{\pgfqpoint{1.592866in}{3.109955in}}%
\pgfpathlineto{\pgfqpoint{1.605119in}{3.209943in}}%
\pgfpathlineto{\pgfqpoint{1.617372in}{3.185535in}}%
\pgfpathlineto{\pgfqpoint{1.629625in}{3.107170in}}%
\pgfpathlineto{\pgfqpoint{1.641877in}{2.957910in}}%
\pgfpathlineto{\pgfqpoint{1.690889in}{2.277545in}}%
\pgfpathlineto{\pgfqpoint{1.703142in}{2.123198in}}%
\pgfpathlineto{\pgfqpoint{1.727648in}{1.873131in}}%
\pgfpathlineto{\pgfqpoint{1.752154in}{1.670574in}}%
\pgfpathlineto{\pgfqpoint{1.764407in}{1.585354in}}%
\pgfpathlineto{\pgfqpoint{1.776660in}{1.510468in}}%
\pgfpathlineto{\pgfqpoint{1.788913in}{1.455043in}}%
\pgfpathlineto{\pgfqpoint{1.813419in}{1.364815in}}%
\pgfpathlineto{\pgfqpoint{1.825672in}{1.328239in}}%
\pgfpathlineto{\pgfqpoint{1.862431in}{1.248227in}}%
\pgfpathlineto{\pgfqpoint{1.874684in}{1.229983in}}%
\pgfpathlineto{\pgfqpoint{1.886937in}{1.222881in}}%
\pgfpathlineto{\pgfqpoint{1.899190in}{1.212504in}}%
\pgfpathlineto{\pgfqpoint{1.911443in}{1.196470in}}%
\pgfpathlineto{\pgfqpoint{1.923696in}{1.188402in}}%
\pgfpathlineto{\pgfqpoint{1.935949in}{1.185368in}}%
\pgfpathlineto{\pgfqpoint{1.948202in}{1.180873in}}%
\pgfpathlineto{\pgfqpoint{1.972708in}{1.184927in}}%
\pgfpathlineto{\pgfqpoint{1.984960in}{1.184684in}}%
\pgfpathlineto{\pgfqpoint{2.009466in}{1.212166in}}%
\pgfpathlineto{\pgfqpoint{2.021719in}{1.223266in}}%
\pgfpathlineto{\pgfqpoint{2.046225in}{1.257104in}}%
\pgfpathlineto{\pgfqpoint{2.082984in}{1.329205in}}%
\pgfpathlineto{\pgfqpoint{2.107490in}{1.382445in}}%
\pgfpathlineto{\pgfqpoint{2.119743in}{1.414471in}}%
\pgfpathlineto{\pgfqpoint{2.144249in}{1.472764in}}%
\pgfpathlineto{\pgfqpoint{2.168755in}{1.538652in}}%
\pgfpathlineto{\pgfqpoint{2.181008in}{1.566161in}}%
\pgfpathlineto{\pgfqpoint{2.193261in}{1.586873in}}%
\pgfpathlineto{\pgfqpoint{2.205514in}{1.613569in}}%
\pgfpathlineto{\pgfqpoint{2.242273in}{1.684017in}}%
\pgfpathlineto{\pgfqpoint{2.266779in}{1.718697in}}%
\pgfpathlineto{\pgfqpoint{2.279032in}{1.734105in}}%
\pgfpathlineto{\pgfqpoint{2.291285in}{1.752462in}}%
\pgfpathlineto{\pgfqpoint{2.303538in}{1.757817in}}%
\pgfpathlineto{\pgfqpoint{2.315791in}{1.764601in}}%
\pgfpathlineto{\pgfqpoint{2.328043in}{1.769201in}}%
\pgfpathlineto{\pgfqpoint{2.340296in}{1.765747in}}%
\pgfpathlineto{\pgfqpoint{2.352549in}{1.757302in}}%
\pgfpathlineto{\pgfqpoint{2.389308in}{1.719731in}}%
\pgfpathlineto{\pgfqpoint{2.413814in}{1.680835in}}%
\pgfpathlineto{\pgfqpoint{2.426067in}{1.656499in}}%
\pgfpathlineto{\pgfqpoint{2.438320in}{1.628274in}}%
\pgfpathlineto{\pgfqpoint{2.450573in}{1.605655in}}%
\pgfpathlineto{\pgfqpoint{2.475079in}{1.565409in}}%
\pgfpathlineto{\pgfqpoint{2.499585in}{1.515797in}}%
\pgfpathlineto{\pgfqpoint{2.511838in}{1.498011in}}%
\pgfpathlineto{\pgfqpoint{2.548597in}{1.451570in}}%
\pgfpathlineto{\pgfqpoint{2.560850in}{1.438122in}}%
\pgfpathlineto{\pgfqpoint{2.573103in}{1.427020in}}%
\pgfpathlineto{\pgfqpoint{2.585356in}{1.413513in}}%
\pgfpathlineto{\pgfqpoint{2.597609in}{1.408698in}}%
\pgfpathlineto{\pgfqpoint{2.609862in}{1.399826in}}%
\pgfpathlineto{\pgfqpoint{2.634368in}{1.396004in}}%
\pgfpathlineto{\pgfqpoint{2.671126in}{1.398159in}}%
\pgfpathlineto{\pgfqpoint{2.683379in}{1.400004in}}%
\pgfpathlineto{\pgfqpoint{2.695632in}{1.405088in}}%
\pgfpathlineto{\pgfqpoint{2.707885in}{1.413457in}}%
\pgfpathlineto{\pgfqpoint{2.720138in}{1.418623in}}%
\pgfpathlineto{\pgfqpoint{2.756897in}{1.454650in}}%
\pgfpathlineto{\pgfqpoint{2.769150in}{1.460107in}}%
\pgfpathlineto{\pgfqpoint{2.818162in}{1.510418in}}%
\pgfpathlineto{\pgfqpoint{2.842668in}{1.530604in}}%
\pgfpathlineto{\pgfqpoint{2.854921in}{1.543976in}}%
\pgfpathlineto{\pgfqpoint{2.879427in}{1.565892in}}%
\pgfpathlineto{\pgfqpoint{2.916186in}{1.592054in}}%
\pgfpathlineto{\pgfqpoint{2.928439in}{1.596168in}}%
\pgfpathlineto{\pgfqpoint{2.940692in}{1.597771in}}%
\pgfpathlineto{\pgfqpoint{2.952945in}{1.600812in}}%
\pgfpathlineto{\pgfqpoint{2.965198in}{1.609879in}}%
\pgfpathlineto{\pgfqpoint{2.989704in}{1.605620in}}%
\pgfpathlineto{\pgfqpoint{3.001957in}{1.607175in}}%
\pgfpathlineto{\pgfqpoint{3.014209in}{1.606354in}}%
\pgfpathlineto{\pgfqpoint{3.026462in}{1.600837in}}%
\pgfpathlineto{\pgfqpoint{3.038715in}{1.598575in}}%
\pgfpathlineto{\pgfqpoint{3.050968in}{1.597683in}}%
\pgfpathlineto{\pgfqpoint{3.063221in}{1.591977in}}%
\pgfpathlineto{\pgfqpoint{3.075474in}{1.583475in}}%
\pgfpathlineto{\pgfqpoint{3.087727in}{1.579170in}}%
\pgfpathlineto{\pgfqpoint{3.112233in}{1.565460in}}%
\pgfpathlineto{\pgfqpoint{3.124486in}{1.555901in}}%
\pgfpathlineto{\pgfqpoint{3.148992in}{1.542338in}}%
\pgfpathlineto{\pgfqpoint{3.161245in}{1.534148in}}%
\pgfpathlineto{\pgfqpoint{3.173498in}{1.531250in}}%
\pgfpathlineto{\pgfqpoint{3.185751in}{1.526077in}}%
\pgfpathlineto{\pgfqpoint{3.210257in}{1.512078in}}%
\pgfpathlineto{\pgfqpoint{3.222510in}{1.504123in}}%
\pgfpathlineto{\pgfqpoint{3.271522in}{1.486883in}}%
\pgfpathlineto{\pgfqpoint{3.283775in}{1.483294in}}%
\pgfpathlineto{\pgfqpoint{3.296028in}{1.483862in}}%
\pgfpathlineto{\pgfqpoint{3.320534in}{1.478766in}}%
\pgfpathlineto{\pgfqpoint{3.332787in}{1.480557in}}%
\pgfpathlineto{\pgfqpoint{3.345040in}{1.478497in}}%
\pgfpathlineto{\pgfqpoint{3.394051in}{1.487576in}}%
\pgfpathlineto{\pgfqpoint{3.418557in}{1.496832in}}%
\pgfpathlineto{\pgfqpoint{3.430810in}{1.498701in}}%
\pgfpathlineto{\pgfqpoint{3.443063in}{1.498847in}}%
\pgfpathlineto{\pgfqpoint{3.492075in}{1.516125in}}%
\pgfpathlineto{\pgfqpoint{3.504328in}{1.524251in}}%
\pgfpathlineto{\pgfqpoint{3.516581in}{1.526749in}}%
\pgfpathlineto{\pgfqpoint{3.541087in}{1.534749in}}%
\pgfpathlineto{\pgfqpoint{3.553340in}{1.541228in}}%
\pgfpathlineto{\pgfqpoint{3.565593in}{1.543439in}}%
\pgfpathlineto{\pgfqpoint{3.577846in}{1.548865in}}%
\pgfpathlineto{\pgfqpoint{3.590099in}{1.551991in}}%
\pgfpathlineto{\pgfqpoint{3.602352in}{1.551709in}}%
\pgfpathlineto{\pgfqpoint{3.614605in}{1.554878in}}%
\pgfpathlineto{\pgfqpoint{3.639111in}{1.556529in}}%
\pgfpathlineto{\pgfqpoint{3.651364in}{1.559420in}}%
\pgfpathlineto{\pgfqpoint{3.688123in}{1.554921in}}%
\pgfpathlineto{\pgfqpoint{3.700375in}{1.558404in}}%
\pgfpathlineto{\pgfqpoint{3.724881in}{1.555325in}}%
\pgfpathlineto{\pgfqpoint{3.737134in}{1.552084in}}%
\pgfpathlineto{\pgfqpoint{3.749387in}{1.551133in}}%
\pgfpathlineto{\pgfqpoint{3.798399in}{1.541462in}}%
\pgfpathlineto{\pgfqpoint{3.810652in}{1.537222in}}%
\pgfpathlineto{\pgfqpoint{3.896423in}{1.519470in}}%
\pgfpathlineto{\pgfqpoint{3.908676in}{1.515327in}}%
\pgfpathlineto{\pgfqpoint{3.920929in}{1.516683in}}%
\pgfpathlineto{\pgfqpoint{3.945435in}{1.512312in}}%
\pgfpathlineto{\pgfqpoint{3.969941in}{1.511823in}}%
\pgfpathlineto{\pgfqpoint{3.994447in}{1.512920in}}%
\pgfpathlineto{\pgfqpoint{4.006700in}{1.511937in}}%
\pgfpathlineto{\pgfqpoint{4.018953in}{1.512159in}}%
\pgfpathlineto{\pgfqpoint{4.031206in}{1.511007in}}%
\pgfpathlineto{\pgfqpoint{4.043458in}{1.511978in}}%
\pgfpathlineto{\pgfqpoint{4.055711in}{1.510156in}}%
\pgfpathlineto{\pgfqpoint{4.080217in}{1.509981in}}%
\pgfpathlineto{\pgfqpoint{4.092470in}{1.512868in}}%
\pgfpathlineto{\pgfqpoint{4.104723in}{1.513123in}}%
\pgfpathlineto{\pgfqpoint{4.165988in}{1.522132in}}%
\pgfpathlineto{\pgfqpoint{4.178241in}{1.524466in}}%
\pgfpathlineto{\pgfqpoint{4.190494in}{1.528650in}}%
\pgfpathlineto{\pgfqpoint{4.202747in}{1.528776in}}%
\pgfpathlineto{\pgfqpoint{4.215000in}{1.530433in}}%
\pgfpathlineto{\pgfqpoint{4.227253in}{1.534086in}}%
\pgfpathlineto{\pgfqpoint{4.251759in}{1.538002in}}%
\pgfpathlineto{\pgfqpoint{4.300771in}{1.538394in}}%
\pgfpathlineto{\pgfqpoint{4.313024in}{1.541885in}}%
\pgfpathlineto{\pgfqpoint{4.362036in}{1.540156in}}%
\pgfpathlineto{\pgfqpoint{4.374289in}{1.541212in}}%
\pgfpathlineto{\pgfqpoint{4.386542in}{1.538951in}}%
\pgfpathlineto{\pgfqpoint{4.398794in}{1.538394in}}%
\pgfpathlineto{\pgfqpoint{4.411047in}{1.539178in}}%
\pgfpathlineto{\pgfqpoint{4.423300in}{1.538224in}}%
\pgfpathlineto{\pgfqpoint{4.435553in}{1.535681in}}%
\pgfpathlineto{\pgfqpoint{4.496818in}{1.529352in}}%
\pgfpathlineto{\pgfqpoint{4.496818in}{1.529352in}}%
\pgfusepath{stroke}%
\end{pgfscope}%
\begin{pgfscope}%
\pgfpathrectangle{\pgfqpoint{0.650000in}{0.440000in}}{\pgfqpoint{4.030000in}{3.080000in}}%
\pgfusepath{clip}%
\pgfsetbuttcap%
\pgfsetroundjoin%
\pgfsetlinewidth{1.505625pt}%
\definecolor{currentstroke}{rgb}{0.000000,0.000000,0.000000}%
\pgfsetstrokecolor{currentstroke}%
\pgfsetdash{{5.550000pt}{2.400000pt}}{0.000000pt}%
\pgfpathmoveto{\pgfqpoint{1.617372in}{0.440000in}}%
\pgfpathlineto{\pgfqpoint{1.617372in}{3.185535in}}%
\pgfusepath{stroke}%
\end{pgfscope}%
\begin{pgfscope}%
\pgfpathrectangle{\pgfqpoint{0.650000in}{0.440000in}}{\pgfqpoint{4.030000in}{3.080000in}}%
\pgfusepath{clip}%
\pgfsetbuttcap%
\pgfsetroundjoin%
\pgfsetlinewidth{1.505625pt}%
\definecolor{currentstroke}{rgb}{0.000000,0.000000,0.000000}%
\pgfsetstrokecolor{currentstroke}%
\pgfsetdash{{5.550000pt}{2.400000pt}}{0.000000pt}%
\pgfpathmoveto{\pgfqpoint{2.328043in}{0.440000in}}%
\pgfpathlineto{\pgfqpoint{2.328043in}{1.769201in}}%
\pgfusepath{stroke}%
\end{pgfscope}%
\begin{pgfscope}%
\pgfpathrectangle{\pgfqpoint{0.650000in}{0.440000in}}{\pgfqpoint{4.030000in}{3.080000in}}%
\pgfusepath{clip}%
\pgfsetbuttcap%
\pgfsetroundjoin%
\pgfsetlinewidth{1.505625pt}%
\definecolor{currentstroke}{rgb}{0.000000,0.000000,0.000000}%
\pgfsetstrokecolor{currentstroke}%
\pgfsetdash{{5.550000pt}{2.400000pt}}{0.000000pt}%
\pgfpathmoveto{\pgfqpoint{2.989704in}{0.440000in}}%
\pgfpathlineto{\pgfqpoint{2.989704in}{1.605620in}}%
\pgfusepath{stroke}%
\end{pgfscope}%
\begin{pgfscope}%
\pgfpathrectangle{\pgfqpoint{0.650000in}{0.440000in}}{\pgfqpoint{4.030000in}{3.080000in}}%
\pgfusepath{clip}%
\pgfsetbuttcap%
\pgfsetroundjoin%
\definecolor{currentfill}{rgb}{1.000000,1.000000,0.000000}%
\pgfsetfillcolor{currentfill}%
\pgfsetlinewidth{1.003750pt}%
\definecolor{currentstroke}{rgb}{0.000000,0.000000,0.000000}%
\pgfsetstrokecolor{currentstroke}%
\pgfsetdash{}{0pt}%
\pgfsys@defobject{currentmarker}{\pgfqpoint{-0.041667in}{-0.041667in}}{\pgfqpoint{0.041667in}{0.041667in}}{%
\pgfpathmoveto{\pgfqpoint{0.000000in}{-0.041667in}}%
\pgfpathcurveto{\pgfqpoint{0.011050in}{-0.041667in}}{\pgfqpoint{0.021649in}{-0.037276in}}{\pgfqpoint{0.029463in}{-0.029463in}}%
\pgfpathcurveto{\pgfqpoint{0.037276in}{-0.021649in}}{\pgfqpoint{0.041667in}{-0.011050in}}{\pgfqpoint{0.041667in}{0.000000in}}%
\pgfpathcurveto{\pgfqpoint{0.041667in}{0.011050in}}{\pgfqpoint{0.037276in}{0.021649in}}{\pgfqpoint{0.029463in}{0.029463in}}%
\pgfpathcurveto{\pgfqpoint{0.021649in}{0.037276in}}{\pgfqpoint{0.011050in}{0.041667in}}{\pgfqpoint{0.000000in}{0.041667in}}%
\pgfpathcurveto{\pgfqpoint{-0.011050in}{0.041667in}}{\pgfqpoint{-0.021649in}{0.037276in}}{\pgfqpoint{-0.029463in}{0.029463in}}%
\pgfpathcurveto{\pgfqpoint{-0.037276in}{0.021649in}}{\pgfqpoint{-0.041667in}{0.011050in}}{\pgfqpoint{-0.041667in}{0.000000in}}%
\pgfpathcurveto{\pgfqpoint{-0.041667in}{-0.011050in}}{\pgfqpoint{-0.037276in}{-0.021649in}}{\pgfqpoint{-0.029463in}{-0.029463in}}%
\pgfpathcurveto{\pgfqpoint{-0.021649in}{-0.037276in}}{\pgfqpoint{-0.011050in}{-0.041667in}}{\pgfqpoint{0.000000in}{-0.041667in}}%
\pgfpathlineto{\pgfqpoint{0.000000in}{-0.041667in}}%
\pgfpathclose%
\pgfusepath{stroke,fill}%
}%
\begin{pgfscope}%
\pgfsys@transformshift{1.617372in}{3.185535in}%
\pgfsys@useobject{currentmarker}{}%
\end{pgfscope}%
\begin{pgfscope}%
\pgfsys@transformshift{2.328043in}{1.769201in}%
\pgfsys@useobject{currentmarker}{}%
\end{pgfscope}%
\begin{pgfscope}%
\pgfsys@transformshift{2.989704in}{1.605620in}%
\pgfsys@useobject{currentmarker}{}%
\end{pgfscope}%
\end{pgfscope}%
\begin{pgfscope}%
\pgfsetrectcap%
\pgfsetmiterjoin%
\pgfsetlinewidth{0.803000pt}%
\definecolor{currentstroke}{rgb}{0.000000,0.000000,0.000000}%
\pgfsetstrokecolor{currentstroke}%
\pgfsetdash{}{0pt}%
\pgfpathmoveto{\pgfqpoint{0.650000in}{0.440000in}}%
\pgfpathlineto{\pgfqpoint{0.650000in}{3.520000in}}%
\pgfusepath{stroke}%
\end{pgfscope}%
\begin{pgfscope}%
\pgfsetrectcap%
\pgfsetmiterjoin%
\pgfsetlinewidth{0.803000pt}%
\definecolor{currentstroke}{rgb}{0.000000,0.000000,0.000000}%
\pgfsetstrokecolor{currentstroke}%
\pgfsetdash{}{0pt}%
\pgfpathmoveto{\pgfqpoint{4.680000in}{0.440000in}}%
\pgfpathlineto{\pgfqpoint{4.680000in}{3.520000in}}%
\pgfusepath{stroke}%
\end{pgfscope}%
\begin{pgfscope}%
\pgfsetrectcap%
\pgfsetmiterjoin%
\pgfsetlinewidth{0.803000pt}%
\definecolor{currentstroke}{rgb}{0.000000,0.000000,0.000000}%
\pgfsetstrokecolor{currentstroke}%
\pgfsetdash{}{0pt}%
\pgfpathmoveto{\pgfqpoint{0.650000in}{0.440000in}}%
\pgfpathlineto{\pgfqpoint{4.680000in}{0.440000in}}%
\pgfusepath{stroke}%
\end{pgfscope}%
\begin{pgfscope}%
\pgfsetrectcap%
\pgfsetmiterjoin%
\pgfsetlinewidth{0.803000pt}%
\definecolor{currentstroke}{rgb}{0.000000,0.000000,0.000000}%
\pgfsetstrokecolor{currentstroke}%
\pgfsetdash{}{0pt}%
\pgfpathmoveto{\pgfqpoint{0.650000in}{3.520000in}}%
\pgfpathlineto{\pgfqpoint{4.680000in}{3.520000in}}%
\pgfusepath{stroke}%
\end{pgfscope}%
\begin{pgfscope}%
\definecolor{textcolor}{rgb}{0.000000,0.000000,0.000000}%
\pgfsetstrokecolor{textcolor}%
\pgfsetfillcolor{textcolor}%
\pgftext[x=1.617372in,y=3.324424in,,base]{\color{textcolor}\rmfamily\fontsize{10.000000}{12.000000}\selectfont 3.74 \si{\angstrom}}%
\end{pgfscope}%
\begin{pgfscope}%
\definecolor{textcolor}{rgb}{0.000000,0.000000,0.000000}%
\pgfsetstrokecolor{textcolor}%
\pgfsetfillcolor{textcolor}%
\pgftext[x=2.328043in,y=1.908090in,,base]{\color{textcolor}\rmfamily\fontsize{10.000000}{12.000000}\selectfont 7.10 \si{\angstrom}}%
\end{pgfscope}%
\begin{pgfscope}%
\definecolor{textcolor}{rgb}{0.000000,0.000000,0.000000}%
\pgfsetstrokecolor{textcolor}%
\pgfsetfillcolor{textcolor}%
\pgftext[x=2.989704in,y=1.744509in,,base]{\color{textcolor}\rmfamily\fontsize{10.000000}{12.000000}\selectfont 10.23 \si{\angstrom}}%
\end{pgfscope}%
\begin{pgfscope}%
\pgfsetbuttcap%
\pgfsetmiterjoin%
\definecolor{currentfill}{rgb}{1.000000,1.000000,1.000000}%
\pgfsetfillcolor{currentfill}%
\pgfsetfillopacity{0.800000}%
\pgfsetlinewidth{1.003750pt}%
\definecolor{currentstroke}{rgb}{0.800000,0.800000,0.800000}%
\pgfsetstrokecolor{currentstroke}%
\pgfsetstrokeopacity{0.800000}%
\pgfsetdash{}{0pt}%
\pgfpathmoveto{\pgfqpoint{3.798139in}{3.021556in}}%
\pgfpathlineto{\pgfqpoint{4.582778in}{3.021556in}}%
\pgfpathquadraticcurveto{\pgfqpoint{4.610556in}{3.021556in}}{\pgfqpoint{4.610556in}{3.049334in}}%
\pgfpathlineto{\pgfqpoint{4.610556in}{3.422778in}}%
\pgfpathquadraticcurveto{\pgfqpoint{4.610556in}{3.450556in}}{\pgfqpoint{4.582778in}{3.450556in}}%
\pgfpathlineto{\pgfqpoint{3.798139in}{3.450556in}}%
\pgfpathquadraticcurveto{\pgfqpoint{3.770361in}{3.450556in}}{\pgfqpoint{3.770361in}{3.422778in}}%
\pgfpathlineto{\pgfqpoint{3.770361in}{3.049334in}}%
\pgfpathquadraticcurveto{\pgfqpoint{3.770361in}{3.021556in}}{\pgfqpoint{3.798139in}{3.021556in}}%
\pgfpathlineto{\pgfqpoint{3.798139in}{3.021556in}}%
\pgfpathclose%
\pgfusepath{stroke,fill}%
\end{pgfscope}%
\begin{pgfscope}%
\pgfsetrectcap%
\pgfsetroundjoin%
\pgfsetlinewidth{1.505625pt}%
\definecolor{currentstroke}{rgb}{0.121569,0.466667,0.705882}%
\pgfsetstrokecolor{currentstroke}%
\pgfsetdash{}{0pt}%
\pgfpathmoveto{\pgfqpoint{3.825916in}{3.346389in}}%
\pgfpathlineto{\pgfqpoint{3.964805in}{3.346389in}}%
\pgfpathlineto{\pgfqpoint{4.103694in}{3.346389in}}%
\pgfusepath{stroke}%
\end{pgfscope}%
\begin{pgfscope}%
\definecolor{textcolor}{rgb}{0.000000,0.000000,0.000000}%
\pgfsetstrokecolor{textcolor}%
\pgfsetfillcolor{textcolor}%
\pgftext[x=4.214805in,y=3.297778in,left,base]{\color{textcolor}\rmfamily\fontsize{10.000000}{12.000000}\selectfont data}%
\end{pgfscope}%
\begin{pgfscope}%
\pgfsetbuttcap%
\pgfsetroundjoin%
\definecolor{currentfill}{rgb}{1.000000,1.000000,0.000000}%
\pgfsetfillcolor{currentfill}%
\pgfsetlinewidth{1.003750pt}%
\definecolor{currentstroke}{rgb}{0.000000,0.000000,0.000000}%
\pgfsetstrokecolor{currentstroke}%
\pgfsetdash{}{0pt}%
\pgfsys@defobject{currentmarker}{\pgfqpoint{-0.041667in}{-0.041667in}}{\pgfqpoint{0.041667in}{0.041667in}}{%
\pgfpathmoveto{\pgfqpoint{0.000000in}{-0.041667in}}%
\pgfpathcurveto{\pgfqpoint{0.011050in}{-0.041667in}}{\pgfqpoint{0.021649in}{-0.037276in}}{\pgfqpoint{0.029463in}{-0.029463in}}%
\pgfpathcurveto{\pgfqpoint{0.037276in}{-0.021649in}}{\pgfqpoint{0.041667in}{-0.011050in}}{\pgfqpoint{0.041667in}{0.000000in}}%
\pgfpathcurveto{\pgfqpoint{0.041667in}{0.011050in}}{\pgfqpoint{0.037276in}{0.021649in}}{\pgfqpoint{0.029463in}{0.029463in}}%
\pgfpathcurveto{\pgfqpoint{0.021649in}{0.037276in}}{\pgfqpoint{0.011050in}{0.041667in}}{\pgfqpoint{0.000000in}{0.041667in}}%
\pgfpathcurveto{\pgfqpoint{-0.011050in}{0.041667in}}{\pgfqpoint{-0.021649in}{0.037276in}}{\pgfqpoint{-0.029463in}{0.029463in}}%
\pgfpathcurveto{\pgfqpoint{-0.037276in}{0.021649in}}{\pgfqpoint{-0.041667in}{0.011050in}}{\pgfqpoint{-0.041667in}{0.000000in}}%
\pgfpathcurveto{\pgfqpoint{-0.041667in}{-0.011050in}}{\pgfqpoint{-0.037276in}{-0.021649in}}{\pgfqpoint{-0.029463in}{-0.029463in}}%
\pgfpathcurveto{\pgfqpoint{-0.021649in}{-0.037276in}}{\pgfqpoint{-0.011050in}{-0.041667in}}{\pgfqpoint{0.000000in}{-0.041667in}}%
\pgfpathlineto{\pgfqpoint{0.000000in}{-0.041667in}}%
\pgfpathclose%
\pgfusepath{stroke,fill}%
}%
\begin{pgfscope}%
\pgfsys@transformshift{3.964805in}{3.152723in}%
\pgfsys@useobject{currentmarker}{}%
\end{pgfscope}%
\end{pgfscope}%
\begin{pgfscope}%
\definecolor{textcolor}{rgb}{0.000000,0.000000,0.000000}%
\pgfsetstrokecolor{textcolor}%
\pgfsetfillcolor{textcolor}%
\pgftext[x=4.214805in,y=3.104112in,left,base]{\color{textcolor}\rmfamily\fontsize{10.000000}{12.000000}\selectfont peaks}%
\end{pgfscope}%
\end{pgfpicture}%
\makeatother%
\endgroup%

        }
        \caption{Plot of $g(r)$}
        \label{step2_gr}
    \end{subfigure}
    \begin{subfigure}{0.5\textwidth}
        \resizebox{\textwidth}{!}{
            %% Creator: Matplotlib, PGF backend
%%
%% To include the figure in your LaTeX document, write
%%   \input{<filename>.pgf}
%%
%% Make sure the required packages are loaded in your preamble
%%   \usepackage{pgf}
%%
%% Also ensure that all the required font packages are loaded; for instance,
%% the lmodern package is sometimes necessary when using math font.
%%   \usepackage{lmodern}
%%
%% Figures using additional raster images can only be included by \input if
%% they are in the same directory as the main LaTeX file. For loading figures
%% from other directories you can use the `import` package
%%   \usepackage{import}
%%
%% and then include the figures with
%%   \import{<path to file>}{<filename>.pgf}
%%
%% Matplotlib used the following preamble
%%   \usepackage[utf8]{inputenc}
%%   \usepackage[T1]{fontenc}
%%   \usepackage{siunitx}
%%
\begingroup%
\makeatletter%
\begin{pgfpicture}%
\pgfpathrectangle{\pgfpointorigin}{\pgfqpoint{5.200000in}{4.000000in}}%
\pgfusepath{use as bounding box, clip}%
\begin{pgfscope}%
\pgfsetbuttcap%
\pgfsetmiterjoin%
\definecolor{currentfill}{rgb}{1.000000,1.000000,1.000000}%
\pgfsetfillcolor{currentfill}%
\pgfsetlinewidth{0.000000pt}%
\definecolor{currentstroke}{rgb}{1.000000,1.000000,1.000000}%
\pgfsetstrokecolor{currentstroke}%
\pgfsetdash{}{0pt}%
\pgfpathmoveto{\pgfqpoint{0.000000in}{0.000000in}}%
\pgfpathlineto{\pgfqpoint{5.200000in}{0.000000in}}%
\pgfpathlineto{\pgfqpoint{5.200000in}{4.000000in}}%
\pgfpathlineto{\pgfqpoint{0.000000in}{4.000000in}}%
\pgfpathlineto{\pgfqpoint{0.000000in}{0.000000in}}%
\pgfpathclose%
\pgfusepath{fill}%
\end{pgfscope}%
\begin{pgfscope}%
\pgfsetbuttcap%
\pgfsetmiterjoin%
\definecolor{currentfill}{rgb}{1.000000,1.000000,1.000000}%
\pgfsetfillcolor{currentfill}%
\pgfsetlinewidth{0.000000pt}%
\definecolor{currentstroke}{rgb}{0.000000,0.000000,0.000000}%
\pgfsetstrokecolor{currentstroke}%
\pgfsetstrokeopacity{0.000000}%
\pgfsetdash{}{0pt}%
\pgfpathmoveto{\pgfqpoint{0.650000in}{0.440000in}}%
\pgfpathlineto{\pgfqpoint{4.680000in}{0.440000in}}%
\pgfpathlineto{\pgfqpoint{4.680000in}{3.520000in}}%
\pgfpathlineto{\pgfqpoint{0.650000in}{3.520000in}}%
\pgfpathlineto{\pgfqpoint{0.650000in}{0.440000in}}%
\pgfpathclose%
\pgfusepath{fill}%
\end{pgfscope}%
\begin{pgfscope}%
\pgfsetbuttcap%
\pgfsetroundjoin%
\definecolor{currentfill}{rgb}{0.000000,0.000000,0.000000}%
\pgfsetfillcolor{currentfill}%
\pgfsetlinewidth{0.803000pt}%
\definecolor{currentstroke}{rgb}{0.000000,0.000000,0.000000}%
\pgfsetstrokecolor{currentstroke}%
\pgfsetdash{}{0pt}%
\pgfsys@defobject{currentmarker}{\pgfqpoint{0.000000in}{-0.048611in}}{\pgfqpoint{0.000000in}{0.000000in}}{%
\pgfpathmoveto{\pgfqpoint{0.000000in}{0.000000in}}%
\pgfpathlineto{\pgfqpoint{0.000000in}{-0.048611in}}%
\pgfusepath{stroke,fill}%
}%
\begin{pgfscope}%
\pgfsys@transformshift{0.833182in}{0.440000in}%
\pgfsys@useobject{currentmarker}{}%
\end{pgfscope}%
\end{pgfscope}%
\begin{pgfscope}%
\definecolor{textcolor}{rgb}{0.000000,0.000000,0.000000}%
\pgfsetstrokecolor{textcolor}%
\pgfsetfillcolor{textcolor}%
\pgftext[x=0.833182in,y=0.342778in,,top]{\color{textcolor}\rmfamily\fontsize{10.000000}{12.000000}\selectfont \(\displaystyle {0}\)}%
\end{pgfscope}%
\begin{pgfscope}%
\pgfsetbuttcap%
\pgfsetroundjoin%
\definecolor{currentfill}{rgb}{0.000000,0.000000,0.000000}%
\pgfsetfillcolor{currentfill}%
\pgfsetlinewidth{0.803000pt}%
\definecolor{currentstroke}{rgb}{0.000000,0.000000,0.000000}%
\pgfsetstrokecolor{currentstroke}%
\pgfsetdash{}{0pt}%
\pgfsys@defobject{currentmarker}{\pgfqpoint{0.000000in}{-0.048611in}}{\pgfqpoint{0.000000in}{0.000000in}}{%
\pgfpathmoveto{\pgfqpoint{0.000000in}{0.000000in}}%
\pgfpathlineto{\pgfqpoint{0.000000in}{-0.048611in}}%
\pgfusepath{stroke,fill}%
}%
\begin{pgfscope}%
\pgfsys@transformshift{1.604146in}{0.440000in}%
\pgfsys@useobject{currentmarker}{}%
\end{pgfscope}%
\end{pgfscope}%
\begin{pgfscope}%
\definecolor{textcolor}{rgb}{0.000000,0.000000,0.000000}%
\pgfsetstrokecolor{textcolor}%
\pgfsetfillcolor{textcolor}%
\pgftext[x=1.604146in,y=0.342778in,,top]{\color{textcolor}\rmfamily\fontsize{10.000000}{12.000000}\selectfont \(\displaystyle {2}\)}%
\end{pgfscope}%
\begin{pgfscope}%
\pgfsetbuttcap%
\pgfsetroundjoin%
\definecolor{currentfill}{rgb}{0.000000,0.000000,0.000000}%
\pgfsetfillcolor{currentfill}%
\pgfsetlinewidth{0.803000pt}%
\definecolor{currentstroke}{rgb}{0.000000,0.000000,0.000000}%
\pgfsetstrokecolor{currentstroke}%
\pgfsetdash{}{0pt}%
\pgfsys@defobject{currentmarker}{\pgfqpoint{0.000000in}{-0.048611in}}{\pgfqpoint{0.000000in}{0.000000in}}{%
\pgfpathmoveto{\pgfqpoint{0.000000in}{0.000000in}}%
\pgfpathlineto{\pgfqpoint{0.000000in}{-0.048611in}}%
\pgfusepath{stroke,fill}%
}%
\begin{pgfscope}%
\pgfsys@transformshift{2.375110in}{0.440000in}%
\pgfsys@useobject{currentmarker}{}%
\end{pgfscope}%
\end{pgfscope}%
\begin{pgfscope}%
\definecolor{textcolor}{rgb}{0.000000,0.000000,0.000000}%
\pgfsetstrokecolor{textcolor}%
\pgfsetfillcolor{textcolor}%
\pgftext[x=2.375110in,y=0.342778in,,top]{\color{textcolor}\rmfamily\fontsize{10.000000}{12.000000}\selectfont \(\displaystyle {4}\)}%
\end{pgfscope}%
\begin{pgfscope}%
\pgfsetbuttcap%
\pgfsetroundjoin%
\definecolor{currentfill}{rgb}{0.000000,0.000000,0.000000}%
\pgfsetfillcolor{currentfill}%
\pgfsetlinewidth{0.803000pt}%
\definecolor{currentstroke}{rgb}{0.000000,0.000000,0.000000}%
\pgfsetstrokecolor{currentstroke}%
\pgfsetdash{}{0pt}%
\pgfsys@defobject{currentmarker}{\pgfqpoint{0.000000in}{-0.048611in}}{\pgfqpoint{0.000000in}{0.000000in}}{%
\pgfpathmoveto{\pgfqpoint{0.000000in}{0.000000in}}%
\pgfpathlineto{\pgfqpoint{0.000000in}{-0.048611in}}%
\pgfusepath{stroke,fill}%
}%
\begin{pgfscope}%
\pgfsys@transformshift{3.146075in}{0.440000in}%
\pgfsys@useobject{currentmarker}{}%
\end{pgfscope}%
\end{pgfscope}%
\begin{pgfscope}%
\definecolor{textcolor}{rgb}{0.000000,0.000000,0.000000}%
\pgfsetstrokecolor{textcolor}%
\pgfsetfillcolor{textcolor}%
\pgftext[x=3.146075in,y=0.342778in,,top]{\color{textcolor}\rmfamily\fontsize{10.000000}{12.000000}\selectfont \(\displaystyle {6}\)}%
\end{pgfscope}%
\begin{pgfscope}%
\pgfsetbuttcap%
\pgfsetroundjoin%
\definecolor{currentfill}{rgb}{0.000000,0.000000,0.000000}%
\pgfsetfillcolor{currentfill}%
\pgfsetlinewidth{0.803000pt}%
\definecolor{currentstroke}{rgb}{0.000000,0.000000,0.000000}%
\pgfsetstrokecolor{currentstroke}%
\pgfsetdash{}{0pt}%
\pgfsys@defobject{currentmarker}{\pgfqpoint{0.000000in}{-0.048611in}}{\pgfqpoint{0.000000in}{0.000000in}}{%
\pgfpathmoveto{\pgfqpoint{0.000000in}{0.000000in}}%
\pgfpathlineto{\pgfqpoint{0.000000in}{-0.048611in}}%
\pgfusepath{stroke,fill}%
}%
\begin{pgfscope}%
\pgfsys@transformshift{3.917039in}{0.440000in}%
\pgfsys@useobject{currentmarker}{}%
\end{pgfscope}%
\end{pgfscope}%
\begin{pgfscope}%
\definecolor{textcolor}{rgb}{0.000000,0.000000,0.000000}%
\pgfsetstrokecolor{textcolor}%
\pgfsetfillcolor{textcolor}%
\pgftext[x=3.917039in,y=0.342778in,,top]{\color{textcolor}\rmfamily\fontsize{10.000000}{12.000000}\selectfont \(\displaystyle {8}\)}%
\end{pgfscope}%
\begin{pgfscope}%
\definecolor{textcolor}{rgb}{0.000000,0.000000,0.000000}%
\pgfsetstrokecolor{textcolor}%
\pgfsetfillcolor{textcolor}%
\pgftext[x=2.665000in,y=0.164567in,,top]{\color{textcolor}\rmfamily\fontsize{10.000000}{12.000000}\selectfont \(\displaystyle k [\si{\angstrom^{-1}}]\)}%
\end{pgfscope}%
\begin{pgfscope}%
\pgfsetbuttcap%
\pgfsetroundjoin%
\definecolor{currentfill}{rgb}{0.000000,0.000000,0.000000}%
\pgfsetfillcolor{currentfill}%
\pgfsetlinewidth{0.803000pt}%
\definecolor{currentstroke}{rgb}{0.000000,0.000000,0.000000}%
\pgfsetstrokecolor{currentstroke}%
\pgfsetdash{}{0pt}%
\pgfsys@defobject{currentmarker}{\pgfqpoint{-0.048611in}{0.000000in}}{\pgfqpoint{-0.000000in}{0.000000in}}{%
\pgfpathmoveto{\pgfqpoint{-0.000000in}{0.000000in}}%
\pgfpathlineto{\pgfqpoint{-0.048611in}{0.000000in}}%
\pgfusepath{stroke,fill}%
}%
\begin{pgfscope}%
\pgfsys@transformshift{0.650000in}{0.596610in}%
\pgfsys@useobject{currentmarker}{}%
\end{pgfscope}%
\end{pgfscope}%
\begin{pgfscope}%
\definecolor{textcolor}{rgb}{0.000000,0.000000,0.000000}%
\pgfsetstrokecolor{textcolor}%
\pgfsetfillcolor{textcolor}%
\pgftext[x=0.375308in, y=0.548782in, left, base]{\color{textcolor}\rmfamily\fontsize{10.000000}{12.000000}\selectfont \(\displaystyle {0.0}\)}%
\end{pgfscope}%
\begin{pgfscope}%
\pgfsetbuttcap%
\pgfsetroundjoin%
\definecolor{currentfill}{rgb}{0.000000,0.000000,0.000000}%
\pgfsetfillcolor{currentfill}%
\pgfsetlinewidth{0.803000pt}%
\definecolor{currentstroke}{rgb}{0.000000,0.000000,0.000000}%
\pgfsetstrokecolor{currentstroke}%
\pgfsetdash{}{0pt}%
\pgfsys@defobject{currentmarker}{\pgfqpoint{-0.048611in}{0.000000in}}{\pgfqpoint{-0.000000in}{0.000000in}}{%
\pgfpathmoveto{\pgfqpoint{-0.000000in}{0.000000in}}%
\pgfpathlineto{\pgfqpoint{-0.048611in}{0.000000in}}%
\pgfusepath{stroke,fill}%
}%
\begin{pgfscope}%
\pgfsys@transformshift{0.650000in}{1.118644in}%
\pgfsys@useobject{currentmarker}{}%
\end{pgfscope}%
\end{pgfscope}%
\begin{pgfscope}%
\definecolor{textcolor}{rgb}{0.000000,0.000000,0.000000}%
\pgfsetstrokecolor{textcolor}%
\pgfsetfillcolor{textcolor}%
\pgftext[x=0.375308in, y=1.070816in, left, base]{\color{textcolor}\rmfamily\fontsize{10.000000}{12.000000}\selectfont \(\displaystyle {0.5}\)}%
\end{pgfscope}%
\begin{pgfscope}%
\pgfsetbuttcap%
\pgfsetroundjoin%
\definecolor{currentfill}{rgb}{0.000000,0.000000,0.000000}%
\pgfsetfillcolor{currentfill}%
\pgfsetlinewidth{0.803000pt}%
\definecolor{currentstroke}{rgb}{0.000000,0.000000,0.000000}%
\pgfsetstrokecolor{currentstroke}%
\pgfsetdash{}{0pt}%
\pgfsys@defobject{currentmarker}{\pgfqpoint{-0.048611in}{0.000000in}}{\pgfqpoint{-0.000000in}{0.000000in}}{%
\pgfpathmoveto{\pgfqpoint{-0.000000in}{0.000000in}}%
\pgfpathlineto{\pgfqpoint{-0.048611in}{0.000000in}}%
\pgfusepath{stroke,fill}%
}%
\begin{pgfscope}%
\pgfsys@transformshift{0.650000in}{1.640678in}%
\pgfsys@useobject{currentmarker}{}%
\end{pgfscope}%
\end{pgfscope}%
\begin{pgfscope}%
\definecolor{textcolor}{rgb}{0.000000,0.000000,0.000000}%
\pgfsetstrokecolor{textcolor}%
\pgfsetfillcolor{textcolor}%
\pgftext[x=0.375308in, y=1.592850in, left, base]{\color{textcolor}\rmfamily\fontsize{10.000000}{12.000000}\selectfont \(\displaystyle {1.0}\)}%
\end{pgfscope}%
\begin{pgfscope}%
\pgfsetbuttcap%
\pgfsetroundjoin%
\definecolor{currentfill}{rgb}{0.000000,0.000000,0.000000}%
\pgfsetfillcolor{currentfill}%
\pgfsetlinewidth{0.803000pt}%
\definecolor{currentstroke}{rgb}{0.000000,0.000000,0.000000}%
\pgfsetstrokecolor{currentstroke}%
\pgfsetdash{}{0pt}%
\pgfsys@defobject{currentmarker}{\pgfqpoint{-0.048611in}{0.000000in}}{\pgfqpoint{-0.000000in}{0.000000in}}{%
\pgfpathmoveto{\pgfqpoint{-0.000000in}{0.000000in}}%
\pgfpathlineto{\pgfqpoint{-0.048611in}{0.000000in}}%
\pgfusepath{stroke,fill}%
}%
\begin{pgfscope}%
\pgfsys@transformshift{0.650000in}{2.162712in}%
\pgfsys@useobject{currentmarker}{}%
\end{pgfscope}%
\end{pgfscope}%
\begin{pgfscope}%
\definecolor{textcolor}{rgb}{0.000000,0.000000,0.000000}%
\pgfsetstrokecolor{textcolor}%
\pgfsetfillcolor{textcolor}%
\pgftext[x=0.375308in, y=2.114884in, left, base]{\color{textcolor}\rmfamily\fontsize{10.000000}{12.000000}\selectfont \(\displaystyle {1.5}\)}%
\end{pgfscope}%
\begin{pgfscope}%
\pgfsetbuttcap%
\pgfsetroundjoin%
\definecolor{currentfill}{rgb}{0.000000,0.000000,0.000000}%
\pgfsetfillcolor{currentfill}%
\pgfsetlinewidth{0.803000pt}%
\definecolor{currentstroke}{rgb}{0.000000,0.000000,0.000000}%
\pgfsetstrokecolor{currentstroke}%
\pgfsetdash{}{0pt}%
\pgfsys@defobject{currentmarker}{\pgfqpoint{-0.048611in}{0.000000in}}{\pgfqpoint{-0.000000in}{0.000000in}}{%
\pgfpathmoveto{\pgfqpoint{-0.000000in}{0.000000in}}%
\pgfpathlineto{\pgfqpoint{-0.048611in}{0.000000in}}%
\pgfusepath{stroke,fill}%
}%
\begin{pgfscope}%
\pgfsys@transformshift{0.650000in}{2.684746in}%
\pgfsys@useobject{currentmarker}{}%
\end{pgfscope}%
\end{pgfscope}%
\begin{pgfscope}%
\definecolor{textcolor}{rgb}{0.000000,0.000000,0.000000}%
\pgfsetstrokecolor{textcolor}%
\pgfsetfillcolor{textcolor}%
\pgftext[x=0.375308in, y=2.636918in, left, base]{\color{textcolor}\rmfamily\fontsize{10.000000}{12.000000}\selectfont \(\displaystyle {2.0}\)}%
\end{pgfscope}%
\begin{pgfscope}%
\pgfsetbuttcap%
\pgfsetroundjoin%
\definecolor{currentfill}{rgb}{0.000000,0.000000,0.000000}%
\pgfsetfillcolor{currentfill}%
\pgfsetlinewidth{0.803000pt}%
\definecolor{currentstroke}{rgb}{0.000000,0.000000,0.000000}%
\pgfsetstrokecolor{currentstroke}%
\pgfsetdash{}{0pt}%
\pgfsys@defobject{currentmarker}{\pgfqpoint{-0.048611in}{0.000000in}}{\pgfqpoint{-0.000000in}{0.000000in}}{%
\pgfpathmoveto{\pgfqpoint{-0.000000in}{0.000000in}}%
\pgfpathlineto{\pgfqpoint{-0.048611in}{0.000000in}}%
\pgfusepath{stroke,fill}%
}%
\begin{pgfscope}%
\pgfsys@transformshift{0.650000in}{3.206780in}%
\pgfsys@useobject{currentmarker}{}%
\end{pgfscope}%
\end{pgfscope}%
\begin{pgfscope}%
\definecolor{textcolor}{rgb}{0.000000,0.000000,0.000000}%
\pgfsetstrokecolor{textcolor}%
\pgfsetfillcolor{textcolor}%
\pgftext[x=0.375308in, y=3.158952in, left, base]{\color{textcolor}\rmfamily\fontsize{10.000000}{12.000000}\selectfont \(\displaystyle {2.5}\)}%
\end{pgfscope}%
\begin{pgfscope}%
\definecolor{textcolor}{rgb}{0.000000,0.000000,0.000000}%
\pgfsetstrokecolor{textcolor}%
\pgfsetfillcolor{textcolor}%
\pgftext[x=0.319753in,y=1.980000in,,bottom,rotate=90.000000]{\color{textcolor}\rmfamily\fontsize{10.000000}{12.000000}\selectfont \(\displaystyle S(k)\)}%
\end{pgfscope}%
\begin{pgfscope}%
\pgfpathrectangle{\pgfqpoint{0.650000in}{0.440000in}}{\pgfqpoint{4.030000in}{3.080000in}}%
\pgfusepath{clip}%
\pgfsetbuttcap%
\pgfsetroundjoin%
\pgfsetlinewidth{1.505625pt}%
\definecolor{currentstroke}{rgb}{0.000000,0.000000,0.000000}%
\pgfsetstrokecolor{currentstroke}%
\pgfsetdash{{5.550000pt}{2.400000pt}}{0.000000pt}%
\pgfpathmoveto{\pgfqpoint{1.595349in}{0.440000in}}%
\pgfpathlineto{\pgfqpoint{1.595349in}{3.169882in}}%
\pgfusepath{stroke}%
\end{pgfscope}%
\begin{pgfscope}%
\pgfpathrectangle{\pgfqpoint{0.650000in}{0.440000in}}{\pgfqpoint{4.030000in}{3.080000in}}%
\pgfusepath{clip}%
\pgfsetbuttcap%
\pgfsetroundjoin%
\pgfsetlinewidth{1.505625pt}%
\definecolor{currentstroke}{rgb}{0.000000,0.000000,0.000000}%
\pgfsetstrokecolor{currentstroke}%
\pgfsetdash{{5.550000pt}{2.400000pt}}{0.000000pt}%
\pgfpathmoveto{\pgfqpoint{2.243760in}{0.440000in}}%
\pgfpathlineto{\pgfqpoint{2.243760in}{1.906687in}}%
\pgfusepath{stroke}%
\end{pgfscope}%
\begin{pgfscope}%
\pgfpathrectangle{\pgfqpoint{0.650000in}{0.440000in}}{\pgfqpoint{4.030000in}{3.080000in}}%
\pgfusepath{clip}%
\pgfsetbuttcap%
\pgfsetroundjoin%
\pgfsetlinewidth{1.505625pt}%
\definecolor{currentstroke}{rgb}{0.000000,0.000000,0.000000}%
\pgfsetstrokecolor{currentstroke}%
\pgfsetdash{{5.550000pt}{2.400000pt}}{0.000000pt}%
\pgfpathmoveto{\pgfqpoint{2.914922in}{0.440000in}}%
\pgfpathlineto{\pgfqpoint{2.914922in}{1.746400in}}%
\pgfusepath{stroke}%
\end{pgfscope}%
\begin{pgfscope}%
\pgfpathrectangle{\pgfqpoint{0.650000in}{0.440000in}}{\pgfqpoint{4.030000in}{3.080000in}}%
\pgfusepath{clip}%
\pgfsetbuttcap%
\pgfsetroundjoin%
\pgfsetlinewidth{1.505625pt}%
\definecolor{currentstroke}{rgb}{0.000000,0.000000,0.000000}%
\pgfsetstrokecolor{currentstroke}%
\pgfsetdash{{5.550000pt}{2.400000pt}}{0.000000pt}%
\pgfpathmoveto{\pgfqpoint{3.608835in}{0.440000in}}%
\pgfpathlineto{\pgfqpoint{3.608835in}{1.684647in}}%
\pgfusepath{stroke}%
\end{pgfscope}%
\begin{pgfscope}%
\pgfpathrectangle{\pgfqpoint{0.650000in}{0.440000in}}{\pgfqpoint{4.030000in}{3.080000in}}%
\pgfusepath{clip}%
\pgfsetrectcap%
\pgfsetroundjoin%
\pgfsetlinewidth{1.505625pt}%
\definecolor{currentstroke}{rgb}{1.000000,0.647059,0.000000}%
\pgfsetstrokecolor{currentstroke}%
\pgfsetdash{}{0pt}%
\pgfpathmoveto{\pgfqpoint{0.833182in}{1.581220in}}%
\pgfpathlineto{\pgfqpoint{0.844557in}{1.549136in}}%
\pgfpathlineto{\pgfqpoint{0.855933in}{1.457319in}}%
\pgfpathlineto{\pgfqpoint{0.867309in}{1.318322in}}%
\pgfpathlineto{\pgfqpoint{0.901436in}{0.814286in}}%
\pgfpathlineto{\pgfqpoint{0.912811in}{0.683164in}}%
\pgfpathlineto{\pgfqpoint{0.924187in}{0.592953in}}%
\pgfpathlineto{\pgfqpoint{0.935562in}{0.546789in}}%
\pgfpathlineto{\pgfqpoint{0.946938in}{0.540594in}}%
\pgfpathlineto{\pgfqpoint{0.958314in}{0.564579in}}%
\pgfpathlineto{\pgfqpoint{0.992441in}{0.686519in}}%
\pgfpathlineto{\pgfqpoint{1.003816in}{0.707209in}}%
\pgfpathlineto{\pgfqpoint{1.015192in}{0.709564in}}%
\pgfpathlineto{\pgfqpoint{1.026568in}{0.695686in}}%
\pgfpathlineto{\pgfqpoint{1.037943in}{0.671284in}}%
\pgfpathlineto{\pgfqpoint{1.049319in}{0.643902in}}%
\pgfpathlineto{\pgfqpoint{1.060694in}{0.620937in}}%
\pgfpathlineto{\pgfqpoint{1.072070in}{0.607913in}}%
\pgfpathlineto{\pgfqpoint{1.083446in}{0.607355in}}%
\pgfpathlineto{\pgfqpoint{1.094821in}{0.618472in}}%
\pgfpathlineto{\pgfqpoint{1.106197in}{0.637656in}}%
\pgfpathlineto{\pgfqpoint{1.117573in}{0.659608in}}%
\pgfpathlineto{\pgfqpoint{1.128948in}{0.678787in}}%
\pgfpathlineto{\pgfqpoint{1.140324in}{0.690814in}}%
\pgfpathlineto{\pgfqpoint{1.151699in}{0.693498in}}%
\pgfpathlineto{\pgfqpoint{1.163075in}{0.687257in}}%
\pgfpathlineto{\pgfqpoint{1.174451in}{0.674869in}}%
\pgfpathlineto{\pgfqpoint{1.185826in}{0.660642in}}%
\pgfpathlineto{\pgfqpoint{1.197202in}{0.649248in}}%
\pgfpathlineto{\pgfqpoint{1.208578in}{0.644510in}}%
\pgfpathlineto{\pgfqpoint{1.219953in}{0.648456in}}%
\pgfpathlineto{\pgfqpoint{1.231329in}{0.660866in}}%
\pgfpathlineto{\pgfqpoint{1.242704in}{0.679410in}}%
\pgfpathlineto{\pgfqpoint{1.265456in}{0.719465in}}%
\pgfpathlineto{\pgfqpoint{1.276831in}{0.733418in}}%
\pgfpathlineto{\pgfqpoint{1.288207in}{0.740443in}}%
\pgfpathlineto{\pgfqpoint{1.299583in}{0.740962in}}%
\pgfpathlineto{\pgfqpoint{1.322334in}{0.734007in}}%
\pgfpathlineto{\pgfqpoint{1.333709in}{0.734992in}}%
\pgfpathlineto{\pgfqpoint{1.345085in}{0.744248in}}%
\pgfpathlineto{\pgfqpoint{1.356461in}{0.763961in}}%
\pgfpathlineto{\pgfqpoint{1.367836in}{0.794146in}}%
\pgfpathlineto{\pgfqpoint{1.379212in}{0.832702in}}%
\pgfpathlineto{\pgfqpoint{1.424714in}{1.002699in}}%
\pgfpathlineto{\pgfqpoint{1.436090in}{1.042652in}}%
\pgfpathlineto{\pgfqpoint{1.447466in}{1.088321in}}%
\pgfpathlineto{\pgfqpoint{1.458841in}{1.147892in}}%
\pgfpathlineto{\pgfqpoint{1.470217in}{1.230643in}}%
\pgfpathlineto{\pgfqpoint{1.481593in}{1.345047in}}%
\pgfpathlineto{\pgfqpoint{1.492968in}{1.496742in}}%
\pgfpathlineto{\pgfqpoint{1.504344in}{1.686778in}}%
\pgfpathlineto{\pgfqpoint{1.515720in}{1.910519in}}%
\pgfpathlineto{\pgfqpoint{1.549846in}{2.655732in}}%
\pgfpathlineto{\pgfqpoint{1.561222in}{2.868714in}}%
\pgfpathlineto{\pgfqpoint{1.572598in}{3.033334in}}%
\pgfpathlineto{\pgfqpoint{1.583973in}{3.136157in}}%
\pgfpathlineto{\pgfqpoint{1.595349in}{3.169882in}}%
\pgfpathlineto{\pgfqpoint{1.606725in}{3.134254in}}%
\pgfpathlineto{\pgfqpoint{1.618100in}{3.035908in}}%
\pgfpathlineto{\pgfqpoint{1.629476in}{2.887208in}}%
\pgfpathlineto{\pgfqpoint{1.652227in}{2.504798in}}%
\pgfpathlineto{\pgfqpoint{1.674978in}{2.119607in}}%
\pgfpathlineto{\pgfqpoint{1.686354in}{1.957089in}}%
\pgfpathlineto{\pgfqpoint{1.697730in}{1.822445in}}%
\pgfpathlineto{\pgfqpoint{1.709105in}{1.715869in}}%
\pgfpathlineto{\pgfqpoint{1.720481in}{1.634045in}}%
\pgfpathlineto{\pgfqpoint{1.731856in}{1.571505in}}%
\pgfpathlineto{\pgfqpoint{1.743232in}{1.522108in}}%
\pgfpathlineto{\pgfqpoint{1.765983in}{1.442165in}}%
\pgfpathlineto{\pgfqpoint{1.788735in}{1.370211in}}%
\pgfpathlineto{\pgfqpoint{1.800110in}{1.337280in}}%
\pgfpathlineto{\pgfqpoint{1.811486in}{1.308502in}}%
\pgfpathlineto{\pgfqpoint{1.822861in}{1.285555in}}%
\pgfpathlineto{\pgfqpoint{1.834237in}{1.269464in}}%
\pgfpathlineto{\pgfqpoint{1.845613in}{1.260285in}}%
\pgfpathlineto{\pgfqpoint{1.856988in}{1.257098in}}%
\pgfpathlineto{\pgfqpoint{1.868364in}{1.258258in}}%
\pgfpathlineto{\pgfqpoint{1.925242in}{1.275168in}}%
\pgfpathlineto{\pgfqpoint{1.936618in}{1.278401in}}%
\pgfpathlineto{\pgfqpoint{1.947993in}{1.283811in}}%
\pgfpathlineto{\pgfqpoint{1.959369in}{1.292772in}}%
\pgfpathlineto{\pgfqpoint{1.970745in}{1.306258in}}%
\pgfpathlineto{\pgfqpoint{1.982120in}{1.324631in}}%
\pgfpathlineto{\pgfqpoint{1.993496in}{1.347585in}}%
\pgfpathlineto{\pgfqpoint{2.016247in}{1.403459in}}%
\pgfpathlineto{\pgfqpoint{2.095877in}{1.618296in}}%
\pgfpathlineto{\pgfqpoint{2.130003in}{1.720407in}}%
\pgfpathlineto{\pgfqpoint{2.152755in}{1.787708in}}%
\pgfpathlineto{\pgfqpoint{2.164130in}{1.817684in}}%
\pgfpathlineto{\pgfqpoint{2.175506in}{1.843718in}}%
\pgfpathlineto{\pgfqpoint{2.186882in}{1.865105in}}%
\pgfpathlineto{\pgfqpoint{2.198257in}{1.881597in}}%
\pgfpathlineto{\pgfqpoint{2.209633in}{1.893390in}}%
\pgfpathlineto{\pgfqpoint{2.221008in}{1.901028in}}%
\pgfpathlineto{\pgfqpoint{2.232384in}{1.905227in}}%
\pgfpathlineto{\pgfqpoint{2.243760in}{1.906687in}}%
\pgfpathlineto{\pgfqpoint{2.255135in}{1.905931in}}%
\pgfpathlineto{\pgfqpoint{2.266511in}{1.903194in}}%
\pgfpathlineto{\pgfqpoint{2.277887in}{1.898410in}}%
\pgfpathlineto{\pgfqpoint{2.289262in}{1.891264in}}%
\pgfpathlineto{\pgfqpoint{2.300638in}{1.881326in}}%
\pgfpathlineto{\pgfqpoint{2.312014in}{1.868203in}}%
\pgfpathlineto{\pgfqpoint{2.323389in}{1.851684in}}%
\pgfpathlineto{\pgfqpoint{2.334765in}{1.831851in}}%
\pgfpathlineto{\pgfqpoint{2.357516in}{1.784150in}}%
\pgfpathlineto{\pgfqpoint{2.403019in}{1.680067in}}%
\pgfpathlineto{\pgfqpoint{2.425770in}{1.634979in}}%
\pgfpathlineto{\pgfqpoint{2.448521in}{1.596350in}}%
\pgfpathlineto{\pgfqpoint{2.471272in}{1.562562in}}%
\pgfpathlineto{\pgfqpoint{2.494024in}{1.532785in}}%
\pgfpathlineto{\pgfqpoint{2.516775in}{1.508236in}}%
\pgfpathlineto{\pgfqpoint{2.528150in}{1.498643in}}%
\pgfpathlineto{\pgfqpoint{2.539526in}{1.491169in}}%
\pgfpathlineto{\pgfqpoint{2.550902in}{1.485925in}}%
\pgfpathlineto{\pgfqpoint{2.562277in}{1.482886in}}%
\pgfpathlineto{\pgfqpoint{2.573653in}{1.481897in}}%
\pgfpathlineto{\pgfqpoint{2.585029in}{1.482717in}}%
\pgfpathlineto{\pgfqpoint{2.596404in}{1.485070in}}%
\pgfpathlineto{\pgfqpoint{2.619155in}{1.493430in}}%
\pgfpathlineto{\pgfqpoint{2.641907in}{1.505841in}}%
\pgfpathlineto{\pgfqpoint{2.664658in}{1.522337in}}%
\pgfpathlineto{\pgfqpoint{2.687409in}{1.543275in}}%
\pgfpathlineto{\pgfqpoint{2.710161in}{1.568298in}}%
\pgfpathlineto{\pgfqpoint{2.744287in}{1.610358in}}%
\pgfpathlineto{\pgfqpoint{2.778414in}{1.652531in}}%
\pgfpathlineto{\pgfqpoint{2.801166in}{1.678442in}}%
\pgfpathlineto{\pgfqpoint{2.823917in}{1.701403in}}%
\pgfpathlineto{\pgfqpoint{2.846668in}{1.720377in}}%
\pgfpathlineto{\pgfqpoint{2.869419in}{1.734426in}}%
\pgfpathlineto{\pgfqpoint{2.880795in}{1.739427in}}%
\pgfpathlineto{\pgfqpoint{2.892171in}{1.743063in}}%
\pgfpathlineto{\pgfqpoint{2.903546in}{1.745369in}}%
\pgfpathlineto{\pgfqpoint{2.914922in}{1.746400in}}%
\pgfpathlineto{\pgfqpoint{2.926297in}{1.746216in}}%
\pgfpathlineto{\pgfqpoint{2.937673in}{1.744872in}}%
\pgfpathlineto{\pgfqpoint{2.960424in}{1.738840in}}%
\pgfpathlineto{\pgfqpoint{2.983176in}{1.728518in}}%
\pgfpathlineto{\pgfqpoint{3.005927in}{1.714364in}}%
\pgfpathlineto{\pgfqpoint{3.040054in}{1.688577in}}%
\pgfpathlineto{\pgfqpoint{3.085556in}{1.654087in}}%
\pgfpathlineto{\pgfqpoint{3.131059in}{1.623739in}}%
\pgfpathlineto{\pgfqpoint{3.165186in}{1.602654in}}%
\pgfpathlineto{\pgfqpoint{3.187937in}{1.590854in}}%
\pgfpathlineto{\pgfqpoint{3.210688in}{1.582673in}}%
\pgfpathlineto{\pgfqpoint{3.233439in}{1.578832in}}%
\pgfpathlineto{\pgfqpoint{3.256191in}{1.578600in}}%
\pgfpathlineto{\pgfqpoint{3.290318in}{1.581463in}}%
\pgfpathlineto{\pgfqpoint{3.324444in}{1.585718in}}%
\pgfpathlineto{\pgfqpoint{3.347196in}{1.590537in}}%
\pgfpathlineto{\pgfqpoint{3.369947in}{1.598317in}}%
\pgfpathlineto{\pgfqpoint{3.392698in}{1.609089in}}%
\pgfpathlineto{\pgfqpoint{3.449576in}{1.639273in}}%
\pgfpathlineto{\pgfqpoint{3.483703in}{1.653527in}}%
\pgfpathlineto{\pgfqpoint{3.529206in}{1.669446in}}%
\pgfpathlineto{\pgfqpoint{3.563333in}{1.679268in}}%
\pgfpathlineto{\pgfqpoint{3.586084in}{1.683295in}}%
\pgfpathlineto{\pgfqpoint{3.608835in}{1.684647in}}%
\pgfpathlineto{\pgfqpoint{3.631586in}{1.683577in}}%
\pgfpathlineto{\pgfqpoint{3.665713in}{1.678881in}}%
\pgfpathlineto{\pgfqpoint{3.699840in}{1.671607in}}%
\pgfpathlineto{\pgfqpoint{3.733967in}{1.661766in}}%
\pgfpathlineto{\pgfqpoint{3.813596in}{1.636590in}}%
\pgfpathlineto{\pgfqpoint{3.859099in}{1.626134in}}%
\pgfpathlineto{\pgfqpoint{3.904601in}{1.617406in}}%
\pgfpathlineto{\pgfqpoint{3.938728in}{1.613064in}}%
\pgfpathlineto{\pgfqpoint{3.961480in}{1.611995in}}%
\pgfpathlineto{\pgfqpoint{3.995607in}{1.613254in}}%
\pgfpathlineto{\pgfqpoint{4.029733in}{1.617204in}}%
\pgfpathlineto{\pgfqpoint{4.075236in}{1.624988in}}%
\pgfpathlineto{\pgfqpoint{4.234495in}{1.654650in}}%
\pgfpathlineto{\pgfqpoint{4.234495in}{1.654650in}}%
\pgfusepath{stroke}%
\end{pgfscope}%
\begin{pgfscope}%
\pgfpathrectangle{\pgfqpoint{0.650000in}{0.440000in}}{\pgfqpoint{4.030000in}{3.080000in}}%
\pgfusepath{clip}%
\pgfsetbuttcap%
\pgfsetbeveljoin%
\definecolor{currentfill}{rgb}{0.501961,0.000000,0.501961}%
\pgfsetfillcolor{currentfill}%
\pgfsetlinewidth{1.003750pt}%
\definecolor{currentstroke}{rgb}{0.501961,0.000000,0.501961}%
\pgfsetstrokecolor{currentstroke}%
\pgfsetdash{}{0pt}%
\pgfsys@defobject{currentmarker}{\pgfqpoint{-0.026418in}{-0.022473in}}{\pgfqpoint{0.026418in}{0.027778in}}{%
\pgfpathmoveto{\pgfqpoint{0.000000in}{0.027778in}}%
\pgfpathlineto{\pgfqpoint{-0.006236in}{0.008584in}}%
\pgfpathlineto{\pgfqpoint{-0.026418in}{0.008584in}}%
\pgfpathlineto{\pgfqpoint{-0.010091in}{-0.003279in}}%
\pgfpathlineto{\pgfqpoint{-0.016327in}{-0.022473in}}%
\pgfpathlineto{\pgfqpoint{-0.000000in}{-0.010610in}}%
\pgfpathlineto{\pgfqpoint{0.016327in}{-0.022473in}}%
\pgfpathlineto{\pgfqpoint{0.010091in}{-0.003279in}}%
\pgfpathlineto{\pgfqpoint{0.026418in}{0.008584in}}%
\pgfpathlineto{\pgfqpoint{0.006236in}{0.008584in}}%
\pgfpathlineto{\pgfqpoint{0.000000in}{0.027778in}}%
\pgfpathclose%
\pgfusepath{stroke,fill}%
}%
\begin{pgfscope}%
\pgfsys@transformshift{0.833182in}{0.644434in}%
\pgfsys@useobject{currentmarker}{}%
\end{pgfscope}%
\begin{pgfscope}%
\pgfsys@transformshift{0.939890in}{0.647938in}%
\pgfsys@useobject{currentmarker}{}%
\end{pgfscope}%
\begin{pgfscope}%
\pgfsys@transformshift{1.011028in}{0.651332in}%
\pgfsys@useobject{currentmarker}{}%
\end{pgfscope}%
\begin{pgfscope}%
\pgfsys@transformshift{1.082167in}{0.650273in}%
\pgfsys@useobject{currentmarker}{}%
\end{pgfscope}%
\begin{pgfscope}%
\pgfsys@transformshift{1.153305in}{0.659139in}%
\pgfsys@useobject{currentmarker}{}%
\end{pgfscope}%
\begin{pgfscope}%
\pgfsys@transformshift{1.224444in}{0.680006in}%
\pgfsys@useobject{currentmarker}{}%
\end{pgfscope}%
\begin{pgfscope}%
\pgfsys@transformshift{1.295583in}{0.714496in}%
\pgfsys@useobject{currentmarker}{}%
\end{pgfscope}%
\begin{pgfscope}%
\pgfsys@transformshift{1.366721in}{0.793345in}%
\pgfsys@useobject{currentmarker}{}%
\end{pgfscope}%
\begin{pgfscope}%
\pgfsys@transformshift{1.437860in}{0.998239in}%
\pgfsys@useobject{currentmarker}{}%
\end{pgfscope}%
\begin{pgfscope}%
\pgfsys@transformshift{1.508998in}{1.613358in}%
\pgfsys@useobject{currentmarker}{}%
\end{pgfscope}%
\begin{pgfscope}%
\pgfsys@transformshift{1.580137in}{2.838380in}%
\pgfsys@useobject{currentmarker}{}%
\end{pgfscope}%
\begin{pgfscope}%
\pgfsys@transformshift{1.651275in}{2.779850in}%
\pgfsys@useobject{currentmarker}{}%
\end{pgfscope}%
\begin{pgfscope}%
\pgfsys@transformshift{1.722414in}{1.827208in}%
\pgfsys@useobject{currentmarker}{}%
\end{pgfscope}%
\begin{pgfscope}%
\pgfsys@transformshift{1.793553in}{1.410362in}%
\pgfsys@useobject{currentmarker}{}%
\end{pgfscope}%
\begin{pgfscope}%
\pgfsys@transformshift{1.864691in}{1.277379in}%
\pgfsys@useobject{currentmarker}{}%
\end{pgfscope}%
\begin{pgfscope}%
\pgfsys@transformshift{1.935830in}{1.265366in}%
\pgfsys@useobject{currentmarker}{}%
\end{pgfscope}%
\begin{pgfscope}%
\pgfsys@transformshift{2.006968in}{1.339677in}%
\pgfsys@useobject{currentmarker}{}%
\end{pgfscope}%
\begin{pgfscope}%
\pgfsys@transformshift{2.078107in}{1.502720in}%
\pgfsys@useobject{currentmarker}{}%
\end{pgfscope}%
\begin{pgfscope}%
\pgfsys@transformshift{2.149245in}{1.708875in}%
\pgfsys@useobject{currentmarker}{}%
\end{pgfscope}%
\begin{pgfscope}%
\pgfsys@transformshift{2.220384in}{1.867178in}%
\pgfsys@useobject{currentmarker}{}%
\end{pgfscope}%
\begin{pgfscope}%
\pgfsys@transformshift{2.291523in}{1.901828in}%
\pgfsys@useobject{currentmarker}{}%
\end{pgfscope}%
\begin{pgfscope}%
\pgfsys@transformshift{2.362661in}{1.829305in}%
\pgfsys@useobject{currentmarker}{}%
\end{pgfscope}%
\begin{pgfscope}%
\pgfsys@transformshift{2.433800in}{1.685352in}%
\pgfsys@useobject{currentmarker}{}%
\end{pgfscope}%
\begin{pgfscope}%
\pgfsys@transformshift{2.504938in}{1.559563in}%
\pgfsys@useobject{currentmarker}{}%
\end{pgfscope}%
\begin{pgfscope}%
\pgfsys@transformshift{2.576077in}{1.494800in}%
\pgfsys@useobject{currentmarker}{}%
\end{pgfscope}%
\begin{pgfscope}%
\pgfsys@transformshift{2.647215in}{1.491411in}%
\pgfsys@useobject{currentmarker}{}%
\end{pgfscope}%
\begin{pgfscope}%
\pgfsys@transformshift{2.718354in}{1.539804in}%
\pgfsys@useobject{currentmarker}{}%
\end{pgfscope}%
\begin{pgfscope}%
\pgfsys@transformshift{2.789492in}{1.620639in}%
\pgfsys@useobject{currentmarker}{}%
\end{pgfscope}%
\begin{pgfscope}%
\pgfsys@transformshift{2.860631in}{1.696947in}%
\pgfsys@useobject{currentmarker}{}%
\end{pgfscope}%
\begin{pgfscope}%
\pgfsys@transformshift{2.931770in}{1.740659in}%
\pgfsys@useobject{currentmarker}{}%
\end{pgfscope}%
\begin{pgfscope}%
\pgfsys@transformshift{3.002908in}{1.737154in}%
\pgfsys@useobject{currentmarker}{}%
\end{pgfscope}%
\begin{pgfscope}%
\pgfsys@transformshift{3.074047in}{1.695551in}%
\pgfsys@useobject{currentmarker}{}%
\end{pgfscope}%
\begin{pgfscope}%
\pgfsys@transformshift{3.145185in}{1.644960in}%
\pgfsys@useobject{currentmarker}{}%
\end{pgfscope}%
\begin{pgfscope}%
\pgfsys@transformshift{3.216324in}{1.598738in}%
\pgfsys@useobject{currentmarker}{}%
\end{pgfscope}%
\begin{pgfscope}%
\pgfsys@transformshift{3.287462in}{1.580566in}%
\pgfsys@useobject{currentmarker}{}%
\end{pgfscope}%
\begin{pgfscope}%
\pgfsys@transformshift{3.358601in}{1.583894in}%
\pgfsys@useobject{currentmarker}{}%
\end{pgfscope}%
\begin{pgfscope}%
\pgfsys@transformshift{3.429740in}{1.604575in}%
\pgfsys@useobject{currentmarker}{}%
\end{pgfscope}%
\begin{pgfscope}%
\pgfsys@transformshift{3.500878in}{1.637470in}%
\pgfsys@useobject{currentmarker}{}%
\end{pgfscope}%
\begin{pgfscope}%
\pgfsys@transformshift{3.572017in}{1.665496in}%
\pgfsys@useobject{currentmarker}{}%
\end{pgfscope}%
\begin{pgfscope}%
\pgfsys@transformshift{3.643155in}{1.681134in}%
\pgfsys@useobject{currentmarker}{}%
\end{pgfscope}%
\begin{pgfscope}%
\pgfsys@transformshift{3.714294in}{1.679357in}%
\pgfsys@useobject{currentmarker}{}%
\end{pgfscope}%
\begin{pgfscope}%
\pgfsys@transformshift{3.785432in}{1.662984in}%
\pgfsys@useobject{currentmarker}{}%
\end{pgfscope}%
\begin{pgfscope}%
\pgfsys@transformshift{3.856571in}{1.646704in}%
\pgfsys@useobject{currentmarker}{}%
\end{pgfscope}%
\begin{pgfscope}%
\pgfsys@transformshift{3.927710in}{1.629205in}%
\pgfsys@useobject{currentmarker}{}%
\end{pgfscope}%
\begin{pgfscope}%
\pgfsys@transformshift{3.998848in}{1.614013in}%
\pgfsys@useobject{currentmarker}{}%
\end{pgfscope}%
\begin{pgfscope}%
\pgfsys@transformshift{4.069987in}{1.614187in}%
\pgfsys@useobject{currentmarker}{}%
\end{pgfscope}%
\begin{pgfscope}%
\pgfsys@transformshift{4.141125in}{1.624185in}%
\pgfsys@useobject{currentmarker}{}%
\end{pgfscope}%
\begin{pgfscope}%
\pgfsys@transformshift{4.212264in}{1.643481in}%
\pgfsys@useobject{currentmarker}{}%
\end{pgfscope}%
\begin{pgfscope}%
\pgfsys@transformshift{4.283402in}{1.657499in}%
\pgfsys@useobject{currentmarker}{}%
\end{pgfscope}%
\begin{pgfscope}%
\pgfsys@transformshift{4.354541in}{1.636405in}%
\pgfsys@useobject{currentmarker}{}%
\end{pgfscope}%
\begin{pgfscope}%
\pgfsys@transformshift{4.425680in}{1.678596in}%
\pgfsys@useobject{currentmarker}{}%
\end{pgfscope}%
\begin{pgfscope}%
\pgfsys@transformshift{4.496818in}{1.572795in}%
\pgfsys@useobject{currentmarker}{}%
\end{pgfscope}%
\end{pgfscope}%
\begin{pgfscope}%
\pgfpathrectangle{\pgfqpoint{0.650000in}{0.440000in}}{\pgfqpoint{4.030000in}{3.080000in}}%
\pgfusepath{clip}%
\pgfsetbuttcap%
\pgfsetroundjoin%
\definecolor{currentfill}{rgb}{0.000000,0.501961,0.000000}%
\pgfsetfillcolor{currentfill}%
\pgfsetlinewidth{1.003750pt}%
\definecolor{currentstroke}{rgb}{0.000000,0.000000,0.000000}%
\pgfsetstrokecolor{currentstroke}%
\pgfsetdash{}{0pt}%
\pgfsys@defobject{currentmarker}{\pgfqpoint{-0.041667in}{-0.041667in}}{\pgfqpoint{0.041667in}{0.041667in}}{%
\pgfpathmoveto{\pgfqpoint{0.000000in}{-0.041667in}}%
\pgfpathcurveto{\pgfqpoint{0.011050in}{-0.041667in}}{\pgfqpoint{0.021649in}{-0.037276in}}{\pgfqpoint{0.029463in}{-0.029463in}}%
\pgfpathcurveto{\pgfqpoint{0.037276in}{-0.021649in}}{\pgfqpoint{0.041667in}{-0.011050in}}{\pgfqpoint{0.041667in}{0.000000in}}%
\pgfpathcurveto{\pgfqpoint{0.041667in}{0.011050in}}{\pgfqpoint{0.037276in}{0.021649in}}{\pgfqpoint{0.029463in}{0.029463in}}%
\pgfpathcurveto{\pgfqpoint{0.021649in}{0.037276in}}{\pgfqpoint{0.011050in}{0.041667in}}{\pgfqpoint{0.000000in}{0.041667in}}%
\pgfpathcurveto{\pgfqpoint{-0.011050in}{0.041667in}}{\pgfqpoint{-0.021649in}{0.037276in}}{\pgfqpoint{-0.029463in}{0.029463in}}%
\pgfpathcurveto{\pgfqpoint{-0.037276in}{0.021649in}}{\pgfqpoint{-0.041667in}{0.011050in}}{\pgfqpoint{-0.041667in}{0.000000in}}%
\pgfpathcurveto{\pgfqpoint{-0.041667in}{-0.011050in}}{\pgfqpoint{-0.037276in}{-0.021649in}}{\pgfqpoint{-0.029463in}{-0.029463in}}%
\pgfpathcurveto{\pgfqpoint{-0.021649in}{-0.037276in}}{\pgfqpoint{-0.011050in}{-0.041667in}}{\pgfqpoint{0.000000in}{-0.041667in}}%
\pgfpathlineto{\pgfqpoint{0.000000in}{-0.041667in}}%
\pgfpathclose%
\pgfusepath{stroke,fill}%
}%
\begin{pgfscope}%
\pgfsys@transformshift{1.595349in}{3.169882in}%
\pgfsys@useobject{currentmarker}{}%
\end{pgfscope}%
\begin{pgfscope}%
\pgfsys@transformshift{2.243760in}{1.906687in}%
\pgfsys@useobject{currentmarker}{}%
\end{pgfscope}%
\begin{pgfscope}%
\pgfsys@transformshift{2.914922in}{1.746400in}%
\pgfsys@useobject{currentmarker}{}%
\end{pgfscope}%
\begin{pgfscope}%
\pgfsys@transformshift{3.608835in}{1.684647in}%
\pgfsys@useobject{currentmarker}{}%
\end{pgfscope}%
\end{pgfscope}%
\begin{pgfscope}%
\pgfsetrectcap%
\pgfsetmiterjoin%
\pgfsetlinewidth{0.803000pt}%
\definecolor{currentstroke}{rgb}{0.000000,0.000000,0.000000}%
\pgfsetstrokecolor{currentstroke}%
\pgfsetdash{}{0pt}%
\pgfpathmoveto{\pgfqpoint{0.650000in}{0.440000in}}%
\pgfpathlineto{\pgfqpoint{0.650000in}{3.520000in}}%
\pgfusepath{stroke}%
\end{pgfscope}%
\begin{pgfscope}%
\pgfsetrectcap%
\pgfsetmiterjoin%
\pgfsetlinewidth{0.803000pt}%
\definecolor{currentstroke}{rgb}{0.000000,0.000000,0.000000}%
\pgfsetstrokecolor{currentstroke}%
\pgfsetdash{}{0pt}%
\pgfpathmoveto{\pgfqpoint{4.680000in}{0.440000in}}%
\pgfpathlineto{\pgfqpoint{4.680000in}{3.520000in}}%
\pgfusepath{stroke}%
\end{pgfscope}%
\begin{pgfscope}%
\pgfsetrectcap%
\pgfsetmiterjoin%
\pgfsetlinewidth{0.803000pt}%
\definecolor{currentstroke}{rgb}{0.000000,0.000000,0.000000}%
\pgfsetstrokecolor{currentstroke}%
\pgfsetdash{}{0pt}%
\pgfpathmoveto{\pgfqpoint{0.650000in}{0.440000in}}%
\pgfpathlineto{\pgfqpoint{4.680000in}{0.440000in}}%
\pgfusepath{stroke}%
\end{pgfscope}%
\begin{pgfscope}%
\pgfsetrectcap%
\pgfsetmiterjoin%
\pgfsetlinewidth{0.803000pt}%
\definecolor{currentstroke}{rgb}{0.000000,0.000000,0.000000}%
\pgfsetstrokecolor{currentstroke}%
\pgfsetdash{}{0pt}%
\pgfpathmoveto{\pgfqpoint{0.650000in}{3.520000in}}%
\pgfpathlineto{\pgfqpoint{4.680000in}{3.520000in}}%
\pgfusepath{stroke}%
\end{pgfscope}%
\begin{pgfscope}%
\definecolor{textcolor}{rgb}{0.000000,0.000000,0.000000}%
\pgfsetstrokecolor{textcolor}%
\pgfsetfillcolor{textcolor}%
\pgftext[x=1.595349in,y=3.308771in,,base]{\color{textcolor}\rmfamily\fontsize{10.000000}{12.000000}\selectfont \(\displaystyle 1.98 \si{\angstrom^{-1}}\)}%
\end{pgfscope}%
\begin{pgfscope}%
\definecolor{textcolor}{rgb}{0.000000,0.000000,0.000000}%
\pgfsetstrokecolor{textcolor}%
\pgfsetfillcolor{textcolor}%
\pgftext[x=2.243760in,y=2.045576in,,base]{\color{textcolor}\rmfamily\fontsize{10.000000}{12.000000}\selectfont \(\displaystyle 3.66 \si{\angstrom^{-1}}\)}%
\end{pgfscope}%
\begin{pgfscope}%
\definecolor{textcolor}{rgb}{0.000000,0.000000,0.000000}%
\pgfsetstrokecolor{textcolor}%
\pgfsetfillcolor{textcolor}%
\pgftext[x=2.914922in,y=1.885289in,,base]{\color{textcolor}\rmfamily\fontsize{10.000000}{12.000000}\selectfont \(\displaystyle 5.40 \si{\angstrom^{-1}}\)}%
\end{pgfscope}%
\begin{pgfscope}%
\definecolor{textcolor}{rgb}{0.000000,0.000000,0.000000}%
\pgfsetstrokecolor{textcolor}%
\pgfsetfillcolor{textcolor}%
\pgftext[x=3.608835in,y=1.823536in,,base]{\color{textcolor}\rmfamily\fontsize{10.000000}{12.000000}\selectfont \(\displaystyle 7.20 \si{\angstrom^{-1}}\)}%
\end{pgfscope}%
\begin{pgfscope}%
\pgfsetbuttcap%
\pgfsetmiterjoin%
\definecolor{currentfill}{rgb}{1.000000,1.000000,1.000000}%
\pgfsetfillcolor{currentfill}%
\pgfsetfillopacity{0.800000}%
\pgfsetlinewidth{1.003750pt}%
\definecolor{currentstroke}{rgb}{0.800000,0.800000,0.800000}%
\pgfsetstrokecolor{currentstroke}%
\pgfsetstrokeopacity{0.800000}%
\pgfsetdash{}{0pt}%
\pgfpathmoveto{\pgfqpoint{3.065679in}{2.827890in}}%
\pgfpathlineto{\pgfqpoint{4.582778in}{2.827890in}}%
\pgfpathquadraticcurveto{\pgfqpoint{4.610556in}{2.827890in}}{\pgfqpoint{4.610556in}{2.855668in}}%
\pgfpathlineto{\pgfqpoint{4.610556in}{3.422778in}}%
\pgfpathquadraticcurveto{\pgfqpoint{4.610556in}{3.450556in}}{\pgfqpoint{4.582778in}{3.450556in}}%
\pgfpathlineto{\pgfqpoint{3.065679in}{3.450556in}}%
\pgfpathquadraticcurveto{\pgfqpoint{3.037901in}{3.450556in}}{\pgfqpoint{3.037901in}{3.422778in}}%
\pgfpathlineto{\pgfqpoint{3.037901in}{2.855668in}}%
\pgfpathquadraticcurveto{\pgfqpoint{3.037901in}{2.827890in}}{\pgfqpoint{3.065679in}{2.827890in}}%
\pgfpathlineto{\pgfqpoint{3.065679in}{2.827890in}}%
\pgfpathclose%
\pgfusepath{stroke,fill}%
\end{pgfscope}%
\begin{pgfscope}%
\pgfsetrectcap%
\pgfsetroundjoin%
\pgfsetlinewidth{1.505625pt}%
\definecolor{currentstroke}{rgb}{1.000000,0.647059,0.000000}%
\pgfsetstrokecolor{currentstroke}%
\pgfsetdash{}{0pt}%
\pgfpathmoveto{\pgfqpoint{3.093456in}{3.346389in}}%
\pgfpathlineto{\pgfqpoint{3.232345in}{3.346389in}}%
\pgfpathlineto{\pgfqpoint{3.371234in}{3.346389in}}%
\pgfusepath{stroke}%
\end{pgfscope}%
\begin{pgfscope}%
\definecolor{textcolor}{rgb}{0.000000,0.000000,0.000000}%
\pgfsetstrokecolor{textcolor}%
\pgfsetfillcolor{textcolor}%
\pgftext[x=3.482345in,y=3.297778in,left,base]{\color{textcolor}\rmfamily\fontsize{10.000000}{12.000000}\selectfont Fourier transform}%
\end{pgfscope}%
\begin{pgfscope}%
\pgfsetbuttcap%
\pgfsetroundjoin%
\definecolor{currentfill}{rgb}{0.000000,0.501961,0.000000}%
\pgfsetfillcolor{currentfill}%
\pgfsetlinewidth{1.003750pt}%
\definecolor{currentstroke}{rgb}{0.000000,0.000000,0.000000}%
\pgfsetstrokecolor{currentstroke}%
\pgfsetdash{}{0pt}%
\pgfsys@defobject{currentmarker}{\pgfqpoint{-0.041667in}{-0.041667in}}{\pgfqpoint{0.041667in}{0.041667in}}{%
\pgfpathmoveto{\pgfqpoint{0.000000in}{-0.041667in}}%
\pgfpathcurveto{\pgfqpoint{0.011050in}{-0.041667in}}{\pgfqpoint{0.021649in}{-0.037276in}}{\pgfqpoint{0.029463in}{-0.029463in}}%
\pgfpathcurveto{\pgfqpoint{0.037276in}{-0.021649in}}{\pgfqpoint{0.041667in}{-0.011050in}}{\pgfqpoint{0.041667in}{0.000000in}}%
\pgfpathcurveto{\pgfqpoint{0.041667in}{0.011050in}}{\pgfqpoint{0.037276in}{0.021649in}}{\pgfqpoint{0.029463in}{0.029463in}}%
\pgfpathcurveto{\pgfqpoint{0.021649in}{0.037276in}}{\pgfqpoint{0.011050in}{0.041667in}}{\pgfqpoint{0.000000in}{0.041667in}}%
\pgfpathcurveto{\pgfqpoint{-0.011050in}{0.041667in}}{\pgfqpoint{-0.021649in}{0.037276in}}{\pgfqpoint{-0.029463in}{0.029463in}}%
\pgfpathcurveto{\pgfqpoint{-0.037276in}{0.021649in}}{\pgfqpoint{-0.041667in}{0.011050in}}{\pgfqpoint{-0.041667in}{0.000000in}}%
\pgfpathcurveto{\pgfqpoint{-0.041667in}{-0.011050in}}{\pgfqpoint{-0.037276in}{-0.021649in}}{\pgfqpoint{-0.029463in}{-0.029463in}}%
\pgfpathcurveto{\pgfqpoint{-0.021649in}{-0.037276in}}{\pgfqpoint{-0.011050in}{-0.041667in}}{\pgfqpoint{0.000000in}{-0.041667in}}%
\pgfpathlineto{\pgfqpoint{0.000000in}{-0.041667in}}%
\pgfpathclose%
\pgfusepath{stroke,fill}%
}%
\begin{pgfscope}%
\pgfsys@transformshift{3.232345in}{3.152723in}%
\pgfsys@useobject{currentmarker}{}%
\end{pgfscope}%
\end{pgfscope}%
\begin{pgfscope}%
\definecolor{textcolor}{rgb}{0.000000,0.000000,0.000000}%
\pgfsetstrokecolor{textcolor}%
\pgfsetfillcolor{textcolor}%
\pgftext[x=3.482345in,y=3.104112in,left,base]{\color{textcolor}\rmfamily\fontsize{10.000000}{12.000000}\selectfont peaks}%
\end{pgfscope}%
\begin{pgfscope}%
\pgfsetbuttcap%
\pgfsetbeveljoin%
\definecolor{currentfill}{rgb}{0.501961,0.000000,0.501961}%
\pgfsetfillcolor{currentfill}%
\pgfsetlinewidth{1.003750pt}%
\definecolor{currentstroke}{rgb}{0.501961,0.000000,0.501961}%
\pgfsetstrokecolor{currentstroke}%
\pgfsetdash{}{0pt}%
\pgfsys@defobject{currentmarker}{\pgfqpoint{-0.026418in}{-0.022473in}}{\pgfqpoint{0.026418in}{0.027778in}}{%
\pgfpathmoveto{\pgfqpoint{0.000000in}{0.027778in}}%
\pgfpathlineto{\pgfqpoint{-0.006236in}{0.008584in}}%
\pgfpathlineto{\pgfqpoint{-0.026418in}{0.008584in}}%
\pgfpathlineto{\pgfqpoint{-0.010091in}{-0.003279in}}%
\pgfpathlineto{\pgfqpoint{-0.016327in}{-0.022473in}}%
\pgfpathlineto{\pgfqpoint{-0.000000in}{-0.010610in}}%
\pgfpathlineto{\pgfqpoint{0.016327in}{-0.022473in}}%
\pgfpathlineto{\pgfqpoint{0.010091in}{-0.003279in}}%
\pgfpathlineto{\pgfqpoint{0.026418in}{0.008584in}}%
\pgfpathlineto{\pgfqpoint{0.006236in}{0.008584in}}%
\pgfpathlineto{\pgfqpoint{0.000000in}{0.027778in}}%
\pgfpathclose%
\pgfusepath{stroke,fill}%
}%
\begin{pgfscope}%
\pgfsys@transformshift{3.232345in}{2.959056in}%
\pgfsys@useobject{currentmarker}{}%
\end{pgfscope}%
\end{pgfscope}%
\begin{pgfscope}%
\definecolor{textcolor}{rgb}{0.000000,0.000000,0.000000}%
\pgfsetstrokecolor{textcolor}%
\pgfsetfillcolor{textcolor}%
\pgftext[x=3.482345in,y=2.910445in,left,base]{\color{textcolor}\rmfamily\fontsize{10.000000}{12.000000}\selectfont Direct sampling}%
\end{pgfscope}%
\end{pgfpicture}%
\makeatother%
\endgroup%

        }
        \caption{Plot of $S(k)$ for two different method of sampling.}
        \label{step2_sk}
    \end{subfigure}
    \caption{Plot of the radial pair correlation function $g(r)$ and the structure factor $S(k)$. This result is obtained after $2000$ steps  in \textit{NVE} regime starting from an equilibration temperature of $T = 94.4$ \si{K}.}
\end{figure}

The peaks on the radial pair correlation function $g(r)$ correspond to the presence of particles in the region. It can be observed in figure \ref{step2_gr} the these values are $3.74$ \si{\angstrom}, $7.10$ \si{\angstrom} and $10.23$ \si{\angstrom}, which are matching with the results of Rahman \cite{Rahman} having peaks at $3.7$ \si{\angstrom}, $7.0$ \si{\angstrom} and $10.4$ \si{\angstrom}.

In an analog way, the structure factor can be obtained by computing the discrete fourier transform of $g(r)$ or rather sampling it directly using an estimate.
It turns out that the fourier transform method is really precise but unstable as $k$ tends to zero. From figure \ref{step2_sk} one sees the oscillation which are produced between $k = 0$ and $k = 1$ \si{\angstrom^{-1}}.
On the other hand, the direct sampling produces a smoother output, but it loses in precision. 
For the peak detection, the fourier transform were used and all peaks lying before $k = 2$ were ignored. Another possibility would have been a polynomial (or signal) interpolation of the direct sampling $S(k)$ and then peak detection.

The found peaks are indeed located at $1.98$, $3.66$, $5.4$, $7.2$ \si{\angstrom^{-1}} which are much close to the one found by Rahman \cite{Rahman}, hence $2.00$, $3.62$, $5.75$, $7.18$ \si{\angstrom^{-1}}.

\subsection{Changing the equilibration temperature}
% study g, S as function of T (for equilibration)

Here we want to check whether the structural properties change with respect to the temperature.
This task will be performed observing the structural factor $S(k)$ and how its peaks vary.

\par
\noindent
Considering the canonical ensemble, it's possible to find an approximation of $g(r)$ making the following assumptions:

\begin{align}\label{pmf}
     U(r) \approx 
    \begin{cases}
        \infty & r \le r_0 \\
        - \varepsilon & r_0 < r \le r_{cutoff} \\
        0 & r > r_{cutoff}
    \end{cases} \\
     g(r) \approx e^{-\frac{U(r)}{k_B T}} \text{   i.e. Argon is a dilute system \cite{Chandler}}
\end{align}

which leads using eq. (\ref{correlation}) to a structure factor approximation of

\begin{equation} \label{Sapprox}
    S(k) = 1 + C \cdot \left[\exp\left(\frac{\varepsilon}{k_B T}\right) - 1\right]
\end{equation}

for some constant $C$.

From the graph in figure (\ref{step2_sk_changeT}) that the temperature doesn't shift the structural factor but it damps it instead. From figure (\ref{step2_sk_changeT_peaks}), we see that the structure factor fits with a good approximation the identity (\ref{Sapprox}), because the peaks locates inside the fit uncertainty interval. The found coefficients are

\begin{align}
    &a_{fit} = 100 \pm 16.6 \si{K} \approx \varepsilon/k_b = 120 \si{K}
    &C_{fit} = -2.35 \pm 0.19
\end{align}

Indeed, it's not guaranteed that the approximation would work with higher temperatures, as $g(r)$ is supposed to take a much complex form compared to the one described in eq. \ref{pmf}.

\par
\noindent
Looking at the structure factor in figure (\ref{step2_sk_changeT}), notice that low temperatures produce large oscillations on the result. This is the same numerical phenomenon observed in figure (\ref{step2_sk}) for the \textit{fourier transform} computation. It is particularly remarkable for $T = 47.2$ \si{K}, where the result oscillates on a factor $\approx 0.25$.

\begin{figure}
    \begin{subfigure}{0.5\textwidth}
        \resizebox{\textwidth}{!}{
            %% Creator: Matplotlib, PGF backend
%%
%% To include the figure in your LaTeX document, write
%%   \input{<filename>.pgf}
%%
%% Make sure the required packages are loaded in your preamble
%%   \usepackage{pgf}
%%
%% and, on pdftex
%%   \usepackage[utf8]{inputenc}\DeclareUnicodeCharacter{2212}{-}
%%
%% or, on luatex and xetex
%%   \usepackage{unicode-math}
%%
%% Figures using additional raster images can only be included by \input if
%% they are in the same directory as the main LaTeX file. For loading figures
%% from other directories you can use the `import` package
%%   \usepackage{import}
%%
%% and then include the figures with
%%   \import{<path to file>}{<filename>.pgf}
%%
%% Matplotlib used the following preamble
%%   \usepackage[utf8]{inputenc}
%%   \usepackage[T1]{fontenc}
%%   \usepackage{siunitx}
%%
\begingroup%
\makeatletter%
\begin{pgfpicture}%
\pgfpathrectangle{\pgfpointorigin}{\pgfqpoint{4.500000in}{3.700000in}}%
\pgfusepath{use as bounding box, clip}%
\begin{pgfscope}%
\pgfsetbuttcap%
\pgfsetmiterjoin%
\definecolor{currentfill}{rgb}{1.000000,1.000000,1.000000}%
\pgfsetfillcolor{currentfill}%
\pgfsetlinewidth{0.000000pt}%
\definecolor{currentstroke}{rgb}{1.000000,1.000000,1.000000}%
\pgfsetstrokecolor{currentstroke}%
\pgfsetdash{}{0pt}%
\pgfpathmoveto{\pgfqpoint{0.000000in}{0.000000in}}%
\pgfpathlineto{\pgfqpoint{4.500000in}{0.000000in}}%
\pgfpathlineto{\pgfqpoint{4.500000in}{3.700000in}}%
\pgfpathlineto{\pgfqpoint{0.000000in}{3.700000in}}%
\pgfpathclose%
\pgfusepath{fill}%
\end{pgfscope}%
\begin{pgfscope}%
\pgfsetbuttcap%
\pgfsetmiterjoin%
\definecolor{currentfill}{rgb}{1.000000,1.000000,1.000000}%
\pgfsetfillcolor{currentfill}%
\pgfsetlinewidth{0.000000pt}%
\definecolor{currentstroke}{rgb}{0.000000,0.000000,0.000000}%
\pgfsetstrokecolor{currentstroke}%
\pgfsetstrokeopacity{0.000000}%
\pgfsetdash{}{0pt}%
\pgfpathmoveto{\pgfqpoint{0.562500in}{0.407000in}}%
\pgfpathlineto{\pgfqpoint{4.050000in}{0.407000in}}%
\pgfpathlineto{\pgfqpoint{4.050000in}{3.256000in}}%
\pgfpathlineto{\pgfqpoint{0.562500in}{3.256000in}}%
\pgfpathclose%
\pgfusepath{fill}%
\end{pgfscope}%
\begin{pgfscope}%
\pgfsetbuttcap%
\pgfsetroundjoin%
\definecolor{currentfill}{rgb}{0.000000,0.000000,0.000000}%
\pgfsetfillcolor{currentfill}%
\pgfsetlinewidth{0.803000pt}%
\definecolor{currentstroke}{rgb}{0.000000,0.000000,0.000000}%
\pgfsetstrokecolor{currentstroke}%
\pgfsetdash{}{0pt}%
\pgfsys@defobject{currentmarker}{\pgfqpoint{0.000000in}{-0.048611in}}{\pgfqpoint{0.000000in}{0.000000in}}{%
\pgfpathmoveto{\pgfqpoint{0.000000in}{0.000000in}}%
\pgfpathlineto{\pgfqpoint{0.000000in}{-0.048611in}}%
\pgfusepath{stroke,fill}%
}%
\begin{pgfscope}%
\pgfsys@transformshift{0.721023in}{0.407000in}%
\pgfsys@useobject{currentmarker}{}%
\end{pgfscope}%
\end{pgfscope}%
\begin{pgfscope}%
\definecolor{textcolor}{rgb}{0.000000,0.000000,0.000000}%
\pgfsetstrokecolor{textcolor}%
\pgfsetfillcolor{textcolor}%
\pgftext[x=0.721023in,y=0.309778in,,top]{\color{textcolor}\rmfamily\fontsize{10.000000}{12.000000}\selectfont \(\displaystyle {0}\)}%
\end{pgfscope}%
\begin{pgfscope}%
\pgfsetbuttcap%
\pgfsetroundjoin%
\definecolor{currentfill}{rgb}{0.000000,0.000000,0.000000}%
\pgfsetfillcolor{currentfill}%
\pgfsetlinewidth{0.803000pt}%
\definecolor{currentstroke}{rgb}{0.000000,0.000000,0.000000}%
\pgfsetstrokecolor{currentstroke}%
\pgfsetdash{}{0pt}%
\pgfsys@defobject{currentmarker}{\pgfqpoint{0.000000in}{-0.048611in}}{\pgfqpoint{0.000000in}{0.000000in}}{%
\pgfpathmoveto{\pgfqpoint{0.000000in}{0.000000in}}%
\pgfpathlineto{\pgfqpoint{0.000000in}{-0.048611in}}%
\pgfusepath{stroke,fill}%
}%
\begin{pgfscope}%
\pgfsys@transformshift{1.439659in}{0.407000in}%
\pgfsys@useobject{currentmarker}{}%
\end{pgfscope}%
\end{pgfscope}%
\begin{pgfscope}%
\definecolor{textcolor}{rgb}{0.000000,0.000000,0.000000}%
\pgfsetstrokecolor{textcolor}%
\pgfsetfillcolor{textcolor}%
\pgftext[x=1.439659in,y=0.309778in,,top]{\color{textcolor}\rmfamily\fontsize{10.000000}{12.000000}\selectfont \(\displaystyle {2}\)}%
\end{pgfscope}%
\begin{pgfscope}%
\pgfsetbuttcap%
\pgfsetroundjoin%
\definecolor{currentfill}{rgb}{0.000000,0.000000,0.000000}%
\pgfsetfillcolor{currentfill}%
\pgfsetlinewidth{0.803000pt}%
\definecolor{currentstroke}{rgb}{0.000000,0.000000,0.000000}%
\pgfsetstrokecolor{currentstroke}%
\pgfsetdash{}{0pt}%
\pgfsys@defobject{currentmarker}{\pgfqpoint{0.000000in}{-0.048611in}}{\pgfqpoint{0.000000in}{0.000000in}}{%
\pgfpathmoveto{\pgfqpoint{0.000000in}{0.000000in}}%
\pgfpathlineto{\pgfqpoint{0.000000in}{-0.048611in}}%
\pgfusepath{stroke,fill}%
}%
\begin{pgfscope}%
\pgfsys@transformshift{2.158295in}{0.407000in}%
\pgfsys@useobject{currentmarker}{}%
\end{pgfscope}%
\end{pgfscope}%
\begin{pgfscope}%
\definecolor{textcolor}{rgb}{0.000000,0.000000,0.000000}%
\pgfsetstrokecolor{textcolor}%
\pgfsetfillcolor{textcolor}%
\pgftext[x=2.158295in,y=0.309778in,,top]{\color{textcolor}\rmfamily\fontsize{10.000000}{12.000000}\selectfont \(\displaystyle {4}\)}%
\end{pgfscope}%
\begin{pgfscope}%
\pgfsetbuttcap%
\pgfsetroundjoin%
\definecolor{currentfill}{rgb}{0.000000,0.000000,0.000000}%
\pgfsetfillcolor{currentfill}%
\pgfsetlinewidth{0.803000pt}%
\definecolor{currentstroke}{rgb}{0.000000,0.000000,0.000000}%
\pgfsetstrokecolor{currentstroke}%
\pgfsetdash{}{0pt}%
\pgfsys@defobject{currentmarker}{\pgfqpoint{0.000000in}{-0.048611in}}{\pgfqpoint{0.000000in}{0.000000in}}{%
\pgfpathmoveto{\pgfqpoint{0.000000in}{0.000000in}}%
\pgfpathlineto{\pgfqpoint{0.000000in}{-0.048611in}}%
\pgfusepath{stroke,fill}%
}%
\begin{pgfscope}%
\pgfsys@transformshift{2.876932in}{0.407000in}%
\pgfsys@useobject{currentmarker}{}%
\end{pgfscope}%
\end{pgfscope}%
\begin{pgfscope}%
\definecolor{textcolor}{rgb}{0.000000,0.000000,0.000000}%
\pgfsetstrokecolor{textcolor}%
\pgfsetfillcolor{textcolor}%
\pgftext[x=2.876932in,y=0.309778in,,top]{\color{textcolor}\rmfamily\fontsize{10.000000}{12.000000}\selectfont \(\displaystyle {6}\)}%
\end{pgfscope}%
\begin{pgfscope}%
\pgfsetbuttcap%
\pgfsetroundjoin%
\definecolor{currentfill}{rgb}{0.000000,0.000000,0.000000}%
\pgfsetfillcolor{currentfill}%
\pgfsetlinewidth{0.803000pt}%
\definecolor{currentstroke}{rgb}{0.000000,0.000000,0.000000}%
\pgfsetstrokecolor{currentstroke}%
\pgfsetdash{}{0pt}%
\pgfsys@defobject{currentmarker}{\pgfqpoint{0.000000in}{-0.048611in}}{\pgfqpoint{0.000000in}{0.000000in}}{%
\pgfpathmoveto{\pgfqpoint{0.000000in}{0.000000in}}%
\pgfpathlineto{\pgfqpoint{0.000000in}{-0.048611in}}%
\pgfusepath{stroke,fill}%
}%
\begin{pgfscope}%
\pgfsys@transformshift{3.595568in}{0.407000in}%
\pgfsys@useobject{currentmarker}{}%
\end{pgfscope}%
\end{pgfscope}%
\begin{pgfscope}%
\definecolor{textcolor}{rgb}{0.000000,0.000000,0.000000}%
\pgfsetstrokecolor{textcolor}%
\pgfsetfillcolor{textcolor}%
\pgftext[x=3.595568in,y=0.309778in,,top]{\color{textcolor}\rmfamily\fontsize{10.000000}{12.000000}\selectfont \(\displaystyle {8}\)}%
\end{pgfscope}%
\begin{pgfscope}%
\definecolor{textcolor}{rgb}{0.000000,0.000000,0.000000}%
\pgfsetstrokecolor{textcolor}%
\pgfsetfillcolor{textcolor}%
\pgftext[x=2.306250in,y=0.131567in,,top]{\color{textcolor}\rmfamily\fontsize{10.000000}{12.000000}\selectfont \(\displaystyle k [\si{\angstrom^{-1}}]\)}%
\end{pgfscope}%
\begin{pgfscope}%
\pgfsetbuttcap%
\pgfsetroundjoin%
\definecolor{currentfill}{rgb}{0.000000,0.000000,0.000000}%
\pgfsetfillcolor{currentfill}%
\pgfsetlinewidth{0.803000pt}%
\definecolor{currentstroke}{rgb}{0.000000,0.000000,0.000000}%
\pgfsetstrokecolor{currentstroke}%
\pgfsetdash{}{0pt}%
\pgfsys@defobject{currentmarker}{\pgfqpoint{-0.048611in}{0.000000in}}{\pgfqpoint{0.000000in}{0.000000in}}{%
\pgfpathmoveto{\pgfqpoint{0.000000in}{0.000000in}}%
\pgfpathlineto{\pgfqpoint{-0.048611in}{0.000000in}}%
\pgfusepath{stroke,fill}%
}%
\begin{pgfscope}%
\pgfsys@transformshift{0.562500in}{0.666568in}%
\pgfsys@useobject{currentmarker}{}%
\end{pgfscope}%
\end{pgfscope}%
\begin{pgfscope}%
\definecolor{textcolor}{rgb}{0.000000,0.000000,0.000000}%
\pgfsetstrokecolor{textcolor}%
\pgfsetfillcolor{textcolor}%
\pgftext[x=0.287808in, y=0.618740in, left, base]{\color{textcolor}\rmfamily\fontsize{10.000000}{12.000000}\selectfont \(\displaystyle {0.0}\)}%
\end{pgfscope}%
\begin{pgfscope}%
\pgfsetbuttcap%
\pgfsetroundjoin%
\definecolor{currentfill}{rgb}{0.000000,0.000000,0.000000}%
\pgfsetfillcolor{currentfill}%
\pgfsetlinewidth{0.803000pt}%
\definecolor{currentstroke}{rgb}{0.000000,0.000000,0.000000}%
\pgfsetstrokecolor{currentstroke}%
\pgfsetdash{}{0pt}%
\pgfsys@defobject{currentmarker}{\pgfqpoint{-0.048611in}{0.000000in}}{\pgfqpoint{0.000000in}{0.000000in}}{%
\pgfpathmoveto{\pgfqpoint{0.000000in}{0.000000in}}%
\pgfpathlineto{\pgfqpoint{-0.048611in}{0.000000in}}%
\pgfusepath{stroke,fill}%
}%
\begin{pgfscope}%
\pgfsys@transformshift{0.562500in}{1.060003in}%
\pgfsys@useobject{currentmarker}{}%
\end{pgfscope}%
\end{pgfscope}%
\begin{pgfscope}%
\definecolor{textcolor}{rgb}{0.000000,0.000000,0.000000}%
\pgfsetstrokecolor{textcolor}%
\pgfsetfillcolor{textcolor}%
\pgftext[x=0.287808in, y=1.012175in, left, base]{\color{textcolor}\rmfamily\fontsize{10.000000}{12.000000}\selectfont \(\displaystyle {0.5}\)}%
\end{pgfscope}%
\begin{pgfscope}%
\pgfsetbuttcap%
\pgfsetroundjoin%
\definecolor{currentfill}{rgb}{0.000000,0.000000,0.000000}%
\pgfsetfillcolor{currentfill}%
\pgfsetlinewidth{0.803000pt}%
\definecolor{currentstroke}{rgb}{0.000000,0.000000,0.000000}%
\pgfsetstrokecolor{currentstroke}%
\pgfsetdash{}{0pt}%
\pgfsys@defobject{currentmarker}{\pgfqpoint{-0.048611in}{0.000000in}}{\pgfqpoint{0.000000in}{0.000000in}}{%
\pgfpathmoveto{\pgfqpoint{0.000000in}{0.000000in}}%
\pgfpathlineto{\pgfqpoint{-0.048611in}{0.000000in}}%
\pgfusepath{stroke,fill}%
}%
\begin{pgfscope}%
\pgfsys@transformshift{0.562500in}{1.453437in}%
\pgfsys@useobject{currentmarker}{}%
\end{pgfscope}%
\end{pgfscope}%
\begin{pgfscope}%
\definecolor{textcolor}{rgb}{0.000000,0.000000,0.000000}%
\pgfsetstrokecolor{textcolor}%
\pgfsetfillcolor{textcolor}%
\pgftext[x=0.287808in, y=1.405609in, left, base]{\color{textcolor}\rmfamily\fontsize{10.000000}{12.000000}\selectfont \(\displaystyle {1.0}\)}%
\end{pgfscope}%
\begin{pgfscope}%
\pgfsetbuttcap%
\pgfsetroundjoin%
\definecolor{currentfill}{rgb}{0.000000,0.000000,0.000000}%
\pgfsetfillcolor{currentfill}%
\pgfsetlinewidth{0.803000pt}%
\definecolor{currentstroke}{rgb}{0.000000,0.000000,0.000000}%
\pgfsetstrokecolor{currentstroke}%
\pgfsetdash{}{0pt}%
\pgfsys@defobject{currentmarker}{\pgfqpoint{-0.048611in}{0.000000in}}{\pgfqpoint{0.000000in}{0.000000in}}{%
\pgfpathmoveto{\pgfqpoint{0.000000in}{0.000000in}}%
\pgfpathlineto{\pgfqpoint{-0.048611in}{0.000000in}}%
\pgfusepath{stroke,fill}%
}%
\begin{pgfscope}%
\pgfsys@transformshift{0.562500in}{1.846872in}%
\pgfsys@useobject{currentmarker}{}%
\end{pgfscope}%
\end{pgfscope}%
\begin{pgfscope}%
\definecolor{textcolor}{rgb}{0.000000,0.000000,0.000000}%
\pgfsetstrokecolor{textcolor}%
\pgfsetfillcolor{textcolor}%
\pgftext[x=0.287808in, y=1.799044in, left, base]{\color{textcolor}\rmfamily\fontsize{10.000000}{12.000000}\selectfont \(\displaystyle {1.5}\)}%
\end{pgfscope}%
\begin{pgfscope}%
\pgfsetbuttcap%
\pgfsetroundjoin%
\definecolor{currentfill}{rgb}{0.000000,0.000000,0.000000}%
\pgfsetfillcolor{currentfill}%
\pgfsetlinewidth{0.803000pt}%
\definecolor{currentstroke}{rgb}{0.000000,0.000000,0.000000}%
\pgfsetstrokecolor{currentstroke}%
\pgfsetdash{}{0pt}%
\pgfsys@defobject{currentmarker}{\pgfqpoint{-0.048611in}{0.000000in}}{\pgfqpoint{0.000000in}{0.000000in}}{%
\pgfpathmoveto{\pgfqpoint{0.000000in}{0.000000in}}%
\pgfpathlineto{\pgfqpoint{-0.048611in}{0.000000in}}%
\pgfusepath{stroke,fill}%
}%
\begin{pgfscope}%
\pgfsys@transformshift{0.562500in}{2.240307in}%
\pgfsys@useobject{currentmarker}{}%
\end{pgfscope}%
\end{pgfscope}%
\begin{pgfscope}%
\definecolor{textcolor}{rgb}{0.000000,0.000000,0.000000}%
\pgfsetstrokecolor{textcolor}%
\pgfsetfillcolor{textcolor}%
\pgftext[x=0.287808in, y=2.192479in, left, base]{\color{textcolor}\rmfamily\fontsize{10.000000}{12.000000}\selectfont \(\displaystyle {2.0}\)}%
\end{pgfscope}%
\begin{pgfscope}%
\pgfsetbuttcap%
\pgfsetroundjoin%
\definecolor{currentfill}{rgb}{0.000000,0.000000,0.000000}%
\pgfsetfillcolor{currentfill}%
\pgfsetlinewidth{0.803000pt}%
\definecolor{currentstroke}{rgb}{0.000000,0.000000,0.000000}%
\pgfsetstrokecolor{currentstroke}%
\pgfsetdash{}{0pt}%
\pgfsys@defobject{currentmarker}{\pgfqpoint{-0.048611in}{0.000000in}}{\pgfqpoint{0.000000in}{0.000000in}}{%
\pgfpathmoveto{\pgfqpoint{0.000000in}{0.000000in}}%
\pgfpathlineto{\pgfqpoint{-0.048611in}{0.000000in}}%
\pgfusepath{stroke,fill}%
}%
\begin{pgfscope}%
\pgfsys@transformshift{0.562500in}{2.633741in}%
\pgfsys@useobject{currentmarker}{}%
\end{pgfscope}%
\end{pgfscope}%
\begin{pgfscope}%
\definecolor{textcolor}{rgb}{0.000000,0.000000,0.000000}%
\pgfsetstrokecolor{textcolor}%
\pgfsetfillcolor{textcolor}%
\pgftext[x=0.287808in, y=2.585913in, left, base]{\color{textcolor}\rmfamily\fontsize{10.000000}{12.000000}\selectfont \(\displaystyle {2.5}\)}%
\end{pgfscope}%
\begin{pgfscope}%
\pgfsetbuttcap%
\pgfsetroundjoin%
\definecolor{currentfill}{rgb}{0.000000,0.000000,0.000000}%
\pgfsetfillcolor{currentfill}%
\pgfsetlinewidth{0.803000pt}%
\definecolor{currentstroke}{rgb}{0.000000,0.000000,0.000000}%
\pgfsetstrokecolor{currentstroke}%
\pgfsetdash{}{0pt}%
\pgfsys@defobject{currentmarker}{\pgfqpoint{-0.048611in}{0.000000in}}{\pgfqpoint{0.000000in}{0.000000in}}{%
\pgfpathmoveto{\pgfqpoint{0.000000in}{0.000000in}}%
\pgfpathlineto{\pgfqpoint{-0.048611in}{0.000000in}}%
\pgfusepath{stroke,fill}%
}%
\begin{pgfscope}%
\pgfsys@transformshift{0.562500in}{3.027176in}%
\pgfsys@useobject{currentmarker}{}%
\end{pgfscope}%
\end{pgfscope}%
\begin{pgfscope}%
\definecolor{textcolor}{rgb}{0.000000,0.000000,0.000000}%
\pgfsetstrokecolor{textcolor}%
\pgfsetfillcolor{textcolor}%
\pgftext[x=0.287808in, y=2.979348in, left, base]{\color{textcolor}\rmfamily\fontsize{10.000000}{12.000000}\selectfont \(\displaystyle {3.0}\)}%
\end{pgfscope}%
\begin{pgfscope}%
\definecolor{textcolor}{rgb}{0.000000,0.000000,0.000000}%
\pgfsetstrokecolor{textcolor}%
\pgfsetfillcolor{textcolor}%
\pgftext[x=0.232253in,y=1.831500in,,bottom,rotate=90.000000]{\color{textcolor}\rmfamily\fontsize{10.000000}{12.000000}\selectfont \(\displaystyle S(k)\)}%
\end{pgfscope}%
\begin{pgfscope}%
\pgfpathrectangle{\pgfqpoint{0.562500in}{0.407000in}}{\pgfqpoint{3.487500in}{2.849000in}}%
\pgfusepath{clip}%
\pgfsetrectcap%
\pgfsetroundjoin%
\pgfsetlinewidth{1.505625pt}%
\definecolor{currentstroke}{rgb}{0.121569,0.466667,0.705882}%
\pgfsetstrokecolor{currentstroke}%
\pgfsetdash{}{0pt}%
\pgfpathmoveto{\pgfqpoint{0.721023in}{0.678628in}}%
\pgfpathlineto{\pgfqpoint{0.731626in}{0.699404in}}%
\pgfpathlineto{\pgfqpoint{0.742230in}{0.758387in}}%
\pgfpathlineto{\pgfqpoint{0.752833in}{0.846057in}}%
\pgfpathlineto{\pgfqpoint{0.774040in}{1.047974in}}%
\pgfpathlineto{\pgfqpoint{0.784644in}{1.128795in}}%
\pgfpathlineto{\pgfqpoint{0.795247in}{1.176702in}}%
\pgfpathlineto{\pgfqpoint{0.805851in}{1.182743in}}%
\pgfpathlineto{\pgfqpoint{0.816454in}{1.144446in}}%
\pgfpathlineto{\pgfqpoint{0.827058in}{1.066295in}}%
\pgfpathlineto{\pgfqpoint{0.837662in}{0.959076in}}%
\pgfpathlineto{\pgfqpoint{0.858869in}{0.720964in}}%
\pgfpathlineto{\pgfqpoint{0.869472in}{0.624045in}}%
\pgfpathlineto{\pgfqpoint{0.880076in}{0.560248in}}%
\pgfpathlineto{\pgfqpoint{0.890679in}{0.536500in}}%
\pgfpathlineto{\pgfqpoint{0.901283in}{0.552620in}}%
\pgfpathlineto{\pgfqpoint{0.911886in}{0.601420in}}%
\pgfpathlineto{\pgfqpoint{0.943697in}{0.803434in}}%
\pgfpathlineto{\pgfqpoint{0.954300in}{0.839258in}}%
\pgfpathlineto{\pgfqpoint{0.964904in}{0.842998in}}%
\pgfpathlineto{\pgfqpoint{0.975507in}{0.814451in}}%
\pgfpathlineto{\pgfqpoint{0.986111in}{0.760432in}}%
\pgfpathlineto{\pgfqpoint{1.007318in}{0.628266in}}%
\pgfpathlineto{\pgfqpoint{1.017921in}{0.580145in}}%
\pgfpathlineto{\pgfqpoint{1.028525in}{0.560034in}}%
\pgfpathlineto{\pgfqpoint{1.039129in}{0.572840in}}%
\pgfpathlineto{\pgfqpoint{1.049732in}{0.616076in}}%
\pgfpathlineto{\pgfqpoint{1.081543in}{0.812196in}}%
\pgfpathlineto{\pgfqpoint{1.092146in}{0.849321in}}%
\pgfpathlineto{\pgfqpoint{1.102750in}{0.853341in}}%
\pgfpathlineto{\pgfqpoint{1.113353in}{0.822684in}}%
\pgfpathlineto{\pgfqpoint{1.123957in}{0.763973in}}%
\pgfpathlineto{\pgfqpoint{1.145164in}{0.621016in}}%
\pgfpathlineto{\pgfqpoint{1.155767in}{0.572515in}}%
\pgfpathlineto{\pgfqpoint{1.166371in}{0.559327in}}%
\pgfpathlineto{\pgfqpoint{1.176974in}{0.587990in}}%
\pgfpathlineto{\pgfqpoint{1.187578in}{0.655612in}}%
\pgfpathlineto{\pgfqpoint{1.219388in}{0.938379in}}%
\pgfpathlineto{\pgfqpoint{1.229992in}{0.989504in}}%
\pgfpathlineto{\pgfqpoint{1.240596in}{0.991439in}}%
\pgfpathlineto{\pgfqpoint{1.251199in}{0.941277in}}%
\pgfpathlineto{\pgfqpoint{1.261803in}{0.848641in}}%
\pgfpathlineto{\pgfqpoint{1.283010in}{0.630271in}}%
\pgfpathlineto{\pgfqpoint{1.293613in}{0.567864in}}%
\pgfpathlineto{\pgfqpoint{1.304217in}{0.578122in}}%
\pgfpathlineto{\pgfqpoint{1.314820in}{0.682375in}}%
\pgfpathlineto{\pgfqpoint{1.325424in}{0.888228in}}%
\pgfpathlineto{\pgfqpoint{1.336027in}{1.187239in}}%
\pgfpathlineto{\pgfqpoint{1.357234in}{1.957017in}}%
\pgfpathlineto{\pgfqpoint{1.367838in}{2.349420in}}%
\pgfpathlineto{\pgfqpoint{1.378441in}{2.690747in}}%
\pgfpathlineto{\pgfqpoint{1.389045in}{2.946222in}}%
\pgfpathlineto{\pgfqpoint{1.399648in}{3.093652in}}%
\pgfpathlineto{\pgfqpoint{1.410252in}{3.126500in}}%
\pgfpathlineto{\pgfqpoint{1.420856in}{3.054106in}}%
\pgfpathlineto{\pgfqpoint{1.431459in}{2.899110in}}%
\pgfpathlineto{\pgfqpoint{1.452666in}{2.468431in}}%
\pgfpathlineto{\pgfqpoint{1.463270in}{2.256565in}}%
\pgfpathlineto{\pgfqpoint{1.473873in}{2.078847in}}%
\pgfpathlineto{\pgfqpoint{1.484477in}{1.945824in}}%
\pgfpathlineto{\pgfqpoint{1.495080in}{1.856393in}}%
\pgfpathlineto{\pgfqpoint{1.505684in}{1.799744in}}%
\pgfpathlineto{\pgfqpoint{1.526891in}{1.716332in}}%
\pgfpathlineto{\pgfqpoint{1.537494in}{1.656498in}}%
\pgfpathlineto{\pgfqpoint{1.548098in}{1.571225in}}%
\pgfpathlineto{\pgfqpoint{1.558701in}{1.460403in}}%
\pgfpathlineto{\pgfqpoint{1.590512in}{1.078627in}}%
\pgfpathlineto{\pgfqpoint{1.601115in}{0.984749in}}%
\pgfpathlineto{\pgfqpoint{1.611719in}{0.927635in}}%
\pgfpathlineto{\pgfqpoint{1.622323in}{0.910416in}}%
\pgfpathlineto{\pgfqpoint{1.632926in}{0.928749in}}%
\pgfpathlineto{\pgfqpoint{1.643530in}{0.971917in}}%
\pgfpathlineto{\pgfqpoint{1.664737in}{1.073650in}}%
\pgfpathlineto{\pgfqpoint{1.675340in}{1.104389in}}%
\pgfpathlineto{\pgfqpoint{1.685944in}{1.110264in}}%
\pgfpathlineto{\pgfqpoint{1.696547in}{1.090762in}}%
\pgfpathlineto{\pgfqpoint{1.707151in}{1.052040in}}%
\pgfpathlineto{\pgfqpoint{1.717754in}{1.005433in}}%
\pgfpathlineto{\pgfqpoint{1.728358in}{0.964907in}}%
\pgfpathlineto{\pgfqpoint{1.738961in}{0.944014in}}%
\pgfpathlineto{\pgfqpoint{1.749565in}{0.953018in}}%
\pgfpathlineto{\pgfqpoint{1.760168in}{0.996770in}}%
\pgfpathlineto{\pgfqpoint{1.770772in}{1.073741in}}%
\pgfpathlineto{\pgfqpoint{1.781375in}{1.176345in}}%
\pgfpathlineto{\pgfqpoint{1.802582in}{1.407683in}}%
\pgfpathlineto{\pgfqpoint{1.813186in}{1.508074in}}%
\pgfpathlineto{\pgfqpoint{1.823790in}{1.582462in}}%
\pgfpathlineto{\pgfqpoint{1.834393in}{1.624252in}}%
\pgfpathlineto{\pgfqpoint{1.844997in}{1.632233in}}%
\pgfpathlineto{\pgfqpoint{1.855600in}{1.610375in}}%
\pgfpathlineto{\pgfqpoint{1.866204in}{1.566746in}}%
\pgfpathlineto{\pgfqpoint{1.887411in}{1.456723in}}%
\pgfpathlineto{\pgfqpoint{1.898014in}{1.411207in}}%
\pgfpathlineto{\pgfqpoint{1.908618in}{1.382635in}}%
\pgfpathlineto{\pgfqpoint{1.919221in}{1.375143in}}%
\pgfpathlineto{\pgfqpoint{1.929825in}{1.389586in}}%
\pgfpathlineto{\pgfqpoint{1.940428in}{1.423990in}}%
\pgfpathlineto{\pgfqpoint{1.951032in}{1.474324in}}%
\pgfpathlineto{\pgfqpoint{1.972239in}{1.601677in}}%
\pgfpathlineto{\pgfqpoint{1.993446in}{1.729483in}}%
\pgfpathlineto{\pgfqpoint{2.004049in}{1.782616in}}%
\pgfpathlineto{\pgfqpoint{2.014653in}{1.824337in}}%
\pgfpathlineto{\pgfqpoint{2.025257in}{1.852465in}}%
\pgfpathlineto{\pgfqpoint{2.035860in}{1.865596in}}%
\pgfpathlineto{\pgfqpoint{2.046464in}{1.863149in}}%
\pgfpathlineto{\pgfqpoint{2.057067in}{1.845450in}}%
\pgfpathlineto{\pgfqpoint{2.067671in}{1.813819in}}%
\pgfpathlineto{\pgfqpoint{2.078274in}{1.770573in}}%
\pgfpathlineto{\pgfqpoint{2.099481in}{1.662548in}}%
\pgfpathlineto{\pgfqpoint{2.120688in}{1.551195in}}%
\pgfpathlineto{\pgfqpoint{2.131292in}{1.502557in}}%
\pgfpathlineto{\pgfqpoint{2.141895in}{1.461188in}}%
\pgfpathlineto{\pgfqpoint{2.152499in}{1.427471in}}%
\pgfpathlineto{\pgfqpoint{2.163102in}{1.400638in}}%
\pgfpathlineto{\pgfqpoint{2.173706in}{1.379104in}}%
\pgfpathlineto{\pgfqpoint{2.194913in}{1.344379in}}%
\pgfpathlineto{\pgfqpoint{2.226724in}{1.296510in}}%
\pgfpathlineto{\pgfqpoint{2.237327in}{1.282414in}}%
\pgfpathlineto{\pgfqpoint{2.247931in}{1.271086in}}%
\pgfpathlineto{\pgfqpoint{2.258534in}{1.263557in}}%
\pgfpathlineto{\pgfqpoint{2.269138in}{1.260273in}}%
\pgfpathlineto{\pgfqpoint{2.279741in}{1.260912in}}%
\pgfpathlineto{\pgfqpoint{2.290345in}{1.264410in}}%
\pgfpathlineto{\pgfqpoint{2.311552in}{1.273752in}}%
\pgfpathlineto{\pgfqpoint{2.322155in}{1.276789in}}%
\pgfpathlineto{\pgfqpoint{2.332759in}{1.277936in}}%
\pgfpathlineto{\pgfqpoint{2.353966in}{1.277980in}}%
\pgfpathlineto{\pgfqpoint{2.364569in}{1.280722in}}%
\pgfpathlineto{\pgfqpoint{2.375173in}{1.288468in}}%
\pgfpathlineto{\pgfqpoint{2.385776in}{1.303249in}}%
\pgfpathlineto{\pgfqpoint{2.396380in}{1.326144in}}%
\pgfpathlineto{\pgfqpoint{2.406984in}{1.356920in}}%
\pgfpathlineto{\pgfqpoint{2.428191in}{1.434303in}}%
\pgfpathlineto{\pgfqpoint{2.449398in}{1.510607in}}%
\pgfpathlineto{\pgfqpoint{2.460001in}{1.539631in}}%
\pgfpathlineto{\pgfqpoint{2.470605in}{1.559437in}}%
\pgfpathlineto{\pgfqpoint{2.481208in}{1.569388in}}%
\pgfpathlineto{\pgfqpoint{2.491812in}{1.570326in}}%
\pgfpathlineto{\pgfqpoint{2.502415in}{1.564321in}}%
\pgfpathlineto{\pgfqpoint{2.534226in}{1.533098in}}%
\pgfpathlineto{\pgfqpoint{2.544829in}{1.526405in}}%
\pgfpathlineto{\pgfqpoint{2.555433in}{1.523492in}}%
\pgfpathlineto{\pgfqpoint{2.566036in}{1.523961in}}%
\pgfpathlineto{\pgfqpoint{2.597847in}{1.532481in}}%
\pgfpathlineto{\pgfqpoint{2.608451in}{1.533122in}}%
\pgfpathlineto{\pgfqpoint{2.619054in}{1.531512in}}%
\pgfpathlineto{\pgfqpoint{2.640261in}{1.523290in}}%
\pgfpathlineto{\pgfqpoint{2.661468in}{1.514821in}}%
\pgfpathlineto{\pgfqpoint{2.672072in}{1.512592in}}%
\pgfpathlineto{\pgfqpoint{2.693279in}{1.512305in}}%
\pgfpathlineto{\pgfqpoint{2.714486in}{1.512073in}}%
\pgfpathlineto{\pgfqpoint{2.725089in}{1.509289in}}%
\pgfpathlineto{\pgfqpoint{2.735693in}{1.503738in}}%
\pgfpathlineto{\pgfqpoint{2.746296in}{1.495319in}}%
\pgfpathlineto{\pgfqpoint{2.767503in}{1.472197in}}%
\pgfpathlineto{\pgfqpoint{2.788710in}{1.447629in}}%
\pgfpathlineto{\pgfqpoint{2.799314in}{1.437236in}}%
\pgfpathlineto{\pgfqpoint{2.809918in}{1.428691in}}%
\pgfpathlineto{\pgfqpoint{2.831125in}{1.416084in}}%
\pgfpathlineto{\pgfqpoint{2.852332in}{1.404920in}}%
\pgfpathlineto{\pgfqpoint{2.873539in}{1.390835in}}%
\pgfpathlineto{\pgfqpoint{2.894746in}{1.375217in}}%
\pgfpathlineto{\pgfqpoint{2.905349in}{1.368799in}}%
\pgfpathlineto{\pgfqpoint{2.915953in}{1.364523in}}%
\pgfpathlineto{\pgfqpoint{2.926556in}{1.363021in}}%
\pgfpathlineto{\pgfqpoint{2.937160in}{1.364504in}}%
\pgfpathlineto{\pgfqpoint{2.947763in}{1.368719in}}%
\pgfpathlineto{\pgfqpoint{2.968970in}{1.382522in}}%
\pgfpathlineto{\pgfqpoint{2.990177in}{1.397755in}}%
\pgfpathlineto{\pgfqpoint{3.011385in}{1.410397in}}%
\pgfpathlineto{\pgfqpoint{3.032592in}{1.422005in}}%
\pgfpathlineto{\pgfqpoint{3.053799in}{1.436979in}}%
\pgfpathlineto{\pgfqpoint{3.075006in}{1.457058in}}%
\pgfpathlineto{\pgfqpoint{3.096213in}{1.478485in}}%
\pgfpathlineto{\pgfqpoint{3.106816in}{1.487586in}}%
\pgfpathlineto{\pgfqpoint{3.117420in}{1.494617in}}%
\pgfpathlineto{\pgfqpoint{3.128023in}{1.499221in}}%
\pgfpathlineto{\pgfqpoint{3.138627in}{1.501457in}}%
\pgfpathlineto{\pgfqpoint{3.149230in}{1.501770in}}%
\pgfpathlineto{\pgfqpoint{3.191644in}{1.498207in}}%
\pgfpathlineto{\pgfqpoint{3.244662in}{1.499812in}}%
\pgfpathlineto{\pgfqpoint{3.255266in}{1.497401in}}%
\pgfpathlineto{\pgfqpoint{3.276473in}{1.488609in}}%
\pgfpathlineto{\pgfqpoint{3.308283in}{1.473519in}}%
\pgfpathlineto{\pgfqpoint{3.318887in}{1.470247in}}%
\pgfpathlineto{\pgfqpoint{3.340094in}{1.467013in}}%
\pgfpathlineto{\pgfqpoint{3.361301in}{1.465116in}}%
\pgfpathlineto{\pgfqpoint{3.371904in}{1.463127in}}%
\pgfpathlineto{\pgfqpoint{3.382508in}{1.459866in}}%
\pgfpathlineto{\pgfqpoint{3.403715in}{1.449625in}}%
\pgfpathlineto{\pgfqpoint{3.424922in}{1.437543in}}%
\pgfpathlineto{\pgfqpoint{3.435526in}{1.432494in}}%
\pgfpathlineto{\pgfqpoint{3.446129in}{1.428835in}}%
\pgfpathlineto{\pgfqpoint{3.456733in}{1.426773in}}%
\pgfpathlineto{\pgfqpoint{3.477940in}{1.426640in}}%
\pgfpathlineto{\pgfqpoint{3.509750in}{1.428388in}}%
\pgfpathlineto{\pgfqpoint{3.530957in}{1.425938in}}%
\pgfpathlineto{\pgfqpoint{3.562768in}{1.420267in}}%
\pgfpathlineto{\pgfqpoint{3.573371in}{1.419917in}}%
\pgfpathlineto{\pgfqpoint{3.583975in}{1.420939in}}%
\pgfpathlineto{\pgfqpoint{3.605182in}{1.426780in}}%
\pgfpathlineto{\pgfqpoint{3.636993in}{1.438679in}}%
\pgfpathlineto{\pgfqpoint{3.658200in}{1.443892in}}%
\pgfpathlineto{\pgfqpoint{3.700614in}{1.450722in}}%
\pgfpathlineto{\pgfqpoint{3.721821in}{1.457373in}}%
\pgfpathlineto{\pgfqpoint{3.753631in}{1.469685in}}%
\pgfpathlineto{\pgfqpoint{3.774838in}{1.474606in}}%
\pgfpathlineto{\pgfqpoint{3.796046in}{1.475197in}}%
\pgfpathlineto{\pgfqpoint{3.838460in}{1.473040in}}%
\pgfpathlineto{\pgfqpoint{3.891477in}{1.476464in}}%
\pgfpathlineto{\pgfqpoint{3.891477in}{1.476464in}}%
\pgfusepath{stroke}%
\end{pgfscope}%
\begin{pgfscope}%
\pgfpathrectangle{\pgfqpoint{0.562500in}{0.407000in}}{\pgfqpoint{3.487500in}{2.849000in}}%
\pgfusepath{clip}%
\pgfsetrectcap%
\pgfsetroundjoin%
\pgfsetlinewidth{1.505625pt}%
\definecolor{currentstroke}{rgb}{1.000000,0.498039,0.054902}%
\pgfsetstrokecolor{currentstroke}%
\pgfsetdash{}{0pt}%
\pgfpathmoveto{\pgfqpoint{0.721023in}{1.129823in}}%
\pgfpathlineto{\pgfqpoint{0.731626in}{1.120348in}}%
\pgfpathlineto{\pgfqpoint{0.742230in}{1.093214in}}%
\pgfpathlineto{\pgfqpoint{0.752833in}{1.052033in}}%
\pgfpathlineto{\pgfqpoint{0.795247in}{0.852488in}}%
\pgfpathlineto{\pgfqpoint{0.805851in}{0.814907in}}%
\pgfpathlineto{\pgfqpoint{0.816454in}{0.784928in}}%
\pgfpathlineto{\pgfqpoint{0.827058in}{0.761019in}}%
\pgfpathlineto{\pgfqpoint{0.848265in}{0.722755in}}%
\pgfpathlineto{\pgfqpoint{0.869472in}{0.688675in}}%
\pgfpathlineto{\pgfqpoint{0.880076in}{0.674390in}}%
\pgfpathlineto{\pgfqpoint{0.890679in}{0.664552in}}%
\pgfpathlineto{\pgfqpoint{0.901283in}{0.661283in}}%
\pgfpathlineto{\pgfqpoint{0.911886in}{0.665852in}}%
\pgfpathlineto{\pgfqpoint{0.922490in}{0.678034in}}%
\pgfpathlineto{\pgfqpoint{0.954300in}{0.733524in}}%
\pgfpathlineto{\pgfqpoint{0.964904in}{0.744858in}}%
\pgfpathlineto{\pgfqpoint{0.975507in}{0.746834in}}%
\pgfpathlineto{\pgfqpoint{0.986111in}{0.738551in}}%
\pgfpathlineto{\pgfqpoint{0.996714in}{0.721562in}}%
\pgfpathlineto{\pgfqpoint{1.017921in}{0.678138in}}%
\pgfpathlineto{\pgfqpoint{1.028525in}{0.662507in}}%
\pgfpathlineto{\pgfqpoint{1.039129in}{0.657222in}}%
\pgfpathlineto{\pgfqpoint{1.049732in}{0.664483in}}%
\pgfpathlineto{\pgfqpoint{1.060336in}{0.683559in}}%
\pgfpathlineto{\pgfqpoint{1.092146in}{0.765626in}}%
\pgfpathlineto{\pgfqpoint{1.102750in}{0.780584in}}%
\pgfpathlineto{\pgfqpoint{1.113353in}{0.781629in}}%
\pgfpathlineto{\pgfqpoint{1.123957in}{0.768483in}}%
\pgfpathlineto{\pgfqpoint{1.134560in}{0.744530in}}%
\pgfpathlineto{\pgfqpoint{1.145164in}{0.716239in}}%
\pgfpathlineto{\pgfqpoint{1.155767in}{0.691791in}}%
\pgfpathlineto{\pgfqpoint{1.166371in}{0.679188in}}%
\pgfpathlineto{\pgfqpoint{1.176974in}{0.684279in}}%
\pgfpathlineto{\pgfqpoint{1.187578in}{0.709168in}}%
\pgfpathlineto{\pgfqpoint{1.198181in}{0.751412in}}%
\pgfpathlineto{\pgfqpoint{1.219388in}{0.858098in}}%
\pgfpathlineto{\pgfqpoint{1.229992in}{0.902369in}}%
\pgfpathlineto{\pgfqpoint{1.240596in}{0.928411in}}%
\pgfpathlineto{\pgfqpoint{1.251199in}{0.931756in}}%
\pgfpathlineto{\pgfqpoint{1.261803in}{0.913882in}}%
\pgfpathlineto{\pgfqpoint{1.283010in}{0.852140in}}%
\pgfpathlineto{\pgfqpoint{1.293613in}{0.839255in}}%
\pgfpathlineto{\pgfqpoint{1.304217in}{0.861942in}}%
\pgfpathlineto{\pgfqpoint{1.314820in}{0.935081in}}%
\pgfpathlineto{\pgfqpoint{1.325424in}{1.067436in}}%
\pgfpathlineto{\pgfqpoint{1.336027in}{1.259397in}}%
\pgfpathlineto{\pgfqpoint{1.346631in}{1.502125in}}%
\pgfpathlineto{\pgfqpoint{1.378441in}{2.334484in}}%
\pgfpathlineto{\pgfqpoint{1.389045in}{2.563179in}}%
\pgfpathlineto{\pgfqpoint{1.399648in}{2.730447in}}%
\pgfpathlineto{\pgfqpoint{1.410252in}{2.823889in}}%
\pgfpathlineto{\pgfqpoint{1.420856in}{2.840352in}}%
\pgfpathlineto{\pgfqpoint{1.431459in}{2.785834in}}%
\pgfpathlineto{\pgfqpoint{1.442063in}{2.673897in}}%
\pgfpathlineto{\pgfqpoint{1.452666in}{2.522936in}}%
\pgfpathlineto{\pgfqpoint{1.484477in}{2.024266in}}%
\pgfpathlineto{\pgfqpoint{1.495080in}{1.887938in}}%
\pgfpathlineto{\pgfqpoint{1.505684in}{1.775299in}}%
\pgfpathlineto{\pgfqpoint{1.516287in}{1.683603in}}%
\pgfpathlineto{\pgfqpoint{1.537494in}{1.538054in}}%
\pgfpathlineto{\pgfqpoint{1.569305in}{1.332152in}}%
\pgfpathlineto{\pgfqpoint{1.590512in}{1.198238in}}%
\pgfpathlineto{\pgfqpoint{1.601115in}{1.144725in}}%
\pgfpathlineto{\pgfqpoint{1.611719in}{1.106236in}}%
\pgfpathlineto{\pgfqpoint{1.622323in}{1.084912in}}%
\pgfpathlineto{\pgfqpoint{1.632926in}{1.080089in}}%
\pgfpathlineto{\pgfqpoint{1.643530in}{1.088367in}}%
\pgfpathlineto{\pgfqpoint{1.675340in}{1.135286in}}%
\pgfpathlineto{\pgfqpoint{1.685944in}{1.140841in}}%
\pgfpathlineto{\pgfqpoint{1.696547in}{1.137302in}}%
\pgfpathlineto{\pgfqpoint{1.707151in}{1.126212in}}%
\pgfpathlineto{\pgfqpoint{1.728358in}{1.098157in}}%
\pgfpathlineto{\pgfqpoint{1.738961in}{1.091775in}}%
\pgfpathlineto{\pgfqpoint{1.749565in}{1.096569in}}%
\pgfpathlineto{\pgfqpoint{1.760168in}{1.114825in}}%
\pgfpathlineto{\pgfqpoint{1.770772in}{1.146356in}}%
\pgfpathlineto{\pgfqpoint{1.781375in}{1.188585in}}%
\pgfpathlineto{\pgfqpoint{1.813186in}{1.332462in}}%
\pgfpathlineto{\pgfqpoint{1.823790in}{1.370376in}}%
\pgfpathlineto{\pgfqpoint{1.834393in}{1.398494in}}%
\pgfpathlineto{\pgfqpoint{1.844997in}{1.416867in}}%
\pgfpathlineto{\pgfqpoint{1.855600in}{1.427397in}}%
\pgfpathlineto{\pgfqpoint{1.887411in}{1.445434in}}%
\pgfpathlineto{\pgfqpoint{1.898014in}{1.457527in}}%
\pgfpathlineto{\pgfqpoint{1.908618in}{1.475518in}}%
\pgfpathlineto{\pgfqpoint{1.919221in}{1.499110in}}%
\pgfpathlineto{\pgfqpoint{1.940428in}{1.556698in}}%
\pgfpathlineto{\pgfqpoint{1.961635in}{1.613783in}}%
\pgfpathlineto{\pgfqpoint{1.972239in}{1.637771in}}%
\pgfpathlineto{\pgfqpoint{1.982842in}{1.657685in}}%
\pgfpathlineto{\pgfqpoint{1.993446in}{1.673548in}}%
\pgfpathlineto{\pgfqpoint{2.004049in}{1.685755in}}%
\pgfpathlineto{\pgfqpoint{2.014653in}{1.694766in}}%
\pgfpathlineto{\pgfqpoint{2.025257in}{1.700843in}}%
\pgfpathlineto{\pgfqpoint{2.035860in}{1.703898in}}%
\pgfpathlineto{\pgfqpoint{2.046464in}{1.703495in}}%
\pgfpathlineto{\pgfqpoint{2.057067in}{1.698990in}}%
\pgfpathlineto{\pgfqpoint{2.067671in}{1.689767in}}%
\pgfpathlineto{\pgfqpoint{2.078274in}{1.675496in}}%
\pgfpathlineto{\pgfqpoint{2.088878in}{1.656320in}}%
\pgfpathlineto{\pgfqpoint{2.099481in}{1.632945in}}%
\pgfpathlineto{\pgfqpoint{2.131292in}{1.550890in}}%
\pgfpathlineto{\pgfqpoint{2.152499in}{1.500180in}}%
\pgfpathlineto{\pgfqpoint{2.173706in}{1.458864in}}%
\pgfpathlineto{\pgfqpoint{2.205516in}{1.407005in}}%
\pgfpathlineto{\pgfqpoint{2.237327in}{1.356988in}}%
\pgfpathlineto{\pgfqpoint{2.247931in}{1.342161in}}%
\pgfpathlineto{\pgfqpoint{2.258534in}{1.329613in}}%
\pgfpathlineto{\pgfqpoint{2.269138in}{1.320024in}}%
\pgfpathlineto{\pgfqpoint{2.279741in}{1.313703in}}%
\pgfpathlineto{\pgfqpoint{2.290345in}{1.310502in}}%
\pgfpathlineto{\pgfqpoint{2.300948in}{1.309849in}}%
\pgfpathlineto{\pgfqpoint{2.322155in}{1.312707in}}%
\pgfpathlineto{\pgfqpoint{2.375173in}{1.321971in}}%
\pgfpathlineto{\pgfqpoint{2.385776in}{1.327000in}}%
\pgfpathlineto{\pgfqpoint{2.396380in}{1.334963in}}%
\pgfpathlineto{\pgfqpoint{2.406984in}{1.346251in}}%
\pgfpathlineto{\pgfqpoint{2.417587in}{1.360736in}}%
\pgfpathlineto{\pgfqpoint{2.438794in}{1.396233in}}%
\pgfpathlineto{\pgfqpoint{2.460001in}{1.432435in}}%
\pgfpathlineto{\pgfqpoint{2.470605in}{1.447942in}}%
\pgfpathlineto{\pgfqpoint{2.481208in}{1.460880in}}%
\pgfpathlineto{\pgfqpoint{2.491812in}{1.471242in}}%
\pgfpathlineto{\pgfqpoint{2.513019in}{1.486341in}}%
\pgfpathlineto{\pgfqpoint{2.544829in}{1.506143in}}%
\pgfpathlineto{\pgfqpoint{2.576640in}{1.528615in}}%
\pgfpathlineto{\pgfqpoint{2.587243in}{1.534654in}}%
\pgfpathlineto{\pgfqpoint{2.597847in}{1.538909in}}%
\pgfpathlineto{\pgfqpoint{2.608451in}{1.541052in}}%
\pgfpathlineto{\pgfqpoint{2.619054in}{1.541104in}}%
\pgfpathlineto{\pgfqpoint{2.640261in}{1.536627in}}%
\pgfpathlineto{\pgfqpoint{2.672072in}{1.527765in}}%
\pgfpathlineto{\pgfqpoint{2.714486in}{1.519238in}}%
\pgfpathlineto{\pgfqpoint{2.725089in}{1.515008in}}%
\pgfpathlineto{\pgfqpoint{2.735693in}{1.509069in}}%
\pgfpathlineto{\pgfqpoint{2.746296in}{1.501422in}}%
\pgfpathlineto{\pgfqpoint{2.767503in}{1.482556in}}%
\pgfpathlineto{\pgfqpoint{2.788710in}{1.463267in}}%
\pgfpathlineto{\pgfqpoint{2.809918in}{1.447976in}}%
\pgfpathlineto{\pgfqpoint{2.831125in}{1.437220in}}%
\pgfpathlineto{\pgfqpoint{2.862935in}{1.422762in}}%
\pgfpathlineto{\pgfqpoint{2.915953in}{1.394209in}}%
\pgfpathlineto{\pgfqpoint{2.926556in}{1.391016in}}%
\pgfpathlineto{\pgfqpoint{2.937160in}{1.389509in}}%
\pgfpathlineto{\pgfqpoint{2.947763in}{1.389639in}}%
\pgfpathlineto{\pgfqpoint{2.968970in}{1.393493in}}%
\pgfpathlineto{\pgfqpoint{3.021988in}{1.406300in}}%
\pgfpathlineto{\pgfqpoint{3.043195in}{1.412267in}}%
\pgfpathlineto{\pgfqpoint{3.064402in}{1.422063in}}%
\pgfpathlineto{\pgfqpoint{3.085609in}{1.435787in}}%
\pgfpathlineto{\pgfqpoint{3.117420in}{1.457107in}}%
\pgfpathlineto{\pgfqpoint{3.138627in}{1.467201in}}%
\pgfpathlineto{\pgfqpoint{3.159834in}{1.473682in}}%
\pgfpathlineto{\pgfqpoint{3.202248in}{1.485019in}}%
\pgfpathlineto{\pgfqpoint{3.234059in}{1.493908in}}%
\pgfpathlineto{\pgfqpoint{3.255266in}{1.496030in}}%
\pgfpathlineto{\pgfqpoint{3.276473in}{1.493716in}}%
\pgfpathlineto{\pgfqpoint{3.340094in}{1.480616in}}%
\pgfpathlineto{\pgfqpoint{3.371904in}{1.475804in}}%
\pgfpathlineto{\pgfqpoint{3.393112in}{1.469493in}}%
\pgfpathlineto{\pgfqpoint{3.456733in}{1.444425in}}%
\pgfpathlineto{\pgfqpoint{3.477940in}{1.440626in}}%
\pgfpathlineto{\pgfqpoint{3.520354in}{1.435009in}}%
\pgfpathlineto{\pgfqpoint{3.573371in}{1.424758in}}%
\pgfpathlineto{\pgfqpoint{3.594579in}{1.424664in}}%
\pgfpathlineto{\pgfqpoint{3.615786in}{1.427560in}}%
\pgfpathlineto{\pgfqpoint{3.658200in}{1.434791in}}%
\pgfpathlineto{\pgfqpoint{3.700614in}{1.440224in}}%
\pgfpathlineto{\pgfqpoint{3.721821in}{1.444974in}}%
\pgfpathlineto{\pgfqpoint{3.774838in}{1.459329in}}%
\pgfpathlineto{\pgfqpoint{3.796046in}{1.462275in}}%
\pgfpathlineto{\pgfqpoint{3.849063in}{1.467199in}}%
\pgfpathlineto{\pgfqpoint{3.891477in}{1.473010in}}%
\pgfpathlineto{\pgfqpoint{3.891477in}{1.473010in}}%
\pgfusepath{stroke}%
\end{pgfscope}%
\begin{pgfscope}%
\pgfpathrectangle{\pgfqpoint{0.562500in}{0.407000in}}{\pgfqpoint{3.487500in}{2.849000in}}%
\pgfusepath{clip}%
\pgfsetrectcap%
\pgfsetroundjoin%
\pgfsetlinewidth{1.505625pt}%
\definecolor{currentstroke}{rgb}{0.172549,0.627451,0.172549}%
\pgfsetstrokecolor{currentstroke}%
\pgfsetdash{}{0pt}%
\pgfpathmoveto{\pgfqpoint{0.721023in}{1.351258in}}%
\pgfpathlineto{\pgfqpoint{0.731626in}{1.329601in}}%
\pgfpathlineto{\pgfqpoint{0.742230in}{1.267631in}}%
\pgfpathlineto{\pgfqpoint{0.752833in}{1.173816in}}%
\pgfpathlineto{\pgfqpoint{0.784644in}{0.833211in}}%
\pgfpathlineto{\pgfqpoint{0.795247in}{0.744070in}}%
\pgfpathlineto{\pgfqpoint{0.805851in}{0.682017in}}%
\pgfpathlineto{\pgfqpoint{0.816454in}{0.649019in}}%
\pgfpathlineto{\pgfqpoint{0.827058in}{0.642263in}}%
\pgfpathlineto{\pgfqpoint{0.837662in}{0.655209in}}%
\pgfpathlineto{\pgfqpoint{0.869472in}{0.726751in}}%
\pgfpathlineto{\pgfqpoint{0.880076in}{0.738278in}}%
\pgfpathlineto{\pgfqpoint{0.890679in}{0.738723in}}%
\pgfpathlineto{\pgfqpoint{0.901283in}{0.729711in}}%
\pgfpathlineto{\pgfqpoint{0.933093in}{0.687480in}}%
\pgfpathlineto{\pgfqpoint{0.943697in}{0.682044in}}%
\pgfpathlineto{\pgfqpoint{0.954300in}{0.684198in}}%
\pgfpathlineto{\pgfqpoint{0.964904in}{0.692871in}}%
\pgfpathlineto{\pgfqpoint{0.986111in}{0.718208in}}%
\pgfpathlineto{\pgfqpoint{0.996714in}{0.728080in}}%
\pgfpathlineto{\pgfqpoint{1.007318in}{0.732726in}}%
\pgfpathlineto{\pgfqpoint{1.017921in}{0.731477in}}%
\pgfpathlineto{\pgfqpoint{1.028525in}{0.725355in}}%
\pgfpathlineto{\pgfqpoint{1.049732in}{0.708670in}}%
\pgfpathlineto{\pgfqpoint{1.060336in}{0.704036in}}%
\pgfpathlineto{\pgfqpoint{1.070939in}{0.704784in}}%
\pgfpathlineto{\pgfqpoint{1.081543in}{0.711431in}}%
\pgfpathlineto{\pgfqpoint{1.092146in}{0.722947in}}%
\pgfpathlineto{\pgfqpoint{1.113353in}{0.750893in}}%
\pgfpathlineto{\pgfqpoint{1.123957in}{0.761759in}}%
\pgfpathlineto{\pgfqpoint{1.134560in}{0.767923in}}%
\pgfpathlineto{\pgfqpoint{1.145164in}{0.769109in}}%
\pgfpathlineto{\pgfqpoint{1.176974in}{0.761876in}}%
\pgfpathlineto{\pgfqpoint{1.187578in}{0.765976in}}%
\pgfpathlineto{\pgfqpoint{1.198181in}{0.777552in}}%
\pgfpathlineto{\pgfqpoint{1.208785in}{0.797158in}}%
\pgfpathlineto{\pgfqpoint{1.219388in}{0.823623in}}%
\pgfpathlineto{\pgfqpoint{1.251199in}{0.915997in}}%
\pgfpathlineto{\pgfqpoint{1.272406in}{0.966347in}}%
\pgfpathlineto{\pgfqpoint{1.283010in}{0.990914in}}%
\pgfpathlineto{\pgfqpoint{1.293613in}{1.021932in}}%
\pgfpathlineto{\pgfqpoint{1.304217in}{1.066688in}}%
\pgfpathlineto{\pgfqpoint{1.314820in}{1.132693in}}%
\pgfpathlineto{\pgfqpoint{1.325424in}{1.226093in}}%
\pgfpathlineto{\pgfqpoint{1.336027in}{1.350136in}}%
\pgfpathlineto{\pgfqpoint{1.346631in}{1.504009in}}%
\pgfpathlineto{\pgfqpoint{1.367838in}{1.875198in}}%
\pgfpathlineto{\pgfqpoint{1.389045in}{2.250500in}}%
\pgfpathlineto{\pgfqpoint{1.399648in}{2.403368in}}%
\pgfpathlineto{\pgfqpoint{1.410252in}{2.515795in}}%
\pgfpathlineto{\pgfqpoint{1.420856in}{2.579360in}}%
\pgfpathlineto{\pgfqpoint{1.431459in}{2.590679in}}%
\pgfpathlineto{\pgfqpoint{1.442063in}{2.551705in}}%
\pgfpathlineto{\pgfqpoint{1.452666in}{2.469233in}}%
\pgfpathlineto{\pgfqpoint{1.463270in}{2.353715in}}%
\pgfpathlineto{\pgfqpoint{1.505684in}{1.804349in}}%
\pgfpathlineto{\pgfqpoint{1.516287in}{1.692906in}}%
\pgfpathlineto{\pgfqpoint{1.526891in}{1.600563in}}%
\pgfpathlineto{\pgfqpoint{1.537494in}{1.526323in}}%
\pgfpathlineto{\pgfqpoint{1.548098in}{1.467192in}}%
\pgfpathlineto{\pgfqpoint{1.558701in}{1.419237in}}%
\pgfpathlineto{\pgfqpoint{1.579908in}{1.342160in}}%
\pgfpathlineto{\pgfqpoint{1.601115in}{1.276112in}}%
\pgfpathlineto{\pgfqpoint{1.611719in}{1.246498in}}%
\pgfpathlineto{\pgfqpoint{1.622323in}{1.220437in}}%
\pgfpathlineto{\pgfqpoint{1.632926in}{1.199088in}}%
\pgfpathlineto{\pgfqpoint{1.643530in}{1.183220in}}%
\pgfpathlineto{\pgfqpoint{1.654133in}{1.172945in}}%
\pgfpathlineto{\pgfqpoint{1.664737in}{1.167658in}}%
\pgfpathlineto{\pgfqpoint{1.675340in}{1.166191in}}%
\pgfpathlineto{\pgfqpoint{1.696547in}{1.169045in}}%
\pgfpathlineto{\pgfqpoint{1.749565in}{1.179452in}}%
\pgfpathlineto{\pgfqpoint{1.760168in}{1.185295in}}%
\pgfpathlineto{\pgfqpoint{1.770772in}{1.194242in}}%
\pgfpathlineto{\pgfqpoint{1.781375in}{1.206556in}}%
\pgfpathlineto{\pgfqpoint{1.791979in}{1.222035in}}%
\pgfpathlineto{\pgfqpoint{1.813186in}{1.259872in}}%
\pgfpathlineto{\pgfqpoint{1.876807in}{1.386172in}}%
\pgfpathlineto{\pgfqpoint{1.898014in}{1.433686in}}%
\pgfpathlineto{\pgfqpoint{1.961635in}{1.586397in}}%
\pgfpathlineto{\pgfqpoint{1.972239in}{1.605288in}}%
\pgfpathlineto{\pgfqpoint{1.982842in}{1.620749in}}%
\pgfpathlineto{\pgfqpoint{1.993446in}{1.632818in}}%
\pgfpathlineto{\pgfqpoint{2.004049in}{1.641773in}}%
\pgfpathlineto{\pgfqpoint{2.014653in}{1.648014in}}%
\pgfpathlineto{\pgfqpoint{2.025257in}{1.651932in}}%
\pgfpathlineto{\pgfqpoint{2.035860in}{1.653807in}}%
\pgfpathlineto{\pgfqpoint{2.046464in}{1.653748in}}%
\pgfpathlineto{\pgfqpoint{2.057067in}{1.651690in}}%
\pgfpathlineto{\pgfqpoint{2.067671in}{1.647454in}}%
\pgfpathlineto{\pgfqpoint{2.078274in}{1.640829in}}%
\pgfpathlineto{\pgfqpoint{2.088878in}{1.631679in}}%
\pgfpathlineto{\pgfqpoint{2.099481in}{1.620024in}}%
\pgfpathlineto{\pgfqpoint{2.110085in}{1.606082in}}%
\pgfpathlineto{\pgfqpoint{2.131292in}{1.573121in}}%
\pgfpathlineto{\pgfqpoint{2.184309in}{1.485843in}}%
\pgfpathlineto{\pgfqpoint{2.216120in}{1.439454in}}%
\pgfpathlineto{\pgfqpoint{2.247931in}{1.396826in}}%
\pgfpathlineto{\pgfqpoint{2.269138in}{1.372732in}}%
\pgfpathlineto{\pgfqpoint{2.279741in}{1.362911in}}%
\pgfpathlineto{\pgfqpoint{2.290345in}{1.354867in}}%
\pgfpathlineto{\pgfqpoint{2.300948in}{1.348649in}}%
\pgfpathlineto{\pgfqpoint{2.311552in}{1.344159in}}%
\pgfpathlineto{\pgfqpoint{2.322155in}{1.341178in}}%
\pgfpathlineto{\pgfqpoint{2.343362in}{1.338617in}}%
\pgfpathlineto{\pgfqpoint{2.364569in}{1.339135in}}%
\pgfpathlineto{\pgfqpoint{2.385776in}{1.342467in}}%
\pgfpathlineto{\pgfqpoint{2.396380in}{1.345544in}}%
\pgfpathlineto{\pgfqpoint{2.406984in}{1.349817in}}%
\pgfpathlineto{\pgfqpoint{2.417587in}{1.355414in}}%
\pgfpathlineto{\pgfqpoint{2.438794in}{1.370536in}}%
\pgfpathlineto{\pgfqpoint{2.460001in}{1.389685in}}%
\pgfpathlineto{\pgfqpoint{2.523622in}{1.450870in}}%
\pgfpathlineto{\pgfqpoint{2.566036in}{1.488199in}}%
\pgfpathlineto{\pgfqpoint{2.587243in}{1.504713in}}%
\pgfpathlineto{\pgfqpoint{2.608451in}{1.517814in}}%
\pgfpathlineto{\pgfqpoint{2.629658in}{1.526326in}}%
\pgfpathlineto{\pgfqpoint{2.650865in}{1.530222in}}%
\pgfpathlineto{\pgfqpoint{2.672072in}{1.530338in}}%
\pgfpathlineto{\pgfqpoint{2.693279in}{1.527514in}}%
\pgfpathlineto{\pgfqpoint{2.714486in}{1.522026in}}%
\pgfpathlineto{\pgfqpoint{2.735693in}{1.513818in}}%
\pgfpathlineto{\pgfqpoint{2.756900in}{1.503186in}}%
\pgfpathlineto{\pgfqpoint{2.894746in}{1.426605in}}%
\pgfpathlineto{\pgfqpoint{2.915953in}{1.417345in}}%
\pgfpathlineto{\pgfqpoint{2.937160in}{1.411130in}}%
\pgfpathlineto{\pgfqpoint{2.958367in}{1.408023in}}%
\pgfpathlineto{\pgfqpoint{2.990177in}{1.406877in}}%
\pgfpathlineto{\pgfqpoint{3.021988in}{1.407380in}}%
\pgfpathlineto{\pgfqpoint{3.043195in}{1.409304in}}%
\pgfpathlineto{\pgfqpoint{3.064402in}{1.413760in}}%
\pgfpathlineto{\pgfqpoint{3.085609in}{1.421013in}}%
\pgfpathlineto{\pgfqpoint{3.159834in}{1.450886in}}%
\pgfpathlineto{\pgfqpoint{3.255266in}{1.481066in}}%
\pgfpathlineto{\pgfqpoint{3.276473in}{1.485139in}}%
\pgfpathlineto{\pgfqpoint{3.297680in}{1.486919in}}%
\pgfpathlineto{\pgfqpoint{3.329490in}{1.486262in}}%
\pgfpathlineto{\pgfqpoint{3.361301in}{1.483047in}}%
\pgfpathlineto{\pgfqpoint{3.393112in}{1.477329in}}%
\pgfpathlineto{\pgfqpoint{3.435526in}{1.466451in}}%
\pgfpathlineto{\pgfqpoint{3.488543in}{1.453324in}}%
\pgfpathlineto{\pgfqpoint{3.552164in}{1.439605in}}%
\pgfpathlineto{\pgfqpoint{3.583975in}{1.435232in}}%
\pgfpathlineto{\pgfqpoint{3.615786in}{1.433784in}}%
\pgfpathlineto{\pgfqpoint{3.690010in}{1.434108in}}%
\pgfpathlineto{\pgfqpoint{3.721821in}{1.437223in}}%
\pgfpathlineto{\pgfqpoint{3.891477in}{1.463874in}}%
\pgfpathlineto{\pgfqpoint{3.891477in}{1.463874in}}%
\pgfusepath{stroke}%
\end{pgfscope}%
\begin{pgfscope}%
\pgfpathrectangle{\pgfqpoint{0.562500in}{0.407000in}}{\pgfqpoint{3.487500in}{2.849000in}}%
\pgfusepath{clip}%
\pgfsetrectcap%
\pgfsetroundjoin%
\pgfsetlinewidth{1.505625pt}%
\definecolor{currentstroke}{rgb}{0.839216,0.152941,0.156863}%
\pgfsetstrokecolor{currentstroke}%
\pgfsetdash{}{0pt}%
\pgfpathmoveto{\pgfqpoint{0.721023in}{1.237547in}}%
\pgfpathlineto{\pgfqpoint{0.731626in}{1.221030in}}%
\pgfpathlineto{\pgfqpoint{0.742230in}{1.173643in}}%
\pgfpathlineto{\pgfqpoint{0.752833in}{1.101523in}}%
\pgfpathlineto{\pgfqpoint{0.795247in}{0.759656in}}%
\pgfpathlineto{\pgfqpoint{0.805851in}{0.705525in}}%
\pgfpathlineto{\pgfqpoint{0.816454in}{0.673094in}}%
\pgfpathlineto{\pgfqpoint{0.827058in}{0.661039in}}%
\pgfpathlineto{\pgfqpoint{0.837662in}{0.665214in}}%
\pgfpathlineto{\pgfqpoint{0.848265in}{0.679736in}}%
\pgfpathlineto{\pgfqpoint{0.869472in}{0.715208in}}%
\pgfpathlineto{\pgfqpoint{0.880076in}{0.726655in}}%
\pgfpathlineto{\pgfqpoint{0.890679in}{0.730831in}}%
\pgfpathlineto{\pgfqpoint{0.901283in}{0.728095in}}%
\pgfpathlineto{\pgfqpoint{0.911886in}{0.720475in}}%
\pgfpathlineto{\pgfqpoint{0.933093in}{0.702456in}}%
\pgfpathlineto{\pgfqpoint{0.943697in}{0.697424in}}%
\pgfpathlineto{\pgfqpoint{0.954300in}{0.696990in}}%
\pgfpathlineto{\pgfqpoint{0.964904in}{0.700994in}}%
\pgfpathlineto{\pgfqpoint{0.986111in}{0.716424in}}%
\pgfpathlineto{\pgfqpoint{0.996714in}{0.723791in}}%
\pgfpathlineto{\pgfqpoint{1.007318in}{0.728630in}}%
\pgfpathlineto{\pgfqpoint{1.017921in}{0.730207in}}%
\pgfpathlineto{\pgfqpoint{1.028525in}{0.728804in}}%
\pgfpathlineto{\pgfqpoint{1.049732in}{0.722270in}}%
\pgfpathlineto{\pgfqpoint{1.060336in}{0.720590in}}%
\pgfpathlineto{\pgfqpoint{1.070939in}{0.721879in}}%
\pgfpathlineto{\pgfqpoint{1.081543in}{0.726696in}}%
\pgfpathlineto{\pgfqpoint{1.092146in}{0.734704in}}%
\pgfpathlineto{\pgfqpoint{1.123957in}{0.764965in}}%
\pgfpathlineto{\pgfqpoint{1.134560in}{0.772388in}}%
\pgfpathlineto{\pgfqpoint{1.145164in}{0.777352in}}%
\pgfpathlineto{\pgfqpoint{1.176974in}{0.787164in}}%
\pgfpathlineto{\pgfqpoint{1.187578in}{0.794517in}}%
\pgfpathlineto{\pgfqpoint{1.198181in}{0.806587in}}%
\pgfpathlineto{\pgfqpoint{1.208785in}{0.823992in}}%
\pgfpathlineto{\pgfqpoint{1.219388in}{0.846419in}}%
\pgfpathlineto{\pgfqpoint{1.240596in}{0.901540in}}%
\pgfpathlineto{\pgfqpoint{1.272406in}{0.993621in}}%
\pgfpathlineto{\pgfqpoint{1.283010in}{1.028834in}}%
\pgfpathlineto{\pgfqpoint{1.293613in}{1.071055in}}%
\pgfpathlineto{\pgfqpoint{1.304217in}{1.124670in}}%
\pgfpathlineto{\pgfqpoint{1.314820in}{1.194116in}}%
\pgfpathlineto{\pgfqpoint{1.325424in}{1.282915in}}%
\pgfpathlineto{\pgfqpoint{1.336027in}{1.392726in}}%
\pgfpathlineto{\pgfqpoint{1.346631in}{1.522601in}}%
\pgfpathlineto{\pgfqpoint{1.367838in}{1.824001in}}%
\pgfpathlineto{\pgfqpoint{1.389045in}{2.125353in}}%
\pgfpathlineto{\pgfqpoint{1.399648in}{2.250694in}}%
\pgfpathlineto{\pgfqpoint{1.410252in}{2.346712in}}%
\pgfpathlineto{\pgfqpoint{1.420856in}{2.406893in}}%
\pgfpathlineto{\pgfqpoint{1.431459in}{2.428003in}}%
\pgfpathlineto{\pgfqpoint{1.442063in}{2.410421in}}%
\pgfpathlineto{\pgfqpoint{1.452666in}{2.357953in}}%
\pgfpathlineto{\pgfqpoint{1.463270in}{2.277183in}}%
\pgfpathlineto{\pgfqpoint{1.473873in}{2.176478in}}%
\pgfpathlineto{\pgfqpoint{1.516287in}{1.741569in}}%
\pgfpathlineto{\pgfqpoint{1.526891in}{1.654342in}}%
\pgfpathlineto{\pgfqpoint{1.537494in}{1.580359in}}%
\pgfpathlineto{\pgfqpoint{1.548098in}{1.518696in}}%
\pgfpathlineto{\pgfqpoint{1.558701in}{1.467379in}}%
\pgfpathlineto{\pgfqpoint{1.569305in}{1.424029in}}%
\pgfpathlineto{\pgfqpoint{1.590512in}{1.352857in}}%
\pgfpathlineto{\pgfqpoint{1.611719in}{1.294637in}}%
\pgfpathlineto{\pgfqpoint{1.622323in}{1.269886in}}%
\pgfpathlineto{\pgfqpoint{1.632926in}{1.248549in}}%
\pgfpathlineto{\pgfqpoint{1.643530in}{1.231037in}}%
\pgfpathlineto{\pgfqpoint{1.654133in}{1.217528in}}%
\pgfpathlineto{\pgfqpoint{1.664737in}{1.207865in}}%
\pgfpathlineto{\pgfqpoint{1.675340in}{1.201569in}}%
\pgfpathlineto{\pgfqpoint{1.685944in}{1.197943in}}%
\pgfpathlineto{\pgfqpoint{1.696547in}{1.196230in}}%
\pgfpathlineto{\pgfqpoint{1.717754in}{1.196185in}}%
\pgfpathlineto{\pgfqpoint{1.738961in}{1.199347in}}%
\pgfpathlineto{\pgfqpoint{1.749565in}{1.202622in}}%
\pgfpathlineto{\pgfqpoint{1.760168in}{1.207576in}}%
\pgfpathlineto{\pgfqpoint{1.770772in}{1.214582in}}%
\pgfpathlineto{\pgfqpoint{1.781375in}{1.223859in}}%
\pgfpathlineto{\pgfqpoint{1.791979in}{1.235413in}}%
\pgfpathlineto{\pgfqpoint{1.813186in}{1.264387in}}%
\pgfpathlineto{\pgfqpoint{1.834393in}{1.298558in}}%
\pgfpathlineto{\pgfqpoint{1.866204in}{1.354196in}}%
\pgfpathlineto{\pgfqpoint{1.898014in}{1.414253in}}%
\pgfpathlineto{\pgfqpoint{1.961635in}{1.539666in}}%
\pgfpathlineto{\pgfqpoint{1.982842in}{1.573264in}}%
\pgfpathlineto{\pgfqpoint{1.993446in}{1.587217in}}%
\pgfpathlineto{\pgfqpoint{2.004049in}{1.599173in}}%
\pgfpathlineto{\pgfqpoint{2.014653in}{1.609125in}}%
\pgfpathlineto{\pgfqpoint{2.025257in}{1.617071in}}%
\pgfpathlineto{\pgfqpoint{2.035860in}{1.622980in}}%
\pgfpathlineto{\pgfqpoint{2.046464in}{1.626784in}}%
\pgfpathlineto{\pgfqpoint{2.057067in}{1.628384in}}%
\pgfpathlineto{\pgfqpoint{2.067671in}{1.627683in}}%
\pgfpathlineto{\pgfqpoint{2.078274in}{1.624623in}}%
\pgfpathlineto{\pgfqpoint{2.088878in}{1.619226in}}%
\pgfpathlineto{\pgfqpoint{2.099481in}{1.611619in}}%
\pgfpathlineto{\pgfqpoint{2.110085in}{1.602043in}}%
\pgfpathlineto{\pgfqpoint{2.131292in}{1.578370in}}%
\pgfpathlineto{\pgfqpoint{2.163102in}{1.537212in}}%
\pgfpathlineto{\pgfqpoint{2.237327in}{1.439736in}}%
\pgfpathlineto{\pgfqpoint{2.258534in}{1.414171in}}%
\pgfpathlineto{\pgfqpoint{2.279741in}{1.392435in}}%
\pgfpathlineto{\pgfqpoint{2.300948in}{1.376025in}}%
\pgfpathlineto{\pgfqpoint{2.311552in}{1.369923in}}%
\pgfpathlineto{\pgfqpoint{2.332759in}{1.361423in}}%
\pgfpathlineto{\pgfqpoint{2.353966in}{1.356840in}}%
\pgfpathlineto{\pgfqpoint{2.375173in}{1.355384in}}%
\pgfpathlineto{\pgfqpoint{2.396380in}{1.357160in}}%
\pgfpathlineto{\pgfqpoint{2.417587in}{1.362727in}}%
\pgfpathlineto{\pgfqpoint{2.438794in}{1.372215in}}%
\pgfpathlineto{\pgfqpoint{2.460001in}{1.384859in}}%
\pgfpathlineto{\pgfqpoint{2.491812in}{1.406938in}}%
\pgfpathlineto{\pgfqpoint{2.544829in}{1.446720in}}%
\pgfpathlineto{\pgfqpoint{2.587243in}{1.479381in}}%
\pgfpathlineto{\pgfqpoint{2.608451in}{1.493224in}}%
\pgfpathlineto{\pgfqpoint{2.629658in}{1.503646in}}%
\pgfpathlineto{\pgfqpoint{2.650865in}{1.510304in}}%
\pgfpathlineto{\pgfqpoint{2.672072in}{1.513775in}}%
\pgfpathlineto{\pgfqpoint{2.693279in}{1.514963in}}%
\pgfpathlineto{\pgfqpoint{2.714486in}{1.514407in}}%
\pgfpathlineto{\pgfqpoint{2.735693in}{1.512023in}}%
\pgfpathlineto{\pgfqpoint{2.756900in}{1.507398in}}%
\pgfpathlineto{\pgfqpoint{2.778107in}{1.500306in}}%
\pgfpathlineto{\pgfqpoint{2.799314in}{1.491031in}}%
\pgfpathlineto{\pgfqpoint{2.841728in}{1.469043in}}%
\pgfpathlineto{\pgfqpoint{2.884142in}{1.447788in}}%
\pgfpathlineto{\pgfqpoint{2.915953in}{1.434768in}}%
\pgfpathlineto{\pgfqpoint{2.937160in}{1.427952in}}%
\pgfpathlineto{\pgfqpoint{2.958367in}{1.422731in}}%
\pgfpathlineto{\pgfqpoint{2.979574in}{1.419091in}}%
\pgfpathlineto{\pgfqpoint{3.011385in}{1.416474in}}%
\pgfpathlineto{\pgfqpoint{3.043195in}{1.417013in}}%
\pgfpathlineto{\pgfqpoint{3.075006in}{1.420153in}}%
\pgfpathlineto{\pgfqpoint{3.106816in}{1.425242in}}%
\pgfpathlineto{\pgfqpoint{3.138627in}{1.432182in}}%
\pgfpathlineto{\pgfqpoint{3.170437in}{1.441187in}}%
\pgfpathlineto{\pgfqpoint{3.244662in}{1.463634in}}%
\pgfpathlineto{\pgfqpoint{3.276473in}{1.470282in}}%
\pgfpathlineto{\pgfqpoint{3.318887in}{1.476471in}}%
\pgfpathlineto{\pgfqpoint{3.350697in}{1.478734in}}%
\pgfpathlineto{\pgfqpoint{3.382508in}{1.478072in}}%
\pgfpathlineto{\pgfqpoint{3.424922in}{1.473869in}}%
\pgfpathlineto{\pgfqpoint{3.477940in}{1.466260in}}%
\pgfpathlineto{\pgfqpoint{3.520354in}{1.457119in}}%
\pgfpathlineto{\pgfqpoint{3.562768in}{1.448137in}}%
\pgfpathlineto{\pgfqpoint{3.594579in}{1.443921in}}%
\pgfpathlineto{\pgfqpoint{3.668803in}{1.437215in}}%
\pgfpathlineto{\pgfqpoint{3.700614in}{1.436602in}}%
\pgfpathlineto{\pgfqpoint{3.732424in}{1.438698in}}%
\pgfpathlineto{\pgfqpoint{3.796046in}{1.443667in}}%
\pgfpathlineto{\pgfqpoint{3.827856in}{1.446451in}}%
\pgfpathlineto{\pgfqpoint{3.870270in}{1.453124in}}%
\pgfpathlineto{\pgfqpoint{3.891477in}{1.456495in}}%
\pgfpathlineto{\pgfqpoint{3.891477in}{1.456495in}}%
\pgfusepath{stroke}%
\end{pgfscope}%
\begin{pgfscope}%
\pgfpathrectangle{\pgfqpoint{0.562500in}{0.407000in}}{\pgfqpoint{3.487500in}{2.849000in}}%
\pgfusepath{clip}%
\pgfsetrectcap%
\pgfsetroundjoin%
\pgfsetlinewidth{1.505625pt}%
\definecolor{currentstroke}{rgb}{0.580392,0.403922,0.741176}%
\pgfsetstrokecolor{currentstroke}%
\pgfsetdash{}{0pt}%
\pgfpathmoveto{\pgfqpoint{0.721023in}{1.175444in}}%
\pgfpathlineto{\pgfqpoint{0.731626in}{1.162007in}}%
\pgfpathlineto{\pgfqpoint{0.742230in}{1.123334in}}%
\pgfpathlineto{\pgfqpoint{0.752833in}{1.064084in}}%
\pgfpathlineto{\pgfqpoint{0.774040in}{0.912950in}}%
\pgfpathlineto{\pgfqpoint{0.784644in}{0.837507in}}%
\pgfpathlineto{\pgfqpoint{0.795247in}{0.771997in}}%
\pgfpathlineto{\pgfqpoint{0.805851in}{0.721470in}}%
\pgfpathlineto{\pgfqpoint{0.816454in}{0.688388in}}%
\pgfpathlineto{\pgfqpoint{0.827058in}{0.672539in}}%
\pgfpathlineto{\pgfqpoint{0.837662in}{0.671390in}}%
\pgfpathlineto{\pgfqpoint{0.848265in}{0.680779in}}%
\pgfpathlineto{\pgfqpoint{0.880076in}{0.724843in}}%
\pgfpathlineto{\pgfqpoint{0.890679in}{0.732742in}}%
\pgfpathlineto{\pgfqpoint{0.901283in}{0.734740in}}%
\pgfpathlineto{\pgfqpoint{0.911886in}{0.731591in}}%
\pgfpathlineto{\pgfqpoint{0.943697in}{0.711133in}}%
\pgfpathlineto{\pgfqpoint{0.954300in}{0.707486in}}%
\pgfpathlineto{\pgfqpoint{0.964904in}{0.707373in}}%
\pgfpathlineto{\pgfqpoint{0.975507in}{0.710650in}}%
\pgfpathlineto{\pgfqpoint{0.996714in}{0.723317in}}%
\pgfpathlineto{\pgfqpoint{1.007318in}{0.729945in}}%
\pgfpathlineto{\pgfqpoint{1.017921in}{0.735169in}}%
\pgfpathlineto{\pgfqpoint{1.028525in}{0.738398in}}%
\pgfpathlineto{\pgfqpoint{1.039129in}{0.739665in}}%
\pgfpathlineto{\pgfqpoint{1.070939in}{0.739211in}}%
\pgfpathlineto{\pgfqpoint{1.081543in}{0.740888in}}%
\pgfpathlineto{\pgfqpoint{1.092146in}{0.744608in}}%
\pgfpathlineto{\pgfqpoint{1.102750in}{0.750423in}}%
\pgfpathlineto{\pgfqpoint{1.123957in}{0.766624in}}%
\pgfpathlineto{\pgfqpoint{1.187578in}{0.821929in}}%
\pgfpathlineto{\pgfqpoint{1.198181in}{0.834734in}}%
\pgfpathlineto{\pgfqpoint{1.208785in}{0.850474in}}%
\pgfpathlineto{\pgfqpoint{1.219388in}{0.869601in}}%
\pgfpathlineto{\pgfqpoint{1.229992in}{0.892279in}}%
\pgfpathlineto{\pgfqpoint{1.240596in}{0.918467in}}%
\pgfpathlineto{\pgfqpoint{1.251199in}{0.948111in}}%
\pgfpathlineto{\pgfqpoint{1.261803in}{0.981374in}}%
\pgfpathlineto{\pgfqpoint{1.272406in}{1.018854in}}%
\pgfpathlineto{\pgfqpoint{1.283010in}{1.061712in}}%
\pgfpathlineto{\pgfqpoint{1.293613in}{1.111654in}}%
\pgfpathlineto{\pgfqpoint{1.304217in}{1.170740in}}%
\pgfpathlineto{\pgfqpoint{1.314820in}{1.241028in}}%
\pgfpathlineto{\pgfqpoint{1.325424in}{1.324099in}}%
\pgfpathlineto{\pgfqpoint{1.336027in}{1.420557in}}%
\pgfpathlineto{\pgfqpoint{1.357234in}{1.648660in}}%
\pgfpathlineto{\pgfqpoint{1.389045in}{2.016666in}}%
\pgfpathlineto{\pgfqpoint{1.399648in}{2.121092in}}%
\pgfpathlineto{\pgfqpoint{1.410252in}{2.205167in}}%
\pgfpathlineto{\pgfqpoint{1.420856in}{2.263606in}}%
\pgfpathlineto{\pgfqpoint{1.431459in}{2.293048in}}%
\pgfpathlineto{\pgfqpoint{1.442063in}{2.292459in}}%
\pgfpathlineto{\pgfqpoint{1.452666in}{2.263235in}}%
\pgfpathlineto{\pgfqpoint{1.463270in}{2.208985in}}%
\pgfpathlineto{\pgfqpoint{1.473873in}{2.135039in}}%
\pgfpathlineto{\pgfqpoint{1.495080in}{1.953782in}}%
\pgfpathlineto{\pgfqpoint{1.516287in}{1.769281in}}%
\pgfpathlineto{\pgfqpoint{1.526891in}{1.687468in}}%
\pgfpathlineto{\pgfqpoint{1.537494in}{1.615845in}}%
\pgfpathlineto{\pgfqpoint{1.548098in}{1.554932in}}%
\pgfpathlineto{\pgfqpoint{1.558701in}{1.504036in}}%
\pgfpathlineto{\pgfqpoint{1.569305in}{1.461658in}}%
\pgfpathlineto{\pgfqpoint{1.579908in}{1.425940in}}%
\pgfpathlineto{\pgfqpoint{1.601115in}{1.367543in}}%
\pgfpathlineto{\pgfqpoint{1.622323in}{1.319209in}}%
\pgfpathlineto{\pgfqpoint{1.643530in}{1.278879in}}%
\pgfpathlineto{\pgfqpoint{1.654133in}{1.262397in}}%
\pgfpathlineto{\pgfqpoint{1.664737in}{1.248802in}}%
\pgfpathlineto{\pgfqpoint{1.675340in}{1.238225in}}%
\pgfpathlineto{\pgfqpoint{1.685944in}{1.230579in}}%
\pgfpathlineto{\pgfqpoint{1.696547in}{1.225582in}}%
\pgfpathlineto{\pgfqpoint{1.707151in}{1.222831in}}%
\pgfpathlineto{\pgfqpoint{1.717754in}{1.221899in}}%
\pgfpathlineto{\pgfqpoint{1.728358in}{1.222420in}}%
\pgfpathlineto{\pgfqpoint{1.749565in}{1.227016in}}%
\pgfpathlineto{\pgfqpoint{1.760168in}{1.231045in}}%
\pgfpathlineto{\pgfqpoint{1.770772in}{1.236367in}}%
\pgfpathlineto{\pgfqpoint{1.781375in}{1.243128in}}%
\pgfpathlineto{\pgfqpoint{1.791979in}{1.251427in}}%
\pgfpathlineto{\pgfqpoint{1.813186in}{1.272603in}}%
\pgfpathlineto{\pgfqpoint{1.834393in}{1.298909in}}%
\pgfpathlineto{\pgfqpoint{1.866204in}{1.344318in}}%
\pgfpathlineto{\pgfqpoint{1.898014in}{1.394268in}}%
\pgfpathlineto{\pgfqpoint{1.940428in}{1.466826in}}%
\pgfpathlineto{\pgfqpoint{1.972239in}{1.519424in}}%
\pgfpathlineto{\pgfqpoint{1.993446in}{1.548989in}}%
\pgfpathlineto{\pgfqpoint{2.014653in}{1.572282in}}%
\pgfpathlineto{\pgfqpoint{2.025257in}{1.581461in}}%
\pgfpathlineto{\pgfqpoint{2.035860in}{1.589026in}}%
\pgfpathlineto{\pgfqpoint{2.046464in}{1.595007in}}%
\pgfpathlineto{\pgfqpoint{2.057067in}{1.599405in}}%
\pgfpathlineto{\pgfqpoint{2.067671in}{1.602181in}}%
\pgfpathlineto{\pgfqpoint{2.078274in}{1.603264in}}%
\pgfpathlineto{\pgfqpoint{2.088878in}{1.602566in}}%
\pgfpathlineto{\pgfqpoint{2.099481in}{1.600012in}}%
\pgfpathlineto{\pgfqpoint{2.110085in}{1.595573in}}%
\pgfpathlineto{\pgfqpoint{2.120688in}{1.589288in}}%
\pgfpathlineto{\pgfqpoint{2.131292in}{1.581281in}}%
\pgfpathlineto{\pgfqpoint{2.152499in}{1.560979in}}%
\pgfpathlineto{\pgfqpoint{2.184309in}{1.524232in}}%
\pgfpathlineto{\pgfqpoint{2.226724in}{1.474108in}}%
\pgfpathlineto{\pgfqpoint{2.258534in}{1.440074in}}%
\pgfpathlineto{\pgfqpoint{2.279741in}{1.420216in}}%
\pgfpathlineto{\pgfqpoint{2.300948in}{1.403536in}}%
\pgfpathlineto{\pgfqpoint{2.322155in}{1.390639in}}%
\pgfpathlineto{\pgfqpoint{2.343362in}{1.381557in}}%
\pgfpathlineto{\pgfqpoint{2.364569in}{1.375789in}}%
\pgfpathlineto{\pgfqpoint{2.385776in}{1.372733in}}%
\pgfpathlineto{\pgfqpoint{2.406984in}{1.372162in}}%
\pgfpathlineto{\pgfqpoint{2.428191in}{1.374324in}}%
\pgfpathlineto{\pgfqpoint{2.449398in}{1.379587in}}%
\pgfpathlineto{\pgfqpoint{2.470605in}{1.387955in}}%
\pgfpathlineto{\pgfqpoint{2.491812in}{1.398839in}}%
\pgfpathlineto{\pgfqpoint{2.534226in}{1.424285in}}%
\pgfpathlineto{\pgfqpoint{2.597847in}{1.462621in}}%
\pgfpathlineto{\pgfqpoint{2.629658in}{1.479887in}}%
\pgfpathlineto{\pgfqpoint{2.650865in}{1.489605in}}%
\pgfpathlineto{\pgfqpoint{2.672072in}{1.497187in}}%
\pgfpathlineto{\pgfqpoint{2.693279in}{1.502240in}}%
\pgfpathlineto{\pgfqpoint{2.714486in}{1.504674in}}%
\pgfpathlineto{\pgfqpoint{2.735693in}{1.504680in}}%
\pgfpathlineto{\pgfqpoint{2.756900in}{1.502649in}}%
\pgfpathlineto{\pgfqpoint{2.788710in}{1.496761in}}%
\pgfpathlineto{\pgfqpoint{2.820521in}{1.488346in}}%
\pgfpathlineto{\pgfqpoint{2.852332in}{1.477572in}}%
\pgfpathlineto{\pgfqpoint{2.894746in}{1.460186in}}%
\pgfpathlineto{\pgfqpoint{2.926556in}{1.447366in}}%
\pgfpathlineto{\pgfqpoint{2.958367in}{1.437320in}}%
\pgfpathlineto{\pgfqpoint{2.990177in}{1.430416in}}%
\pgfpathlineto{\pgfqpoint{3.032592in}{1.424279in}}%
\pgfpathlineto{\pgfqpoint{3.064402in}{1.422099in}}%
\pgfpathlineto{\pgfqpoint{3.085609in}{1.422484in}}%
\pgfpathlineto{\pgfqpoint{3.117420in}{1.425891in}}%
\pgfpathlineto{\pgfqpoint{3.170437in}{1.435083in}}%
\pgfpathlineto{\pgfqpoint{3.223455in}{1.445966in}}%
\pgfpathlineto{\pgfqpoint{3.308283in}{1.465550in}}%
\pgfpathlineto{\pgfqpoint{3.340094in}{1.469499in}}%
\pgfpathlineto{\pgfqpoint{3.382508in}{1.472334in}}%
\pgfpathlineto{\pgfqpoint{3.424922in}{1.473105in}}%
\pgfpathlineto{\pgfqpoint{3.456733in}{1.471679in}}%
\pgfpathlineto{\pgfqpoint{3.499147in}{1.467084in}}%
\pgfpathlineto{\pgfqpoint{3.668803in}{1.444357in}}%
\pgfpathlineto{\pgfqpoint{3.721821in}{1.441216in}}%
\pgfpathlineto{\pgfqpoint{3.764235in}{1.440826in}}%
\pgfpathlineto{\pgfqpoint{3.817253in}{1.442808in}}%
\pgfpathlineto{\pgfqpoint{3.870270in}{1.447552in}}%
\pgfpathlineto{\pgfqpoint{3.891477in}{1.449794in}}%
\pgfpathlineto{\pgfqpoint{3.891477in}{1.449794in}}%
\pgfusepath{stroke}%
\end{pgfscope}%
\begin{pgfscope}%
\pgfsetrectcap%
\pgfsetmiterjoin%
\pgfsetlinewidth{0.803000pt}%
\definecolor{currentstroke}{rgb}{0.000000,0.000000,0.000000}%
\pgfsetstrokecolor{currentstroke}%
\pgfsetdash{}{0pt}%
\pgfpathmoveto{\pgfqpoint{0.562500in}{0.407000in}}%
\pgfpathlineto{\pgfqpoint{0.562500in}{3.256000in}}%
\pgfusepath{stroke}%
\end{pgfscope}%
\begin{pgfscope}%
\pgfsetrectcap%
\pgfsetmiterjoin%
\pgfsetlinewidth{0.803000pt}%
\definecolor{currentstroke}{rgb}{0.000000,0.000000,0.000000}%
\pgfsetstrokecolor{currentstroke}%
\pgfsetdash{}{0pt}%
\pgfpathmoveto{\pgfqpoint{4.050000in}{0.407000in}}%
\pgfpathlineto{\pgfqpoint{4.050000in}{3.256000in}}%
\pgfusepath{stroke}%
\end{pgfscope}%
\begin{pgfscope}%
\pgfsetrectcap%
\pgfsetmiterjoin%
\pgfsetlinewidth{0.803000pt}%
\definecolor{currentstroke}{rgb}{0.000000,0.000000,0.000000}%
\pgfsetstrokecolor{currentstroke}%
\pgfsetdash{}{0pt}%
\pgfpathmoveto{\pgfqpoint{0.562500in}{0.407000in}}%
\pgfpathlineto{\pgfqpoint{4.050000in}{0.407000in}}%
\pgfusepath{stroke}%
\end{pgfscope}%
\begin{pgfscope}%
\pgfsetrectcap%
\pgfsetmiterjoin%
\pgfsetlinewidth{0.803000pt}%
\definecolor{currentstroke}{rgb}{0.000000,0.000000,0.000000}%
\pgfsetstrokecolor{currentstroke}%
\pgfsetdash{}{0pt}%
\pgfpathmoveto{\pgfqpoint{0.562500in}{3.256000in}}%
\pgfpathlineto{\pgfqpoint{4.050000in}{3.256000in}}%
\pgfusepath{stroke}%
\end{pgfscope}%
\begin{pgfscope}%
\pgfsetbuttcap%
\pgfsetmiterjoin%
\definecolor{currentfill}{rgb}{1.000000,1.000000,1.000000}%
\pgfsetfillcolor{currentfill}%
\pgfsetfillopacity{0.800000}%
\pgfsetlinewidth{1.003750pt}%
\definecolor{currentstroke}{rgb}{0.800000,0.800000,0.800000}%
\pgfsetstrokecolor{currentstroke}%
\pgfsetstrokeopacity{0.800000}%
\pgfsetdash{}{0pt}%
\pgfpathmoveto{\pgfqpoint{2.659748in}{2.103392in}}%
\pgfpathlineto{\pgfqpoint{3.952778in}{2.103392in}}%
\pgfpathquadraticcurveto{\pgfqpoint{3.980556in}{2.103392in}}{\pgfqpoint{3.980556in}{2.131169in}}%
\pgfpathlineto{\pgfqpoint{3.980556in}{3.158778in}}%
\pgfpathquadraticcurveto{\pgfqpoint{3.980556in}{3.186556in}}{\pgfqpoint{3.952778in}{3.186556in}}%
\pgfpathlineto{\pgfqpoint{2.659748in}{3.186556in}}%
\pgfpathquadraticcurveto{\pgfqpoint{2.631971in}{3.186556in}}{\pgfqpoint{2.631971in}{3.158778in}}%
\pgfpathlineto{\pgfqpoint{2.631971in}{2.131169in}}%
\pgfpathquadraticcurveto{\pgfqpoint{2.631971in}{2.103392in}}{\pgfqpoint{2.659748in}{2.103392in}}%
\pgfpathclose%
\pgfusepath{stroke,fill}%
\end{pgfscope}%
\begin{pgfscope}%
\pgfsetrectcap%
\pgfsetroundjoin%
\pgfsetlinewidth{1.505625pt}%
\definecolor{currentstroke}{rgb}{0.121569,0.466667,0.705882}%
\pgfsetstrokecolor{currentstroke}%
\pgfsetdash{}{0pt}%
\pgfpathmoveto{\pgfqpoint{2.687526in}{3.075470in}}%
\pgfpathlineto{\pgfqpoint{2.965304in}{3.075470in}}%
\pgfusepath{stroke}%
\end{pgfscope}%
\begin{pgfscope}%
\definecolor{textcolor}{rgb}{0.000000,0.000000,0.000000}%
\pgfsetstrokecolor{textcolor}%
\pgfsetfillcolor{textcolor}%
\pgftext[x=3.076415in,y=3.026859in,left,base]{\color{textcolor}\rmfamily\fontsize{10.000000}{12.000000}\selectfont T = 47.2 [\si{K}]}%
\end{pgfscope}%
\begin{pgfscope}%
\pgfsetrectcap%
\pgfsetroundjoin%
\pgfsetlinewidth{1.505625pt}%
\definecolor{currentstroke}{rgb}{1.000000,0.498039,0.054902}%
\pgfsetstrokecolor{currentstroke}%
\pgfsetdash{}{0pt}%
\pgfpathmoveto{\pgfqpoint{2.687526in}{2.867170in}}%
\pgfpathlineto{\pgfqpoint{2.965304in}{2.867170in}}%
\pgfusepath{stroke}%
\end{pgfscope}%
\begin{pgfscope}%
\definecolor{textcolor}{rgb}{0.000000,0.000000,0.000000}%
\pgfsetstrokecolor{textcolor}%
\pgfsetfillcolor{textcolor}%
\pgftext[x=3.076415in,y=2.818559in,left,base]{\color{textcolor}\rmfamily\fontsize{10.000000}{12.000000}\selectfont T = 66.8 [\si{K}]}%
\end{pgfscope}%
\begin{pgfscope}%
\pgfsetrectcap%
\pgfsetroundjoin%
\pgfsetlinewidth{1.505625pt}%
\definecolor{currentstroke}{rgb}{0.172549,0.627451,0.172549}%
\pgfsetstrokecolor{currentstroke}%
\pgfsetdash{}{0pt}%
\pgfpathmoveto{\pgfqpoint{2.687526in}{2.658871in}}%
\pgfpathlineto{\pgfqpoint{2.965304in}{2.658871in}}%
\pgfusepath{stroke}%
\end{pgfscope}%
\begin{pgfscope}%
\definecolor{textcolor}{rgb}{0.000000,0.000000,0.000000}%
\pgfsetstrokecolor{textcolor}%
\pgfsetfillcolor{textcolor}%
\pgftext[x=3.076415in,y=2.610260in,left,base]{\color{textcolor}\rmfamily\fontsize{10.000000}{12.000000}\selectfont T = 94.4 [\si{K}]}%
\end{pgfscope}%
\begin{pgfscope}%
\pgfsetrectcap%
\pgfsetroundjoin%
\pgfsetlinewidth{1.505625pt}%
\definecolor{currentstroke}{rgb}{0.839216,0.152941,0.156863}%
\pgfsetstrokecolor{currentstroke}%
\pgfsetdash{}{0pt}%
\pgfpathmoveto{\pgfqpoint{2.687526in}{2.450572in}}%
\pgfpathlineto{\pgfqpoint{2.965304in}{2.450572in}}%
\pgfusepath{stroke}%
\end{pgfscope}%
\begin{pgfscope}%
\definecolor{textcolor}{rgb}{0.000000,0.000000,0.000000}%
\pgfsetstrokecolor{textcolor}%
\pgfsetfillcolor{textcolor}%
\pgftext[x=3.076415in,y=2.401960in,left,base]{\color{textcolor}\rmfamily\fontsize{10.000000}{12.000000}\selectfont T = 133.5 [\si{K}]}%
\end{pgfscope}%
\begin{pgfscope}%
\pgfsetrectcap%
\pgfsetroundjoin%
\pgfsetlinewidth{1.505625pt}%
\definecolor{currentstroke}{rgb}{0.580392,0.403922,0.741176}%
\pgfsetstrokecolor{currentstroke}%
\pgfsetdash{}{0pt}%
\pgfpathmoveto{\pgfqpoint{2.687526in}{2.242272in}}%
\pgfpathlineto{\pgfqpoint{2.965304in}{2.242272in}}%
\pgfusepath{stroke}%
\end{pgfscope}%
\begin{pgfscope}%
\definecolor{textcolor}{rgb}{0.000000,0.000000,0.000000}%
\pgfsetstrokecolor{textcolor}%
\pgfsetfillcolor{textcolor}%
\pgftext[x=3.076415in,y=2.193661in,left,base]{\color{textcolor}\rmfamily\fontsize{10.000000}{12.000000}\selectfont T = 188.8 [\si{K}]}%
\end{pgfscope}%
\end{pgfpicture}%
\makeatother%
\endgroup%

        }
        \caption{Structure factor.}
        \label{step2_sk_changeT}
    \end{subfigure}
    %\hspace{0.05\textwidth}
    \begin{subfigure}{0.5\textwidth}
        \resizebox{\textwidth}{!}{
            %% Creator: Matplotlib, PGF backend
%%
%% To include the figure in your LaTeX document, write
%%   \input{<filename>.pgf}
%%
%% Make sure the required packages are loaded in your preamble
%%   \usepackage{pgf}
%%
%% and, on pdftex
%%   \usepackage[utf8]{inputenc}\DeclareUnicodeCharacter{2212}{-}
%%
%% or, on luatex and xetex
%%   \usepackage{unicode-math}
%%
%% Figures using additional raster images can only be included by \input if
%% they are in the same directory as the main LaTeX file. For loading figures
%% from other directories you can use the `import` package
%%   \usepackage{import}
%%
%% and then include the figures with
%%   \import{<path to file>}{<filename>.pgf}
%%
%% Matplotlib used the following preamble
%%   \usepackage[utf8]{inputenc}
%%   \usepackage[T1]{fontenc}
%%   \usepackage{siunitx}
%%
\begingroup%
\makeatletter%
\begin{pgfpicture}%
\pgfpathrectangle{\pgfpointorigin}{\pgfqpoint{4.500000in}{3.700000in}}%
\pgfusepath{use as bounding box, clip}%
\begin{pgfscope}%
\pgfsetbuttcap%
\pgfsetmiterjoin%
\definecolor{currentfill}{rgb}{1.000000,1.000000,1.000000}%
\pgfsetfillcolor{currentfill}%
\pgfsetlinewidth{0.000000pt}%
\definecolor{currentstroke}{rgb}{1.000000,1.000000,1.000000}%
\pgfsetstrokecolor{currentstroke}%
\pgfsetdash{}{0pt}%
\pgfpathmoveto{\pgfqpoint{0.000000in}{0.000000in}}%
\pgfpathlineto{\pgfqpoint{4.500000in}{0.000000in}}%
\pgfpathlineto{\pgfqpoint{4.500000in}{3.700000in}}%
\pgfpathlineto{\pgfqpoint{0.000000in}{3.700000in}}%
\pgfpathclose%
\pgfusepath{fill}%
\end{pgfscope}%
\begin{pgfscope}%
\pgfsetbuttcap%
\pgfsetmiterjoin%
\definecolor{currentfill}{rgb}{1.000000,1.000000,1.000000}%
\pgfsetfillcolor{currentfill}%
\pgfsetlinewidth{0.000000pt}%
\definecolor{currentstroke}{rgb}{0.000000,0.000000,0.000000}%
\pgfsetstrokecolor{currentstroke}%
\pgfsetstrokeopacity{0.000000}%
\pgfsetdash{}{0pt}%
\pgfpathmoveto{\pgfqpoint{0.562500in}{0.407000in}}%
\pgfpathlineto{\pgfqpoint{4.050000in}{0.407000in}}%
\pgfpathlineto{\pgfqpoint{4.050000in}{3.256000in}}%
\pgfpathlineto{\pgfqpoint{0.562500in}{3.256000in}}%
\pgfpathclose%
\pgfusepath{fill}%
\end{pgfscope}%
\begin{pgfscope}%
\pgfpathrectangle{\pgfqpoint{0.562500in}{0.407000in}}{\pgfqpoint{3.487500in}{2.849000in}}%
\pgfusepath{clip}%
\pgfsetbuttcap%
\pgfsetroundjoin%
\definecolor{currentfill}{rgb}{1.000000,1.000000,0.000000}%
\pgfsetfillcolor{currentfill}%
\pgfsetfillopacity{0.500000}%
\pgfsetlinewidth{0.000000pt}%
\definecolor{currentstroke}{rgb}{0.000000,0.000000,0.000000}%
\pgfsetstrokecolor{currentstroke}%
\pgfsetdash{}{0pt}%
\pgfpathmoveto{\pgfqpoint{0.721023in}{3.126500in}}%
\pgfpathlineto{\pgfqpoint{0.721023in}{2.233866in}}%
\pgfpathlineto{\pgfqpoint{1.158771in}{1.813505in}}%
\pgfpathlineto{\pgfqpoint{1.777841in}{1.355866in}}%
\pgfpathlineto{\pgfqpoint{2.653338in}{0.918430in}}%
\pgfpathlineto{\pgfqpoint{3.891477in}{0.536500in}}%
\pgfpathlineto{\pgfqpoint{3.891477in}{1.199775in}}%
\pgfpathlineto{\pgfqpoint{3.891477in}{1.199775in}}%
\pgfpathlineto{\pgfqpoint{2.653338in}{1.710904in}}%
\pgfpathlineto{\pgfqpoint{1.777841in}{2.246849in}}%
\pgfpathlineto{\pgfqpoint{1.158771in}{2.742739in}}%
\pgfpathlineto{\pgfqpoint{0.721023in}{3.126500in}}%
\pgfpathclose%
\pgfusepath{fill}%
\end{pgfscope}%
\begin{pgfscope}%
\pgfsetbuttcap%
\pgfsetroundjoin%
\definecolor{currentfill}{rgb}{0.000000,0.000000,0.000000}%
\pgfsetfillcolor{currentfill}%
\pgfsetlinewidth{0.803000pt}%
\definecolor{currentstroke}{rgb}{0.000000,0.000000,0.000000}%
\pgfsetstrokecolor{currentstroke}%
\pgfsetdash{}{0pt}%
\pgfsys@defobject{currentmarker}{\pgfqpoint{0.000000in}{-0.048611in}}{\pgfqpoint{0.000000in}{0.000000in}}{%
\pgfpathmoveto{\pgfqpoint{0.000000in}{0.000000in}}%
\pgfpathlineto{\pgfqpoint{0.000000in}{-0.048611in}}%
\pgfusepath{stroke,fill}%
}%
\begin{pgfscope}%
\pgfsys@transformshift{1.007617in}{0.407000in}%
\pgfsys@useobject{currentmarker}{}%
\end{pgfscope}%
\end{pgfscope}%
\begin{pgfscope}%
\definecolor{textcolor}{rgb}{0.000000,0.000000,0.000000}%
\pgfsetstrokecolor{textcolor}%
\pgfsetfillcolor{textcolor}%
\pgftext[x=1.007617in,y=0.309778in,,top]{\color{textcolor}\rmfamily\fontsize{10.000000}{12.000000}\selectfont \(\displaystyle {60}\)}%
\end{pgfscope}%
\begin{pgfscope}%
\pgfsetbuttcap%
\pgfsetroundjoin%
\definecolor{currentfill}{rgb}{0.000000,0.000000,0.000000}%
\pgfsetfillcolor{currentfill}%
\pgfsetlinewidth{0.803000pt}%
\definecolor{currentstroke}{rgb}{0.000000,0.000000,0.000000}%
\pgfsetstrokecolor{currentstroke}%
\pgfsetdash{}{0pt}%
\pgfsys@defobject{currentmarker}{\pgfqpoint{0.000000in}{-0.048611in}}{\pgfqpoint{0.000000in}{0.000000in}}{%
\pgfpathmoveto{\pgfqpoint{0.000000in}{0.000000in}}%
\pgfpathlineto{\pgfqpoint{0.000000in}{-0.048611in}}%
\pgfusepath{stroke,fill}%
}%
\begin{pgfscope}%
\pgfsys@transformshift{1.455422in}{0.407000in}%
\pgfsys@useobject{currentmarker}{}%
\end{pgfscope}%
\end{pgfscope}%
\begin{pgfscope}%
\definecolor{textcolor}{rgb}{0.000000,0.000000,0.000000}%
\pgfsetstrokecolor{textcolor}%
\pgfsetfillcolor{textcolor}%
\pgftext[x=1.455422in,y=0.309778in,,top]{\color{textcolor}\rmfamily\fontsize{10.000000}{12.000000}\selectfont \(\displaystyle {80}\)}%
\end{pgfscope}%
\begin{pgfscope}%
\pgfsetbuttcap%
\pgfsetroundjoin%
\definecolor{currentfill}{rgb}{0.000000,0.000000,0.000000}%
\pgfsetfillcolor{currentfill}%
\pgfsetlinewidth{0.803000pt}%
\definecolor{currentstroke}{rgb}{0.000000,0.000000,0.000000}%
\pgfsetstrokecolor{currentstroke}%
\pgfsetdash{}{0pt}%
\pgfsys@defobject{currentmarker}{\pgfqpoint{0.000000in}{-0.048611in}}{\pgfqpoint{0.000000in}{0.000000in}}{%
\pgfpathmoveto{\pgfqpoint{0.000000in}{0.000000in}}%
\pgfpathlineto{\pgfqpoint{0.000000in}{-0.048611in}}%
\pgfusepath{stroke,fill}%
}%
\begin{pgfscope}%
\pgfsys@transformshift{1.903226in}{0.407000in}%
\pgfsys@useobject{currentmarker}{}%
\end{pgfscope}%
\end{pgfscope}%
\begin{pgfscope}%
\definecolor{textcolor}{rgb}{0.000000,0.000000,0.000000}%
\pgfsetstrokecolor{textcolor}%
\pgfsetfillcolor{textcolor}%
\pgftext[x=1.903226in,y=0.309778in,,top]{\color{textcolor}\rmfamily\fontsize{10.000000}{12.000000}\selectfont \(\displaystyle {100}\)}%
\end{pgfscope}%
\begin{pgfscope}%
\pgfsetbuttcap%
\pgfsetroundjoin%
\definecolor{currentfill}{rgb}{0.000000,0.000000,0.000000}%
\pgfsetfillcolor{currentfill}%
\pgfsetlinewidth{0.803000pt}%
\definecolor{currentstroke}{rgb}{0.000000,0.000000,0.000000}%
\pgfsetstrokecolor{currentstroke}%
\pgfsetdash{}{0pt}%
\pgfsys@defobject{currentmarker}{\pgfqpoint{0.000000in}{-0.048611in}}{\pgfqpoint{0.000000in}{0.000000in}}{%
\pgfpathmoveto{\pgfqpoint{0.000000in}{0.000000in}}%
\pgfpathlineto{\pgfqpoint{0.000000in}{-0.048611in}}%
\pgfusepath{stroke,fill}%
}%
\begin{pgfscope}%
\pgfsys@transformshift{2.351030in}{0.407000in}%
\pgfsys@useobject{currentmarker}{}%
\end{pgfscope}%
\end{pgfscope}%
\begin{pgfscope}%
\definecolor{textcolor}{rgb}{0.000000,0.000000,0.000000}%
\pgfsetstrokecolor{textcolor}%
\pgfsetfillcolor{textcolor}%
\pgftext[x=2.351030in,y=0.309778in,,top]{\color{textcolor}\rmfamily\fontsize{10.000000}{12.000000}\selectfont \(\displaystyle {120}\)}%
\end{pgfscope}%
\begin{pgfscope}%
\pgfsetbuttcap%
\pgfsetroundjoin%
\definecolor{currentfill}{rgb}{0.000000,0.000000,0.000000}%
\pgfsetfillcolor{currentfill}%
\pgfsetlinewidth{0.803000pt}%
\definecolor{currentstroke}{rgb}{0.000000,0.000000,0.000000}%
\pgfsetstrokecolor{currentstroke}%
\pgfsetdash{}{0pt}%
\pgfsys@defobject{currentmarker}{\pgfqpoint{0.000000in}{-0.048611in}}{\pgfqpoint{0.000000in}{0.000000in}}{%
\pgfpathmoveto{\pgfqpoint{0.000000in}{0.000000in}}%
\pgfpathlineto{\pgfqpoint{0.000000in}{-0.048611in}}%
\pgfusepath{stroke,fill}%
}%
\begin{pgfscope}%
\pgfsys@transformshift{2.798835in}{0.407000in}%
\pgfsys@useobject{currentmarker}{}%
\end{pgfscope}%
\end{pgfscope}%
\begin{pgfscope}%
\definecolor{textcolor}{rgb}{0.000000,0.000000,0.000000}%
\pgfsetstrokecolor{textcolor}%
\pgfsetfillcolor{textcolor}%
\pgftext[x=2.798835in,y=0.309778in,,top]{\color{textcolor}\rmfamily\fontsize{10.000000}{12.000000}\selectfont \(\displaystyle {140}\)}%
\end{pgfscope}%
\begin{pgfscope}%
\pgfsetbuttcap%
\pgfsetroundjoin%
\definecolor{currentfill}{rgb}{0.000000,0.000000,0.000000}%
\pgfsetfillcolor{currentfill}%
\pgfsetlinewidth{0.803000pt}%
\definecolor{currentstroke}{rgb}{0.000000,0.000000,0.000000}%
\pgfsetstrokecolor{currentstroke}%
\pgfsetdash{}{0pt}%
\pgfsys@defobject{currentmarker}{\pgfqpoint{0.000000in}{-0.048611in}}{\pgfqpoint{0.000000in}{0.000000in}}{%
\pgfpathmoveto{\pgfqpoint{0.000000in}{0.000000in}}%
\pgfpathlineto{\pgfqpoint{0.000000in}{-0.048611in}}%
\pgfusepath{stroke,fill}%
}%
\begin{pgfscope}%
\pgfsys@transformshift{3.246639in}{0.407000in}%
\pgfsys@useobject{currentmarker}{}%
\end{pgfscope}%
\end{pgfscope}%
\begin{pgfscope}%
\definecolor{textcolor}{rgb}{0.000000,0.000000,0.000000}%
\pgfsetstrokecolor{textcolor}%
\pgfsetfillcolor{textcolor}%
\pgftext[x=3.246639in,y=0.309778in,,top]{\color{textcolor}\rmfamily\fontsize{10.000000}{12.000000}\selectfont \(\displaystyle {160}\)}%
\end{pgfscope}%
\begin{pgfscope}%
\pgfsetbuttcap%
\pgfsetroundjoin%
\definecolor{currentfill}{rgb}{0.000000,0.000000,0.000000}%
\pgfsetfillcolor{currentfill}%
\pgfsetlinewidth{0.803000pt}%
\definecolor{currentstroke}{rgb}{0.000000,0.000000,0.000000}%
\pgfsetstrokecolor{currentstroke}%
\pgfsetdash{}{0pt}%
\pgfsys@defobject{currentmarker}{\pgfqpoint{0.000000in}{-0.048611in}}{\pgfqpoint{0.000000in}{0.000000in}}{%
\pgfpathmoveto{\pgfqpoint{0.000000in}{0.000000in}}%
\pgfpathlineto{\pgfqpoint{0.000000in}{-0.048611in}}%
\pgfusepath{stroke,fill}%
}%
\begin{pgfscope}%
\pgfsys@transformshift{3.694443in}{0.407000in}%
\pgfsys@useobject{currentmarker}{}%
\end{pgfscope}%
\end{pgfscope}%
\begin{pgfscope}%
\definecolor{textcolor}{rgb}{0.000000,0.000000,0.000000}%
\pgfsetstrokecolor{textcolor}%
\pgfsetfillcolor{textcolor}%
\pgftext[x=3.694443in,y=0.309778in,,top]{\color{textcolor}\rmfamily\fontsize{10.000000}{12.000000}\selectfont \(\displaystyle {180}\)}%
\end{pgfscope}%
\begin{pgfscope}%
\definecolor{textcolor}{rgb}{0.000000,0.000000,0.000000}%
\pgfsetstrokecolor{textcolor}%
\pgfsetfillcolor{textcolor}%
\pgftext[x=2.306250in,y=0.131567in,,top]{\color{textcolor}\rmfamily\fontsize{10.000000}{12.000000}\selectfont \(\displaystyle T\) [\si{K}]}%
\end{pgfscope}%
\begin{pgfscope}%
\pgfsetbuttcap%
\pgfsetroundjoin%
\definecolor{currentfill}{rgb}{0.000000,0.000000,0.000000}%
\pgfsetfillcolor{currentfill}%
\pgfsetlinewidth{0.803000pt}%
\definecolor{currentstroke}{rgb}{0.000000,0.000000,0.000000}%
\pgfsetstrokecolor{currentstroke}%
\pgfsetdash{}{0pt}%
\pgfsys@defobject{currentmarker}{\pgfqpoint{-0.048611in}{0.000000in}}{\pgfqpoint{0.000000in}{0.000000in}}{%
\pgfpathmoveto{\pgfqpoint{0.000000in}{0.000000in}}%
\pgfpathlineto{\pgfqpoint{-0.048611in}{0.000000in}}%
\pgfusepath{stroke,fill}%
}%
\begin{pgfscope}%
\pgfsys@transformshift{0.562500in}{0.603731in}%
\pgfsys@useobject{currentmarker}{}%
\end{pgfscope}%
\end{pgfscope}%
\begin{pgfscope}%
\definecolor{textcolor}{rgb}{0.000000,0.000000,0.000000}%
\pgfsetstrokecolor{textcolor}%
\pgfsetfillcolor{textcolor}%
\pgftext[x=0.287808in, y=0.555903in, left, base]{\color{textcolor}\rmfamily\fontsize{10.000000}{12.000000}\selectfont \(\displaystyle {1.8}\)}%
\end{pgfscope}%
\begin{pgfscope}%
\pgfsetbuttcap%
\pgfsetroundjoin%
\definecolor{currentfill}{rgb}{0.000000,0.000000,0.000000}%
\pgfsetfillcolor{currentfill}%
\pgfsetlinewidth{0.803000pt}%
\definecolor{currentstroke}{rgb}{0.000000,0.000000,0.000000}%
\pgfsetstrokecolor{currentstroke}%
\pgfsetdash{}{0pt}%
\pgfsys@defobject{currentmarker}{\pgfqpoint{-0.048611in}{0.000000in}}{\pgfqpoint{0.000000in}{0.000000in}}{%
\pgfpathmoveto{\pgfqpoint{0.000000in}{0.000000in}}%
\pgfpathlineto{\pgfqpoint{-0.048611in}{0.000000in}}%
\pgfusepath{stroke,fill}%
}%
\begin{pgfscope}%
\pgfsys@transformshift{0.562500in}{0.931955in}%
\pgfsys@useobject{currentmarker}{}%
\end{pgfscope}%
\end{pgfscope}%
\begin{pgfscope}%
\definecolor{textcolor}{rgb}{0.000000,0.000000,0.000000}%
\pgfsetstrokecolor{textcolor}%
\pgfsetfillcolor{textcolor}%
\pgftext[x=0.287808in, y=0.884127in, left, base]{\color{textcolor}\rmfamily\fontsize{10.000000}{12.000000}\selectfont \(\displaystyle {2.0}\)}%
\end{pgfscope}%
\begin{pgfscope}%
\pgfsetbuttcap%
\pgfsetroundjoin%
\definecolor{currentfill}{rgb}{0.000000,0.000000,0.000000}%
\pgfsetfillcolor{currentfill}%
\pgfsetlinewidth{0.803000pt}%
\definecolor{currentstroke}{rgb}{0.000000,0.000000,0.000000}%
\pgfsetstrokecolor{currentstroke}%
\pgfsetdash{}{0pt}%
\pgfsys@defobject{currentmarker}{\pgfqpoint{-0.048611in}{0.000000in}}{\pgfqpoint{0.000000in}{0.000000in}}{%
\pgfpathmoveto{\pgfqpoint{0.000000in}{0.000000in}}%
\pgfpathlineto{\pgfqpoint{-0.048611in}{0.000000in}}%
\pgfusepath{stroke,fill}%
}%
\begin{pgfscope}%
\pgfsys@transformshift{0.562500in}{1.260179in}%
\pgfsys@useobject{currentmarker}{}%
\end{pgfscope}%
\end{pgfscope}%
\begin{pgfscope}%
\definecolor{textcolor}{rgb}{0.000000,0.000000,0.000000}%
\pgfsetstrokecolor{textcolor}%
\pgfsetfillcolor{textcolor}%
\pgftext[x=0.287808in, y=1.212351in, left, base]{\color{textcolor}\rmfamily\fontsize{10.000000}{12.000000}\selectfont \(\displaystyle {2.2}\)}%
\end{pgfscope}%
\begin{pgfscope}%
\pgfsetbuttcap%
\pgfsetroundjoin%
\definecolor{currentfill}{rgb}{0.000000,0.000000,0.000000}%
\pgfsetfillcolor{currentfill}%
\pgfsetlinewidth{0.803000pt}%
\definecolor{currentstroke}{rgb}{0.000000,0.000000,0.000000}%
\pgfsetstrokecolor{currentstroke}%
\pgfsetdash{}{0pt}%
\pgfsys@defobject{currentmarker}{\pgfqpoint{-0.048611in}{0.000000in}}{\pgfqpoint{0.000000in}{0.000000in}}{%
\pgfpathmoveto{\pgfqpoint{0.000000in}{0.000000in}}%
\pgfpathlineto{\pgfqpoint{-0.048611in}{0.000000in}}%
\pgfusepath{stroke,fill}%
}%
\begin{pgfscope}%
\pgfsys@transformshift{0.562500in}{1.588403in}%
\pgfsys@useobject{currentmarker}{}%
\end{pgfscope}%
\end{pgfscope}%
\begin{pgfscope}%
\definecolor{textcolor}{rgb}{0.000000,0.000000,0.000000}%
\pgfsetstrokecolor{textcolor}%
\pgfsetfillcolor{textcolor}%
\pgftext[x=0.287808in, y=1.540575in, left, base]{\color{textcolor}\rmfamily\fontsize{10.000000}{12.000000}\selectfont \(\displaystyle {2.4}\)}%
\end{pgfscope}%
\begin{pgfscope}%
\pgfsetbuttcap%
\pgfsetroundjoin%
\definecolor{currentfill}{rgb}{0.000000,0.000000,0.000000}%
\pgfsetfillcolor{currentfill}%
\pgfsetlinewidth{0.803000pt}%
\definecolor{currentstroke}{rgb}{0.000000,0.000000,0.000000}%
\pgfsetstrokecolor{currentstroke}%
\pgfsetdash{}{0pt}%
\pgfsys@defobject{currentmarker}{\pgfqpoint{-0.048611in}{0.000000in}}{\pgfqpoint{0.000000in}{0.000000in}}{%
\pgfpathmoveto{\pgfqpoint{0.000000in}{0.000000in}}%
\pgfpathlineto{\pgfqpoint{-0.048611in}{0.000000in}}%
\pgfusepath{stroke,fill}%
}%
\begin{pgfscope}%
\pgfsys@transformshift{0.562500in}{1.916627in}%
\pgfsys@useobject{currentmarker}{}%
\end{pgfscope}%
\end{pgfscope}%
\begin{pgfscope}%
\definecolor{textcolor}{rgb}{0.000000,0.000000,0.000000}%
\pgfsetstrokecolor{textcolor}%
\pgfsetfillcolor{textcolor}%
\pgftext[x=0.287808in, y=1.868799in, left, base]{\color{textcolor}\rmfamily\fontsize{10.000000}{12.000000}\selectfont \(\displaystyle {2.6}\)}%
\end{pgfscope}%
\begin{pgfscope}%
\pgfsetbuttcap%
\pgfsetroundjoin%
\definecolor{currentfill}{rgb}{0.000000,0.000000,0.000000}%
\pgfsetfillcolor{currentfill}%
\pgfsetlinewidth{0.803000pt}%
\definecolor{currentstroke}{rgb}{0.000000,0.000000,0.000000}%
\pgfsetstrokecolor{currentstroke}%
\pgfsetdash{}{0pt}%
\pgfsys@defobject{currentmarker}{\pgfqpoint{-0.048611in}{0.000000in}}{\pgfqpoint{0.000000in}{0.000000in}}{%
\pgfpathmoveto{\pgfqpoint{0.000000in}{0.000000in}}%
\pgfpathlineto{\pgfqpoint{-0.048611in}{0.000000in}}%
\pgfusepath{stroke,fill}%
}%
\begin{pgfscope}%
\pgfsys@transformshift{0.562500in}{2.244851in}%
\pgfsys@useobject{currentmarker}{}%
\end{pgfscope}%
\end{pgfscope}%
\begin{pgfscope}%
\definecolor{textcolor}{rgb}{0.000000,0.000000,0.000000}%
\pgfsetstrokecolor{textcolor}%
\pgfsetfillcolor{textcolor}%
\pgftext[x=0.287808in, y=2.197023in, left, base]{\color{textcolor}\rmfamily\fontsize{10.000000}{12.000000}\selectfont \(\displaystyle {2.8}\)}%
\end{pgfscope}%
\begin{pgfscope}%
\pgfsetbuttcap%
\pgfsetroundjoin%
\definecolor{currentfill}{rgb}{0.000000,0.000000,0.000000}%
\pgfsetfillcolor{currentfill}%
\pgfsetlinewidth{0.803000pt}%
\definecolor{currentstroke}{rgb}{0.000000,0.000000,0.000000}%
\pgfsetstrokecolor{currentstroke}%
\pgfsetdash{}{0pt}%
\pgfsys@defobject{currentmarker}{\pgfqpoint{-0.048611in}{0.000000in}}{\pgfqpoint{0.000000in}{0.000000in}}{%
\pgfpathmoveto{\pgfqpoint{0.000000in}{0.000000in}}%
\pgfpathlineto{\pgfqpoint{-0.048611in}{0.000000in}}%
\pgfusepath{stroke,fill}%
}%
\begin{pgfscope}%
\pgfsys@transformshift{0.562500in}{2.573075in}%
\pgfsys@useobject{currentmarker}{}%
\end{pgfscope}%
\end{pgfscope}%
\begin{pgfscope}%
\definecolor{textcolor}{rgb}{0.000000,0.000000,0.000000}%
\pgfsetstrokecolor{textcolor}%
\pgfsetfillcolor{textcolor}%
\pgftext[x=0.287808in, y=2.525247in, left, base]{\color{textcolor}\rmfamily\fontsize{10.000000}{12.000000}\selectfont \(\displaystyle {3.0}\)}%
\end{pgfscope}%
\begin{pgfscope}%
\pgfsetbuttcap%
\pgfsetroundjoin%
\definecolor{currentfill}{rgb}{0.000000,0.000000,0.000000}%
\pgfsetfillcolor{currentfill}%
\pgfsetlinewidth{0.803000pt}%
\definecolor{currentstroke}{rgb}{0.000000,0.000000,0.000000}%
\pgfsetstrokecolor{currentstroke}%
\pgfsetdash{}{0pt}%
\pgfsys@defobject{currentmarker}{\pgfqpoint{-0.048611in}{0.000000in}}{\pgfqpoint{0.000000in}{0.000000in}}{%
\pgfpathmoveto{\pgfqpoint{0.000000in}{0.000000in}}%
\pgfpathlineto{\pgfqpoint{-0.048611in}{0.000000in}}%
\pgfusepath{stroke,fill}%
}%
\begin{pgfscope}%
\pgfsys@transformshift{0.562500in}{2.901299in}%
\pgfsys@useobject{currentmarker}{}%
\end{pgfscope}%
\end{pgfscope}%
\begin{pgfscope}%
\definecolor{textcolor}{rgb}{0.000000,0.000000,0.000000}%
\pgfsetstrokecolor{textcolor}%
\pgfsetfillcolor{textcolor}%
\pgftext[x=0.287808in, y=2.853471in, left, base]{\color{textcolor}\rmfamily\fontsize{10.000000}{12.000000}\selectfont \(\displaystyle {3.2}\)}%
\end{pgfscope}%
\begin{pgfscope}%
\pgfsetbuttcap%
\pgfsetroundjoin%
\definecolor{currentfill}{rgb}{0.000000,0.000000,0.000000}%
\pgfsetfillcolor{currentfill}%
\pgfsetlinewidth{0.803000pt}%
\definecolor{currentstroke}{rgb}{0.000000,0.000000,0.000000}%
\pgfsetstrokecolor{currentstroke}%
\pgfsetdash{}{0pt}%
\pgfsys@defobject{currentmarker}{\pgfqpoint{-0.048611in}{0.000000in}}{\pgfqpoint{0.000000in}{0.000000in}}{%
\pgfpathmoveto{\pgfqpoint{0.000000in}{0.000000in}}%
\pgfpathlineto{\pgfqpoint{-0.048611in}{0.000000in}}%
\pgfusepath{stroke,fill}%
}%
\begin{pgfscope}%
\pgfsys@transformshift{0.562500in}{3.229523in}%
\pgfsys@useobject{currentmarker}{}%
\end{pgfscope}%
\end{pgfscope}%
\begin{pgfscope}%
\definecolor{textcolor}{rgb}{0.000000,0.000000,0.000000}%
\pgfsetstrokecolor{textcolor}%
\pgfsetfillcolor{textcolor}%
\pgftext[x=0.287808in, y=3.181695in, left, base]{\color{textcolor}\rmfamily\fontsize{10.000000}{12.000000}\selectfont \(\displaystyle {3.4}\)}%
\end{pgfscope}%
\begin{pgfscope}%
\definecolor{textcolor}{rgb}{0.000000,0.000000,0.000000}%
\pgfsetstrokecolor{textcolor}%
\pgfsetfillcolor{textcolor}%
\pgftext[x=0.232253in,y=1.831500in,,bottom,rotate=90.000000]{\color{textcolor}\rmfamily\fontsize{10.000000}{12.000000}\selectfont \(\displaystyle \max_k|S(k)|\)}%
\end{pgfscope}%
\begin{pgfscope}%
\pgfpathrectangle{\pgfqpoint{0.562500in}{0.407000in}}{\pgfqpoint{3.487500in}{2.849000in}}%
\pgfusepath{clip}%
\pgfsetrectcap%
\pgfsetroundjoin%
\pgfsetlinewidth{1.505625pt}%
\definecolor{currentstroke}{rgb}{0.121569,0.466667,0.705882}%
\pgfsetstrokecolor{currentstroke}%
\pgfsetdash{}{0pt}%
\pgfpathmoveto{\pgfqpoint{0.721023in}{2.780229in}}%
\pgfpathlineto{\pgfqpoint{1.158771in}{2.183429in}}%
\pgfpathlineto{\pgfqpoint{1.777841in}{1.662704in}}%
\pgfpathlineto{\pgfqpoint{2.653338in}{1.323421in}}%
\pgfpathlineto{\pgfqpoint{3.891477in}{1.041954in}}%
\pgfusepath{stroke}%
\end{pgfscope}%
\begin{pgfscope}%
\pgfpathrectangle{\pgfqpoint{0.562500in}{0.407000in}}{\pgfqpoint{3.487500in}{2.849000in}}%
\pgfusepath{clip}%
\pgfsetbuttcap%
\pgfsetroundjoin%
\definecolor{currentfill}{rgb}{0.121569,0.466667,0.705882}%
\pgfsetfillcolor{currentfill}%
\pgfsetlinewidth{1.003750pt}%
\definecolor{currentstroke}{rgb}{0.000000,0.000000,0.000000}%
\pgfsetstrokecolor{currentstroke}%
\pgfsetdash{}{0pt}%
\pgfsys@defobject{currentmarker}{\pgfqpoint{-0.041667in}{-0.041667in}}{\pgfqpoint{0.041667in}{0.041667in}}{%
\pgfpathmoveto{\pgfqpoint{0.000000in}{-0.041667in}}%
\pgfpathcurveto{\pgfqpoint{0.011050in}{-0.041667in}}{\pgfqpoint{0.021649in}{-0.037276in}}{\pgfqpoint{0.029463in}{-0.029463in}}%
\pgfpathcurveto{\pgfqpoint{0.037276in}{-0.021649in}}{\pgfqpoint{0.041667in}{-0.011050in}}{\pgfqpoint{0.041667in}{0.000000in}}%
\pgfpathcurveto{\pgfqpoint{0.041667in}{0.011050in}}{\pgfqpoint{0.037276in}{0.021649in}}{\pgfqpoint{0.029463in}{0.029463in}}%
\pgfpathcurveto{\pgfqpoint{0.021649in}{0.037276in}}{\pgfqpoint{0.011050in}{0.041667in}}{\pgfqpoint{0.000000in}{0.041667in}}%
\pgfpathcurveto{\pgfqpoint{-0.011050in}{0.041667in}}{\pgfqpoint{-0.021649in}{0.037276in}}{\pgfqpoint{-0.029463in}{0.029463in}}%
\pgfpathcurveto{\pgfqpoint{-0.037276in}{0.021649in}}{\pgfqpoint{-0.041667in}{0.011050in}}{\pgfqpoint{-0.041667in}{0.000000in}}%
\pgfpathcurveto{\pgfqpoint{-0.041667in}{-0.011050in}}{\pgfqpoint{-0.037276in}{-0.021649in}}{\pgfqpoint{-0.029463in}{-0.029463in}}%
\pgfpathcurveto{\pgfqpoint{-0.021649in}{-0.037276in}}{\pgfqpoint{-0.011050in}{-0.041667in}}{\pgfqpoint{0.000000in}{-0.041667in}}%
\pgfpathclose%
\pgfusepath{stroke,fill}%
}%
\begin{pgfscope}%
\pgfsys@transformshift{0.721023in}{2.780229in}%
\pgfsys@useobject{currentmarker}{}%
\end{pgfscope}%
\begin{pgfscope}%
\pgfsys@transformshift{1.158771in}{2.183429in}%
\pgfsys@useobject{currentmarker}{}%
\end{pgfscope}%
\begin{pgfscope}%
\pgfsys@transformshift{1.777841in}{1.662704in}%
\pgfsys@useobject{currentmarker}{}%
\end{pgfscope}%
\begin{pgfscope}%
\pgfsys@transformshift{2.653338in}{1.323421in}%
\pgfsys@useobject{currentmarker}{}%
\end{pgfscope}%
\begin{pgfscope}%
\pgfsys@transformshift{3.891477in}{1.041954in}%
\pgfsys@useobject{currentmarker}{}%
\end{pgfscope}%
\end{pgfscope}%
\begin{pgfscope}%
\pgfpathrectangle{\pgfqpoint{0.562500in}{0.407000in}}{\pgfqpoint{3.487500in}{2.849000in}}%
\pgfusepath{clip}%
\pgfsetbuttcap%
\pgfsetroundjoin%
\pgfsetlinewidth{1.505625pt}%
\definecolor{currentstroke}{rgb}{1.000000,0.000000,0.000000}%
\pgfsetstrokecolor{currentstroke}%
\pgfsetdash{{5.550000pt}{2.400000pt}}{0.000000pt}%
\pgfpathmoveto{\pgfqpoint{0.721023in}{2.680183in}}%
\pgfpathlineto{\pgfqpoint{1.158771in}{2.278122in}}%
\pgfpathlineto{\pgfqpoint{1.777841in}{1.801358in}}%
\pgfpathlineto{\pgfqpoint{2.653338in}{1.314667in}}%
\pgfpathlineto{\pgfqpoint{3.891477in}{0.868138in}}%
\pgfusepath{stroke}%
\end{pgfscope}%
\begin{pgfscope}%
\pgfsetrectcap%
\pgfsetmiterjoin%
\pgfsetlinewidth{0.803000pt}%
\definecolor{currentstroke}{rgb}{0.000000,0.000000,0.000000}%
\pgfsetstrokecolor{currentstroke}%
\pgfsetdash{}{0pt}%
\pgfpathmoveto{\pgfqpoint{0.562500in}{0.407000in}}%
\pgfpathlineto{\pgfqpoint{0.562500in}{3.256000in}}%
\pgfusepath{stroke}%
\end{pgfscope}%
\begin{pgfscope}%
\pgfsetrectcap%
\pgfsetmiterjoin%
\pgfsetlinewidth{0.803000pt}%
\definecolor{currentstroke}{rgb}{0.000000,0.000000,0.000000}%
\pgfsetstrokecolor{currentstroke}%
\pgfsetdash{}{0pt}%
\pgfpathmoveto{\pgfqpoint{4.050000in}{0.407000in}}%
\pgfpathlineto{\pgfqpoint{4.050000in}{3.256000in}}%
\pgfusepath{stroke}%
\end{pgfscope}%
\begin{pgfscope}%
\pgfsetrectcap%
\pgfsetmiterjoin%
\pgfsetlinewidth{0.803000pt}%
\definecolor{currentstroke}{rgb}{0.000000,0.000000,0.000000}%
\pgfsetstrokecolor{currentstroke}%
\pgfsetdash{}{0pt}%
\pgfpathmoveto{\pgfqpoint{0.562500in}{0.407000in}}%
\pgfpathlineto{\pgfqpoint{4.050000in}{0.407000in}}%
\pgfusepath{stroke}%
\end{pgfscope}%
\begin{pgfscope}%
\pgfsetrectcap%
\pgfsetmiterjoin%
\pgfsetlinewidth{0.803000pt}%
\definecolor{currentstroke}{rgb}{0.000000,0.000000,0.000000}%
\pgfsetstrokecolor{currentstroke}%
\pgfsetdash{}{0pt}%
\pgfpathmoveto{\pgfqpoint{0.562500in}{3.256000in}}%
\pgfpathlineto{\pgfqpoint{4.050000in}{3.256000in}}%
\pgfusepath{stroke}%
\end{pgfscope}%
\begin{pgfscope}%
\pgfsetbuttcap%
\pgfsetmiterjoin%
\definecolor{currentfill}{rgb}{1.000000,1.000000,1.000000}%
\pgfsetfillcolor{currentfill}%
\pgfsetfillopacity{0.800000}%
\pgfsetlinewidth{1.003750pt}%
\definecolor{currentstroke}{rgb}{0.800000,0.800000,0.800000}%
\pgfsetstrokecolor{currentstroke}%
\pgfsetstrokeopacity{0.800000}%
\pgfsetdash{}{0pt}%
\pgfpathmoveto{\pgfqpoint{2.304236in}{2.528820in}}%
\pgfpathlineto{\pgfqpoint{3.952778in}{2.528820in}}%
\pgfpathquadraticcurveto{\pgfqpoint{3.980556in}{2.528820in}}{\pgfqpoint{3.980556in}{2.556598in}}%
\pgfpathlineto{\pgfqpoint{3.980556in}{3.158778in}}%
\pgfpathquadraticcurveto{\pgfqpoint{3.980556in}{3.186556in}}{\pgfqpoint{3.952778in}{3.186556in}}%
\pgfpathlineto{\pgfqpoint{2.304236in}{3.186556in}}%
\pgfpathquadraticcurveto{\pgfqpoint{2.276458in}{3.186556in}}{\pgfqpoint{2.276458in}{3.158778in}}%
\pgfpathlineto{\pgfqpoint{2.276458in}{2.556598in}}%
\pgfpathquadraticcurveto{\pgfqpoint{2.276458in}{2.528820in}}{\pgfqpoint{2.304236in}{2.528820in}}%
\pgfpathclose%
\pgfusepath{stroke,fill}%
\end{pgfscope}%
\begin{pgfscope}%
\pgfsetrectcap%
\pgfsetroundjoin%
\pgfsetlinewidth{1.505625pt}%
\definecolor{currentstroke}{rgb}{0.121569,0.466667,0.705882}%
\pgfsetstrokecolor{currentstroke}%
\pgfsetdash{}{0pt}%
\pgfpathmoveto{\pgfqpoint{2.332014in}{3.082389in}}%
\pgfpathlineto{\pgfqpoint{2.609792in}{3.082389in}}%
\pgfusepath{stroke}%
\end{pgfscope}%
\begin{pgfscope}%
\pgfsetbuttcap%
\pgfsetroundjoin%
\definecolor{currentfill}{rgb}{0.121569,0.466667,0.705882}%
\pgfsetfillcolor{currentfill}%
\pgfsetlinewidth{1.003750pt}%
\definecolor{currentstroke}{rgb}{0.000000,0.000000,0.000000}%
\pgfsetstrokecolor{currentstroke}%
\pgfsetdash{}{0pt}%
\pgfsys@defobject{currentmarker}{\pgfqpoint{-0.041667in}{-0.041667in}}{\pgfqpoint{0.041667in}{0.041667in}}{%
\pgfpathmoveto{\pgfqpoint{0.000000in}{-0.041667in}}%
\pgfpathcurveto{\pgfqpoint{0.011050in}{-0.041667in}}{\pgfqpoint{0.021649in}{-0.037276in}}{\pgfqpoint{0.029463in}{-0.029463in}}%
\pgfpathcurveto{\pgfqpoint{0.037276in}{-0.021649in}}{\pgfqpoint{0.041667in}{-0.011050in}}{\pgfqpoint{0.041667in}{0.000000in}}%
\pgfpathcurveto{\pgfqpoint{0.041667in}{0.011050in}}{\pgfqpoint{0.037276in}{0.021649in}}{\pgfqpoint{0.029463in}{0.029463in}}%
\pgfpathcurveto{\pgfqpoint{0.021649in}{0.037276in}}{\pgfqpoint{0.011050in}{0.041667in}}{\pgfqpoint{0.000000in}{0.041667in}}%
\pgfpathcurveto{\pgfqpoint{-0.011050in}{0.041667in}}{\pgfqpoint{-0.021649in}{0.037276in}}{\pgfqpoint{-0.029463in}{0.029463in}}%
\pgfpathcurveto{\pgfqpoint{-0.037276in}{0.021649in}}{\pgfqpoint{-0.041667in}{0.011050in}}{\pgfqpoint{-0.041667in}{0.000000in}}%
\pgfpathcurveto{\pgfqpoint{-0.041667in}{-0.011050in}}{\pgfqpoint{-0.037276in}{-0.021649in}}{\pgfqpoint{-0.029463in}{-0.029463in}}%
\pgfpathcurveto{\pgfqpoint{-0.021649in}{-0.037276in}}{\pgfqpoint{-0.011050in}{-0.041667in}}{\pgfqpoint{0.000000in}{-0.041667in}}%
\pgfpathclose%
\pgfusepath{stroke,fill}%
}%
\begin{pgfscope}%
\pgfsys@transformshift{2.470903in}{3.082389in}%
\pgfsys@useobject{currentmarker}{}%
\end{pgfscope}%
\end{pgfscope}%
\begin{pgfscope}%
\definecolor{textcolor}{rgb}{0.000000,0.000000,0.000000}%
\pgfsetstrokecolor{textcolor}%
\pgfsetfillcolor{textcolor}%
\pgftext[x=2.720903in,y=3.033778in,left,base]{\color{textcolor}\rmfamily\fontsize{10.000000}{12.000000}\selectfont peaks}%
\end{pgfscope}%
\begin{pgfscope}%
\pgfsetbuttcap%
\pgfsetroundjoin%
\pgfsetlinewidth{1.505625pt}%
\definecolor{currentstroke}{rgb}{1.000000,0.000000,0.000000}%
\pgfsetstrokecolor{currentstroke}%
\pgfsetdash{{5.550000pt}{2.400000pt}}{0.000000pt}%
\pgfpathmoveto{\pgfqpoint{2.332014in}{2.861375in}}%
\pgfpathlineto{\pgfqpoint{2.609792in}{2.861375in}}%
\pgfusepath{stroke}%
\end{pgfscope}%
\begin{pgfscope}%
\definecolor{textcolor}{rgb}{0.000000,0.000000,0.000000}%
\pgfsetstrokecolor{textcolor}%
\pgfsetfillcolor{textcolor}%
\pgftext[x=2.720903in,y=2.812764in,left,base]{\color{textcolor}\rmfamily\fontsize{10.000000}{12.000000}\selectfont fit \(\displaystyle 1 + C \cdot [e^{-\frac{a}{T}} - 1]\)}%
\end{pgfscope}%
\begin{pgfscope}%
\pgfsetbuttcap%
\pgfsetmiterjoin%
\definecolor{currentfill}{rgb}{1.000000,1.000000,0.000000}%
\pgfsetfillcolor{currentfill}%
\pgfsetfillopacity{0.500000}%
\pgfsetlinewidth{0.000000pt}%
\definecolor{currentstroke}{rgb}{0.000000,0.000000,0.000000}%
\pgfsetstrokecolor{currentstroke}%
\pgfsetstrokeopacity{0.500000}%
\pgfsetdash{}{0pt}%
\pgfpathmoveto{\pgfqpoint{2.332014in}{2.611375in}}%
\pgfpathlineto{\pgfqpoint{2.609792in}{2.611375in}}%
\pgfpathlineto{\pgfqpoint{2.609792in}{2.708598in}}%
\pgfpathlineto{\pgfqpoint{2.332014in}{2.708598in}}%
\pgfpathclose%
\pgfusepath{fill}%
\end{pgfscope}%
\begin{pgfscope}%
\definecolor{textcolor}{rgb}{0.000000,0.000000,0.000000}%
\pgfsetstrokecolor{textcolor}%
\pgfsetfillcolor{textcolor}%
\pgftext[x=2.720903in,y=2.611375in,left,base]{\color{textcolor}\rmfamily\fontsize{10.000000}{12.000000}\selectfont fit uncertainty}%
\end{pgfscope}%
\end{pgfpicture}%
\makeatother%
\endgroup%

        }
        \caption{First (or highest) peak of $S(k)$ as function of the temperature.}
        \label{step2_sk_changeT_peaks}
    \end{subfigure}
    \caption{Structure factor $S(k)$ evaluated for different temperatures.}
\end{figure}

\subsection{Variation of $g(r)$ as function of $N_{cells}$ and $r_{cutoff}$ term}
% More atoms

Here the purpose is to obtain some further relations concerning $g(r)$. From figure (\ref{step2_changeN-gofr}) we find that the number of cells doesn't influence at all the pair correlation function. This result is expected, because the positions are simulated within periodic boundary conditions, thus increasing the number of periods is not supposed to change the density.  On the other hand, having access to a larger range of values of $r$, one notices that the radial pair correlation function is well behaved in the limit of $r \to \infty$, hence $g(\infty) = = 1$. This means essentially that a particle is correlated to another if their reciprocal distance is at most $r \approx 20$ \si{\angstrom}.

Changing on the other hand the distance of cut-off, i.e. the maximum distance of interaction among particles, or rather $r > r_{cutoff} \implies V(r) = 0$, we notice some more interesting behaviour concerning the correlation function. Above $r = \sigma = 3.4$ \si{\angstrom}, $g(r)$ doesn't change significantly among different distances of cut-off. However, cutting off the Lennard-Jones potential for $r < \sigma$ it causes the particles not to find an equilibrium state, thus the pair correlation function presents no peak because there are no stable configurations for the particles.
That's what it's observed in figure (\ref{step2_changeR-gofr}).

Notice that $r = \sigma$ is the point where the Lennard-Jones potential reaches zero, all values on the left are positive and all on the right are negative.
Cutting off the values with $r_{cutoff}$, only the values on the left are kept, setting to zero the others.
The case $r_{cutoff} < \sigma$ corresponds to simulate repulsive balls which have no attraction term, because the potential $V(r) \ge 0$. Hence, their interaction is limited to a set of collisions. The more $r_{cutoff}$ decreases, the sharpest the potential gap becomes, involving then a stronger repulsion force. 
On the other hand $r_{cutoff} > \sigma$ admits negative values in the potential, implying that the particles could find an equilibrium state. 
Here the structure factor is not shown because it contains no more informations than the radial pair correlation function in this case.

\begin{figure}
    \begin{subfigure}{0.5\textwidth}
        \resizebox{\textwidth}{!}{
            %% Creator: Matplotlib, PGF backend
%%
%% To include the figure in your LaTeX document, write
%%   \input{<filename>.pgf}
%%
%% Make sure the required packages are loaded in your preamble
%%   \usepackage{pgf}
%%
%% and, on pdftex
%%   \usepackage[utf8]{inputenc}\DeclareUnicodeCharacter{2212}{-}
%%
%% or, on luatex and xetex
%%   \usepackage{unicode-math}
%%
%% Figures using additional raster images can only be included by \input if
%% they are in the same directory as the main LaTeX file. For loading figures
%% from other directories you can use the `import` package
%%   \usepackage{import}
%%
%% and then include the figures with
%%   \import{<path to file>}{<filename>.pgf}
%%
%% Matplotlib used the following preamble
%%   \usepackage[utf8]{inputenc}
%%   \usepackage[T1]{fontenc}
%%   \usepackage{siunitx}
%%
\begingroup%
\makeatletter%
\begin{pgfpicture}%
\pgfpathrectangle{\pgfpointorigin}{\pgfqpoint{4.500000in}{3.700000in}}%
\pgfusepath{use as bounding box, clip}%
\begin{pgfscope}%
\pgfsetbuttcap%
\pgfsetmiterjoin%
\definecolor{currentfill}{rgb}{1.000000,1.000000,1.000000}%
\pgfsetfillcolor{currentfill}%
\pgfsetlinewidth{0.000000pt}%
\definecolor{currentstroke}{rgb}{1.000000,1.000000,1.000000}%
\pgfsetstrokecolor{currentstroke}%
\pgfsetdash{}{0pt}%
\pgfpathmoveto{\pgfqpoint{0.000000in}{0.000000in}}%
\pgfpathlineto{\pgfqpoint{4.500000in}{0.000000in}}%
\pgfpathlineto{\pgfqpoint{4.500000in}{3.700000in}}%
\pgfpathlineto{\pgfqpoint{0.000000in}{3.700000in}}%
\pgfpathclose%
\pgfusepath{fill}%
\end{pgfscope}%
\begin{pgfscope}%
\pgfsetbuttcap%
\pgfsetmiterjoin%
\definecolor{currentfill}{rgb}{1.000000,1.000000,1.000000}%
\pgfsetfillcolor{currentfill}%
\pgfsetlinewidth{0.000000pt}%
\definecolor{currentstroke}{rgb}{0.000000,0.000000,0.000000}%
\pgfsetstrokecolor{currentstroke}%
\pgfsetstrokeopacity{0.000000}%
\pgfsetdash{}{0pt}%
\pgfpathmoveto{\pgfqpoint{0.562500in}{0.407000in}}%
\pgfpathlineto{\pgfqpoint{4.050000in}{0.407000in}}%
\pgfpathlineto{\pgfqpoint{4.050000in}{3.256000in}}%
\pgfpathlineto{\pgfqpoint{0.562500in}{3.256000in}}%
\pgfpathclose%
\pgfusepath{fill}%
\end{pgfscope}%
\begin{pgfscope}%
\pgfsetbuttcap%
\pgfsetroundjoin%
\definecolor{currentfill}{rgb}{0.000000,0.000000,0.000000}%
\pgfsetfillcolor{currentfill}%
\pgfsetlinewidth{0.803000pt}%
\definecolor{currentstroke}{rgb}{0.000000,0.000000,0.000000}%
\pgfsetstrokecolor{currentstroke}%
\pgfsetdash{}{0pt}%
\pgfsys@defobject{currentmarker}{\pgfqpoint{0.000000in}{-0.048611in}}{\pgfqpoint{0.000000in}{0.000000in}}{%
\pgfpathmoveto{\pgfqpoint{0.000000in}{0.000000in}}%
\pgfpathlineto{\pgfqpoint{0.000000in}{-0.048611in}}%
\pgfusepath{stroke,fill}%
}%
\begin{pgfscope}%
\pgfsys@transformshift{0.718374in}{0.407000in}%
\pgfsys@useobject{currentmarker}{}%
\end{pgfscope}%
\end{pgfscope}%
\begin{pgfscope}%
\definecolor{textcolor}{rgb}{0.000000,0.000000,0.000000}%
\pgfsetstrokecolor{textcolor}%
\pgfsetfillcolor{textcolor}%
\pgftext[x=0.718374in,y=0.309778in,,top]{\color{textcolor}\rmfamily\fontsize{10.000000}{12.000000}\selectfont \(\displaystyle {0}\)}%
\end{pgfscope}%
\begin{pgfscope}%
\pgfsetbuttcap%
\pgfsetroundjoin%
\definecolor{currentfill}{rgb}{0.000000,0.000000,0.000000}%
\pgfsetfillcolor{currentfill}%
\pgfsetlinewidth{0.803000pt}%
\definecolor{currentstroke}{rgb}{0.000000,0.000000,0.000000}%
\pgfsetstrokecolor{currentstroke}%
\pgfsetdash{}{0pt}%
\pgfsys@defobject{currentmarker}{\pgfqpoint{0.000000in}{-0.048611in}}{\pgfqpoint{0.000000in}{0.000000in}}{%
\pgfpathmoveto{\pgfqpoint{0.000000in}{0.000000in}}%
\pgfpathlineto{\pgfqpoint{0.000000in}{-0.048611in}}%
\pgfusepath{stroke,fill}%
}%
\begin{pgfscope}%
\pgfsys@transformshift{1.175331in}{0.407000in}%
\pgfsys@useobject{currentmarker}{}%
\end{pgfscope}%
\end{pgfscope}%
\begin{pgfscope}%
\definecolor{textcolor}{rgb}{0.000000,0.000000,0.000000}%
\pgfsetstrokecolor{textcolor}%
\pgfsetfillcolor{textcolor}%
\pgftext[x=1.175331in,y=0.309778in,,top]{\color{textcolor}\rmfamily\fontsize{10.000000}{12.000000}\selectfont \(\displaystyle {5}\)}%
\end{pgfscope}%
\begin{pgfscope}%
\pgfsetbuttcap%
\pgfsetroundjoin%
\definecolor{currentfill}{rgb}{0.000000,0.000000,0.000000}%
\pgfsetfillcolor{currentfill}%
\pgfsetlinewidth{0.803000pt}%
\definecolor{currentstroke}{rgb}{0.000000,0.000000,0.000000}%
\pgfsetstrokecolor{currentstroke}%
\pgfsetdash{}{0pt}%
\pgfsys@defobject{currentmarker}{\pgfqpoint{0.000000in}{-0.048611in}}{\pgfqpoint{0.000000in}{0.000000in}}{%
\pgfpathmoveto{\pgfqpoint{0.000000in}{0.000000in}}%
\pgfpathlineto{\pgfqpoint{0.000000in}{-0.048611in}}%
\pgfusepath{stroke,fill}%
}%
\begin{pgfscope}%
\pgfsys@transformshift{1.632287in}{0.407000in}%
\pgfsys@useobject{currentmarker}{}%
\end{pgfscope}%
\end{pgfscope}%
\begin{pgfscope}%
\definecolor{textcolor}{rgb}{0.000000,0.000000,0.000000}%
\pgfsetstrokecolor{textcolor}%
\pgfsetfillcolor{textcolor}%
\pgftext[x=1.632287in,y=0.309778in,,top]{\color{textcolor}\rmfamily\fontsize{10.000000}{12.000000}\selectfont \(\displaystyle {10}\)}%
\end{pgfscope}%
\begin{pgfscope}%
\pgfsetbuttcap%
\pgfsetroundjoin%
\definecolor{currentfill}{rgb}{0.000000,0.000000,0.000000}%
\pgfsetfillcolor{currentfill}%
\pgfsetlinewidth{0.803000pt}%
\definecolor{currentstroke}{rgb}{0.000000,0.000000,0.000000}%
\pgfsetstrokecolor{currentstroke}%
\pgfsetdash{}{0pt}%
\pgfsys@defobject{currentmarker}{\pgfqpoint{0.000000in}{-0.048611in}}{\pgfqpoint{0.000000in}{0.000000in}}{%
\pgfpathmoveto{\pgfqpoint{0.000000in}{0.000000in}}%
\pgfpathlineto{\pgfqpoint{0.000000in}{-0.048611in}}%
\pgfusepath{stroke,fill}%
}%
\begin{pgfscope}%
\pgfsys@transformshift{2.089244in}{0.407000in}%
\pgfsys@useobject{currentmarker}{}%
\end{pgfscope}%
\end{pgfscope}%
\begin{pgfscope}%
\definecolor{textcolor}{rgb}{0.000000,0.000000,0.000000}%
\pgfsetstrokecolor{textcolor}%
\pgfsetfillcolor{textcolor}%
\pgftext[x=2.089244in,y=0.309778in,,top]{\color{textcolor}\rmfamily\fontsize{10.000000}{12.000000}\selectfont \(\displaystyle {15}\)}%
\end{pgfscope}%
\begin{pgfscope}%
\pgfsetbuttcap%
\pgfsetroundjoin%
\definecolor{currentfill}{rgb}{0.000000,0.000000,0.000000}%
\pgfsetfillcolor{currentfill}%
\pgfsetlinewidth{0.803000pt}%
\definecolor{currentstroke}{rgb}{0.000000,0.000000,0.000000}%
\pgfsetstrokecolor{currentstroke}%
\pgfsetdash{}{0pt}%
\pgfsys@defobject{currentmarker}{\pgfqpoint{0.000000in}{-0.048611in}}{\pgfqpoint{0.000000in}{0.000000in}}{%
\pgfpathmoveto{\pgfqpoint{0.000000in}{0.000000in}}%
\pgfpathlineto{\pgfqpoint{0.000000in}{-0.048611in}}%
\pgfusepath{stroke,fill}%
}%
\begin{pgfscope}%
\pgfsys@transformshift{2.546201in}{0.407000in}%
\pgfsys@useobject{currentmarker}{}%
\end{pgfscope}%
\end{pgfscope}%
\begin{pgfscope}%
\definecolor{textcolor}{rgb}{0.000000,0.000000,0.000000}%
\pgfsetstrokecolor{textcolor}%
\pgfsetfillcolor{textcolor}%
\pgftext[x=2.546201in,y=0.309778in,,top]{\color{textcolor}\rmfamily\fontsize{10.000000}{12.000000}\selectfont \(\displaystyle {20}\)}%
\end{pgfscope}%
\begin{pgfscope}%
\pgfsetbuttcap%
\pgfsetroundjoin%
\definecolor{currentfill}{rgb}{0.000000,0.000000,0.000000}%
\pgfsetfillcolor{currentfill}%
\pgfsetlinewidth{0.803000pt}%
\definecolor{currentstroke}{rgb}{0.000000,0.000000,0.000000}%
\pgfsetstrokecolor{currentstroke}%
\pgfsetdash{}{0pt}%
\pgfsys@defobject{currentmarker}{\pgfqpoint{0.000000in}{-0.048611in}}{\pgfqpoint{0.000000in}{0.000000in}}{%
\pgfpathmoveto{\pgfqpoint{0.000000in}{0.000000in}}%
\pgfpathlineto{\pgfqpoint{0.000000in}{-0.048611in}}%
\pgfusepath{stroke,fill}%
}%
\begin{pgfscope}%
\pgfsys@transformshift{3.003157in}{0.407000in}%
\pgfsys@useobject{currentmarker}{}%
\end{pgfscope}%
\end{pgfscope}%
\begin{pgfscope}%
\definecolor{textcolor}{rgb}{0.000000,0.000000,0.000000}%
\pgfsetstrokecolor{textcolor}%
\pgfsetfillcolor{textcolor}%
\pgftext[x=3.003157in,y=0.309778in,,top]{\color{textcolor}\rmfamily\fontsize{10.000000}{12.000000}\selectfont \(\displaystyle {25}\)}%
\end{pgfscope}%
\begin{pgfscope}%
\pgfsetbuttcap%
\pgfsetroundjoin%
\definecolor{currentfill}{rgb}{0.000000,0.000000,0.000000}%
\pgfsetfillcolor{currentfill}%
\pgfsetlinewidth{0.803000pt}%
\definecolor{currentstroke}{rgb}{0.000000,0.000000,0.000000}%
\pgfsetstrokecolor{currentstroke}%
\pgfsetdash{}{0pt}%
\pgfsys@defobject{currentmarker}{\pgfqpoint{0.000000in}{-0.048611in}}{\pgfqpoint{0.000000in}{0.000000in}}{%
\pgfpathmoveto{\pgfqpoint{0.000000in}{0.000000in}}%
\pgfpathlineto{\pgfqpoint{0.000000in}{-0.048611in}}%
\pgfusepath{stroke,fill}%
}%
\begin{pgfscope}%
\pgfsys@transformshift{3.460114in}{0.407000in}%
\pgfsys@useobject{currentmarker}{}%
\end{pgfscope}%
\end{pgfscope}%
\begin{pgfscope}%
\definecolor{textcolor}{rgb}{0.000000,0.000000,0.000000}%
\pgfsetstrokecolor{textcolor}%
\pgfsetfillcolor{textcolor}%
\pgftext[x=3.460114in,y=0.309778in,,top]{\color{textcolor}\rmfamily\fontsize{10.000000}{12.000000}\selectfont \(\displaystyle {30}\)}%
\end{pgfscope}%
\begin{pgfscope}%
\pgfsetbuttcap%
\pgfsetroundjoin%
\definecolor{currentfill}{rgb}{0.000000,0.000000,0.000000}%
\pgfsetfillcolor{currentfill}%
\pgfsetlinewidth{0.803000pt}%
\definecolor{currentstroke}{rgb}{0.000000,0.000000,0.000000}%
\pgfsetstrokecolor{currentstroke}%
\pgfsetdash{}{0pt}%
\pgfsys@defobject{currentmarker}{\pgfqpoint{0.000000in}{-0.048611in}}{\pgfqpoint{0.000000in}{0.000000in}}{%
\pgfpathmoveto{\pgfqpoint{0.000000in}{0.000000in}}%
\pgfpathlineto{\pgfqpoint{0.000000in}{-0.048611in}}%
\pgfusepath{stroke,fill}%
}%
\begin{pgfscope}%
\pgfsys@transformshift{3.917071in}{0.407000in}%
\pgfsys@useobject{currentmarker}{}%
\end{pgfscope}%
\end{pgfscope}%
\begin{pgfscope}%
\definecolor{textcolor}{rgb}{0.000000,0.000000,0.000000}%
\pgfsetstrokecolor{textcolor}%
\pgfsetfillcolor{textcolor}%
\pgftext[x=3.917071in,y=0.309778in,,top]{\color{textcolor}\rmfamily\fontsize{10.000000}{12.000000}\selectfont \(\displaystyle {35}\)}%
\end{pgfscope}%
\begin{pgfscope}%
\definecolor{textcolor}{rgb}{0.000000,0.000000,0.000000}%
\pgfsetstrokecolor{textcolor}%
\pgfsetfillcolor{textcolor}%
\pgftext[x=2.306250in,y=0.131567in,,top]{\color{textcolor}\rmfamily\fontsize{10.000000}{12.000000}\selectfont \(\displaystyle r\) [\si{\angstrom}]}%
\end{pgfscope}%
\begin{pgfscope}%
\pgfsetbuttcap%
\pgfsetroundjoin%
\definecolor{currentfill}{rgb}{0.000000,0.000000,0.000000}%
\pgfsetfillcolor{currentfill}%
\pgfsetlinewidth{0.803000pt}%
\definecolor{currentstroke}{rgb}{0.000000,0.000000,0.000000}%
\pgfsetstrokecolor{currentstroke}%
\pgfsetdash{}{0pt}%
\pgfsys@defobject{currentmarker}{\pgfqpoint{-0.048611in}{0.000000in}}{\pgfqpoint{0.000000in}{0.000000in}}{%
\pgfpathmoveto{\pgfqpoint{0.000000in}{0.000000in}}%
\pgfpathlineto{\pgfqpoint{-0.048611in}{0.000000in}}%
\pgfusepath{stroke,fill}%
}%
\begin{pgfscope}%
\pgfsys@transformshift{0.562500in}{0.536500in}%
\pgfsys@useobject{currentmarker}{}%
\end{pgfscope}%
\end{pgfscope}%
\begin{pgfscope}%
\definecolor{textcolor}{rgb}{0.000000,0.000000,0.000000}%
\pgfsetstrokecolor{textcolor}%
\pgfsetfillcolor{textcolor}%
\pgftext[x=0.287808in, y=0.488672in, left, base]{\color{textcolor}\rmfamily\fontsize{10.000000}{12.000000}\selectfont \(\displaystyle {0.0}\)}%
\end{pgfscope}%
\begin{pgfscope}%
\pgfsetbuttcap%
\pgfsetroundjoin%
\definecolor{currentfill}{rgb}{0.000000,0.000000,0.000000}%
\pgfsetfillcolor{currentfill}%
\pgfsetlinewidth{0.803000pt}%
\definecolor{currentstroke}{rgb}{0.000000,0.000000,0.000000}%
\pgfsetstrokecolor{currentstroke}%
\pgfsetdash{}{0pt}%
\pgfsys@defobject{currentmarker}{\pgfqpoint{-0.048611in}{0.000000in}}{\pgfqpoint{0.000000in}{0.000000in}}{%
\pgfpathmoveto{\pgfqpoint{0.000000in}{0.000000in}}%
\pgfpathlineto{\pgfqpoint{-0.048611in}{0.000000in}}%
\pgfusepath{stroke,fill}%
}%
\begin{pgfscope}%
\pgfsys@transformshift{0.562500in}{0.994710in}%
\pgfsys@useobject{currentmarker}{}%
\end{pgfscope}%
\end{pgfscope}%
\begin{pgfscope}%
\definecolor{textcolor}{rgb}{0.000000,0.000000,0.000000}%
\pgfsetstrokecolor{textcolor}%
\pgfsetfillcolor{textcolor}%
\pgftext[x=0.287808in, y=0.946882in, left, base]{\color{textcolor}\rmfamily\fontsize{10.000000}{12.000000}\selectfont \(\displaystyle {0.5}\)}%
\end{pgfscope}%
\begin{pgfscope}%
\pgfsetbuttcap%
\pgfsetroundjoin%
\definecolor{currentfill}{rgb}{0.000000,0.000000,0.000000}%
\pgfsetfillcolor{currentfill}%
\pgfsetlinewidth{0.803000pt}%
\definecolor{currentstroke}{rgb}{0.000000,0.000000,0.000000}%
\pgfsetstrokecolor{currentstroke}%
\pgfsetdash{}{0pt}%
\pgfsys@defobject{currentmarker}{\pgfqpoint{-0.048611in}{0.000000in}}{\pgfqpoint{0.000000in}{0.000000in}}{%
\pgfpathmoveto{\pgfqpoint{0.000000in}{0.000000in}}%
\pgfpathlineto{\pgfqpoint{-0.048611in}{0.000000in}}%
\pgfusepath{stroke,fill}%
}%
\begin{pgfscope}%
\pgfsys@transformshift{0.562500in}{1.452919in}%
\pgfsys@useobject{currentmarker}{}%
\end{pgfscope}%
\end{pgfscope}%
\begin{pgfscope}%
\definecolor{textcolor}{rgb}{0.000000,0.000000,0.000000}%
\pgfsetstrokecolor{textcolor}%
\pgfsetfillcolor{textcolor}%
\pgftext[x=0.287808in, y=1.405091in, left, base]{\color{textcolor}\rmfamily\fontsize{10.000000}{12.000000}\selectfont \(\displaystyle {1.0}\)}%
\end{pgfscope}%
\begin{pgfscope}%
\pgfsetbuttcap%
\pgfsetroundjoin%
\definecolor{currentfill}{rgb}{0.000000,0.000000,0.000000}%
\pgfsetfillcolor{currentfill}%
\pgfsetlinewidth{0.803000pt}%
\definecolor{currentstroke}{rgb}{0.000000,0.000000,0.000000}%
\pgfsetstrokecolor{currentstroke}%
\pgfsetdash{}{0pt}%
\pgfsys@defobject{currentmarker}{\pgfqpoint{-0.048611in}{0.000000in}}{\pgfqpoint{0.000000in}{0.000000in}}{%
\pgfpathmoveto{\pgfqpoint{0.000000in}{0.000000in}}%
\pgfpathlineto{\pgfqpoint{-0.048611in}{0.000000in}}%
\pgfusepath{stroke,fill}%
}%
\begin{pgfscope}%
\pgfsys@transformshift{0.562500in}{1.911129in}%
\pgfsys@useobject{currentmarker}{}%
\end{pgfscope}%
\end{pgfscope}%
\begin{pgfscope}%
\definecolor{textcolor}{rgb}{0.000000,0.000000,0.000000}%
\pgfsetstrokecolor{textcolor}%
\pgfsetfillcolor{textcolor}%
\pgftext[x=0.287808in, y=1.863301in, left, base]{\color{textcolor}\rmfamily\fontsize{10.000000}{12.000000}\selectfont \(\displaystyle {1.5}\)}%
\end{pgfscope}%
\begin{pgfscope}%
\pgfsetbuttcap%
\pgfsetroundjoin%
\definecolor{currentfill}{rgb}{0.000000,0.000000,0.000000}%
\pgfsetfillcolor{currentfill}%
\pgfsetlinewidth{0.803000pt}%
\definecolor{currentstroke}{rgb}{0.000000,0.000000,0.000000}%
\pgfsetstrokecolor{currentstroke}%
\pgfsetdash{}{0pt}%
\pgfsys@defobject{currentmarker}{\pgfqpoint{-0.048611in}{0.000000in}}{\pgfqpoint{0.000000in}{0.000000in}}{%
\pgfpathmoveto{\pgfqpoint{0.000000in}{0.000000in}}%
\pgfpathlineto{\pgfqpoint{-0.048611in}{0.000000in}}%
\pgfusepath{stroke,fill}%
}%
\begin{pgfscope}%
\pgfsys@transformshift{0.562500in}{2.369339in}%
\pgfsys@useobject{currentmarker}{}%
\end{pgfscope}%
\end{pgfscope}%
\begin{pgfscope}%
\definecolor{textcolor}{rgb}{0.000000,0.000000,0.000000}%
\pgfsetstrokecolor{textcolor}%
\pgfsetfillcolor{textcolor}%
\pgftext[x=0.287808in, y=2.321511in, left, base]{\color{textcolor}\rmfamily\fontsize{10.000000}{12.000000}\selectfont \(\displaystyle {2.0}\)}%
\end{pgfscope}%
\begin{pgfscope}%
\pgfsetbuttcap%
\pgfsetroundjoin%
\definecolor{currentfill}{rgb}{0.000000,0.000000,0.000000}%
\pgfsetfillcolor{currentfill}%
\pgfsetlinewidth{0.803000pt}%
\definecolor{currentstroke}{rgb}{0.000000,0.000000,0.000000}%
\pgfsetstrokecolor{currentstroke}%
\pgfsetdash{}{0pt}%
\pgfsys@defobject{currentmarker}{\pgfqpoint{-0.048611in}{0.000000in}}{\pgfqpoint{0.000000in}{0.000000in}}{%
\pgfpathmoveto{\pgfqpoint{0.000000in}{0.000000in}}%
\pgfpathlineto{\pgfqpoint{-0.048611in}{0.000000in}}%
\pgfusepath{stroke,fill}%
}%
\begin{pgfscope}%
\pgfsys@transformshift{0.562500in}{2.827548in}%
\pgfsys@useobject{currentmarker}{}%
\end{pgfscope}%
\end{pgfscope}%
\begin{pgfscope}%
\definecolor{textcolor}{rgb}{0.000000,0.000000,0.000000}%
\pgfsetstrokecolor{textcolor}%
\pgfsetfillcolor{textcolor}%
\pgftext[x=0.287808in, y=2.779720in, left, base]{\color{textcolor}\rmfamily\fontsize{10.000000}{12.000000}\selectfont \(\displaystyle {2.5}\)}%
\end{pgfscope}%
\begin{pgfscope}%
\definecolor{textcolor}{rgb}{0.000000,0.000000,0.000000}%
\pgfsetstrokecolor{textcolor}%
\pgfsetfillcolor{textcolor}%
\pgftext[x=0.232253in,y=1.831500in,,bottom,rotate=90.000000]{\color{textcolor}\rmfamily\fontsize{10.000000}{12.000000}\selectfont \(\displaystyle g(r)\)}%
\end{pgfscope}%
\begin{pgfscope}%
\pgfpathrectangle{\pgfqpoint{0.562500in}{0.407000in}}{\pgfqpoint{3.487500in}{2.849000in}}%
\pgfusepath{clip}%
\pgfsetrectcap%
\pgfsetroundjoin%
\pgfsetlinewidth{1.505625pt}%
\definecolor{currentstroke}{rgb}{0.121569,0.466667,0.705882}%
\pgfsetstrokecolor{currentstroke}%
\pgfsetdash{}{0pt}%
\pgfpathmoveto{\pgfqpoint{0.721023in}{0.536500in}}%
\pgfpathlineto{\pgfqpoint{0.996484in}{0.536653in}}%
\pgfpathlineto{\pgfqpoint{1.001781in}{0.539386in}}%
\pgfpathlineto{\pgfqpoint{1.007079in}{0.558719in}}%
\pgfpathlineto{\pgfqpoint{1.012376in}{0.626449in}}%
\pgfpathlineto{\pgfqpoint{1.017673in}{0.798064in}}%
\pgfpathlineto{\pgfqpoint{1.022971in}{1.115550in}}%
\pgfpathlineto{\pgfqpoint{1.044160in}{2.842693in}}%
\pgfpathlineto{\pgfqpoint{1.049457in}{3.040647in}}%
\pgfpathlineto{\pgfqpoint{1.054755in}{3.126500in}}%
\pgfpathlineto{\pgfqpoint{1.060052in}{3.099461in}}%
\pgfpathlineto{\pgfqpoint{1.065349in}{3.005644in}}%
\pgfpathlineto{\pgfqpoint{1.070647in}{2.883392in}}%
\pgfpathlineto{\pgfqpoint{1.102431in}{1.928993in}}%
\pgfpathlineto{\pgfqpoint{1.118323in}{1.603670in}}%
\pgfpathlineto{\pgfqpoint{1.134215in}{1.389870in}}%
\pgfpathlineto{\pgfqpoint{1.144809in}{1.293293in}}%
\pgfpathlineto{\pgfqpoint{1.155404in}{1.223492in}}%
\pgfpathlineto{\pgfqpoint{1.160701in}{1.191325in}}%
\pgfpathlineto{\pgfqpoint{1.165999in}{1.175223in}}%
\pgfpathlineto{\pgfqpoint{1.176593in}{1.130149in}}%
\pgfpathlineto{\pgfqpoint{1.181891in}{1.122529in}}%
\pgfpathlineto{\pgfqpoint{1.187188in}{1.108802in}}%
\pgfpathlineto{\pgfqpoint{1.192485in}{1.108145in}}%
\pgfpathlineto{\pgfqpoint{1.197783in}{1.110081in}}%
\pgfpathlineto{\pgfqpoint{1.203080in}{1.108172in}}%
\pgfpathlineto{\pgfqpoint{1.208377in}{1.101960in}}%
\pgfpathlineto{\pgfqpoint{1.213675in}{1.103010in}}%
\pgfpathlineto{\pgfqpoint{1.218972in}{1.111579in}}%
\pgfpathlineto{\pgfqpoint{1.224269in}{1.124092in}}%
\pgfpathlineto{\pgfqpoint{1.229567in}{1.129230in}}%
\pgfpathlineto{\pgfqpoint{1.234864in}{1.145544in}}%
\pgfpathlineto{\pgfqpoint{1.261351in}{1.250651in}}%
\pgfpathlineto{\pgfqpoint{1.266648in}{1.278858in}}%
\pgfpathlineto{\pgfqpoint{1.277243in}{1.344466in}}%
\pgfpathlineto{\pgfqpoint{1.314324in}{1.544849in}}%
\pgfpathlineto{\pgfqpoint{1.330216in}{1.608045in}}%
\pgfpathlineto{\pgfqpoint{1.346108in}{1.662156in}}%
\pgfpathlineto{\pgfqpoint{1.351406in}{1.673095in}}%
\pgfpathlineto{\pgfqpoint{1.356703in}{1.680640in}}%
\pgfpathlineto{\pgfqpoint{1.362000in}{1.680986in}}%
\pgfpathlineto{\pgfqpoint{1.367298in}{1.690815in}}%
\pgfpathlineto{\pgfqpoint{1.372595in}{1.689601in}}%
\pgfpathlineto{\pgfqpoint{1.383190in}{1.672096in}}%
\pgfpathlineto{\pgfqpoint{1.393784in}{1.646919in}}%
\pgfpathlineto{\pgfqpoint{1.409676in}{1.584381in}}%
\pgfpathlineto{\pgfqpoint{1.430866in}{1.496266in}}%
\pgfpathlineto{\pgfqpoint{1.436163in}{1.469459in}}%
\pgfpathlineto{\pgfqpoint{1.467947in}{1.360809in}}%
\pgfpathlineto{\pgfqpoint{1.473244in}{1.348763in}}%
\pgfpathlineto{\pgfqpoint{1.478542in}{1.343136in}}%
\pgfpathlineto{\pgfqpoint{1.483839in}{1.330101in}}%
\pgfpathlineto{\pgfqpoint{1.489136in}{1.326409in}}%
\pgfpathlineto{\pgfqpoint{1.494434in}{1.317690in}}%
\pgfpathlineto{\pgfqpoint{1.499731in}{1.317260in}}%
\pgfpathlineto{\pgfqpoint{1.505028in}{1.318533in}}%
\pgfpathlineto{\pgfqpoint{1.510326in}{1.314849in}}%
\pgfpathlineto{\pgfqpoint{1.515623in}{1.319582in}}%
\pgfpathlineto{\pgfqpoint{1.536812in}{1.346648in}}%
\pgfpathlineto{\pgfqpoint{1.547407in}{1.364850in}}%
\pgfpathlineto{\pgfqpoint{1.552704in}{1.377447in}}%
\pgfpathlineto{\pgfqpoint{1.558002in}{1.384107in}}%
\pgfpathlineto{\pgfqpoint{1.563299in}{1.394615in}}%
\pgfpathlineto{\pgfqpoint{1.568596in}{1.410735in}}%
\pgfpathlineto{\pgfqpoint{1.595083in}{1.468063in}}%
\pgfpathlineto{\pgfqpoint{1.600380in}{1.484581in}}%
\pgfpathlineto{\pgfqpoint{1.605678in}{1.497354in}}%
\pgfpathlineto{\pgfqpoint{1.610975in}{1.502754in}}%
\pgfpathlineto{\pgfqpoint{1.616272in}{1.506137in}}%
\pgfpathlineto{\pgfqpoint{1.621570in}{1.515714in}}%
\pgfpathlineto{\pgfqpoint{1.632164in}{1.528562in}}%
\pgfpathlineto{\pgfqpoint{1.658651in}{1.537167in}}%
\pgfpathlineto{\pgfqpoint{1.669246in}{1.527511in}}%
\pgfpathlineto{\pgfqpoint{1.674543in}{1.525362in}}%
\pgfpathlineto{\pgfqpoint{1.685138in}{1.517436in}}%
\pgfpathlineto{\pgfqpoint{1.695732in}{1.503924in}}%
\pgfpathlineto{\pgfqpoint{1.701030in}{1.498990in}}%
\pgfpathlineto{\pgfqpoint{1.711624in}{1.483036in}}%
\pgfpathlineto{\pgfqpoint{1.722219in}{1.470694in}}%
\pgfpathlineto{\pgfqpoint{1.727516in}{1.468785in}}%
\pgfpathlineto{\pgfqpoint{1.732814in}{1.455931in}}%
\pgfpathlineto{\pgfqpoint{1.738111in}{1.447458in}}%
\pgfpathlineto{\pgfqpoint{1.743408in}{1.446065in}}%
\pgfpathlineto{\pgfqpoint{1.748706in}{1.442818in}}%
\pgfpathlineto{\pgfqpoint{1.754003in}{1.432708in}}%
\pgfpathlineto{\pgfqpoint{1.759300in}{1.427123in}}%
\pgfpathlineto{\pgfqpoint{1.764598in}{1.425004in}}%
\pgfpathlineto{\pgfqpoint{1.769895in}{1.421181in}}%
\pgfpathlineto{\pgfqpoint{1.775192in}{1.415593in}}%
\pgfpathlineto{\pgfqpoint{1.785787in}{1.408581in}}%
\pgfpathlineto{\pgfqpoint{1.801679in}{1.404501in}}%
\pgfpathlineto{\pgfqpoint{1.806976in}{1.404286in}}%
\pgfpathlineto{\pgfqpoint{1.812274in}{1.402421in}}%
\pgfpathlineto{\pgfqpoint{1.828166in}{1.410432in}}%
\pgfpathlineto{\pgfqpoint{1.833463in}{1.411157in}}%
\pgfpathlineto{\pgfqpoint{1.838760in}{1.414235in}}%
\pgfpathlineto{\pgfqpoint{1.865247in}{1.438975in}}%
\pgfpathlineto{\pgfqpoint{1.875842in}{1.446958in}}%
\pgfpathlineto{\pgfqpoint{1.881139in}{1.453362in}}%
\pgfpathlineto{\pgfqpoint{1.897031in}{1.466118in}}%
\pgfpathlineto{\pgfqpoint{1.902328in}{1.472335in}}%
\pgfpathlineto{\pgfqpoint{1.912923in}{1.480163in}}%
\pgfpathlineto{\pgfqpoint{1.923518in}{1.483388in}}%
\pgfpathlineto{\pgfqpoint{1.928815in}{1.485965in}}%
\pgfpathlineto{\pgfqpoint{1.934112in}{1.484120in}}%
\pgfpathlineto{\pgfqpoint{1.939410in}{1.487873in}}%
\pgfpathlineto{\pgfqpoint{1.950004in}{1.486116in}}%
\pgfpathlineto{\pgfqpoint{1.960599in}{1.480281in}}%
\pgfpathlineto{\pgfqpoint{1.965896in}{1.481298in}}%
\pgfpathlineto{\pgfqpoint{1.987086in}{1.475351in}}%
\pgfpathlineto{\pgfqpoint{1.997680in}{1.466705in}}%
\pgfpathlineto{\pgfqpoint{2.018870in}{1.459661in}}%
\pgfpathlineto{\pgfqpoint{2.034762in}{1.449353in}}%
\pgfpathlineto{\pgfqpoint{2.040059in}{1.448445in}}%
\pgfpathlineto{\pgfqpoint{2.045356in}{1.445191in}}%
\pgfpathlineto{\pgfqpoint{2.050654in}{1.443322in}}%
\pgfpathlineto{\pgfqpoint{2.055951in}{1.442742in}}%
\pgfpathlineto{\pgfqpoint{2.061248in}{1.440711in}}%
\pgfpathlineto{\pgfqpoint{2.066546in}{1.441276in}}%
\pgfpathlineto{\pgfqpoint{2.082438in}{1.435261in}}%
\pgfpathlineto{\pgfqpoint{2.087735in}{1.433792in}}%
\pgfpathlineto{\pgfqpoint{2.093032in}{1.434827in}}%
\pgfpathlineto{\pgfqpoint{2.098330in}{1.432490in}}%
\pgfpathlineto{\pgfqpoint{2.103627in}{1.434999in}}%
\pgfpathlineto{\pgfqpoint{2.108924in}{1.433498in}}%
\pgfpathlineto{\pgfqpoint{2.114222in}{1.433812in}}%
\pgfpathlineto{\pgfqpoint{2.119519in}{1.436517in}}%
\pgfpathlineto{\pgfqpoint{2.135411in}{1.437799in}}%
\pgfpathlineto{\pgfqpoint{2.146006in}{1.446380in}}%
\pgfpathlineto{\pgfqpoint{2.151303in}{1.445058in}}%
\pgfpathlineto{\pgfqpoint{2.161898in}{1.450683in}}%
\pgfpathlineto{\pgfqpoint{2.167195in}{1.452547in}}%
\pgfpathlineto{\pgfqpoint{2.172492in}{1.452592in}}%
\pgfpathlineto{\pgfqpoint{2.177790in}{1.454542in}}%
\pgfpathlineto{\pgfqpoint{2.183087in}{1.458643in}}%
\pgfpathlineto{\pgfqpoint{2.188384in}{1.458355in}}%
\pgfpathlineto{\pgfqpoint{2.193682in}{1.461138in}}%
\pgfpathlineto{\pgfqpoint{2.198979in}{1.462634in}}%
\pgfpathlineto{\pgfqpoint{2.204276in}{1.461638in}}%
\pgfpathlineto{\pgfqpoint{2.209574in}{1.464610in}}%
\pgfpathlineto{\pgfqpoint{2.220168in}{1.467916in}}%
\pgfpathlineto{\pgfqpoint{2.225466in}{1.468040in}}%
\pgfpathlineto{\pgfqpoint{2.230763in}{1.466739in}}%
\pgfpathlineto{\pgfqpoint{2.241358in}{1.467359in}}%
\pgfpathlineto{\pgfqpoint{2.246655in}{1.464740in}}%
\pgfpathlineto{\pgfqpoint{2.251952in}{1.466650in}}%
\pgfpathlineto{\pgfqpoint{2.257250in}{1.464644in}}%
\pgfpathlineto{\pgfqpoint{2.262547in}{1.467211in}}%
\pgfpathlineto{\pgfqpoint{2.273142in}{1.464181in}}%
\pgfpathlineto{\pgfqpoint{2.283736in}{1.459610in}}%
\pgfpathlineto{\pgfqpoint{2.294331in}{1.457982in}}%
\pgfpathlineto{\pgfqpoint{2.299628in}{1.454672in}}%
\pgfpathlineto{\pgfqpoint{2.304926in}{1.452854in}}%
\pgfpathlineto{\pgfqpoint{2.304926in}{1.452854in}}%
\pgfusepath{stroke}%
\end{pgfscope}%
\begin{pgfscope}%
\pgfpathrectangle{\pgfqpoint{0.562500in}{0.407000in}}{\pgfqpoint{3.487500in}{2.849000in}}%
\pgfusepath{clip}%
\pgfsetbuttcap%
\pgfsetroundjoin%
\definecolor{currentfill}{rgb}{0.000000,0.000000,0.000000}%
\pgfsetfillcolor{currentfill}%
\pgfsetlinewidth{1.003750pt}%
\definecolor{currentstroke}{rgb}{0.000000,0.000000,0.000000}%
\pgfsetstrokecolor{currentstroke}%
\pgfsetdash{}{0pt}%
\pgfsys@defobject{currentmarker}{\pgfqpoint{-0.041667in}{-0.041667in}}{\pgfqpoint{0.041667in}{0.041667in}}{%
\pgfpathmoveto{\pgfqpoint{0.000000in}{-0.041667in}}%
\pgfpathcurveto{\pgfqpoint{0.011050in}{-0.041667in}}{\pgfqpoint{0.021649in}{-0.037276in}}{\pgfqpoint{0.029463in}{-0.029463in}}%
\pgfpathcurveto{\pgfqpoint{0.037276in}{-0.021649in}}{\pgfqpoint{0.041667in}{-0.011050in}}{\pgfqpoint{0.041667in}{0.000000in}}%
\pgfpathcurveto{\pgfqpoint{0.041667in}{0.011050in}}{\pgfqpoint{0.037276in}{0.021649in}}{\pgfqpoint{0.029463in}{0.029463in}}%
\pgfpathcurveto{\pgfqpoint{0.021649in}{0.037276in}}{\pgfqpoint{0.011050in}{0.041667in}}{\pgfqpoint{0.000000in}{0.041667in}}%
\pgfpathcurveto{\pgfqpoint{-0.011050in}{0.041667in}}{\pgfqpoint{-0.021649in}{0.037276in}}{\pgfqpoint{-0.029463in}{0.029463in}}%
\pgfpathcurveto{\pgfqpoint{-0.037276in}{0.021649in}}{\pgfqpoint{-0.041667in}{0.011050in}}{\pgfqpoint{-0.041667in}{0.000000in}}%
\pgfpathcurveto{\pgfqpoint{-0.041667in}{-0.011050in}}{\pgfqpoint{-0.037276in}{-0.021649in}}{\pgfqpoint{-0.029463in}{-0.029463in}}%
\pgfpathcurveto{\pgfqpoint{-0.021649in}{-0.037276in}}{\pgfqpoint{-0.011050in}{-0.041667in}}{\pgfqpoint{0.000000in}{-0.041667in}}%
\pgfpathclose%
\pgfusepath{stroke,fill}%
}%
\begin{pgfscope}%
\pgfsys@transformshift{1.060052in}{3.099461in}%
\pgfsys@useobject{currentmarker}{}%
\end{pgfscope}%
\begin{pgfscope}%
\pgfsys@transformshift{1.367298in}{1.690815in}%
\pgfsys@useobject{currentmarker}{}%
\end{pgfscope}%
\begin{pgfscope}%
\pgfsys@transformshift{1.653354in}{1.535588in}%
\pgfsys@useobject{currentmarker}{}%
\end{pgfscope}%
\end{pgfscope}%
\begin{pgfscope}%
\pgfpathrectangle{\pgfqpoint{0.562500in}{0.407000in}}{\pgfqpoint{3.487500in}{2.849000in}}%
\pgfusepath{clip}%
\pgfsetrectcap%
\pgfsetroundjoin%
\pgfsetlinewidth{1.505625pt}%
\definecolor{currentstroke}{rgb}{1.000000,0.498039,0.054902}%
\pgfsetstrokecolor{currentstroke}%
\pgfsetdash{}{0pt}%
\pgfpathmoveto{\pgfqpoint{0.723671in}{0.536500in}}%
\pgfpathlineto{\pgfqpoint{0.988538in}{0.536508in}}%
\pgfpathlineto{\pgfqpoint{0.999133in}{0.537839in}}%
\pgfpathlineto{\pgfqpoint{1.009727in}{0.585686in}}%
\pgfpathlineto{\pgfqpoint{1.020322in}{0.938596in}}%
\pgfpathlineto{\pgfqpoint{1.041511in}{2.666536in}}%
\pgfpathlineto{\pgfqpoint{1.052106in}{3.107128in}}%
\pgfpathlineto{\pgfqpoint{1.062701in}{3.081130in}}%
\pgfpathlineto{\pgfqpoint{1.073295in}{2.806037in}}%
\pgfpathlineto{\pgfqpoint{1.094485in}{2.134001in}}%
\pgfpathlineto{\pgfqpoint{1.105079in}{1.864102in}}%
\pgfpathlineto{\pgfqpoint{1.115674in}{1.648944in}}%
\pgfpathlineto{\pgfqpoint{1.126269in}{1.487916in}}%
\pgfpathlineto{\pgfqpoint{1.136863in}{1.364704in}}%
\pgfpathlineto{\pgfqpoint{1.147458in}{1.274020in}}%
\pgfpathlineto{\pgfqpoint{1.158053in}{1.208865in}}%
\pgfpathlineto{\pgfqpoint{1.168647in}{1.160780in}}%
\pgfpathlineto{\pgfqpoint{1.179242in}{1.125938in}}%
\pgfpathlineto{\pgfqpoint{1.189837in}{1.106921in}}%
\pgfpathlineto{\pgfqpoint{1.200431in}{1.100203in}}%
\pgfpathlineto{\pgfqpoint{1.211026in}{1.103981in}}%
\pgfpathlineto{\pgfqpoint{1.221621in}{1.115032in}}%
\pgfpathlineto{\pgfqpoint{1.232215in}{1.138414in}}%
\pgfpathlineto{\pgfqpoint{1.242810in}{1.172028in}}%
\pgfpathlineto{\pgfqpoint{1.253405in}{1.214507in}}%
\pgfpathlineto{\pgfqpoint{1.274594in}{1.325134in}}%
\pgfpathlineto{\pgfqpoint{1.295783in}{1.446314in}}%
\pgfpathlineto{\pgfqpoint{1.316973in}{1.555702in}}%
\pgfpathlineto{\pgfqpoint{1.327567in}{1.599165in}}%
\pgfpathlineto{\pgfqpoint{1.338162in}{1.637461in}}%
\pgfpathlineto{\pgfqpoint{1.348757in}{1.667324in}}%
\pgfpathlineto{\pgfqpoint{1.359352in}{1.686182in}}%
\pgfpathlineto{\pgfqpoint{1.369946in}{1.689211in}}%
\pgfpathlineto{\pgfqpoint{1.380541in}{1.678076in}}%
\pgfpathlineto{\pgfqpoint{1.391136in}{1.651732in}}%
\pgfpathlineto{\pgfqpoint{1.401730in}{1.615324in}}%
\pgfpathlineto{\pgfqpoint{1.422920in}{1.522961in}}%
\pgfpathlineto{\pgfqpoint{1.444109in}{1.435219in}}%
\pgfpathlineto{\pgfqpoint{1.454704in}{1.399594in}}%
\pgfpathlineto{\pgfqpoint{1.465298in}{1.368245in}}%
\pgfpathlineto{\pgfqpoint{1.475893in}{1.345949in}}%
\pgfpathlineto{\pgfqpoint{1.486488in}{1.329958in}}%
\pgfpathlineto{\pgfqpoint{1.497082in}{1.320214in}}%
\pgfpathlineto{\pgfqpoint{1.507677in}{1.318401in}}%
\pgfpathlineto{\pgfqpoint{1.518272in}{1.322120in}}%
\pgfpathlineto{\pgfqpoint{1.528866in}{1.333030in}}%
\pgfpathlineto{\pgfqpoint{1.539461in}{1.349004in}}%
\pgfpathlineto{\pgfqpoint{1.550056in}{1.368625in}}%
\pgfpathlineto{\pgfqpoint{1.603029in}{1.485941in}}%
\pgfpathlineto{\pgfqpoint{1.613624in}{1.503214in}}%
\pgfpathlineto{\pgfqpoint{1.624218in}{1.516977in}}%
\pgfpathlineto{\pgfqpoint{1.634813in}{1.526639in}}%
\pgfpathlineto{\pgfqpoint{1.645408in}{1.532324in}}%
\pgfpathlineto{\pgfqpoint{1.656002in}{1.531824in}}%
\pgfpathlineto{\pgfqpoint{1.666597in}{1.529211in}}%
\pgfpathlineto{\pgfqpoint{1.677192in}{1.522938in}}%
\pgfpathlineto{\pgfqpoint{1.687786in}{1.513561in}}%
\pgfpathlineto{\pgfqpoint{1.708976in}{1.488403in}}%
\pgfpathlineto{\pgfqpoint{1.730165in}{1.461072in}}%
\pgfpathlineto{\pgfqpoint{1.740760in}{1.449159in}}%
\pgfpathlineto{\pgfqpoint{1.751354in}{1.435013in}}%
\pgfpathlineto{\pgfqpoint{1.761949in}{1.422979in}}%
\pgfpathlineto{\pgfqpoint{1.772544in}{1.415052in}}%
\pgfpathlineto{\pgfqpoint{1.793733in}{1.404666in}}%
\pgfpathlineto{\pgfqpoint{1.804328in}{1.403429in}}%
\pgfpathlineto{\pgfqpoint{1.814922in}{1.404144in}}%
\pgfpathlineto{\pgfqpoint{1.825517in}{1.406926in}}%
\pgfpathlineto{\pgfqpoint{1.846706in}{1.420687in}}%
\pgfpathlineto{\pgfqpoint{1.857301in}{1.428988in}}%
\pgfpathlineto{\pgfqpoint{1.889085in}{1.458241in}}%
\pgfpathlineto{\pgfqpoint{1.910274in}{1.473551in}}%
\pgfpathlineto{\pgfqpoint{1.931464in}{1.484293in}}%
\pgfpathlineto{\pgfqpoint{1.952653in}{1.485649in}}%
\pgfpathlineto{\pgfqpoint{1.963248in}{1.484434in}}%
\pgfpathlineto{\pgfqpoint{1.973842in}{1.480784in}}%
\pgfpathlineto{\pgfqpoint{2.058600in}{1.440051in}}%
\pgfpathlineto{\pgfqpoint{2.090384in}{1.432962in}}%
\pgfpathlineto{\pgfqpoint{2.100978in}{1.432512in}}%
\pgfpathlineto{\pgfqpoint{2.111573in}{1.433402in}}%
\pgfpathlineto{\pgfqpoint{2.132762in}{1.437858in}}%
\pgfpathlineto{\pgfqpoint{2.196330in}{1.458945in}}%
\pgfpathlineto{\pgfqpoint{2.217520in}{1.464007in}}%
\pgfpathlineto{\pgfqpoint{2.238709in}{1.466297in}}%
\pgfpathlineto{\pgfqpoint{2.249304in}{1.466778in}}%
\pgfpathlineto{\pgfqpoint{2.291682in}{1.461304in}}%
\pgfpathlineto{\pgfqpoint{2.323466in}{1.452390in}}%
\pgfpathlineto{\pgfqpoint{2.376440in}{1.445288in}}%
\pgfpathlineto{\pgfqpoint{2.387034in}{1.444406in}}%
\pgfpathlineto{\pgfqpoint{2.429413in}{1.446024in}}%
\pgfpathlineto{\pgfqpoint{2.492981in}{1.456182in}}%
\pgfpathlineto{\pgfqpoint{2.524765in}{1.458225in}}%
\pgfpathlineto{\pgfqpoint{2.545954in}{1.458693in}}%
\pgfpathlineto{\pgfqpoint{2.588333in}{1.456158in}}%
\pgfpathlineto{\pgfqpoint{2.620117in}{1.453477in}}%
\pgfpathlineto{\pgfqpoint{2.630712in}{1.451648in}}%
\pgfpathlineto{\pgfqpoint{2.683685in}{1.449830in}}%
\pgfpathlineto{\pgfqpoint{2.726064in}{1.450129in}}%
\pgfpathlineto{\pgfqpoint{2.810821in}{1.455310in}}%
\pgfpathlineto{\pgfqpoint{2.874389in}{1.454515in}}%
\pgfpathlineto{\pgfqpoint{2.990930in}{1.451394in}}%
\pgfpathlineto{\pgfqpoint{3.043904in}{1.452512in}}%
\pgfpathlineto{\pgfqpoint{3.096877in}{1.453776in}}%
\pgfpathlineto{\pgfqpoint{3.171040in}{1.453643in}}%
\pgfpathlineto{\pgfqpoint{3.213418in}{1.453116in}}%
\pgfpathlineto{\pgfqpoint{3.245202in}{1.452024in}}%
\pgfpathlineto{\pgfqpoint{3.298176in}{1.452624in}}%
\pgfpathlineto{\pgfqpoint{3.414717in}{1.453330in}}%
\pgfpathlineto{\pgfqpoint{3.467691in}{1.453915in}}%
\pgfpathlineto{\pgfqpoint{3.520664in}{1.452639in}}%
\pgfpathlineto{\pgfqpoint{3.573637in}{1.452536in}}%
\pgfpathlineto{\pgfqpoint{3.658395in}{1.453283in}}%
\pgfpathlineto{\pgfqpoint{3.827909in}{1.452712in}}%
\pgfpathlineto{\pgfqpoint{3.891477in}{1.453341in}}%
\pgfpathlineto{\pgfqpoint{3.891477in}{1.453341in}}%
\pgfusepath{stroke}%
\end{pgfscope}%
\begin{pgfscope}%
\pgfpathrectangle{\pgfqpoint{0.562500in}{0.407000in}}{\pgfqpoint{3.487500in}{2.849000in}}%
\pgfusepath{clip}%
\pgfsetbuttcap%
\pgfsetroundjoin%
\definecolor{currentfill}{rgb}{0.000000,0.000000,0.000000}%
\pgfsetfillcolor{currentfill}%
\pgfsetlinewidth{1.003750pt}%
\definecolor{currentstroke}{rgb}{0.000000,0.000000,0.000000}%
\pgfsetstrokecolor{currentstroke}%
\pgfsetdash{}{0pt}%
\pgfsys@defobject{currentmarker}{\pgfqpoint{-0.041667in}{-0.041667in}}{\pgfqpoint{0.041667in}{0.041667in}}{%
\pgfpathmoveto{\pgfqpoint{0.000000in}{-0.041667in}}%
\pgfpathcurveto{\pgfqpoint{0.011050in}{-0.041667in}}{\pgfqpoint{0.021649in}{-0.037276in}}{\pgfqpoint{0.029463in}{-0.029463in}}%
\pgfpathcurveto{\pgfqpoint{0.037276in}{-0.021649in}}{\pgfqpoint{0.041667in}{-0.011050in}}{\pgfqpoint{0.041667in}{0.000000in}}%
\pgfpathcurveto{\pgfqpoint{0.041667in}{0.011050in}}{\pgfqpoint{0.037276in}{0.021649in}}{\pgfqpoint{0.029463in}{0.029463in}}%
\pgfpathcurveto{\pgfqpoint{0.021649in}{0.037276in}}{\pgfqpoint{0.011050in}{0.041667in}}{\pgfqpoint{0.000000in}{0.041667in}}%
\pgfpathcurveto{\pgfqpoint{-0.011050in}{0.041667in}}{\pgfqpoint{-0.021649in}{0.037276in}}{\pgfqpoint{-0.029463in}{0.029463in}}%
\pgfpathcurveto{\pgfqpoint{-0.037276in}{0.021649in}}{\pgfqpoint{-0.041667in}{0.011050in}}{\pgfqpoint{-0.041667in}{0.000000in}}%
\pgfpathcurveto{\pgfqpoint{-0.041667in}{-0.011050in}}{\pgfqpoint{-0.037276in}{-0.021649in}}{\pgfqpoint{-0.029463in}{-0.029463in}}%
\pgfpathcurveto{\pgfqpoint{-0.021649in}{-0.037276in}}{\pgfqpoint{-0.011050in}{-0.041667in}}{\pgfqpoint{0.000000in}{-0.041667in}}%
\pgfpathclose%
\pgfusepath{stroke,fill}%
}%
\begin{pgfscope}%
\pgfsys@transformshift{1.073295in}{2.806037in}%
\pgfsys@useobject{currentmarker}{}%
\end{pgfscope}%
\begin{pgfscope}%
\pgfsys@transformshift{1.369946in}{1.689211in}%
\pgfsys@useobject{currentmarker}{}%
\end{pgfscope}%
\begin{pgfscope}%
\pgfsys@transformshift{1.656002in}{1.531824in}%
\pgfsys@useobject{currentmarker}{}%
\end{pgfscope}%
\end{pgfscope}%
\begin{pgfscope}%
\pgfsetrectcap%
\pgfsetmiterjoin%
\pgfsetlinewidth{0.803000pt}%
\definecolor{currentstroke}{rgb}{0.000000,0.000000,0.000000}%
\pgfsetstrokecolor{currentstroke}%
\pgfsetdash{}{0pt}%
\pgfpathmoveto{\pgfqpoint{0.562500in}{0.407000in}}%
\pgfpathlineto{\pgfqpoint{0.562500in}{3.256000in}}%
\pgfusepath{stroke}%
\end{pgfscope}%
\begin{pgfscope}%
\pgfsetrectcap%
\pgfsetmiterjoin%
\pgfsetlinewidth{0.803000pt}%
\definecolor{currentstroke}{rgb}{0.000000,0.000000,0.000000}%
\pgfsetstrokecolor{currentstroke}%
\pgfsetdash{}{0pt}%
\pgfpathmoveto{\pgfqpoint{4.050000in}{0.407000in}}%
\pgfpathlineto{\pgfqpoint{4.050000in}{3.256000in}}%
\pgfusepath{stroke}%
\end{pgfscope}%
\begin{pgfscope}%
\pgfsetrectcap%
\pgfsetmiterjoin%
\pgfsetlinewidth{0.803000pt}%
\definecolor{currentstroke}{rgb}{0.000000,0.000000,0.000000}%
\pgfsetstrokecolor{currentstroke}%
\pgfsetdash{}{0pt}%
\pgfpathmoveto{\pgfqpoint{0.562500in}{0.407000in}}%
\pgfpathlineto{\pgfqpoint{4.050000in}{0.407000in}}%
\pgfusepath{stroke}%
\end{pgfscope}%
\begin{pgfscope}%
\pgfsetrectcap%
\pgfsetmiterjoin%
\pgfsetlinewidth{0.803000pt}%
\definecolor{currentstroke}{rgb}{0.000000,0.000000,0.000000}%
\pgfsetstrokecolor{currentstroke}%
\pgfsetdash{}{0pt}%
\pgfpathmoveto{\pgfqpoint{0.562500in}{3.256000in}}%
\pgfpathlineto{\pgfqpoint{4.050000in}{3.256000in}}%
\pgfusepath{stroke}%
\end{pgfscope}%
\begin{pgfscope}%
\pgfsetbuttcap%
\pgfsetmiterjoin%
\definecolor{currentfill}{rgb}{1.000000,1.000000,1.000000}%
\pgfsetfillcolor{currentfill}%
\pgfsetfillopacity{0.800000}%
\pgfsetlinewidth{1.003750pt}%
\definecolor{currentstroke}{rgb}{0.800000,0.800000,0.800000}%
\pgfsetstrokecolor{currentstroke}%
\pgfsetstrokeopacity{0.800000}%
\pgfsetdash{}{0pt}%
\pgfpathmoveto{\pgfqpoint{2.839596in}{2.757556in}}%
\pgfpathlineto{\pgfqpoint{3.952778in}{2.757556in}}%
\pgfpathquadraticcurveto{\pgfqpoint{3.980556in}{2.757556in}}{\pgfqpoint{3.980556in}{2.785334in}}%
\pgfpathlineto{\pgfqpoint{3.980556in}{3.158778in}}%
\pgfpathquadraticcurveto{\pgfqpoint{3.980556in}{3.186556in}}{\pgfqpoint{3.952778in}{3.186556in}}%
\pgfpathlineto{\pgfqpoint{2.839596in}{3.186556in}}%
\pgfpathquadraticcurveto{\pgfqpoint{2.811819in}{3.186556in}}{\pgfqpoint{2.811819in}{3.158778in}}%
\pgfpathlineto{\pgfqpoint{2.811819in}{2.785334in}}%
\pgfpathquadraticcurveto{\pgfqpoint{2.811819in}{2.757556in}}{\pgfqpoint{2.839596in}{2.757556in}}%
\pgfpathclose%
\pgfusepath{stroke,fill}%
\end{pgfscope}%
\begin{pgfscope}%
\pgfsetrectcap%
\pgfsetroundjoin%
\pgfsetlinewidth{1.505625pt}%
\definecolor{currentstroke}{rgb}{0.121569,0.466667,0.705882}%
\pgfsetstrokecolor{currentstroke}%
\pgfsetdash{}{0pt}%
\pgfpathmoveto{\pgfqpoint{2.867374in}{3.082389in}}%
\pgfpathlineto{\pgfqpoint{3.145152in}{3.082389in}}%
\pgfusepath{stroke}%
\end{pgfscope}%
\begin{pgfscope}%
\definecolor{textcolor}{rgb}{0.000000,0.000000,0.000000}%
\pgfsetstrokecolor{textcolor}%
\pgfsetfillcolor{textcolor}%
\pgftext[x=3.256263in,y=3.033778in,left,base]{\color{textcolor}\rmfamily\fontsize{10.000000}{12.000000}\selectfont \(\displaystyle N_{cells} = 6\)}%
\end{pgfscope}%
\begin{pgfscope}%
\pgfsetrectcap%
\pgfsetroundjoin%
\pgfsetlinewidth{1.505625pt}%
\definecolor{currentstroke}{rgb}{1.000000,0.498039,0.054902}%
\pgfsetstrokecolor{currentstroke}%
\pgfsetdash{}{0pt}%
\pgfpathmoveto{\pgfqpoint{2.867374in}{2.888723in}}%
\pgfpathlineto{\pgfqpoint{3.145152in}{2.888723in}}%
\pgfusepath{stroke}%
\end{pgfscope}%
\begin{pgfscope}%
\definecolor{textcolor}{rgb}{0.000000,0.000000,0.000000}%
\pgfsetstrokecolor{textcolor}%
\pgfsetfillcolor{textcolor}%
\pgftext[x=3.256263in,y=2.840112in,left,base]{\color{textcolor}\rmfamily\fontsize{10.000000}{12.000000}\selectfont \(\displaystyle N_{cells} = 12\)}%
\end{pgfscope}%
\end{pgfpicture}%
\makeatother%
\endgroup%

        }
        \caption{Varying the number of simulated cells.}
        \label{step2_changeN-gofr}
    \end{subfigure}
    \begin{subfigure}{0.5\textwidth}
        \resizebox{\textwidth}{!}{
            %% Creator: Matplotlib, PGF backend
%%
%% To include the figure in your LaTeX document, write
%%   \input{<filename>.pgf}
%%
%% Make sure the required packages are loaded in your preamble
%%   \usepackage{pgf}
%%
%% and, on pdftex
%%   \usepackage[utf8]{inputenc}\DeclareUnicodeCharacter{2212}{-}
%%
%% or, on luatex and xetex
%%   \usepackage{unicode-math}
%%
%% Figures using additional raster images can only be included by \input if
%% they are in the same directory as the main LaTeX file. For loading figures
%% from other directories you can use the `import` package
%%   \usepackage{import}
%%
%% and then include the figures with
%%   \import{<path to file>}{<filename>.pgf}
%%
%% Matplotlib used the following preamble
%%   \usepackage[utf8]{inputenc}
%%   \usepackage[T1]{fontenc}
%%   \usepackage{siunitx}
%%
\begingroup%
\makeatletter%
\begin{pgfpicture}%
\pgfpathrectangle{\pgfpointorigin}{\pgfqpoint{4.500000in}{3.700000in}}%
\pgfusepath{use as bounding box, clip}%
\begin{pgfscope}%
\pgfsetbuttcap%
\pgfsetmiterjoin%
\definecolor{currentfill}{rgb}{1.000000,1.000000,1.000000}%
\pgfsetfillcolor{currentfill}%
\pgfsetlinewidth{0.000000pt}%
\definecolor{currentstroke}{rgb}{1.000000,1.000000,1.000000}%
\pgfsetstrokecolor{currentstroke}%
\pgfsetdash{}{0pt}%
\pgfpathmoveto{\pgfqpoint{0.000000in}{0.000000in}}%
\pgfpathlineto{\pgfqpoint{4.500000in}{0.000000in}}%
\pgfpathlineto{\pgfqpoint{4.500000in}{3.700000in}}%
\pgfpathlineto{\pgfqpoint{0.000000in}{3.700000in}}%
\pgfpathclose%
\pgfusepath{fill}%
\end{pgfscope}%
\begin{pgfscope}%
\pgfsetbuttcap%
\pgfsetmiterjoin%
\definecolor{currentfill}{rgb}{1.000000,1.000000,1.000000}%
\pgfsetfillcolor{currentfill}%
\pgfsetlinewidth{0.000000pt}%
\definecolor{currentstroke}{rgb}{0.000000,0.000000,0.000000}%
\pgfsetstrokecolor{currentstroke}%
\pgfsetstrokeopacity{0.000000}%
\pgfsetdash{}{0pt}%
\pgfpathmoveto{\pgfqpoint{0.562500in}{0.407000in}}%
\pgfpathlineto{\pgfqpoint{4.050000in}{0.407000in}}%
\pgfpathlineto{\pgfqpoint{4.050000in}{3.256000in}}%
\pgfpathlineto{\pgfqpoint{0.562500in}{3.256000in}}%
\pgfpathclose%
\pgfusepath{fill}%
\end{pgfscope}%
\begin{pgfscope}%
\pgfsetbuttcap%
\pgfsetroundjoin%
\definecolor{currentfill}{rgb}{0.000000,0.000000,0.000000}%
\pgfsetfillcolor{currentfill}%
\pgfsetlinewidth{0.803000pt}%
\definecolor{currentstroke}{rgb}{0.000000,0.000000,0.000000}%
\pgfsetstrokecolor{currentstroke}%
\pgfsetdash{}{0pt}%
\pgfsys@defobject{currentmarker}{\pgfqpoint{0.000000in}{-0.048611in}}{\pgfqpoint{0.000000in}{0.000000in}}{%
\pgfpathmoveto{\pgfqpoint{0.000000in}{0.000000in}}%
\pgfpathlineto{\pgfqpoint{0.000000in}{-0.048611in}}%
\pgfusepath{stroke,fill}%
}%
\begin{pgfscope}%
\pgfsys@transformshift{0.715721in}{0.407000in}%
\pgfsys@useobject{currentmarker}{}%
\end{pgfscope}%
\end{pgfscope}%
\begin{pgfscope}%
\definecolor{textcolor}{rgb}{0.000000,0.000000,0.000000}%
\pgfsetstrokecolor{textcolor}%
\pgfsetfillcolor{textcolor}%
\pgftext[x=0.715721in,y=0.309778in,,top]{\color{textcolor}\rmfamily\fontsize{10.000000}{12.000000}\selectfont \(\displaystyle {0.0}\)}%
\end{pgfscope}%
\begin{pgfscope}%
\pgfsetbuttcap%
\pgfsetroundjoin%
\definecolor{currentfill}{rgb}{0.000000,0.000000,0.000000}%
\pgfsetfillcolor{currentfill}%
\pgfsetlinewidth{0.803000pt}%
\definecolor{currentstroke}{rgb}{0.000000,0.000000,0.000000}%
\pgfsetstrokecolor{currentstroke}%
\pgfsetdash{}{0pt}%
\pgfsys@defobject{currentmarker}{\pgfqpoint{0.000000in}{-0.048611in}}{\pgfqpoint{0.000000in}{0.000000in}}{%
\pgfpathmoveto{\pgfqpoint{0.000000in}{0.000000in}}%
\pgfpathlineto{\pgfqpoint{0.000000in}{-0.048611in}}%
\pgfusepath{stroke,fill}%
}%
\begin{pgfscope}%
\pgfsys@transformshift{1.173060in}{0.407000in}%
\pgfsys@useobject{currentmarker}{}%
\end{pgfscope}%
\end{pgfscope}%
\begin{pgfscope}%
\definecolor{textcolor}{rgb}{0.000000,0.000000,0.000000}%
\pgfsetstrokecolor{textcolor}%
\pgfsetfillcolor{textcolor}%
\pgftext[x=1.173060in,y=0.309778in,,top]{\color{textcolor}\rmfamily\fontsize{10.000000}{12.000000}\selectfont \(\displaystyle {2.5}\)}%
\end{pgfscope}%
\begin{pgfscope}%
\pgfsetbuttcap%
\pgfsetroundjoin%
\definecolor{currentfill}{rgb}{0.000000,0.000000,0.000000}%
\pgfsetfillcolor{currentfill}%
\pgfsetlinewidth{0.803000pt}%
\definecolor{currentstroke}{rgb}{0.000000,0.000000,0.000000}%
\pgfsetstrokecolor{currentstroke}%
\pgfsetdash{}{0pt}%
\pgfsys@defobject{currentmarker}{\pgfqpoint{0.000000in}{-0.048611in}}{\pgfqpoint{0.000000in}{0.000000in}}{%
\pgfpathmoveto{\pgfqpoint{0.000000in}{0.000000in}}%
\pgfpathlineto{\pgfqpoint{0.000000in}{-0.048611in}}%
\pgfusepath{stroke,fill}%
}%
\begin{pgfscope}%
\pgfsys@transformshift{1.630398in}{0.407000in}%
\pgfsys@useobject{currentmarker}{}%
\end{pgfscope}%
\end{pgfscope}%
\begin{pgfscope}%
\definecolor{textcolor}{rgb}{0.000000,0.000000,0.000000}%
\pgfsetstrokecolor{textcolor}%
\pgfsetfillcolor{textcolor}%
\pgftext[x=1.630398in,y=0.309778in,,top]{\color{textcolor}\rmfamily\fontsize{10.000000}{12.000000}\selectfont \(\displaystyle {5.0}\)}%
\end{pgfscope}%
\begin{pgfscope}%
\pgfsetbuttcap%
\pgfsetroundjoin%
\definecolor{currentfill}{rgb}{0.000000,0.000000,0.000000}%
\pgfsetfillcolor{currentfill}%
\pgfsetlinewidth{0.803000pt}%
\definecolor{currentstroke}{rgb}{0.000000,0.000000,0.000000}%
\pgfsetstrokecolor{currentstroke}%
\pgfsetdash{}{0pt}%
\pgfsys@defobject{currentmarker}{\pgfqpoint{0.000000in}{-0.048611in}}{\pgfqpoint{0.000000in}{0.000000in}}{%
\pgfpathmoveto{\pgfqpoint{0.000000in}{0.000000in}}%
\pgfpathlineto{\pgfqpoint{0.000000in}{-0.048611in}}%
\pgfusepath{stroke,fill}%
}%
\begin{pgfscope}%
\pgfsys@transformshift{2.087737in}{0.407000in}%
\pgfsys@useobject{currentmarker}{}%
\end{pgfscope}%
\end{pgfscope}%
\begin{pgfscope}%
\definecolor{textcolor}{rgb}{0.000000,0.000000,0.000000}%
\pgfsetstrokecolor{textcolor}%
\pgfsetfillcolor{textcolor}%
\pgftext[x=2.087737in,y=0.309778in,,top]{\color{textcolor}\rmfamily\fontsize{10.000000}{12.000000}\selectfont \(\displaystyle {7.5}\)}%
\end{pgfscope}%
\begin{pgfscope}%
\pgfsetbuttcap%
\pgfsetroundjoin%
\definecolor{currentfill}{rgb}{0.000000,0.000000,0.000000}%
\pgfsetfillcolor{currentfill}%
\pgfsetlinewidth{0.803000pt}%
\definecolor{currentstroke}{rgb}{0.000000,0.000000,0.000000}%
\pgfsetstrokecolor{currentstroke}%
\pgfsetdash{}{0pt}%
\pgfsys@defobject{currentmarker}{\pgfqpoint{0.000000in}{-0.048611in}}{\pgfqpoint{0.000000in}{0.000000in}}{%
\pgfpathmoveto{\pgfqpoint{0.000000in}{0.000000in}}%
\pgfpathlineto{\pgfqpoint{0.000000in}{-0.048611in}}%
\pgfusepath{stroke,fill}%
}%
\begin{pgfscope}%
\pgfsys@transformshift{2.545076in}{0.407000in}%
\pgfsys@useobject{currentmarker}{}%
\end{pgfscope}%
\end{pgfscope}%
\begin{pgfscope}%
\definecolor{textcolor}{rgb}{0.000000,0.000000,0.000000}%
\pgfsetstrokecolor{textcolor}%
\pgfsetfillcolor{textcolor}%
\pgftext[x=2.545076in,y=0.309778in,,top]{\color{textcolor}\rmfamily\fontsize{10.000000}{12.000000}\selectfont \(\displaystyle {10.0}\)}%
\end{pgfscope}%
\begin{pgfscope}%
\pgfsetbuttcap%
\pgfsetroundjoin%
\definecolor{currentfill}{rgb}{0.000000,0.000000,0.000000}%
\pgfsetfillcolor{currentfill}%
\pgfsetlinewidth{0.803000pt}%
\definecolor{currentstroke}{rgb}{0.000000,0.000000,0.000000}%
\pgfsetstrokecolor{currentstroke}%
\pgfsetdash{}{0pt}%
\pgfsys@defobject{currentmarker}{\pgfqpoint{0.000000in}{-0.048611in}}{\pgfqpoint{0.000000in}{0.000000in}}{%
\pgfpathmoveto{\pgfqpoint{0.000000in}{0.000000in}}%
\pgfpathlineto{\pgfqpoint{0.000000in}{-0.048611in}}%
\pgfusepath{stroke,fill}%
}%
\begin{pgfscope}%
\pgfsys@transformshift{3.002415in}{0.407000in}%
\pgfsys@useobject{currentmarker}{}%
\end{pgfscope}%
\end{pgfscope}%
\begin{pgfscope}%
\definecolor{textcolor}{rgb}{0.000000,0.000000,0.000000}%
\pgfsetstrokecolor{textcolor}%
\pgfsetfillcolor{textcolor}%
\pgftext[x=3.002415in,y=0.309778in,,top]{\color{textcolor}\rmfamily\fontsize{10.000000}{12.000000}\selectfont \(\displaystyle {12.5}\)}%
\end{pgfscope}%
\begin{pgfscope}%
\pgfsetbuttcap%
\pgfsetroundjoin%
\definecolor{currentfill}{rgb}{0.000000,0.000000,0.000000}%
\pgfsetfillcolor{currentfill}%
\pgfsetlinewidth{0.803000pt}%
\definecolor{currentstroke}{rgb}{0.000000,0.000000,0.000000}%
\pgfsetstrokecolor{currentstroke}%
\pgfsetdash{}{0pt}%
\pgfsys@defobject{currentmarker}{\pgfqpoint{0.000000in}{-0.048611in}}{\pgfqpoint{0.000000in}{0.000000in}}{%
\pgfpathmoveto{\pgfqpoint{0.000000in}{0.000000in}}%
\pgfpathlineto{\pgfqpoint{0.000000in}{-0.048611in}}%
\pgfusepath{stroke,fill}%
}%
\begin{pgfscope}%
\pgfsys@transformshift{3.459753in}{0.407000in}%
\pgfsys@useobject{currentmarker}{}%
\end{pgfscope}%
\end{pgfscope}%
\begin{pgfscope}%
\definecolor{textcolor}{rgb}{0.000000,0.000000,0.000000}%
\pgfsetstrokecolor{textcolor}%
\pgfsetfillcolor{textcolor}%
\pgftext[x=3.459753in,y=0.309778in,,top]{\color{textcolor}\rmfamily\fontsize{10.000000}{12.000000}\selectfont \(\displaystyle {15.0}\)}%
\end{pgfscope}%
\begin{pgfscope}%
\pgfsetbuttcap%
\pgfsetroundjoin%
\definecolor{currentfill}{rgb}{0.000000,0.000000,0.000000}%
\pgfsetfillcolor{currentfill}%
\pgfsetlinewidth{0.803000pt}%
\definecolor{currentstroke}{rgb}{0.000000,0.000000,0.000000}%
\pgfsetstrokecolor{currentstroke}%
\pgfsetdash{}{0pt}%
\pgfsys@defobject{currentmarker}{\pgfqpoint{0.000000in}{-0.048611in}}{\pgfqpoint{0.000000in}{0.000000in}}{%
\pgfpathmoveto{\pgfqpoint{0.000000in}{0.000000in}}%
\pgfpathlineto{\pgfqpoint{0.000000in}{-0.048611in}}%
\pgfusepath{stroke,fill}%
}%
\begin{pgfscope}%
\pgfsys@transformshift{3.917092in}{0.407000in}%
\pgfsys@useobject{currentmarker}{}%
\end{pgfscope}%
\end{pgfscope}%
\begin{pgfscope}%
\definecolor{textcolor}{rgb}{0.000000,0.000000,0.000000}%
\pgfsetstrokecolor{textcolor}%
\pgfsetfillcolor{textcolor}%
\pgftext[x=3.917092in,y=0.309778in,,top]{\color{textcolor}\rmfamily\fontsize{10.000000}{12.000000}\selectfont \(\displaystyle {17.5}\)}%
\end{pgfscope}%
\begin{pgfscope}%
\definecolor{textcolor}{rgb}{0.000000,0.000000,0.000000}%
\pgfsetstrokecolor{textcolor}%
\pgfsetfillcolor{textcolor}%
\pgftext[x=2.306250in,y=0.131567in,,top]{\color{textcolor}\rmfamily\fontsize{10.000000}{12.000000}\selectfont \(\displaystyle r\) [\si{\angstrom}]}%
\end{pgfscope}%
\begin{pgfscope}%
\pgfsetbuttcap%
\pgfsetroundjoin%
\definecolor{currentfill}{rgb}{0.000000,0.000000,0.000000}%
\pgfsetfillcolor{currentfill}%
\pgfsetlinewidth{0.803000pt}%
\definecolor{currentstroke}{rgb}{0.000000,0.000000,0.000000}%
\pgfsetstrokecolor{currentstroke}%
\pgfsetdash{}{0pt}%
\pgfsys@defobject{currentmarker}{\pgfqpoint{-0.048611in}{0.000000in}}{\pgfqpoint{0.000000in}{0.000000in}}{%
\pgfpathmoveto{\pgfqpoint{0.000000in}{0.000000in}}%
\pgfpathlineto{\pgfqpoint{-0.048611in}{0.000000in}}%
\pgfusepath{stroke,fill}%
}%
\begin{pgfscope}%
\pgfsys@transformshift{0.562500in}{0.536500in}%
\pgfsys@useobject{currentmarker}{}%
\end{pgfscope}%
\end{pgfscope}%
\begin{pgfscope}%
\definecolor{textcolor}{rgb}{0.000000,0.000000,0.000000}%
\pgfsetstrokecolor{textcolor}%
\pgfsetfillcolor{textcolor}%
\pgftext[x=0.287808in, y=0.488672in, left, base]{\color{textcolor}\rmfamily\fontsize{10.000000}{12.000000}\selectfont \(\displaystyle {0.0}\)}%
\end{pgfscope}%
\begin{pgfscope}%
\pgfsetbuttcap%
\pgfsetroundjoin%
\definecolor{currentfill}{rgb}{0.000000,0.000000,0.000000}%
\pgfsetfillcolor{currentfill}%
\pgfsetlinewidth{0.803000pt}%
\definecolor{currentstroke}{rgb}{0.000000,0.000000,0.000000}%
\pgfsetstrokecolor{currentstroke}%
\pgfsetdash{}{0pt}%
\pgfsys@defobject{currentmarker}{\pgfqpoint{-0.048611in}{0.000000in}}{\pgfqpoint{0.000000in}{0.000000in}}{%
\pgfpathmoveto{\pgfqpoint{0.000000in}{0.000000in}}%
\pgfpathlineto{\pgfqpoint{-0.048611in}{0.000000in}}%
\pgfusepath{stroke,fill}%
}%
\begin{pgfscope}%
\pgfsys@transformshift{0.562500in}{0.982474in}%
\pgfsys@useobject{currentmarker}{}%
\end{pgfscope}%
\end{pgfscope}%
\begin{pgfscope}%
\definecolor{textcolor}{rgb}{0.000000,0.000000,0.000000}%
\pgfsetstrokecolor{textcolor}%
\pgfsetfillcolor{textcolor}%
\pgftext[x=0.287808in, y=0.934647in, left, base]{\color{textcolor}\rmfamily\fontsize{10.000000}{12.000000}\selectfont \(\displaystyle {0.5}\)}%
\end{pgfscope}%
\begin{pgfscope}%
\pgfsetbuttcap%
\pgfsetroundjoin%
\definecolor{currentfill}{rgb}{0.000000,0.000000,0.000000}%
\pgfsetfillcolor{currentfill}%
\pgfsetlinewidth{0.803000pt}%
\definecolor{currentstroke}{rgb}{0.000000,0.000000,0.000000}%
\pgfsetstrokecolor{currentstroke}%
\pgfsetdash{}{0pt}%
\pgfsys@defobject{currentmarker}{\pgfqpoint{-0.048611in}{0.000000in}}{\pgfqpoint{0.000000in}{0.000000in}}{%
\pgfpathmoveto{\pgfqpoint{0.000000in}{0.000000in}}%
\pgfpathlineto{\pgfqpoint{-0.048611in}{0.000000in}}%
\pgfusepath{stroke,fill}%
}%
\begin{pgfscope}%
\pgfsys@transformshift{0.562500in}{1.428449in}%
\pgfsys@useobject{currentmarker}{}%
\end{pgfscope}%
\end{pgfscope}%
\begin{pgfscope}%
\definecolor{textcolor}{rgb}{0.000000,0.000000,0.000000}%
\pgfsetstrokecolor{textcolor}%
\pgfsetfillcolor{textcolor}%
\pgftext[x=0.287808in, y=1.380621in, left, base]{\color{textcolor}\rmfamily\fontsize{10.000000}{12.000000}\selectfont \(\displaystyle {1.0}\)}%
\end{pgfscope}%
\begin{pgfscope}%
\pgfsetbuttcap%
\pgfsetroundjoin%
\definecolor{currentfill}{rgb}{0.000000,0.000000,0.000000}%
\pgfsetfillcolor{currentfill}%
\pgfsetlinewidth{0.803000pt}%
\definecolor{currentstroke}{rgb}{0.000000,0.000000,0.000000}%
\pgfsetstrokecolor{currentstroke}%
\pgfsetdash{}{0pt}%
\pgfsys@defobject{currentmarker}{\pgfqpoint{-0.048611in}{0.000000in}}{\pgfqpoint{0.000000in}{0.000000in}}{%
\pgfpathmoveto{\pgfqpoint{0.000000in}{0.000000in}}%
\pgfpathlineto{\pgfqpoint{-0.048611in}{0.000000in}}%
\pgfusepath{stroke,fill}%
}%
\begin{pgfscope}%
\pgfsys@transformshift{0.562500in}{1.874423in}%
\pgfsys@useobject{currentmarker}{}%
\end{pgfscope}%
\end{pgfscope}%
\begin{pgfscope}%
\definecolor{textcolor}{rgb}{0.000000,0.000000,0.000000}%
\pgfsetstrokecolor{textcolor}%
\pgfsetfillcolor{textcolor}%
\pgftext[x=0.287808in, y=1.826595in, left, base]{\color{textcolor}\rmfamily\fontsize{10.000000}{12.000000}\selectfont \(\displaystyle {1.5}\)}%
\end{pgfscope}%
\begin{pgfscope}%
\pgfsetbuttcap%
\pgfsetroundjoin%
\definecolor{currentfill}{rgb}{0.000000,0.000000,0.000000}%
\pgfsetfillcolor{currentfill}%
\pgfsetlinewidth{0.803000pt}%
\definecolor{currentstroke}{rgb}{0.000000,0.000000,0.000000}%
\pgfsetstrokecolor{currentstroke}%
\pgfsetdash{}{0pt}%
\pgfsys@defobject{currentmarker}{\pgfqpoint{-0.048611in}{0.000000in}}{\pgfqpoint{0.000000in}{0.000000in}}{%
\pgfpathmoveto{\pgfqpoint{0.000000in}{0.000000in}}%
\pgfpathlineto{\pgfqpoint{-0.048611in}{0.000000in}}%
\pgfusepath{stroke,fill}%
}%
\begin{pgfscope}%
\pgfsys@transformshift{0.562500in}{2.320397in}%
\pgfsys@useobject{currentmarker}{}%
\end{pgfscope}%
\end{pgfscope}%
\begin{pgfscope}%
\definecolor{textcolor}{rgb}{0.000000,0.000000,0.000000}%
\pgfsetstrokecolor{textcolor}%
\pgfsetfillcolor{textcolor}%
\pgftext[x=0.287808in, y=2.272570in, left, base]{\color{textcolor}\rmfamily\fontsize{10.000000}{12.000000}\selectfont \(\displaystyle {2.0}\)}%
\end{pgfscope}%
\begin{pgfscope}%
\pgfsetbuttcap%
\pgfsetroundjoin%
\definecolor{currentfill}{rgb}{0.000000,0.000000,0.000000}%
\pgfsetfillcolor{currentfill}%
\pgfsetlinewidth{0.803000pt}%
\definecolor{currentstroke}{rgb}{0.000000,0.000000,0.000000}%
\pgfsetstrokecolor{currentstroke}%
\pgfsetdash{}{0pt}%
\pgfsys@defobject{currentmarker}{\pgfqpoint{-0.048611in}{0.000000in}}{\pgfqpoint{0.000000in}{0.000000in}}{%
\pgfpathmoveto{\pgfqpoint{0.000000in}{0.000000in}}%
\pgfpathlineto{\pgfqpoint{-0.048611in}{0.000000in}}%
\pgfusepath{stroke,fill}%
}%
\begin{pgfscope}%
\pgfsys@transformshift{0.562500in}{2.766372in}%
\pgfsys@useobject{currentmarker}{}%
\end{pgfscope}%
\end{pgfscope}%
\begin{pgfscope}%
\definecolor{textcolor}{rgb}{0.000000,0.000000,0.000000}%
\pgfsetstrokecolor{textcolor}%
\pgfsetfillcolor{textcolor}%
\pgftext[x=0.287808in, y=2.718544in, left, base]{\color{textcolor}\rmfamily\fontsize{10.000000}{12.000000}\selectfont \(\displaystyle {2.5}\)}%
\end{pgfscope}%
\begin{pgfscope}%
\pgfsetbuttcap%
\pgfsetroundjoin%
\definecolor{currentfill}{rgb}{0.000000,0.000000,0.000000}%
\pgfsetfillcolor{currentfill}%
\pgfsetlinewidth{0.803000pt}%
\definecolor{currentstroke}{rgb}{0.000000,0.000000,0.000000}%
\pgfsetstrokecolor{currentstroke}%
\pgfsetdash{}{0pt}%
\pgfsys@defobject{currentmarker}{\pgfqpoint{-0.048611in}{0.000000in}}{\pgfqpoint{0.000000in}{0.000000in}}{%
\pgfpathmoveto{\pgfqpoint{0.000000in}{0.000000in}}%
\pgfpathlineto{\pgfqpoint{-0.048611in}{0.000000in}}%
\pgfusepath{stroke,fill}%
}%
\begin{pgfscope}%
\pgfsys@transformshift{0.562500in}{3.212346in}%
\pgfsys@useobject{currentmarker}{}%
\end{pgfscope}%
\end{pgfscope}%
\begin{pgfscope}%
\definecolor{textcolor}{rgb}{0.000000,0.000000,0.000000}%
\pgfsetstrokecolor{textcolor}%
\pgfsetfillcolor{textcolor}%
\pgftext[x=0.287808in, y=3.164518in, left, base]{\color{textcolor}\rmfamily\fontsize{10.000000}{12.000000}\selectfont \(\displaystyle {3.0}\)}%
\end{pgfscope}%
\begin{pgfscope}%
\definecolor{textcolor}{rgb}{0.000000,0.000000,0.000000}%
\pgfsetstrokecolor{textcolor}%
\pgfsetfillcolor{textcolor}%
\pgftext[x=0.232253in,y=1.831500in,,bottom,rotate=90.000000]{\color{textcolor}\rmfamily\fontsize{10.000000}{12.000000}\selectfont \(\displaystyle g(r)\)}%
\end{pgfscope}%
\begin{pgfscope}%
\pgfpathrectangle{\pgfqpoint{0.562500in}{0.407000in}}{\pgfqpoint{3.487500in}{2.849000in}}%
\pgfusepath{clip}%
\pgfsetrectcap%
\pgfsetroundjoin%
\pgfsetlinewidth{1.505625pt}%
\definecolor{currentstroke}{rgb}{0.121569,0.466667,0.705882}%
\pgfsetstrokecolor{currentstroke}%
\pgfsetdash{}{0pt}%
\pgfpathmoveto{\pgfqpoint{0.721023in}{2.016943in}}%
\pgfpathlineto{\pgfqpoint{0.731626in}{1.386384in}}%
\pgfpathlineto{\pgfqpoint{0.742230in}{1.428056in}}%
\pgfpathlineto{\pgfqpoint{0.752833in}{1.451286in}}%
\pgfpathlineto{\pgfqpoint{0.763437in}{1.427000in}}%
\pgfpathlineto{\pgfqpoint{0.774040in}{1.399072in}}%
\pgfpathlineto{\pgfqpoint{0.784644in}{1.455815in}}%
\pgfpathlineto{\pgfqpoint{0.805851in}{1.430401in}}%
\pgfpathlineto{\pgfqpoint{0.816454in}{1.426862in}}%
\pgfpathlineto{\pgfqpoint{0.827058in}{1.409137in}}%
\pgfpathlineto{\pgfqpoint{0.837662in}{1.408254in}}%
\pgfpathlineto{\pgfqpoint{0.848265in}{1.425819in}}%
\pgfpathlineto{\pgfqpoint{0.858869in}{1.421584in}}%
\pgfpathlineto{\pgfqpoint{0.869472in}{1.435152in}}%
\pgfpathlineto{\pgfqpoint{0.880076in}{1.429661in}}%
\pgfpathlineto{\pgfqpoint{0.890679in}{1.425808in}}%
\pgfpathlineto{\pgfqpoint{0.901283in}{1.424296in}}%
\pgfpathlineto{\pgfqpoint{0.911886in}{1.427638in}}%
\pgfpathlineto{\pgfqpoint{0.922490in}{1.417152in}}%
\pgfpathlineto{\pgfqpoint{0.943697in}{1.421020in}}%
\pgfpathlineto{\pgfqpoint{0.954300in}{1.411484in}}%
\pgfpathlineto{\pgfqpoint{0.975507in}{1.417237in}}%
\pgfpathlineto{\pgfqpoint{0.986111in}{1.427934in}}%
\pgfpathlineto{\pgfqpoint{0.996714in}{1.432607in}}%
\pgfpathlineto{\pgfqpoint{1.007318in}{1.423733in}}%
\pgfpathlineto{\pgfqpoint{1.017921in}{1.425824in}}%
\pgfpathlineto{\pgfqpoint{1.028525in}{1.424204in}}%
\pgfpathlineto{\pgfqpoint{1.039129in}{1.420924in}}%
\pgfpathlineto{\pgfqpoint{1.049732in}{1.423954in}}%
\pgfpathlineto{\pgfqpoint{1.060336in}{1.432085in}}%
\pgfpathlineto{\pgfqpoint{1.081543in}{1.429917in}}%
\pgfpathlineto{\pgfqpoint{1.092146in}{1.433824in}}%
\pgfpathlineto{\pgfqpoint{1.102750in}{1.418620in}}%
\pgfpathlineto{\pgfqpoint{1.113353in}{1.430541in}}%
\pgfpathlineto{\pgfqpoint{1.123957in}{1.432504in}}%
\pgfpathlineto{\pgfqpoint{1.134560in}{1.430672in}}%
\pgfpathlineto{\pgfqpoint{1.145164in}{1.433706in}}%
\pgfpathlineto{\pgfqpoint{1.155767in}{1.426507in}}%
\pgfpathlineto{\pgfqpoint{1.166371in}{1.427805in}}%
\pgfpathlineto{\pgfqpoint{1.187578in}{1.427520in}}%
\pgfpathlineto{\pgfqpoint{1.208785in}{1.430974in}}%
\pgfpathlineto{\pgfqpoint{1.219388in}{1.427946in}}%
\pgfpathlineto{\pgfqpoint{1.229992in}{1.429135in}}%
\pgfpathlineto{\pgfqpoint{1.240596in}{1.434670in}}%
\pgfpathlineto{\pgfqpoint{1.251199in}{1.429734in}}%
\pgfpathlineto{\pgfqpoint{1.272406in}{1.426243in}}%
\pgfpathlineto{\pgfqpoint{1.304217in}{1.427678in}}%
\pgfpathlineto{\pgfqpoint{1.314820in}{1.426169in}}%
\pgfpathlineto{\pgfqpoint{1.325424in}{1.431265in}}%
\pgfpathlineto{\pgfqpoint{1.336027in}{1.428664in}}%
\pgfpathlineto{\pgfqpoint{1.346631in}{1.430052in}}%
\pgfpathlineto{\pgfqpoint{1.357234in}{1.429441in}}%
\pgfpathlineto{\pgfqpoint{1.367838in}{1.426994in}}%
\pgfpathlineto{\pgfqpoint{1.378441in}{1.430072in}}%
\pgfpathlineto{\pgfqpoint{1.399648in}{1.426778in}}%
\pgfpathlineto{\pgfqpoint{1.420856in}{1.427304in}}%
\pgfpathlineto{\pgfqpoint{1.431459in}{1.430642in}}%
\pgfpathlineto{\pgfqpoint{1.442063in}{1.427009in}}%
\pgfpathlineto{\pgfqpoint{1.452666in}{1.428240in}}%
\pgfpathlineto{\pgfqpoint{1.463270in}{1.427191in}}%
\pgfpathlineto{\pgfqpoint{1.473873in}{1.429519in}}%
\pgfpathlineto{\pgfqpoint{1.484477in}{1.425388in}}%
\pgfpathlineto{\pgfqpoint{1.505684in}{1.428875in}}%
\pgfpathlineto{\pgfqpoint{1.537494in}{1.428350in}}%
\pgfpathlineto{\pgfqpoint{1.569305in}{1.430998in}}%
\pgfpathlineto{\pgfqpoint{1.579908in}{1.428034in}}%
\pgfpathlineto{\pgfqpoint{1.601115in}{1.425618in}}%
\pgfpathlineto{\pgfqpoint{1.611719in}{1.428205in}}%
\pgfpathlineto{\pgfqpoint{1.622323in}{1.425785in}}%
\pgfpathlineto{\pgfqpoint{1.632926in}{1.428662in}}%
\pgfpathlineto{\pgfqpoint{1.643530in}{1.434222in}}%
\pgfpathlineto{\pgfqpoint{1.654133in}{1.429667in}}%
\pgfpathlineto{\pgfqpoint{1.664737in}{1.429785in}}%
\pgfpathlineto{\pgfqpoint{1.685944in}{1.425338in}}%
\pgfpathlineto{\pgfqpoint{1.696547in}{1.429721in}}%
\pgfpathlineto{\pgfqpoint{1.707151in}{1.430970in}}%
\pgfpathlineto{\pgfqpoint{1.717754in}{1.428225in}}%
\pgfpathlineto{\pgfqpoint{1.728358in}{1.431185in}}%
\pgfpathlineto{\pgfqpoint{1.738961in}{1.425816in}}%
\pgfpathlineto{\pgfqpoint{1.749565in}{1.429832in}}%
\pgfpathlineto{\pgfqpoint{1.760168in}{1.431233in}}%
\pgfpathlineto{\pgfqpoint{1.770772in}{1.426638in}}%
\pgfpathlineto{\pgfqpoint{1.781375in}{1.427631in}}%
\pgfpathlineto{\pgfqpoint{1.791979in}{1.429753in}}%
\pgfpathlineto{\pgfqpoint{1.813186in}{1.428652in}}%
\pgfpathlineto{\pgfqpoint{1.823790in}{1.427984in}}%
\pgfpathlineto{\pgfqpoint{1.834393in}{1.429568in}}%
\pgfpathlineto{\pgfqpoint{1.844997in}{1.427331in}}%
\pgfpathlineto{\pgfqpoint{1.855600in}{1.428789in}}%
\pgfpathlineto{\pgfqpoint{1.866204in}{1.431820in}}%
\pgfpathlineto{\pgfqpoint{1.876807in}{1.428042in}}%
\pgfpathlineto{\pgfqpoint{1.887411in}{1.428835in}}%
\pgfpathlineto{\pgfqpoint{1.898014in}{1.426881in}}%
\pgfpathlineto{\pgfqpoint{1.908618in}{1.430415in}}%
\pgfpathlineto{\pgfqpoint{1.919221in}{1.427338in}}%
\pgfpathlineto{\pgfqpoint{1.929825in}{1.425583in}}%
\pgfpathlineto{\pgfqpoint{1.940428in}{1.428282in}}%
\pgfpathlineto{\pgfqpoint{1.951032in}{1.428680in}}%
\pgfpathlineto{\pgfqpoint{1.961635in}{1.427085in}}%
\pgfpathlineto{\pgfqpoint{1.972239in}{1.428715in}}%
\pgfpathlineto{\pgfqpoint{1.982842in}{1.426814in}}%
\pgfpathlineto{\pgfqpoint{2.004049in}{1.430026in}}%
\pgfpathlineto{\pgfqpoint{2.014653in}{1.429015in}}%
\pgfpathlineto{\pgfqpoint{2.025257in}{1.429566in}}%
\pgfpathlineto{\pgfqpoint{2.035860in}{1.428055in}}%
\pgfpathlineto{\pgfqpoint{2.046464in}{1.429407in}}%
\pgfpathlineto{\pgfqpoint{2.057067in}{1.428978in}}%
\pgfpathlineto{\pgfqpoint{2.067671in}{1.426492in}}%
\pgfpathlineto{\pgfqpoint{2.088878in}{1.427547in}}%
\pgfpathlineto{\pgfqpoint{2.099481in}{1.427667in}}%
\pgfpathlineto{\pgfqpoint{2.110085in}{1.429630in}}%
\pgfpathlineto{\pgfqpoint{2.120688in}{1.429957in}}%
\pgfpathlineto{\pgfqpoint{2.131292in}{1.428609in}}%
\pgfpathlineto{\pgfqpoint{2.141895in}{1.428612in}}%
\pgfpathlineto{\pgfqpoint{2.152499in}{1.427125in}}%
\pgfpathlineto{\pgfqpoint{2.163102in}{1.427647in}}%
\pgfpathlineto{\pgfqpoint{2.173706in}{1.426959in}}%
\pgfpathlineto{\pgfqpoint{2.184309in}{1.429695in}}%
\pgfpathlineto{\pgfqpoint{2.194913in}{1.428757in}}%
\pgfpathlineto{\pgfqpoint{2.216120in}{1.429471in}}%
\pgfpathlineto{\pgfqpoint{2.226724in}{1.428418in}}%
\pgfpathlineto{\pgfqpoint{2.237327in}{1.429701in}}%
\pgfpathlineto{\pgfqpoint{2.247931in}{1.426712in}}%
\pgfpathlineto{\pgfqpoint{2.258534in}{1.428501in}}%
\pgfpathlineto{\pgfqpoint{2.290345in}{1.427899in}}%
\pgfpathlineto{\pgfqpoint{2.300948in}{1.429555in}}%
\pgfpathlineto{\pgfqpoint{2.332759in}{1.429564in}}%
\pgfpathlineto{\pgfqpoint{2.353966in}{1.428559in}}%
\pgfpathlineto{\pgfqpoint{2.385776in}{1.428749in}}%
\pgfpathlineto{\pgfqpoint{2.396380in}{1.427050in}}%
\pgfpathlineto{\pgfqpoint{2.428191in}{1.430469in}}%
\pgfpathlineto{\pgfqpoint{2.438794in}{1.429355in}}%
\pgfpathlineto{\pgfqpoint{2.470605in}{1.429397in}}%
\pgfpathlineto{\pgfqpoint{2.481208in}{1.427250in}}%
\pgfpathlineto{\pgfqpoint{2.491812in}{1.427644in}}%
\pgfpathlineto{\pgfqpoint{2.502415in}{1.429216in}}%
\pgfpathlineto{\pgfqpoint{2.513019in}{1.427314in}}%
\pgfpathlineto{\pgfqpoint{2.523622in}{1.428170in}}%
\pgfpathlineto{\pgfqpoint{2.534226in}{1.427800in}}%
\pgfpathlineto{\pgfqpoint{2.544829in}{1.428755in}}%
\pgfpathlineto{\pgfqpoint{2.555433in}{1.427603in}}%
\pgfpathlineto{\pgfqpoint{2.566036in}{1.428416in}}%
\pgfpathlineto{\pgfqpoint{2.587243in}{1.427709in}}%
\pgfpathlineto{\pgfqpoint{2.608451in}{1.429419in}}%
\pgfpathlineto{\pgfqpoint{2.629658in}{1.428024in}}%
\pgfpathlineto{\pgfqpoint{2.640261in}{1.428415in}}%
\pgfpathlineto{\pgfqpoint{2.650865in}{1.430069in}}%
\pgfpathlineto{\pgfqpoint{2.672072in}{1.428610in}}%
\pgfpathlineto{\pgfqpoint{2.682675in}{1.429211in}}%
\pgfpathlineto{\pgfqpoint{2.693279in}{1.427489in}}%
\pgfpathlineto{\pgfqpoint{2.756900in}{1.429965in}}%
\pgfpathlineto{\pgfqpoint{2.767503in}{1.429790in}}%
\pgfpathlineto{\pgfqpoint{2.778107in}{1.427709in}}%
\pgfpathlineto{\pgfqpoint{2.831125in}{1.428430in}}%
\pgfpathlineto{\pgfqpoint{2.841728in}{1.427854in}}%
\pgfpathlineto{\pgfqpoint{2.852332in}{1.429216in}}%
\pgfpathlineto{\pgfqpoint{2.862935in}{1.428481in}}%
\pgfpathlineto{\pgfqpoint{2.873539in}{1.428957in}}%
\pgfpathlineto{\pgfqpoint{2.884142in}{1.427183in}}%
\pgfpathlineto{\pgfqpoint{2.915953in}{1.429564in}}%
\pgfpathlineto{\pgfqpoint{2.937160in}{1.427737in}}%
\pgfpathlineto{\pgfqpoint{2.947763in}{1.429488in}}%
\pgfpathlineto{\pgfqpoint{2.979574in}{1.427759in}}%
\pgfpathlineto{\pgfqpoint{2.990177in}{1.429011in}}%
\pgfpathlineto{\pgfqpoint{3.011385in}{1.429028in}}%
\pgfpathlineto{\pgfqpoint{3.414319in}{1.428018in}}%
\pgfpathlineto{\pgfqpoint{3.424922in}{1.429267in}}%
\pgfpathlineto{\pgfqpoint{3.435526in}{1.426977in}}%
\pgfpathlineto{\pgfqpoint{3.446129in}{1.428832in}}%
\pgfpathlineto{\pgfqpoint{3.467336in}{1.428100in}}%
\pgfpathlineto{\pgfqpoint{3.488543in}{1.428812in}}%
\pgfpathlineto{\pgfqpoint{3.499147in}{1.428799in}}%
\pgfpathlineto{\pgfqpoint{3.509750in}{1.427395in}}%
\pgfpathlineto{\pgfqpoint{3.520354in}{1.429085in}}%
\pgfpathlineto{\pgfqpoint{3.552164in}{1.428465in}}%
\pgfpathlineto{\pgfqpoint{3.562768in}{1.429086in}}%
\pgfpathlineto{\pgfqpoint{3.583975in}{1.428224in}}%
\pgfpathlineto{\pgfqpoint{3.615786in}{1.428684in}}%
\pgfpathlineto{\pgfqpoint{3.636993in}{1.428127in}}%
\pgfpathlineto{\pgfqpoint{3.658200in}{1.428389in}}%
\pgfpathlineto{\pgfqpoint{3.679407in}{1.429365in}}%
\pgfpathlineto{\pgfqpoint{3.700614in}{1.428630in}}%
\pgfpathlineto{\pgfqpoint{3.721821in}{1.428876in}}%
\pgfpathlineto{\pgfqpoint{3.732424in}{1.428143in}}%
\pgfpathlineto{\pgfqpoint{3.753631in}{1.428891in}}%
\pgfpathlineto{\pgfqpoint{3.774838in}{1.428026in}}%
\pgfpathlineto{\pgfqpoint{3.817253in}{1.428816in}}%
\pgfpathlineto{\pgfqpoint{3.827856in}{1.427890in}}%
\pgfpathlineto{\pgfqpoint{3.849063in}{1.429057in}}%
\pgfpathlineto{\pgfqpoint{3.880874in}{1.428170in}}%
\pgfpathlineto{\pgfqpoint{3.891477in}{1.428621in}}%
\pgfpathlineto{\pgfqpoint{3.891477in}{1.428621in}}%
\pgfusepath{stroke}%
\end{pgfscope}%
\begin{pgfscope}%
\pgfpathrectangle{\pgfqpoint{0.562500in}{0.407000in}}{\pgfqpoint{3.487500in}{2.849000in}}%
\pgfusepath{clip}%
\pgfsetrectcap%
\pgfsetroundjoin%
\pgfsetlinewidth{1.505625pt}%
\definecolor{currentstroke}{rgb}{1.000000,0.498039,0.054902}%
\pgfsetstrokecolor{currentstroke}%
\pgfsetdash{}{0pt}%
\pgfpathmoveto{\pgfqpoint{0.721023in}{1.276722in}}%
\pgfpathlineto{\pgfqpoint{0.731626in}{1.560016in}}%
\pgfpathlineto{\pgfqpoint{0.742230in}{1.487273in}}%
\pgfpathlineto{\pgfqpoint{0.752833in}{1.392539in}}%
\pgfpathlineto{\pgfqpoint{0.763437in}{1.407707in}}%
\pgfpathlineto{\pgfqpoint{0.774040in}{1.419464in}}%
\pgfpathlineto{\pgfqpoint{0.784644in}{1.386222in}}%
\pgfpathlineto{\pgfqpoint{0.795247in}{1.431346in}}%
\pgfpathlineto{\pgfqpoint{0.805851in}{1.400803in}}%
\pgfpathlineto{\pgfqpoint{0.816454in}{1.431191in}}%
\pgfpathlineto{\pgfqpoint{0.827058in}{1.416970in}}%
\pgfpathlineto{\pgfqpoint{0.837662in}{1.413385in}}%
\pgfpathlineto{\pgfqpoint{0.848265in}{1.443716in}}%
\pgfpathlineto{\pgfqpoint{0.858869in}{1.414251in}}%
\pgfpathlineto{\pgfqpoint{0.869472in}{1.434467in}}%
\pgfpathlineto{\pgfqpoint{0.880076in}{1.419134in}}%
\pgfpathlineto{\pgfqpoint{0.890679in}{1.426563in}}%
\pgfpathlineto{\pgfqpoint{0.901283in}{1.410196in}}%
\pgfpathlineto{\pgfqpoint{0.911886in}{1.426496in}}%
\pgfpathlineto{\pgfqpoint{0.922490in}{1.416395in}}%
\pgfpathlineto{\pgfqpoint{0.933093in}{1.414452in}}%
\pgfpathlineto{\pgfqpoint{0.943697in}{1.423600in}}%
\pgfpathlineto{\pgfqpoint{0.964904in}{1.421668in}}%
\pgfpathlineto{\pgfqpoint{0.986111in}{1.431760in}}%
\pgfpathlineto{\pgfqpoint{0.996714in}{1.428479in}}%
\pgfpathlineto{\pgfqpoint{1.017921in}{1.415673in}}%
\pgfpathlineto{\pgfqpoint{1.028525in}{1.426613in}}%
\pgfpathlineto{\pgfqpoint{1.039129in}{1.429213in}}%
\pgfpathlineto{\pgfqpoint{1.049732in}{1.434936in}}%
\pgfpathlineto{\pgfqpoint{1.060336in}{1.431131in}}%
\pgfpathlineto{\pgfqpoint{1.070939in}{1.425935in}}%
\pgfpathlineto{\pgfqpoint{1.081543in}{1.425944in}}%
\pgfpathlineto{\pgfqpoint{1.092146in}{1.427902in}}%
\pgfpathlineto{\pgfqpoint{1.102750in}{1.426923in}}%
\pgfpathlineto{\pgfqpoint{1.113353in}{1.429299in}}%
\pgfpathlineto{\pgfqpoint{1.123957in}{1.427149in}}%
\pgfpathlineto{\pgfqpoint{1.134560in}{1.426468in}}%
\pgfpathlineto{\pgfqpoint{1.155767in}{1.427593in}}%
\pgfpathlineto{\pgfqpoint{1.166371in}{1.430640in}}%
\pgfpathlineto{\pgfqpoint{1.176974in}{1.432500in}}%
\pgfpathlineto{\pgfqpoint{1.187578in}{1.428194in}}%
\pgfpathlineto{\pgfqpoint{1.198181in}{1.429426in}}%
\pgfpathlineto{\pgfqpoint{1.208785in}{1.427341in}}%
\pgfpathlineto{\pgfqpoint{1.219388in}{1.428776in}}%
\pgfpathlineto{\pgfqpoint{1.240596in}{1.424726in}}%
\pgfpathlineto{\pgfqpoint{1.251199in}{1.429911in}}%
\pgfpathlineto{\pgfqpoint{1.261803in}{1.425547in}}%
\pgfpathlineto{\pgfqpoint{1.272406in}{1.429988in}}%
\pgfpathlineto{\pgfqpoint{1.314820in}{1.425750in}}%
\pgfpathlineto{\pgfqpoint{1.325424in}{1.432981in}}%
\pgfpathlineto{\pgfqpoint{1.336027in}{1.428093in}}%
\pgfpathlineto{\pgfqpoint{1.346631in}{1.430378in}}%
\pgfpathlineto{\pgfqpoint{1.357234in}{1.425778in}}%
\pgfpathlineto{\pgfqpoint{1.367838in}{1.428782in}}%
\pgfpathlineto{\pgfqpoint{1.378441in}{1.425587in}}%
\pgfpathlineto{\pgfqpoint{1.389045in}{1.428498in}}%
\pgfpathlineto{\pgfqpoint{1.399648in}{1.426749in}}%
\pgfpathlineto{\pgfqpoint{1.420856in}{1.428038in}}%
\pgfpathlineto{\pgfqpoint{1.431459in}{1.430592in}}%
\pgfpathlineto{\pgfqpoint{1.442063in}{1.424914in}}%
\pgfpathlineto{\pgfqpoint{1.452666in}{1.431471in}}%
\pgfpathlineto{\pgfqpoint{1.463270in}{1.428953in}}%
\pgfpathlineto{\pgfqpoint{1.473873in}{1.428107in}}%
\pgfpathlineto{\pgfqpoint{1.484477in}{1.423627in}}%
\pgfpathlineto{\pgfqpoint{1.495080in}{1.430212in}}%
\pgfpathlineto{\pgfqpoint{1.505684in}{1.427778in}}%
\pgfpathlineto{\pgfqpoint{1.516287in}{1.429698in}}%
\pgfpathlineto{\pgfqpoint{1.548098in}{1.430061in}}%
\pgfpathlineto{\pgfqpoint{1.558701in}{1.426969in}}%
\pgfpathlineto{\pgfqpoint{1.569305in}{1.428828in}}%
\pgfpathlineto{\pgfqpoint{1.601115in}{1.429457in}}%
\pgfpathlineto{\pgfqpoint{1.611719in}{1.426679in}}%
\pgfpathlineto{\pgfqpoint{1.632926in}{1.429852in}}%
\pgfpathlineto{\pgfqpoint{1.643530in}{1.426082in}}%
\pgfpathlineto{\pgfqpoint{1.654133in}{1.428987in}}%
\pgfpathlineto{\pgfqpoint{1.664737in}{1.427716in}}%
\pgfpathlineto{\pgfqpoint{1.675340in}{1.431756in}}%
\pgfpathlineto{\pgfqpoint{1.685944in}{1.425750in}}%
\pgfpathlineto{\pgfqpoint{1.696547in}{1.426830in}}%
\pgfpathlineto{\pgfqpoint{1.707151in}{1.430079in}}%
\pgfpathlineto{\pgfqpoint{1.749565in}{1.428925in}}%
\pgfpathlineto{\pgfqpoint{1.770772in}{1.429520in}}%
\pgfpathlineto{\pgfqpoint{1.781375in}{1.430004in}}%
\pgfpathlineto{\pgfqpoint{1.791979in}{1.428272in}}%
\pgfpathlineto{\pgfqpoint{1.823790in}{1.429408in}}%
\pgfpathlineto{\pgfqpoint{1.834393in}{1.427558in}}%
\pgfpathlineto{\pgfqpoint{1.844997in}{1.427918in}}%
\pgfpathlineto{\pgfqpoint{1.866204in}{1.426624in}}%
\pgfpathlineto{\pgfqpoint{1.887411in}{1.428336in}}%
\pgfpathlineto{\pgfqpoint{1.898014in}{1.426199in}}%
\pgfpathlineto{\pgfqpoint{1.908618in}{1.429888in}}%
\pgfpathlineto{\pgfqpoint{1.951032in}{1.427974in}}%
\pgfpathlineto{\pgfqpoint{1.961635in}{1.430367in}}%
\pgfpathlineto{\pgfqpoint{1.972239in}{1.428543in}}%
\pgfpathlineto{\pgfqpoint{1.982842in}{1.429986in}}%
\pgfpathlineto{\pgfqpoint{2.004049in}{1.430022in}}%
\pgfpathlineto{\pgfqpoint{2.014653in}{1.426861in}}%
\pgfpathlineto{\pgfqpoint{2.025257in}{1.429061in}}%
\pgfpathlineto{\pgfqpoint{2.035860in}{1.427793in}}%
\pgfpathlineto{\pgfqpoint{2.046464in}{1.427818in}}%
\pgfpathlineto{\pgfqpoint{2.067671in}{1.430184in}}%
\pgfpathlineto{\pgfqpoint{2.078274in}{1.427739in}}%
\pgfpathlineto{\pgfqpoint{2.099481in}{1.428102in}}%
\pgfpathlineto{\pgfqpoint{2.141895in}{1.428240in}}%
\pgfpathlineto{\pgfqpoint{2.152499in}{1.429237in}}%
\pgfpathlineto{\pgfqpoint{2.163102in}{1.427973in}}%
\pgfpathlineto{\pgfqpoint{2.173706in}{1.428945in}}%
\pgfpathlineto{\pgfqpoint{2.216120in}{1.429065in}}%
\pgfpathlineto{\pgfqpoint{2.226724in}{1.430229in}}%
\pgfpathlineto{\pgfqpoint{2.258534in}{1.426710in}}%
\pgfpathlineto{\pgfqpoint{2.269138in}{1.427671in}}%
\pgfpathlineto{\pgfqpoint{2.279741in}{1.426422in}}%
\pgfpathlineto{\pgfqpoint{2.300948in}{1.429160in}}%
\pgfpathlineto{\pgfqpoint{2.311552in}{1.429131in}}%
\pgfpathlineto{\pgfqpoint{2.322155in}{1.427289in}}%
\pgfpathlineto{\pgfqpoint{2.332759in}{1.429400in}}%
\pgfpathlineto{\pgfqpoint{2.364569in}{1.427531in}}%
\pgfpathlineto{\pgfqpoint{2.385776in}{1.430097in}}%
\pgfpathlineto{\pgfqpoint{2.428191in}{1.428032in}}%
\pgfpathlineto{\pgfqpoint{2.438794in}{1.428972in}}%
\pgfpathlineto{\pgfqpoint{2.449398in}{1.426593in}}%
\pgfpathlineto{\pgfqpoint{2.481208in}{1.430780in}}%
\pgfpathlineto{\pgfqpoint{2.491812in}{1.429004in}}%
\pgfpathlineto{\pgfqpoint{2.619054in}{1.428261in}}%
\pgfpathlineto{\pgfqpoint{2.629658in}{1.429385in}}%
\pgfpathlineto{\pgfqpoint{2.640261in}{1.427720in}}%
\pgfpathlineto{\pgfqpoint{2.650865in}{1.429837in}}%
\pgfpathlineto{\pgfqpoint{2.661468in}{1.428141in}}%
\pgfpathlineto{\pgfqpoint{2.682675in}{1.428058in}}%
\pgfpathlineto{\pgfqpoint{2.693279in}{1.429198in}}%
\pgfpathlineto{\pgfqpoint{2.703882in}{1.427858in}}%
\pgfpathlineto{\pgfqpoint{2.746296in}{1.429287in}}%
\pgfpathlineto{\pgfqpoint{2.788710in}{1.428011in}}%
\pgfpathlineto{\pgfqpoint{2.809918in}{1.429980in}}%
\pgfpathlineto{\pgfqpoint{2.820521in}{1.427940in}}%
\pgfpathlineto{\pgfqpoint{2.831125in}{1.428151in}}%
\pgfpathlineto{\pgfqpoint{2.841728in}{1.426709in}}%
\pgfpathlineto{\pgfqpoint{2.852332in}{1.428596in}}%
\pgfpathlineto{\pgfqpoint{2.862935in}{1.428088in}}%
\pgfpathlineto{\pgfqpoint{2.873539in}{1.429531in}}%
\pgfpathlineto{\pgfqpoint{2.894746in}{1.427748in}}%
\pgfpathlineto{\pgfqpoint{2.915953in}{1.428402in}}%
\pgfpathlineto{\pgfqpoint{2.968970in}{1.428113in}}%
\pgfpathlineto{\pgfqpoint{2.990177in}{1.427450in}}%
\pgfpathlineto{\pgfqpoint{3.021988in}{1.429014in}}%
\pgfpathlineto{\pgfqpoint{3.032592in}{1.426461in}}%
\pgfpathlineto{\pgfqpoint{3.053799in}{1.429633in}}%
\pgfpathlineto{\pgfqpoint{3.064402in}{1.428021in}}%
\pgfpathlineto{\pgfqpoint{3.085609in}{1.427710in}}%
\pgfpathlineto{\pgfqpoint{3.096213in}{1.429833in}}%
\pgfpathlineto{\pgfqpoint{3.106816in}{1.427881in}}%
\pgfpathlineto{\pgfqpoint{3.117420in}{1.429330in}}%
\pgfpathlineto{\pgfqpoint{3.138627in}{1.427623in}}%
\pgfpathlineto{\pgfqpoint{3.170437in}{1.428564in}}%
\pgfpathlineto{\pgfqpoint{3.297680in}{1.428085in}}%
\pgfpathlineto{\pgfqpoint{3.308283in}{1.428972in}}%
\pgfpathlineto{\pgfqpoint{3.329490in}{1.428583in}}%
\pgfpathlineto{\pgfqpoint{3.361301in}{1.429412in}}%
\pgfpathlineto{\pgfqpoint{3.371904in}{1.428016in}}%
\pgfpathlineto{\pgfqpoint{3.382508in}{1.429055in}}%
\pgfpathlineto{\pgfqpoint{3.446129in}{1.427545in}}%
\pgfpathlineto{\pgfqpoint{3.467336in}{1.429459in}}%
\pgfpathlineto{\pgfqpoint{3.499147in}{1.429006in}}%
\pgfpathlineto{\pgfqpoint{3.509750in}{1.427750in}}%
\pgfpathlineto{\pgfqpoint{3.520354in}{1.428568in}}%
\pgfpathlineto{\pgfqpoint{3.541561in}{1.427653in}}%
\pgfpathlineto{\pgfqpoint{3.562768in}{1.429108in}}%
\pgfpathlineto{\pgfqpoint{3.573371in}{1.427884in}}%
\pgfpathlineto{\pgfqpoint{3.594579in}{1.428782in}}%
\pgfpathlineto{\pgfqpoint{3.615786in}{1.427685in}}%
\pgfpathlineto{\pgfqpoint{3.626389in}{1.429044in}}%
\pgfpathlineto{\pgfqpoint{3.679407in}{1.428756in}}%
\pgfpathlineto{\pgfqpoint{3.732424in}{1.428851in}}%
\pgfpathlineto{\pgfqpoint{3.796046in}{1.428349in}}%
\pgfpathlineto{\pgfqpoint{3.806649in}{1.427811in}}%
\pgfpathlineto{\pgfqpoint{3.827856in}{1.428029in}}%
\pgfpathlineto{\pgfqpoint{3.880874in}{1.428874in}}%
\pgfpathlineto{\pgfqpoint{3.891477in}{1.428070in}}%
\pgfpathlineto{\pgfqpoint{3.891477in}{1.428070in}}%
\pgfusepath{stroke}%
\end{pgfscope}%
\begin{pgfscope}%
\pgfpathrectangle{\pgfqpoint{0.562500in}{0.407000in}}{\pgfqpoint{3.487500in}{2.849000in}}%
\pgfusepath{clip}%
\pgfsetrectcap%
\pgfsetroundjoin%
\pgfsetlinewidth{1.505625pt}%
\definecolor{currentstroke}{rgb}{0.172549,0.627451,0.172549}%
\pgfsetstrokecolor{currentstroke}%
\pgfsetdash{}{0pt}%
\pgfpathmoveto{\pgfqpoint{0.721023in}{0.536500in}}%
\pgfpathlineto{\pgfqpoint{1.187578in}{0.536895in}}%
\pgfpathlineto{\pgfqpoint{1.198181in}{0.539708in}}%
\pgfpathlineto{\pgfqpoint{1.208785in}{0.551049in}}%
\pgfpathlineto{\pgfqpoint{1.219388in}{0.588874in}}%
\pgfpathlineto{\pgfqpoint{1.229992in}{0.685661in}}%
\pgfpathlineto{\pgfqpoint{1.240596in}{0.866310in}}%
\pgfpathlineto{\pgfqpoint{1.251199in}{1.179438in}}%
\pgfpathlineto{\pgfqpoint{1.261803in}{1.632990in}}%
\pgfpathlineto{\pgfqpoint{1.272406in}{2.145405in}}%
\pgfpathlineto{\pgfqpoint{1.283010in}{2.340406in}}%
\pgfpathlineto{\pgfqpoint{1.304217in}{2.189307in}}%
\pgfpathlineto{\pgfqpoint{1.346631in}{1.933273in}}%
\pgfpathlineto{\pgfqpoint{1.378441in}{1.783667in}}%
\pgfpathlineto{\pgfqpoint{1.399648in}{1.700850in}}%
\pgfpathlineto{\pgfqpoint{1.410252in}{1.667970in}}%
\pgfpathlineto{\pgfqpoint{1.420856in}{1.631337in}}%
\pgfpathlineto{\pgfqpoint{1.431459in}{1.601539in}}%
\pgfpathlineto{\pgfqpoint{1.442063in}{1.565942in}}%
\pgfpathlineto{\pgfqpoint{1.473873in}{1.485619in}}%
\pgfpathlineto{\pgfqpoint{1.484477in}{1.470014in}}%
\pgfpathlineto{\pgfqpoint{1.516287in}{1.415987in}}%
\pgfpathlineto{\pgfqpoint{1.537494in}{1.387519in}}%
\pgfpathlineto{\pgfqpoint{1.548098in}{1.375755in}}%
\pgfpathlineto{\pgfqpoint{1.558701in}{1.367590in}}%
\pgfpathlineto{\pgfqpoint{1.569305in}{1.354028in}}%
\pgfpathlineto{\pgfqpoint{1.579908in}{1.349836in}}%
\pgfpathlineto{\pgfqpoint{1.601115in}{1.331940in}}%
\pgfpathlineto{\pgfqpoint{1.622323in}{1.322074in}}%
\pgfpathlineto{\pgfqpoint{1.632926in}{1.319828in}}%
\pgfpathlineto{\pgfqpoint{1.664737in}{1.325344in}}%
\pgfpathlineto{\pgfqpoint{1.675340in}{1.331855in}}%
\pgfpathlineto{\pgfqpoint{1.685944in}{1.332064in}}%
\pgfpathlineto{\pgfqpoint{1.696547in}{1.337513in}}%
\pgfpathlineto{\pgfqpoint{1.707151in}{1.340484in}}%
\pgfpathlineto{\pgfqpoint{1.717754in}{1.348943in}}%
\pgfpathlineto{\pgfqpoint{1.728358in}{1.354568in}}%
\pgfpathlineto{\pgfqpoint{1.749565in}{1.373681in}}%
\pgfpathlineto{\pgfqpoint{1.760168in}{1.385196in}}%
\pgfpathlineto{\pgfqpoint{1.770772in}{1.391827in}}%
\pgfpathlineto{\pgfqpoint{1.823790in}{1.448432in}}%
\pgfpathlineto{\pgfqpoint{1.834393in}{1.458030in}}%
\pgfpathlineto{\pgfqpoint{1.855600in}{1.470103in}}%
\pgfpathlineto{\pgfqpoint{1.866204in}{1.470850in}}%
\pgfpathlineto{\pgfqpoint{1.876807in}{1.474725in}}%
\pgfpathlineto{\pgfqpoint{1.908618in}{1.474244in}}%
\pgfpathlineto{\pgfqpoint{1.919221in}{1.470145in}}%
\pgfpathlineto{\pgfqpoint{1.929825in}{1.472741in}}%
\pgfpathlineto{\pgfqpoint{1.940428in}{1.468529in}}%
\pgfpathlineto{\pgfqpoint{1.961635in}{1.463534in}}%
\pgfpathlineto{\pgfqpoint{1.982842in}{1.458412in}}%
\pgfpathlineto{\pgfqpoint{2.014653in}{1.449241in}}%
\pgfpathlineto{\pgfqpoint{2.025257in}{1.442984in}}%
\pgfpathlineto{\pgfqpoint{2.035860in}{1.443330in}}%
\pgfpathlineto{\pgfqpoint{2.046464in}{1.437727in}}%
\pgfpathlineto{\pgfqpoint{2.078274in}{1.429539in}}%
\pgfpathlineto{\pgfqpoint{2.088878in}{1.429643in}}%
\pgfpathlineto{\pgfqpoint{2.099481in}{1.425321in}}%
\pgfpathlineto{\pgfqpoint{2.120688in}{1.422228in}}%
\pgfpathlineto{\pgfqpoint{2.131292in}{1.420915in}}%
\pgfpathlineto{\pgfqpoint{2.141895in}{1.416805in}}%
\pgfpathlineto{\pgfqpoint{2.152499in}{1.415695in}}%
\pgfpathlineto{\pgfqpoint{2.163102in}{1.416652in}}%
\pgfpathlineto{\pgfqpoint{2.184309in}{1.415672in}}%
\pgfpathlineto{\pgfqpoint{2.194913in}{1.418125in}}%
\pgfpathlineto{\pgfqpoint{2.216120in}{1.415904in}}%
\pgfpathlineto{\pgfqpoint{2.226724in}{1.415107in}}%
\pgfpathlineto{\pgfqpoint{2.237327in}{1.415785in}}%
\pgfpathlineto{\pgfqpoint{2.247931in}{1.414915in}}%
\pgfpathlineto{\pgfqpoint{2.258534in}{1.416411in}}%
\pgfpathlineto{\pgfqpoint{2.269138in}{1.420122in}}%
\pgfpathlineto{\pgfqpoint{2.279741in}{1.418450in}}%
\pgfpathlineto{\pgfqpoint{2.290345in}{1.419919in}}%
\pgfpathlineto{\pgfqpoint{2.300948in}{1.418653in}}%
\pgfpathlineto{\pgfqpoint{2.311552in}{1.419258in}}%
\pgfpathlineto{\pgfqpoint{2.322155in}{1.423254in}}%
\pgfpathlineto{\pgfqpoint{2.364569in}{1.428810in}}%
\pgfpathlineto{\pgfqpoint{2.375173in}{1.428393in}}%
\pgfpathlineto{\pgfqpoint{2.385776in}{1.429723in}}%
\pgfpathlineto{\pgfqpoint{2.406984in}{1.430433in}}%
\pgfpathlineto{\pgfqpoint{2.428191in}{1.434747in}}%
\pgfpathlineto{\pgfqpoint{2.502415in}{1.434871in}}%
\pgfpathlineto{\pgfqpoint{2.513019in}{1.432992in}}%
\pgfpathlineto{\pgfqpoint{2.523622in}{1.433948in}}%
\pgfpathlineto{\pgfqpoint{2.534226in}{1.431730in}}%
\pgfpathlineto{\pgfqpoint{2.544829in}{1.431170in}}%
\pgfpathlineto{\pgfqpoint{2.555433in}{1.433482in}}%
\pgfpathlineto{\pgfqpoint{2.576640in}{1.434605in}}%
\pgfpathlineto{\pgfqpoint{2.597847in}{1.429939in}}%
\pgfpathlineto{\pgfqpoint{2.619054in}{1.432973in}}%
\pgfpathlineto{\pgfqpoint{2.629658in}{1.431641in}}%
\pgfpathlineto{\pgfqpoint{2.640261in}{1.432573in}}%
\pgfpathlineto{\pgfqpoint{2.661468in}{1.430663in}}%
\pgfpathlineto{\pgfqpoint{2.682675in}{1.430291in}}%
\pgfpathlineto{\pgfqpoint{2.693279in}{1.428741in}}%
\pgfpathlineto{\pgfqpoint{2.703882in}{1.429123in}}%
\pgfpathlineto{\pgfqpoint{2.746296in}{1.425412in}}%
\pgfpathlineto{\pgfqpoint{2.767503in}{1.426665in}}%
\pgfpathlineto{\pgfqpoint{2.788710in}{1.426521in}}%
\pgfpathlineto{\pgfqpoint{2.799314in}{1.427391in}}%
\pgfpathlineto{\pgfqpoint{2.820521in}{1.426483in}}%
\pgfpathlineto{\pgfqpoint{2.862935in}{1.428173in}}%
\pgfpathlineto{\pgfqpoint{2.873539in}{1.426349in}}%
\pgfpathlineto{\pgfqpoint{2.915953in}{1.427591in}}%
\pgfpathlineto{\pgfqpoint{2.926556in}{1.429110in}}%
\pgfpathlineto{\pgfqpoint{2.937160in}{1.428450in}}%
\pgfpathlineto{\pgfqpoint{2.958367in}{1.429372in}}%
\pgfpathlineto{\pgfqpoint{2.990177in}{1.429944in}}%
\pgfpathlineto{\pgfqpoint{3.011385in}{1.431951in}}%
\pgfpathlineto{\pgfqpoint{3.021988in}{1.431041in}}%
\pgfpathlineto{\pgfqpoint{3.032592in}{1.431347in}}%
\pgfpathlineto{\pgfqpoint{3.043195in}{1.430216in}}%
\pgfpathlineto{\pgfqpoint{3.053799in}{1.431263in}}%
\pgfpathlineto{\pgfqpoint{3.064402in}{1.429817in}}%
\pgfpathlineto{\pgfqpoint{3.075006in}{1.430635in}}%
\pgfpathlineto{\pgfqpoint{3.085609in}{1.429849in}}%
\pgfpathlineto{\pgfqpoint{3.128023in}{1.431711in}}%
\pgfpathlineto{\pgfqpoint{3.138627in}{1.430182in}}%
\pgfpathlineto{\pgfqpoint{3.159834in}{1.431071in}}%
\pgfpathlineto{\pgfqpoint{3.170437in}{1.431445in}}%
\pgfpathlineto{\pgfqpoint{3.181041in}{1.430446in}}%
\pgfpathlineto{\pgfqpoint{3.202248in}{1.431351in}}%
\pgfpathlineto{\pgfqpoint{3.212852in}{1.430246in}}%
\pgfpathlineto{\pgfqpoint{3.223455in}{1.430715in}}%
\pgfpathlineto{\pgfqpoint{3.234059in}{1.428308in}}%
\pgfpathlineto{\pgfqpoint{3.244662in}{1.427672in}}%
\pgfpathlineto{\pgfqpoint{3.265869in}{1.429225in}}%
\pgfpathlineto{\pgfqpoint{3.297680in}{1.426867in}}%
\pgfpathlineto{\pgfqpoint{3.308283in}{1.428105in}}%
\pgfpathlineto{\pgfqpoint{3.318887in}{1.426682in}}%
\pgfpathlineto{\pgfqpoint{3.340094in}{1.429896in}}%
\pgfpathlineto{\pgfqpoint{3.350697in}{1.428585in}}%
\pgfpathlineto{\pgfqpoint{3.371904in}{1.429617in}}%
\pgfpathlineto{\pgfqpoint{3.403715in}{1.428864in}}%
\pgfpathlineto{\pgfqpoint{3.424922in}{1.429888in}}%
\pgfpathlineto{\pgfqpoint{3.435526in}{1.428296in}}%
\pgfpathlineto{\pgfqpoint{3.456733in}{1.428391in}}%
\pgfpathlineto{\pgfqpoint{3.467336in}{1.427669in}}%
\pgfpathlineto{\pgfqpoint{3.499147in}{1.429109in}}%
\pgfpathlineto{\pgfqpoint{3.520354in}{1.430559in}}%
\pgfpathlineto{\pgfqpoint{3.541561in}{1.429014in}}%
\pgfpathlineto{\pgfqpoint{3.552164in}{1.427645in}}%
\pgfpathlineto{\pgfqpoint{3.562768in}{1.429042in}}%
\pgfpathlineto{\pgfqpoint{3.573371in}{1.428503in}}%
\pgfpathlineto{\pgfqpoint{3.605182in}{1.429174in}}%
\pgfpathlineto{\pgfqpoint{3.636993in}{1.430247in}}%
\pgfpathlineto{\pgfqpoint{3.647596in}{1.428135in}}%
\pgfpathlineto{\pgfqpoint{3.690010in}{1.430661in}}%
\pgfpathlineto{\pgfqpoint{3.721821in}{1.430011in}}%
\pgfpathlineto{\pgfqpoint{3.764235in}{1.430223in}}%
\pgfpathlineto{\pgfqpoint{3.796046in}{1.429760in}}%
\pgfpathlineto{\pgfqpoint{3.806649in}{1.428771in}}%
\pgfpathlineto{\pgfqpoint{3.827856in}{1.430709in}}%
\pgfpathlineto{\pgfqpoint{3.838460in}{1.430067in}}%
\pgfpathlineto{\pgfqpoint{3.870270in}{1.430444in}}%
\pgfpathlineto{\pgfqpoint{3.880874in}{1.430486in}}%
\pgfpathlineto{\pgfqpoint{3.891477in}{1.428464in}}%
\pgfpathlineto{\pgfqpoint{3.891477in}{1.428464in}}%
\pgfusepath{stroke}%
\end{pgfscope}%
\begin{pgfscope}%
\pgfpathrectangle{\pgfqpoint{0.562500in}{0.407000in}}{\pgfqpoint{3.487500in}{2.849000in}}%
\pgfusepath{clip}%
\pgfsetrectcap%
\pgfsetroundjoin%
\pgfsetlinewidth{1.505625pt}%
\definecolor{currentstroke}{rgb}{0.839216,0.152941,0.156863}%
\pgfsetstrokecolor{currentstroke}%
\pgfsetdash{}{0pt}%
\pgfpathmoveto{\pgfqpoint{0.721023in}{0.536500in}}%
\pgfpathlineto{\pgfqpoint{1.261803in}{0.536647in}}%
\pgfpathlineto{\pgfqpoint{1.272406in}{0.538693in}}%
\pgfpathlineto{\pgfqpoint{1.283010in}{0.557599in}}%
\pgfpathlineto{\pgfqpoint{1.293613in}{0.642865in}}%
\pgfpathlineto{\pgfqpoint{1.304217in}{0.898897in}}%
\pgfpathlineto{\pgfqpoint{1.314820in}{1.413609in}}%
\pgfpathlineto{\pgfqpoint{1.336027in}{3.052283in}}%
\pgfpathlineto{\pgfqpoint{1.346631in}{3.126500in}}%
\pgfpathlineto{\pgfqpoint{1.367838in}{2.732734in}}%
\pgfpathlineto{\pgfqpoint{1.378441in}{2.568089in}}%
\pgfpathlineto{\pgfqpoint{1.389045in}{2.433960in}}%
\pgfpathlineto{\pgfqpoint{1.410252in}{2.193275in}}%
\pgfpathlineto{\pgfqpoint{1.431459in}{1.988304in}}%
\pgfpathlineto{\pgfqpoint{1.442063in}{1.896003in}}%
\pgfpathlineto{\pgfqpoint{1.473873in}{1.675314in}}%
\pgfpathlineto{\pgfqpoint{1.484477in}{1.626884in}}%
\pgfpathlineto{\pgfqpoint{1.495080in}{1.571161in}}%
\pgfpathlineto{\pgfqpoint{1.537494in}{1.398457in}}%
\pgfpathlineto{\pgfqpoint{1.548098in}{1.371488in}}%
\pgfpathlineto{\pgfqpoint{1.558701in}{1.350741in}}%
\pgfpathlineto{\pgfqpoint{1.569305in}{1.316282in}}%
\pgfpathlineto{\pgfqpoint{1.579908in}{1.293834in}}%
\pgfpathlineto{\pgfqpoint{1.590512in}{1.279510in}}%
\pgfpathlineto{\pgfqpoint{1.611719in}{1.255464in}}%
\pgfpathlineto{\pgfqpoint{1.622323in}{1.238848in}}%
\pgfpathlineto{\pgfqpoint{1.632926in}{1.229207in}}%
\pgfpathlineto{\pgfqpoint{1.643530in}{1.223326in}}%
\pgfpathlineto{\pgfqpoint{1.654133in}{1.215256in}}%
\pgfpathlineto{\pgfqpoint{1.664737in}{1.210023in}}%
\pgfpathlineto{\pgfqpoint{1.675340in}{1.211160in}}%
\pgfpathlineto{\pgfqpoint{1.685944in}{1.210875in}}%
\pgfpathlineto{\pgfqpoint{1.696547in}{1.207682in}}%
\pgfpathlineto{\pgfqpoint{1.707151in}{1.211510in}}%
\pgfpathlineto{\pgfqpoint{1.717754in}{1.218439in}}%
\pgfpathlineto{\pgfqpoint{1.728358in}{1.226793in}}%
\pgfpathlineto{\pgfqpoint{1.738961in}{1.237330in}}%
\pgfpathlineto{\pgfqpoint{1.749565in}{1.245508in}}%
\pgfpathlineto{\pgfqpoint{1.781375in}{1.283857in}}%
\pgfpathlineto{\pgfqpoint{1.791979in}{1.302951in}}%
\pgfpathlineto{\pgfqpoint{1.813186in}{1.334307in}}%
\pgfpathlineto{\pgfqpoint{1.887411in}{1.462676in}}%
\pgfpathlineto{\pgfqpoint{1.898014in}{1.486461in}}%
\pgfpathlineto{\pgfqpoint{1.908618in}{1.500969in}}%
\pgfpathlineto{\pgfqpoint{1.919221in}{1.525416in}}%
\pgfpathlineto{\pgfqpoint{1.929825in}{1.545902in}}%
\pgfpathlineto{\pgfqpoint{1.940428in}{1.559236in}}%
\pgfpathlineto{\pgfqpoint{1.951032in}{1.570726in}}%
\pgfpathlineto{\pgfqpoint{1.961635in}{1.574827in}}%
\pgfpathlineto{\pgfqpoint{1.972239in}{1.571069in}}%
\pgfpathlineto{\pgfqpoint{1.982842in}{1.572344in}}%
\pgfpathlineto{\pgfqpoint{1.993446in}{1.563744in}}%
\pgfpathlineto{\pgfqpoint{2.014653in}{1.556413in}}%
\pgfpathlineto{\pgfqpoint{2.025257in}{1.543019in}}%
\pgfpathlineto{\pgfqpoint{2.035860in}{1.533075in}}%
\pgfpathlineto{\pgfqpoint{2.046464in}{1.526882in}}%
\pgfpathlineto{\pgfqpoint{2.057067in}{1.517888in}}%
\pgfpathlineto{\pgfqpoint{2.078274in}{1.486049in}}%
\pgfpathlineto{\pgfqpoint{2.088878in}{1.476758in}}%
\pgfpathlineto{\pgfqpoint{2.099481in}{1.461556in}}%
\pgfpathlineto{\pgfqpoint{2.110085in}{1.450893in}}%
\pgfpathlineto{\pgfqpoint{2.120688in}{1.442447in}}%
\pgfpathlineto{\pgfqpoint{2.131292in}{1.437038in}}%
\pgfpathlineto{\pgfqpoint{2.141895in}{1.429095in}}%
\pgfpathlineto{\pgfqpoint{2.152499in}{1.418432in}}%
\pgfpathlineto{\pgfqpoint{2.173706in}{1.401446in}}%
\pgfpathlineto{\pgfqpoint{2.184309in}{1.390327in}}%
\pgfpathlineto{\pgfqpoint{2.194913in}{1.390804in}}%
\pgfpathlineto{\pgfqpoint{2.226724in}{1.375010in}}%
\pgfpathlineto{\pgfqpoint{2.237327in}{1.377799in}}%
\pgfpathlineto{\pgfqpoint{2.247931in}{1.374478in}}%
\pgfpathlineto{\pgfqpoint{2.258534in}{1.377169in}}%
\pgfpathlineto{\pgfqpoint{2.290345in}{1.372324in}}%
\pgfpathlineto{\pgfqpoint{2.300948in}{1.374474in}}%
\pgfpathlineto{\pgfqpoint{2.311552in}{1.382640in}}%
\pgfpathlineto{\pgfqpoint{2.322155in}{1.386173in}}%
\pgfpathlineto{\pgfqpoint{2.332759in}{1.385546in}}%
\pgfpathlineto{\pgfqpoint{2.343362in}{1.388606in}}%
\pgfpathlineto{\pgfqpoint{2.364569in}{1.398392in}}%
\pgfpathlineto{\pgfqpoint{2.385776in}{1.402598in}}%
\pgfpathlineto{\pgfqpoint{2.417587in}{1.418080in}}%
\pgfpathlineto{\pgfqpoint{2.428191in}{1.421615in}}%
\pgfpathlineto{\pgfqpoint{2.438794in}{1.428620in}}%
\pgfpathlineto{\pgfqpoint{2.449398in}{1.429953in}}%
\pgfpathlineto{\pgfqpoint{2.460001in}{1.433344in}}%
\pgfpathlineto{\pgfqpoint{2.470605in}{1.438795in}}%
\pgfpathlineto{\pgfqpoint{2.481208in}{1.442901in}}%
\pgfpathlineto{\pgfqpoint{2.491812in}{1.443481in}}%
\pgfpathlineto{\pgfqpoint{2.513019in}{1.450460in}}%
\pgfpathlineto{\pgfqpoint{2.523622in}{1.453888in}}%
\pgfpathlineto{\pgfqpoint{2.544829in}{1.456645in}}%
\pgfpathlineto{\pgfqpoint{2.555433in}{1.460039in}}%
\pgfpathlineto{\pgfqpoint{2.566036in}{1.460674in}}%
\pgfpathlineto{\pgfqpoint{2.608451in}{1.456182in}}%
\pgfpathlineto{\pgfqpoint{2.619054in}{1.452131in}}%
\pgfpathlineto{\pgfqpoint{2.640261in}{1.451379in}}%
\pgfpathlineto{\pgfqpoint{2.650865in}{1.446673in}}%
\pgfpathlineto{\pgfqpoint{2.672072in}{1.445868in}}%
\pgfpathlineto{\pgfqpoint{2.682675in}{1.441668in}}%
\pgfpathlineto{\pgfqpoint{2.693279in}{1.441652in}}%
\pgfpathlineto{\pgfqpoint{2.703882in}{1.439253in}}%
\pgfpathlineto{\pgfqpoint{2.714486in}{1.434483in}}%
\pgfpathlineto{\pgfqpoint{2.746296in}{1.425792in}}%
\pgfpathlineto{\pgfqpoint{2.756900in}{1.423792in}}%
\pgfpathlineto{\pgfqpoint{2.767503in}{1.426747in}}%
\pgfpathlineto{\pgfqpoint{2.778107in}{1.423026in}}%
\pgfpathlineto{\pgfqpoint{2.820521in}{1.414364in}}%
\pgfpathlineto{\pgfqpoint{2.831125in}{1.415968in}}%
\pgfpathlineto{\pgfqpoint{2.841728in}{1.413574in}}%
\pgfpathlineto{\pgfqpoint{2.852332in}{1.415218in}}%
\pgfpathlineto{\pgfqpoint{2.862935in}{1.413844in}}%
\pgfpathlineto{\pgfqpoint{2.873539in}{1.414859in}}%
\pgfpathlineto{\pgfqpoint{2.894746in}{1.419551in}}%
\pgfpathlineto{\pgfqpoint{2.905349in}{1.418082in}}%
\pgfpathlineto{\pgfqpoint{2.915953in}{1.417839in}}%
\pgfpathlineto{\pgfqpoint{2.926556in}{1.420645in}}%
\pgfpathlineto{\pgfqpoint{2.947763in}{1.422248in}}%
\pgfpathlineto{\pgfqpoint{2.990177in}{1.427821in}}%
\pgfpathlineto{\pgfqpoint{3.011385in}{1.428881in}}%
\pgfpathlineto{\pgfqpoint{3.021988in}{1.430907in}}%
\pgfpathlineto{\pgfqpoint{3.032592in}{1.429860in}}%
\pgfpathlineto{\pgfqpoint{3.085609in}{1.432462in}}%
\pgfpathlineto{\pgfqpoint{3.096213in}{1.429653in}}%
\pgfpathlineto{\pgfqpoint{3.106816in}{1.430769in}}%
\pgfpathlineto{\pgfqpoint{3.117420in}{1.430595in}}%
\pgfpathlineto{\pgfqpoint{3.128023in}{1.432719in}}%
\pgfpathlineto{\pgfqpoint{3.159834in}{1.432953in}}%
\pgfpathlineto{\pgfqpoint{3.181041in}{1.436957in}}%
\pgfpathlineto{\pgfqpoint{3.191644in}{1.434730in}}%
\pgfpathlineto{\pgfqpoint{3.202248in}{1.436177in}}%
\pgfpathlineto{\pgfqpoint{3.234059in}{1.430235in}}%
\pgfpathlineto{\pgfqpoint{3.244662in}{1.431604in}}%
\pgfpathlineto{\pgfqpoint{3.255266in}{1.430865in}}%
\pgfpathlineto{\pgfqpoint{3.276473in}{1.434179in}}%
\pgfpathlineto{\pgfqpoint{3.287076in}{1.432648in}}%
\pgfpathlineto{\pgfqpoint{3.297680in}{1.433521in}}%
\pgfpathlineto{\pgfqpoint{3.308283in}{1.430809in}}%
\pgfpathlineto{\pgfqpoint{3.318887in}{1.431748in}}%
\pgfpathlineto{\pgfqpoint{3.340094in}{1.430430in}}%
\pgfpathlineto{\pgfqpoint{3.361301in}{1.428339in}}%
\pgfpathlineto{\pgfqpoint{3.371904in}{1.427461in}}%
\pgfpathlineto{\pgfqpoint{3.382508in}{1.429114in}}%
\pgfpathlineto{\pgfqpoint{3.393112in}{1.429153in}}%
\pgfpathlineto{\pgfqpoint{3.403715in}{1.427855in}}%
\pgfpathlineto{\pgfqpoint{3.456733in}{1.427821in}}%
\pgfpathlineto{\pgfqpoint{3.467336in}{1.426646in}}%
\pgfpathlineto{\pgfqpoint{3.488543in}{1.425967in}}%
\pgfpathlineto{\pgfqpoint{3.499147in}{1.425234in}}%
\pgfpathlineto{\pgfqpoint{3.509750in}{1.426850in}}%
\pgfpathlineto{\pgfqpoint{3.520354in}{1.425645in}}%
\pgfpathlineto{\pgfqpoint{3.541561in}{1.430089in}}%
\pgfpathlineto{\pgfqpoint{3.562768in}{1.429118in}}%
\pgfpathlineto{\pgfqpoint{3.583975in}{1.429305in}}%
\pgfpathlineto{\pgfqpoint{3.594579in}{1.431063in}}%
\pgfpathlineto{\pgfqpoint{3.605182in}{1.431200in}}%
\pgfpathlineto{\pgfqpoint{3.615786in}{1.428196in}}%
\pgfpathlineto{\pgfqpoint{3.626389in}{1.430142in}}%
\pgfpathlineto{\pgfqpoint{3.690010in}{1.429022in}}%
\pgfpathlineto{\pgfqpoint{3.700614in}{1.429479in}}%
\pgfpathlineto{\pgfqpoint{3.711217in}{1.426983in}}%
\pgfpathlineto{\pgfqpoint{3.721821in}{1.430302in}}%
\pgfpathlineto{\pgfqpoint{3.743028in}{1.429154in}}%
\pgfpathlineto{\pgfqpoint{3.753631in}{1.431206in}}%
\pgfpathlineto{\pgfqpoint{3.764235in}{1.431999in}}%
\pgfpathlineto{\pgfqpoint{3.774838in}{1.429535in}}%
\pgfpathlineto{\pgfqpoint{3.785442in}{1.430217in}}%
\pgfpathlineto{\pgfqpoint{3.806649in}{1.427528in}}%
\pgfpathlineto{\pgfqpoint{3.827856in}{1.429386in}}%
\pgfpathlineto{\pgfqpoint{3.880874in}{1.428922in}}%
\pgfpathlineto{\pgfqpoint{3.891477in}{1.430022in}}%
\pgfpathlineto{\pgfqpoint{3.891477in}{1.430022in}}%
\pgfusepath{stroke}%
\end{pgfscope}%
\begin{pgfscope}%
\pgfpathrectangle{\pgfqpoint{0.562500in}{0.407000in}}{\pgfqpoint{3.487500in}{2.849000in}}%
\pgfusepath{clip}%
\pgfsetrectcap%
\pgfsetroundjoin%
\pgfsetlinewidth{1.505625pt}%
\definecolor{currentstroke}{rgb}{0.580392,0.403922,0.741176}%
\pgfsetstrokecolor{currentstroke}%
\pgfsetdash{}{0pt}%
\pgfpathmoveto{\pgfqpoint{0.721023in}{0.536500in}}%
\pgfpathlineto{\pgfqpoint{1.272406in}{0.536776in}}%
\pgfpathlineto{\pgfqpoint{1.283010in}{0.539057in}}%
\pgfpathlineto{\pgfqpoint{1.293613in}{0.556160in}}%
\pgfpathlineto{\pgfqpoint{1.304217in}{0.624735in}}%
\pgfpathlineto{\pgfqpoint{1.314820in}{0.796426in}}%
\pgfpathlineto{\pgfqpoint{1.325424in}{1.105717in}}%
\pgfpathlineto{\pgfqpoint{1.346631in}{2.012581in}}%
\pgfpathlineto{\pgfqpoint{1.357234in}{2.461130in}}%
\pgfpathlineto{\pgfqpoint{1.367838in}{2.790234in}}%
\pgfpathlineto{\pgfqpoint{1.378441in}{2.983945in}}%
\pgfpathlineto{\pgfqpoint{1.389045in}{3.074026in}}%
\pgfpathlineto{\pgfqpoint{1.399648in}{3.045402in}}%
\pgfpathlineto{\pgfqpoint{1.410252in}{2.943695in}}%
\pgfpathlineto{\pgfqpoint{1.431459in}{2.649659in}}%
\pgfpathlineto{\pgfqpoint{1.442063in}{2.467366in}}%
\pgfpathlineto{\pgfqpoint{1.452666in}{2.322573in}}%
\pgfpathlineto{\pgfqpoint{1.463270in}{2.158646in}}%
\pgfpathlineto{\pgfqpoint{1.473873in}{2.015853in}}%
\pgfpathlineto{\pgfqpoint{1.495080in}{1.765494in}}%
\pgfpathlineto{\pgfqpoint{1.505684in}{1.661082in}}%
\pgfpathlineto{\pgfqpoint{1.526891in}{1.487930in}}%
\pgfpathlineto{\pgfqpoint{1.537494in}{1.422072in}}%
\pgfpathlineto{\pgfqpoint{1.558701in}{1.319067in}}%
\pgfpathlineto{\pgfqpoint{1.569305in}{1.275249in}}%
\pgfpathlineto{\pgfqpoint{1.590512in}{1.206151in}}%
\pgfpathlineto{\pgfqpoint{1.601115in}{1.182926in}}%
\pgfpathlineto{\pgfqpoint{1.611719in}{1.156042in}}%
\pgfpathlineto{\pgfqpoint{1.622323in}{1.140032in}}%
\pgfpathlineto{\pgfqpoint{1.632926in}{1.121382in}}%
\pgfpathlineto{\pgfqpoint{1.643530in}{1.111146in}}%
\pgfpathlineto{\pgfqpoint{1.654133in}{1.096526in}}%
\pgfpathlineto{\pgfqpoint{1.675340in}{1.087228in}}%
\pgfpathlineto{\pgfqpoint{1.685944in}{1.085606in}}%
\pgfpathlineto{\pgfqpoint{1.696547in}{1.086200in}}%
\pgfpathlineto{\pgfqpoint{1.707151in}{1.089583in}}%
\pgfpathlineto{\pgfqpoint{1.717754in}{1.094819in}}%
\pgfpathlineto{\pgfqpoint{1.728358in}{1.108826in}}%
\pgfpathlineto{\pgfqpoint{1.738961in}{1.120806in}}%
\pgfpathlineto{\pgfqpoint{1.760168in}{1.150466in}}%
\pgfpathlineto{\pgfqpoint{1.770772in}{1.166754in}}%
\pgfpathlineto{\pgfqpoint{1.781375in}{1.186975in}}%
\pgfpathlineto{\pgfqpoint{1.791979in}{1.210288in}}%
\pgfpathlineto{\pgfqpoint{1.802582in}{1.241040in}}%
\pgfpathlineto{\pgfqpoint{1.823790in}{1.290305in}}%
\pgfpathlineto{\pgfqpoint{1.844997in}{1.350102in}}%
\pgfpathlineto{\pgfqpoint{1.855600in}{1.377608in}}%
\pgfpathlineto{\pgfqpoint{1.866204in}{1.410873in}}%
\pgfpathlineto{\pgfqpoint{1.876807in}{1.436321in}}%
\pgfpathlineto{\pgfqpoint{1.898014in}{1.495012in}}%
\pgfpathlineto{\pgfqpoint{1.919221in}{1.543580in}}%
\pgfpathlineto{\pgfqpoint{1.929825in}{1.562745in}}%
\pgfpathlineto{\pgfqpoint{1.940428in}{1.578810in}}%
\pgfpathlineto{\pgfqpoint{1.951032in}{1.604773in}}%
\pgfpathlineto{\pgfqpoint{1.961635in}{1.621457in}}%
\pgfpathlineto{\pgfqpoint{1.972239in}{1.632324in}}%
\pgfpathlineto{\pgfqpoint{1.982842in}{1.640394in}}%
\pgfpathlineto{\pgfqpoint{1.993446in}{1.656685in}}%
\pgfpathlineto{\pgfqpoint{2.025257in}{1.659885in}}%
\pgfpathlineto{\pgfqpoint{2.046464in}{1.641001in}}%
\pgfpathlineto{\pgfqpoint{2.057067in}{1.631594in}}%
\pgfpathlineto{\pgfqpoint{2.067671in}{1.615487in}}%
\pgfpathlineto{\pgfqpoint{2.078274in}{1.591447in}}%
\pgfpathlineto{\pgfqpoint{2.110085in}{1.529536in}}%
\pgfpathlineto{\pgfqpoint{2.120688in}{1.508621in}}%
\pgfpathlineto{\pgfqpoint{2.141895in}{1.457598in}}%
\pgfpathlineto{\pgfqpoint{2.152499in}{1.442305in}}%
\pgfpathlineto{\pgfqpoint{2.184309in}{1.383329in}}%
\pgfpathlineto{\pgfqpoint{2.194913in}{1.370509in}}%
\pgfpathlineto{\pgfqpoint{2.205516in}{1.355316in}}%
\pgfpathlineto{\pgfqpoint{2.216120in}{1.336944in}}%
\pgfpathlineto{\pgfqpoint{2.226724in}{1.327844in}}%
\pgfpathlineto{\pgfqpoint{2.247931in}{1.315669in}}%
\pgfpathlineto{\pgfqpoint{2.258534in}{1.304632in}}%
\pgfpathlineto{\pgfqpoint{2.269138in}{1.299921in}}%
\pgfpathlineto{\pgfqpoint{2.279741in}{1.297309in}}%
\pgfpathlineto{\pgfqpoint{2.300948in}{1.300951in}}%
\pgfpathlineto{\pgfqpoint{2.322155in}{1.307422in}}%
\pgfpathlineto{\pgfqpoint{2.332759in}{1.312458in}}%
\pgfpathlineto{\pgfqpoint{2.343362in}{1.315196in}}%
\pgfpathlineto{\pgfqpoint{2.353966in}{1.321046in}}%
\pgfpathlineto{\pgfqpoint{2.364569in}{1.330221in}}%
\pgfpathlineto{\pgfqpoint{2.375173in}{1.344595in}}%
\pgfpathlineto{\pgfqpoint{2.385776in}{1.354068in}}%
\pgfpathlineto{\pgfqpoint{2.396380in}{1.367393in}}%
\pgfpathlineto{\pgfqpoint{2.406984in}{1.373539in}}%
\pgfpathlineto{\pgfqpoint{2.417587in}{1.384926in}}%
\pgfpathlineto{\pgfqpoint{2.428191in}{1.399451in}}%
\pgfpathlineto{\pgfqpoint{2.460001in}{1.432526in}}%
\pgfpathlineto{\pgfqpoint{2.470605in}{1.443381in}}%
\pgfpathlineto{\pgfqpoint{2.481208in}{1.456221in}}%
\pgfpathlineto{\pgfqpoint{2.491812in}{1.465477in}}%
\pgfpathlineto{\pgfqpoint{2.502415in}{1.476676in}}%
\pgfpathlineto{\pgfqpoint{2.513019in}{1.486199in}}%
\pgfpathlineto{\pgfqpoint{2.523622in}{1.492396in}}%
\pgfpathlineto{\pgfqpoint{2.534226in}{1.496214in}}%
\pgfpathlineto{\pgfqpoint{2.544829in}{1.502887in}}%
\pgfpathlineto{\pgfqpoint{2.555433in}{1.506854in}}%
\pgfpathlineto{\pgfqpoint{2.576640in}{1.508991in}}%
\pgfpathlineto{\pgfqpoint{2.587243in}{1.511991in}}%
\pgfpathlineto{\pgfqpoint{2.597847in}{1.505141in}}%
\pgfpathlineto{\pgfqpoint{2.629658in}{1.499413in}}%
\pgfpathlineto{\pgfqpoint{2.640261in}{1.493653in}}%
\pgfpathlineto{\pgfqpoint{2.650865in}{1.489246in}}%
\pgfpathlineto{\pgfqpoint{2.661468in}{1.482230in}}%
\pgfpathlineto{\pgfqpoint{2.672072in}{1.478189in}}%
\pgfpathlineto{\pgfqpoint{2.682675in}{1.472369in}}%
\pgfpathlineto{\pgfqpoint{2.693279in}{1.468144in}}%
\pgfpathlineto{\pgfqpoint{2.703882in}{1.465368in}}%
\pgfpathlineto{\pgfqpoint{2.725089in}{1.450864in}}%
\pgfpathlineto{\pgfqpoint{2.778107in}{1.409917in}}%
\pgfpathlineto{\pgfqpoint{2.788710in}{1.405066in}}%
\pgfpathlineto{\pgfqpoint{2.820521in}{1.394250in}}%
\pgfpathlineto{\pgfqpoint{2.831125in}{1.388655in}}%
\pgfpathlineto{\pgfqpoint{2.841728in}{1.387176in}}%
\pgfpathlineto{\pgfqpoint{2.873539in}{1.377483in}}%
\pgfpathlineto{\pgfqpoint{2.915953in}{1.380567in}}%
\pgfpathlineto{\pgfqpoint{2.926556in}{1.383987in}}%
\pgfpathlineto{\pgfqpoint{2.937160in}{1.389192in}}%
\pgfpathlineto{\pgfqpoint{2.947763in}{1.391993in}}%
\pgfpathlineto{\pgfqpoint{2.958367in}{1.396397in}}%
\pgfpathlineto{\pgfqpoint{2.990177in}{1.402829in}}%
\pgfpathlineto{\pgfqpoint{3.000781in}{1.409191in}}%
\pgfpathlineto{\pgfqpoint{3.011385in}{1.413062in}}%
\pgfpathlineto{\pgfqpoint{3.021988in}{1.419360in}}%
\pgfpathlineto{\pgfqpoint{3.032592in}{1.422001in}}%
\pgfpathlineto{\pgfqpoint{3.043195in}{1.428363in}}%
\pgfpathlineto{\pgfqpoint{3.053799in}{1.431336in}}%
\pgfpathlineto{\pgfqpoint{3.064402in}{1.437314in}}%
\pgfpathlineto{\pgfqpoint{3.085609in}{1.445419in}}%
\pgfpathlineto{\pgfqpoint{3.106816in}{1.450190in}}%
\pgfpathlineto{\pgfqpoint{3.117420in}{1.453837in}}%
\pgfpathlineto{\pgfqpoint{3.149230in}{1.461626in}}%
\pgfpathlineto{\pgfqpoint{3.159834in}{1.464318in}}%
\pgfpathlineto{\pgfqpoint{3.170437in}{1.463988in}}%
\pgfpathlineto{\pgfqpoint{3.181041in}{1.462541in}}%
\pgfpathlineto{\pgfqpoint{3.191644in}{1.463902in}}%
\pgfpathlineto{\pgfqpoint{3.212852in}{1.459068in}}%
\pgfpathlineto{\pgfqpoint{3.223455in}{1.459104in}}%
\pgfpathlineto{\pgfqpoint{3.244662in}{1.454321in}}%
\pgfpathlineto{\pgfqpoint{3.255266in}{1.449434in}}%
\pgfpathlineto{\pgfqpoint{3.265869in}{1.448343in}}%
\pgfpathlineto{\pgfqpoint{3.287076in}{1.441166in}}%
\pgfpathlineto{\pgfqpoint{3.308283in}{1.434715in}}%
\pgfpathlineto{\pgfqpoint{3.318887in}{1.434720in}}%
\pgfpathlineto{\pgfqpoint{3.329490in}{1.433461in}}%
\pgfpathlineto{\pgfqpoint{3.340094in}{1.428577in}}%
\pgfpathlineto{\pgfqpoint{3.350697in}{1.425936in}}%
\pgfpathlineto{\pgfqpoint{3.371904in}{1.424510in}}%
\pgfpathlineto{\pgfqpoint{3.382508in}{1.422228in}}%
\pgfpathlineto{\pgfqpoint{3.393112in}{1.418612in}}%
\pgfpathlineto{\pgfqpoint{3.414319in}{1.416642in}}%
\pgfpathlineto{\pgfqpoint{3.424922in}{1.411753in}}%
\pgfpathlineto{\pgfqpoint{3.435526in}{1.409447in}}%
\pgfpathlineto{\pgfqpoint{3.456733in}{1.408625in}}%
\pgfpathlineto{\pgfqpoint{3.499147in}{1.408635in}}%
\pgfpathlineto{\pgfqpoint{3.509750in}{1.408712in}}%
\pgfpathlineto{\pgfqpoint{3.541561in}{1.412819in}}%
\pgfpathlineto{\pgfqpoint{3.562768in}{1.414191in}}%
\pgfpathlineto{\pgfqpoint{3.573371in}{1.418194in}}%
\pgfpathlineto{\pgfqpoint{3.605182in}{1.419906in}}%
\pgfpathlineto{\pgfqpoint{3.636993in}{1.430527in}}%
\pgfpathlineto{\pgfqpoint{3.658200in}{1.433476in}}%
\pgfpathlineto{\pgfqpoint{3.668803in}{1.434627in}}%
\pgfpathlineto{\pgfqpoint{3.679407in}{1.438010in}}%
\pgfpathlineto{\pgfqpoint{3.700614in}{1.439212in}}%
\pgfpathlineto{\pgfqpoint{3.711217in}{1.442495in}}%
\pgfpathlineto{\pgfqpoint{3.721821in}{1.441503in}}%
\pgfpathlineto{\pgfqpoint{3.732424in}{1.443139in}}%
\pgfpathlineto{\pgfqpoint{3.753631in}{1.444748in}}%
\pgfpathlineto{\pgfqpoint{3.764235in}{1.445647in}}%
\pgfpathlineto{\pgfqpoint{3.796046in}{1.443146in}}%
\pgfpathlineto{\pgfqpoint{3.817253in}{1.442716in}}%
\pgfpathlineto{\pgfqpoint{3.838460in}{1.441964in}}%
\pgfpathlineto{\pgfqpoint{3.849063in}{1.440408in}}%
\pgfpathlineto{\pgfqpoint{3.859667in}{1.437395in}}%
\pgfpathlineto{\pgfqpoint{3.891477in}{1.432855in}}%
\pgfpathlineto{\pgfqpoint{3.891477in}{1.432855in}}%
\pgfusepath{stroke}%
\end{pgfscope}%
\begin{pgfscope}%
\pgfsetrectcap%
\pgfsetmiterjoin%
\pgfsetlinewidth{0.803000pt}%
\definecolor{currentstroke}{rgb}{0.000000,0.000000,0.000000}%
\pgfsetstrokecolor{currentstroke}%
\pgfsetdash{}{0pt}%
\pgfpathmoveto{\pgfqpoint{0.562500in}{0.407000in}}%
\pgfpathlineto{\pgfqpoint{0.562500in}{3.256000in}}%
\pgfusepath{stroke}%
\end{pgfscope}%
\begin{pgfscope}%
\pgfsetrectcap%
\pgfsetmiterjoin%
\pgfsetlinewidth{0.803000pt}%
\definecolor{currentstroke}{rgb}{0.000000,0.000000,0.000000}%
\pgfsetstrokecolor{currentstroke}%
\pgfsetdash{}{0pt}%
\pgfpathmoveto{\pgfqpoint{4.050000in}{0.407000in}}%
\pgfpathlineto{\pgfqpoint{4.050000in}{3.256000in}}%
\pgfusepath{stroke}%
\end{pgfscope}%
\begin{pgfscope}%
\pgfsetrectcap%
\pgfsetmiterjoin%
\pgfsetlinewidth{0.803000pt}%
\definecolor{currentstroke}{rgb}{0.000000,0.000000,0.000000}%
\pgfsetstrokecolor{currentstroke}%
\pgfsetdash{}{0pt}%
\pgfpathmoveto{\pgfqpoint{0.562500in}{0.407000in}}%
\pgfpathlineto{\pgfqpoint{4.050000in}{0.407000in}}%
\pgfusepath{stroke}%
\end{pgfscope}%
\begin{pgfscope}%
\pgfsetrectcap%
\pgfsetmiterjoin%
\pgfsetlinewidth{0.803000pt}%
\definecolor{currentstroke}{rgb}{0.000000,0.000000,0.000000}%
\pgfsetstrokecolor{currentstroke}%
\pgfsetdash{}{0pt}%
\pgfpathmoveto{\pgfqpoint{0.562500in}{3.256000in}}%
\pgfpathlineto{\pgfqpoint{4.050000in}{3.256000in}}%
\pgfusepath{stroke}%
\end{pgfscope}%
\begin{pgfscope}%
\pgfsetbuttcap%
\pgfsetmiterjoin%
\definecolor{currentfill}{rgb}{1.000000,1.000000,1.000000}%
\pgfsetfillcolor{currentfill}%
\pgfsetfillopacity{0.800000}%
\pgfsetlinewidth{1.003750pt}%
\definecolor{currentstroke}{rgb}{0.800000,0.800000,0.800000}%
\pgfsetstrokecolor{currentstroke}%
\pgfsetstrokeopacity{0.800000}%
\pgfsetdash{}{0pt}%
\pgfpathmoveto{\pgfqpoint{2.733880in}{2.112869in}}%
\pgfpathlineto{\pgfqpoint{3.952778in}{2.112869in}}%
\pgfpathquadraticcurveto{\pgfqpoint{3.980556in}{2.112869in}}{\pgfqpoint{3.980556in}{2.140647in}}%
\pgfpathlineto{\pgfqpoint{3.980556in}{3.158778in}}%
\pgfpathquadraticcurveto{\pgfqpoint{3.980556in}{3.186556in}}{\pgfqpoint{3.952778in}{3.186556in}}%
\pgfpathlineto{\pgfqpoint{2.733880in}{3.186556in}}%
\pgfpathquadraticcurveto{\pgfqpoint{2.706102in}{3.186556in}}{\pgfqpoint{2.706102in}{3.158778in}}%
\pgfpathlineto{\pgfqpoint{2.706102in}{2.140647in}}%
\pgfpathquadraticcurveto{\pgfqpoint{2.706102in}{2.112869in}}{\pgfqpoint{2.733880in}{2.112869in}}%
\pgfpathclose%
\pgfusepath{stroke,fill}%
\end{pgfscope}%
\begin{pgfscope}%
\pgfsetrectcap%
\pgfsetroundjoin%
\pgfsetlinewidth{1.505625pt}%
\definecolor{currentstroke}{rgb}{0.121569,0.466667,0.705882}%
\pgfsetstrokecolor{currentstroke}%
\pgfsetdash{}{0pt}%
\pgfpathmoveto{\pgfqpoint{2.761658in}{3.082389in}}%
\pgfpathlineto{\pgfqpoint{3.039435in}{3.082389in}}%
\pgfusepath{stroke}%
\end{pgfscope}%
\begin{pgfscope}%
\definecolor{textcolor}{rgb}{0.000000,0.000000,0.000000}%
\pgfsetstrokecolor{textcolor}%
\pgfsetfillcolor{textcolor}%
\pgftext[x=3.150547in,y=3.033778in,left,base]{\color{textcolor}\rmfamily\fontsize{10.000000}{12.000000}\selectfont \(\displaystyle r_{cutoff} = 0.7\)}%
\end{pgfscope}%
\begin{pgfscope}%
\pgfsetrectcap%
\pgfsetroundjoin%
\pgfsetlinewidth{1.505625pt}%
\definecolor{currentstroke}{rgb}{1.000000,0.498039,0.054902}%
\pgfsetstrokecolor{currentstroke}%
\pgfsetdash{}{0pt}%
\pgfpathmoveto{\pgfqpoint{2.761658in}{2.875985in}}%
\pgfpathlineto{\pgfqpoint{3.039435in}{2.875985in}}%
\pgfusepath{stroke}%
\end{pgfscope}%
\begin{pgfscope}%
\definecolor{textcolor}{rgb}{0.000000,0.000000,0.000000}%
\pgfsetstrokecolor{textcolor}%
\pgfsetfillcolor{textcolor}%
\pgftext[x=3.150547in,y=2.827374in,left,base]{\color{textcolor}\rmfamily\fontsize{10.000000}{12.000000}\selectfont \(\displaystyle r_{cutoff} = 0.8\)}%
\end{pgfscope}%
\begin{pgfscope}%
\pgfsetrectcap%
\pgfsetroundjoin%
\pgfsetlinewidth{1.505625pt}%
\definecolor{currentstroke}{rgb}{0.172549,0.627451,0.172549}%
\pgfsetstrokecolor{currentstroke}%
\pgfsetdash{}{0pt}%
\pgfpathmoveto{\pgfqpoint{2.761658in}{2.669581in}}%
\pgfpathlineto{\pgfqpoint{3.039435in}{2.669581in}}%
\pgfusepath{stroke}%
\end{pgfscope}%
\begin{pgfscope}%
\definecolor{textcolor}{rgb}{0.000000,0.000000,0.000000}%
\pgfsetstrokecolor{textcolor}%
\pgfsetfillcolor{textcolor}%
\pgftext[x=3.150547in,y=2.620970in,left,base]{\color{textcolor}\rmfamily\fontsize{10.000000}{12.000000}\selectfont \(\displaystyle r_{cutoff} = 0.9\)}%
\end{pgfscope}%
\begin{pgfscope}%
\pgfsetrectcap%
\pgfsetroundjoin%
\pgfsetlinewidth{1.505625pt}%
\definecolor{currentstroke}{rgb}{0.839216,0.152941,0.156863}%
\pgfsetstrokecolor{currentstroke}%
\pgfsetdash{}{0pt}%
\pgfpathmoveto{\pgfqpoint{2.761658in}{2.463177in}}%
\pgfpathlineto{\pgfqpoint{3.039435in}{2.463177in}}%
\pgfusepath{stroke}%
\end{pgfscope}%
\begin{pgfscope}%
\definecolor{textcolor}{rgb}{0.000000,0.000000,0.000000}%
\pgfsetstrokecolor{textcolor}%
\pgfsetfillcolor{textcolor}%
\pgftext[x=3.150547in,y=2.414566in,left,base]{\color{textcolor}\rmfamily\fontsize{10.000000}{12.000000}\selectfont \(\displaystyle r_{cutoff} = 1.0\)}%
\end{pgfscope}%
\begin{pgfscope}%
\pgfsetrectcap%
\pgfsetroundjoin%
\pgfsetlinewidth{1.505625pt}%
\definecolor{currentstroke}{rgb}{0.580392,0.403922,0.741176}%
\pgfsetstrokecolor{currentstroke}%
\pgfsetdash{}{0pt}%
\pgfpathmoveto{\pgfqpoint{2.761658in}{2.256773in}}%
\pgfpathlineto{\pgfqpoint{3.039435in}{2.256773in}}%
\pgfusepath{stroke}%
\end{pgfscope}%
\begin{pgfscope}%
\definecolor{textcolor}{rgb}{0.000000,0.000000,0.000000}%
\pgfsetstrokecolor{textcolor}%
\pgfsetfillcolor{textcolor}%
\pgftext[x=3.150547in,y=2.208162in,left,base]{\color{textcolor}\rmfamily\fontsize{10.000000}{12.000000}\selectfont \(\displaystyle r_{cutoff} = 2.5\)}%
\end{pgfscope}%
\end{pgfpicture}%
\makeatother%
\endgroup%

        }
        \caption{Varying $r_{cutoff}$}
        \label{step2_changeR-gofr}
    \end{subfigure}
    \caption{Plot of the radial pair correlation function $g(r)$ at $T = 94.4$ \si{K} for different parameters.}
\end{figure}

% show form of new potential
% plot g(r) as function of r_cutoff

\section{Detecting the diffusion coefficient} \label{sec:diffusion}

The diffusion coefficient $D$ appearing in the diffusion equation \cite{course05}

\begin{equation}\label{eq:diffusion}
    \frac{\partial c}{\partial t}(t,\vec{x}) = D \nabla^2 c(t,\vec{x})
\end{equation}

can be estimated in two ways: (a) using the Einstein relation for the Mean square displacement (eq. \ref{eq:Dmsd}), or (b) integrating the velocity auto-correlation function (eq. \ref{eq:Dvafc}).
Both methods are valid and should theoretically lead to the same result \cite{course05}. The equations for both methods are provided:

\begin{align}
& D_{MSD} = \frac{1}{6} \frac{\partial \langle r^2 \rangle}{\partial t} \label{eq:Dmsd} \\
& D_{VAFC} = \frac{1}{3} \int_0^\infty d\tau \; \langle \vec{v}(\tau) \vec{v}(0) \rangle \label{eq:Dvafc}
\end{align}

\begin{figure}
    \begin{subfigure}{0.5\textwidth}
        \resizebox{\textwidth}{!}{
            %% Creator: Matplotlib, PGF backend
%%
%% To include the figure in your LaTeX document, write
%%   \input{<filename>.pgf}
%%
%% Make sure the required packages are loaded in your preamble
%%   \usepackage{pgf}
%%
%% Also ensure that all the required font packages are loaded; for instance,
%% the lmodern package is sometimes necessary when using math font.
%%   \usepackage{lmodern}
%%
%% Figures using additional raster images can only be included by \input if
%% they are in the same directory as the main LaTeX file. For loading figures
%% from other directories you can use the `import` package
%%   \usepackage{import}
%%
%% and then include the figures with
%%   \import{<path to file>}{<filename>.pgf}
%%
%% Matplotlib used the following preamble
%%   \usepackage[utf8]{inputenc}
%%   \usepackage[T1]{fontenc}
%%   \usepackage{siunitx}
%%
\begingroup%
\makeatletter%
\begin{pgfpicture}%
\pgfpathrectangle{\pgfpointorigin}{\pgfqpoint{4.500000in}{3.700000in}}%
\pgfusepath{use as bounding box, clip}%
\begin{pgfscope}%
\pgfsetbuttcap%
\pgfsetmiterjoin%
\definecolor{currentfill}{rgb}{1.000000,1.000000,1.000000}%
\pgfsetfillcolor{currentfill}%
\pgfsetlinewidth{0.000000pt}%
\definecolor{currentstroke}{rgb}{1.000000,1.000000,1.000000}%
\pgfsetstrokecolor{currentstroke}%
\pgfsetdash{}{0pt}%
\pgfpathmoveto{\pgfqpoint{0.000000in}{0.000000in}}%
\pgfpathlineto{\pgfqpoint{4.500000in}{0.000000in}}%
\pgfpathlineto{\pgfqpoint{4.500000in}{3.700000in}}%
\pgfpathlineto{\pgfqpoint{0.000000in}{3.700000in}}%
\pgfpathlineto{\pgfqpoint{0.000000in}{0.000000in}}%
\pgfpathclose%
\pgfusepath{fill}%
\end{pgfscope}%
\begin{pgfscope}%
\pgfsetbuttcap%
\pgfsetmiterjoin%
\definecolor{currentfill}{rgb}{1.000000,1.000000,1.000000}%
\pgfsetfillcolor{currentfill}%
\pgfsetlinewidth{0.000000pt}%
\definecolor{currentstroke}{rgb}{0.000000,0.000000,0.000000}%
\pgfsetstrokecolor{currentstroke}%
\pgfsetstrokeopacity{0.000000}%
\pgfsetdash{}{0pt}%
\pgfpathmoveto{\pgfqpoint{0.562500in}{0.407000in}}%
\pgfpathlineto{\pgfqpoint{4.050000in}{0.407000in}}%
\pgfpathlineto{\pgfqpoint{4.050000in}{3.256000in}}%
\pgfpathlineto{\pgfqpoint{0.562500in}{3.256000in}}%
\pgfpathlineto{\pgfqpoint{0.562500in}{0.407000in}}%
\pgfpathclose%
\pgfusepath{fill}%
\end{pgfscope}%
\begin{pgfscope}%
\pgfsetbuttcap%
\pgfsetroundjoin%
\definecolor{currentfill}{rgb}{0.000000,0.000000,0.000000}%
\pgfsetfillcolor{currentfill}%
\pgfsetlinewidth{0.803000pt}%
\definecolor{currentstroke}{rgb}{0.000000,0.000000,0.000000}%
\pgfsetstrokecolor{currentstroke}%
\pgfsetdash{}{0pt}%
\pgfsys@defobject{currentmarker}{\pgfqpoint{0.000000in}{-0.048611in}}{\pgfqpoint{0.000000in}{0.000000in}}{%
\pgfpathmoveto{\pgfqpoint{0.000000in}{0.000000in}}%
\pgfpathlineto{\pgfqpoint{0.000000in}{-0.048611in}}%
\pgfusepath{stroke,fill}%
}%
\begin{pgfscope}%
\pgfsys@transformshift{0.713962in}{0.407000in}%
\pgfsys@useobject{currentmarker}{}%
\end{pgfscope}%
\end{pgfscope}%
\begin{pgfscope}%
\definecolor{textcolor}{rgb}{0.000000,0.000000,0.000000}%
\pgfsetstrokecolor{textcolor}%
\pgfsetfillcolor{textcolor}%
\pgftext[x=0.713962in,y=0.309778in,,top]{\color{textcolor}\rmfamily\fontsize{10.000000}{12.000000}\selectfont \(\displaystyle {0.0}\)}%
\end{pgfscope}%
\begin{pgfscope}%
\pgfsetbuttcap%
\pgfsetroundjoin%
\definecolor{currentfill}{rgb}{0.000000,0.000000,0.000000}%
\pgfsetfillcolor{currentfill}%
\pgfsetlinewidth{0.803000pt}%
\definecolor{currentstroke}{rgb}{0.000000,0.000000,0.000000}%
\pgfsetstrokecolor{currentstroke}%
\pgfsetdash{}{0pt}%
\pgfsys@defobject{currentmarker}{\pgfqpoint{0.000000in}{-0.048611in}}{\pgfqpoint{0.000000in}{0.000000in}}{%
\pgfpathmoveto{\pgfqpoint{0.000000in}{0.000000in}}%
\pgfpathlineto{\pgfqpoint{0.000000in}{-0.048611in}}%
\pgfusepath{stroke,fill}%
}%
\begin{pgfscope}%
\pgfsys@transformshift{1.420076in}{0.407000in}%
\pgfsys@useobject{currentmarker}{}%
\end{pgfscope}%
\end{pgfscope}%
\begin{pgfscope}%
\definecolor{textcolor}{rgb}{0.000000,0.000000,0.000000}%
\pgfsetstrokecolor{textcolor}%
\pgfsetfillcolor{textcolor}%
\pgftext[x=1.420076in,y=0.309778in,,top]{\color{textcolor}\rmfamily\fontsize{10.000000}{12.000000}\selectfont \(\displaystyle {0.2}\)}%
\end{pgfscope}%
\begin{pgfscope}%
\pgfsetbuttcap%
\pgfsetroundjoin%
\definecolor{currentfill}{rgb}{0.000000,0.000000,0.000000}%
\pgfsetfillcolor{currentfill}%
\pgfsetlinewidth{0.803000pt}%
\definecolor{currentstroke}{rgb}{0.000000,0.000000,0.000000}%
\pgfsetstrokecolor{currentstroke}%
\pgfsetdash{}{0pt}%
\pgfsys@defobject{currentmarker}{\pgfqpoint{0.000000in}{-0.048611in}}{\pgfqpoint{0.000000in}{0.000000in}}{%
\pgfpathmoveto{\pgfqpoint{0.000000in}{0.000000in}}%
\pgfpathlineto{\pgfqpoint{0.000000in}{-0.048611in}}%
\pgfusepath{stroke,fill}%
}%
\begin{pgfscope}%
\pgfsys@transformshift{2.126191in}{0.407000in}%
\pgfsys@useobject{currentmarker}{}%
\end{pgfscope}%
\end{pgfscope}%
\begin{pgfscope}%
\definecolor{textcolor}{rgb}{0.000000,0.000000,0.000000}%
\pgfsetstrokecolor{textcolor}%
\pgfsetfillcolor{textcolor}%
\pgftext[x=2.126191in,y=0.309778in,,top]{\color{textcolor}\rmfamily\fontsize{10.000000}{12.000000}\selectfont \(\displaystyle {0.4}\)}%
\end{pgfscope}%
\begin{pgfscope}%
\pgfsetbuttcap%
\pgfsetroundjoin%
\definecolor{currentfill}{rgb}{0.000000,0.000000,0.000000}%
\pgfsetfillcolor{currentfill}%
\pgfsetlinewidth{0.803000pt}%
\definecolor{currentstroke}{rgb}{0.000000,0.000000,0.000000}%
\pgfsetstrokecolor{currentstroke}%
\pgfsetdash{}{0pt}%
\pgfsys@defobject{currentmarker}{\pgfqpoint{0.000000in}{-0.048611in}}{\pgfqpoint{0.000000in}{0.000000in}}{%
\pgfpathmoveto{\pgfqpoint{0.000000in}{0.000000in}}%
\pgfpathlineto{\pgfqpoint{0.000000in}{-0.048611in}}%
\pgfusepath{stroke,fill}%
}%
\begin{pgfscope}%
\pgfsys@transformshift{2.832305in}{0.407000in}%
\pgfsys@useobject{currentmarker}{}%
\end{pgfscope}%
\end{pgfscope}%
\begin{pgfscope}%
\definecolor{textcolor}{rgb}{0.000000,0.000000,0.000000}%
\pgfsetstrokecolor{textcolor}%
\pgfsetfillcolor{textcolor}%
\pgftext[x=2.832305in,y=0.309778in,,top]{\color{textcolor}\rmfamily\fontsize{10.000000}{12.000000}\selectfont \(\displaystyle {0.6}\)}%
\end{pgfscope}%
\begin{pgfscope}%
\pgfsetbuttcap%
\pgfsetroundjoin%
\definecolor{currentfill}{rgb}{0.000000,0.000000,0.000000}%
\pgfsetfillcolor{currentfill}%
\pgfsetlinewidth{0.803000pt}%
\definecolor{currentstroke}{rgb}{0.000000,0.000000,0.000000}%
\pgfsetstrokecolor{currentstroke}%
\pgfsetdash{}{0pt}%
\pgfsys@defobject{currentmarker}{\pgfqpoint{0.000000in}{-0.048611in}}{\pgfqpoint{0.000000in}{0.000000in}}{%
\pgfpathmoveto{\pgfqpoint{0.000000in}{0.000000in}}%
\pgfpathlineto{\pgfqpoint{0.000000in}{-0.048611in}}%
\pgfusepath{stroke,fill}%
}%
\begin{pgfscope}%
\pgfsys@transformshift{3.538420in}{0.407000in}%
\pgfsys@useobject{currentmarker}{}%
\end{pgfscope}%
\end{pgfscope}%
\begin{pgfscope}%
\definecolor{textcolor}{rgb}{0.000000,0.000000,0.000000}%
\pgfsetstrokecolor{textcolor}%
\pgfsetfillcolor{textcolor}%
\pgftext[x=3.538420in,y=0.309778in,,top]{\color{textcolor}\rmfamily\fontsize{10.000000}{12.000000}\selectfont \(\displaystyle {0.8}\)}%
\end{pgfscope}%
\begin{pgfscope}%
\definecolor{textcolor}{rgb}{0.000000,0.000000,0.000000}%
\pgfsetstrokecolor{textcolor}%
\pgfsetfillcolor{textcolor}%
\pgftext[x=2.306250in,y=0.131567in,,top]{\color{textcolor}\rmfamily\fontsize{10.000000}{12.000000}\selectfont \(\displaystyle t\)}%
\end{pgfscope}%
\begin{pgfscope}%
\pgfsetbuttcap%
\pgfsetroundjoin%
\definecolor{currentfill}{rgb}{0.000000,0.000000,0.000000}%
\pgfsetfillcolor{currentfill}%
\pgfsetlinewidth{0.803000pt}%
\definecolor{currentstroke}{rgb}{0.000000,0.000000,0.000000}%
\pgfsetstrokecolor{currentstroke}%
\pgfsetdash{}{0pt}%
\pgfsys@defobject{currentmarker}{\pgfqpoint{-0.048611in}{0.000000in}}{\pgfqpoint{-0.000000in}{0.000000in}}{%
\pgfpathmoveto{\pgfqpoint{-0.000000in}{0.000000in}}%
\pgfpathlineto{\pgfqpoint{-0.048611in}{0.000000in}}%
\pgfusepath{stroke,fill}%
}%
\begin{pgfscope}%
\pgfsys@transformshift{0.562500in}{0.536500in}%
\pgfsys@useobject{currentmarker}{}%
\end{pgfscope}%
\end{pgfscope}%
\begin{pgfscope}%
\definecolor{textcolor}{rgb}{0.000000,0.000000,0.000000}%
\pgfsetstrokecolor{textcolor}%
\pgfsetfillcolor{textcolor}%
\pgftext[x=0.287808in, y=0.488672in, left, base]{\color{textcolor}\rmfamily\fontsize{10.000000}{12.000000}\selectfont \(\displaystyle {0.0}\)}%
\end{pgfscope}%
\begin{pgfscope}%
\pgfsetbuttcap%
\pgfsetroundjoin%
\definecolor{currentfill}{rgb}{0.000000,0.000000,0.000000}%
\pgfsetfillcolor{currentfill}%
\pgfsetlinewidth{0.803000pt}%
\definecolor{currentstroke}{rgb}{0.000000,0.000000,0.000000}%
\pgfsetstrokecolor{currentstroke}%
\pgfsetdash{}{0pt}%
\pgfsys@defobject{currentmarker}{\pgfqpoint{-0.048611in}{0.000000in}}{\pgfqpoint{-0.000000in}{0.000000in}}{%
\pgfpathmoveto{\pgfqpoint{-0.000000in}{0.000000in}}%
\pgfpathlineto{\pgfqpoint{-0.048611in}{0.000000in}}%
\pgfusepath{stroke,fill}%
}%
\begin{pgfscope}%
\pgfsys@transformshift{0.562500in}{0.986807in}%
\pgfsys@useobject{currentmarker}{}%
\end{pgfscope}%
\end{pgfscope}%
\begin{pgfscope}%
\definecolor{textcolor}{rgb}{0.000000,0.000000,0.000000}%
\pgfsetstrokecolor{textcolor}%
\pgfsetfillcolor{textcolor}%
\pgftext[x=0.287808in, y=0.938979in, left, base]{\color{textcolor}\rmfamily\fontsize{10.000000}{12.000000}\selectfont \(\displaystyle {0.5}\)}%
\end{pgfscope}%
\begin{pgfscope}%
\pgfsetbuttcap%
\pgfsetroundjoin%
\definecolor{currentfill}{rgb}{0.000000,0.000000,0.000000}%
\pgfsetfillcolor{currentfill}%
\pgfsetlinewidth{0.803000pt}%
\definecolor{currentstroke}{rgb}{0.000000,0.000000,0.000000}%
\pgfsetstrokecolor{currentstroke}%
\pgfsetdash{}{0pt}%
\pgfsys@defobject{currentmarker}{\pgfqpoint{-0.048611in}{0.000000in}}{\pgfqpoint{-0.000000in}{0.000000in}}{%
\pgfpathmoveto{\pgfqpoint{-0.000000in}{0.000000in}}%
\pgfpathlineto{\pgfqpoint{-0.048611in}{0.000000in}}%
\pgfusepath{stroke,fill}%
}%
\begin{pgfscope}%
\pgfsys@transformshift{0.562500in}{1.437114in}%
\pgfsys@useobject{currentmarker}{}%
\end{pgfscope}%
\end{pgfscope}%
\begin{pgfscope}%
\definecolor{textcolor}{rgb}{0.000000,0.000000,0.000000}%
\pgfsetstrokecolor{textcolor}%
\pgfsetfillcolor{textcolor}%
\pgftext[x=0.287808in, y=1.389287in, left, base]{\color{textcolor}\rmfamily\fontsize{10.000000}{12.000000}\selectfont \(\displaystyle {1.0}\)}%
\end{pgfscope}%
\begin{pgfscope}%
\pgfsetbuttcap%
\pgfsetroundjoin%
\definecolor{currentfill}{rgb}{0.000000,0.000000,0.000000}%
\pgfsetfillcolor{currentfill}%
\pgfsetlinewidth{0.803000pt}%
\definecolor{currentstroke}{rgb}{0.000000,0.000000,0.000000}%
\pgfsetstrokecolor{currentstroke}%
\pgfsetdash{}{0pt}%
\pgfsys@defobject{currentmarker}{\pgfqpoint{-0.048611in}{0.000000in}}{\pgfqpoint{-0.000000in}{0.000000in}}{%
\pgfpathmoveto{\pgfqpoint{-0.000000in}{0.000000in}}%
\pgfpathlineto{\pgfqpoint{-0.048611in}{0.000000in}}%
\pgfusepath{stroke,fill}%
}%
\begin{pgfscope}%
\pgfsys@transformshift{0.562500in}{1.887422in}%
\pgfsys@useobject{currentmarker}{}%
\end{pgfscope}%
\end{pgfscope}%
\begin{pgfscope}%
\definecolor{textcolor}{rgb}{0.000000,0.000000,0.000000}%
\pgfsetstrokecolor{textcolor}%
\pgfsetfillcolor{textcolor}%
\pgftext[x=0.287808in, y=1.839594in, left, base]{\color{textcolor}\rmfamily\fontsize{10.000000}{12.000000}\selectfont \(\displaystyle {1.5}\)}%
\end{pgfscope}%
\begin{pgfscope}%
\pgfsetbuttcap%
\pgfsetroundjoin%
\definecolor{currentfill}{rgb}{0.000000,0.000000,0.000000}%
\pgfsetfillcolor{currentfill}%
\pgfsetlinewidth{0.803000pt}%
\definecolor{currentstroke}{rgb}{0.000000,0.000000,0.000000}%
\pgfsetstrokecolor{currentstroke}%
\pgfsetdash{}{0pt}%
\pgfsys@defobject{currentmarker}{\pgfqpoint{-0.048611in}{0.000000in}}{\pgfqpoint{-0.000000in}{0.000000in}}{%
\pgfpathmoveto{\pgfqpoint{-0.000000in}{0.000000in}}%
\pgfpathlineto{\pgfqpoint{-0.048611in}{0.000000in}}%
\pgfusepath{stroke,fill}%
}%
\begin{pgfscope}%
\pgfsys@transformshift{0.562500in}{2.337729in}%
\pgfsys@useobject{currentmarker}{}%
\end{pgfscope}%
\end{pgfscope}%
\begin{pgfscope}%
\definecolor{textcolor}{rgb}{0.000000,0.000000,0.000000}%
\pgfsetstrokecolor{textcolor}%
\pgfsetfillcolor{textcolor}%
\pgftext[x=0.287808in, y=2.289901in, left, base]{\color{textcolor}\rmfamily\fontsize{10.000000}{12.000000}\selectfont \(\displaystyle {2.0}\)}%
\end{pgfscope}%
\begin{pgfscope}%
\pgfsetbuttcap%
\pgfsetroundjoin%
\definecolor{currentfill}{rgb}{0.000000,0.000000,0.000000}%
\pgfsetfillcolor{currentfill}%
\pgfsetlinewidth{0.803000pt}%
\definecolor{currentstroke}{rgb}{0.000000,0.000000,0.000000}%
\pgfsetstrokecolor{currentstroke}%
\pgfsetdash{}{0pt}%
\pgfsys@defobject{currentmarker}{\pgfqpoint{-0.048611in}{0.000000in}}{\pgfqpoint{-0.000000in}{0.000000in}}{%
\pgfpathmoveto{\pgfqpoint{-0.000000in}{0.000000in}}%
\pgfpathlineto{\pgfqpoint{-0.048611in}{0.000000in}}%
\pgfusepath{stroke,fill}%
}%
\begin{pgfscope}%
\pgfsys@transformshift{0.562500in}{2.788036in}%
\pgfsys@useobject{currentmarker}{}%
\end{pgfscope}%
\end{pgfscope}%
\begin{pgfscope}%
\definecolor{textcolor}{rgb}{0.000000,0.000000,0.000000}%
\pgfsetstrokecolor{textcolor}%
\pgfsetfillcolor{textcolor}%
\pgftext[x=0.287808in, y=2.740208in, left, base]{\color{textcolor}\rmfamily\fontsize{10.000000}{12.000000}\selectfont \(\displaystyle {2.5}\)}%
\end{pgfscope}%
\begin{pgfscope}%
\pgfsetbuttcap%
\pgfsetroundjoin%
\definecolor{currentfill}{rgb}{0.000000,0.000000,0.000000}%
\pgfsetfillcolor{currentfill}%
\pgfsetlinewidth{0.803000pt}%
\definecolor{currentstroke}{rgb}{0.000000,0.000000,0.000000}%
\pgfsetstrokecolor{currentstroke}%
\pgfsetdash{}{0pt}%
\pgfsys@defobject{currentmarker}{\pgfqpoint{-0.048611in}{0.000000in}}{\pgfqpoint{-0.000000in}{0.000000in}}{%
\pgfpathmoveto{\pgfqpoint{-0.000000in}{0.000000in}}%
\pgfpathlineto{\pgfqpoint{-0.048611in}{0.000000in}}%
\pgfusepath{stroke,fill}%
}%
\begin{pgfscope}%
\pgfsys@transformshift{0.562500in}{3.238343in}%
\pgfsys@useobject{currentmarker}{}%
\end{pgfscope}%
\end{pgfscope}%
\begin{pgfscope}%
\definecolor{textcolor}{rgb}{0.000000,0.000000,0.000000}%
\pgfsetstrokecolor{textcolor}%
\pgfsetfillcolor{textcolor}%
\pgftext[x=0.287808in, y=3.190515in, left, base]{\color{textcolor}\rmfamily\fontsize{10.000000}{12.000000}\selectfont \(\displaystyle {3.0}\)}%
\end{pgfscope}%
\begin{pgfscope}%
\definecolor{textcolor}{rgb}{0.000000,0.000000,0.000000}%
\pgfsetstrokecolor{textcolor}%
\pgfsetfillcolor{textcolor}%
\pgftext[x=0.232253in,y=1.831500in,,bottom,rotate=90.000000]{\color{textcolor}\rmfamily\fontsize{10.000000}{12.000000}\selectfont MSD  [\(\displaystyle \si{\angstrom ^2}\)]}%
\end{pgfscope}%
\begin{pgfscope}%
\pgfpathrectangle{\pgfqpoint{0.562500in}{0.407000in}}{\pgfqpoint{3.487500in}{2.849000in}}%
\pgfusepath{clip}%
\pgfsetbuttcap%
\pgfsetroundjoin%
\definecolor{currentfill}{rgb}{0.121569,0.466667,0.705882}%
\pgfsetfillcolor{currentfill}%
\pgfsetlinewidth{1.003750pt}%
\definecolor{currentstroke}{rgb}{0.121569,0.466667,0.705882}%
\pgfsetstrokecolor{currentstroke}%
\pgfsetdash{}{0pt}%
\pgfsys@defobject{currentmarker}{\pgfqpoint{-0.020833in}{-0.020833in}}{\pgfqpoint{0.020833in}{0.020833in}}{%
\pgfpathmoveto{\pgfqpoint{0.000000in}{-0.020833in}}%
\pgfpathcurveto{\pgfqpoint{0.005525in}{-0.020833in}}{\pgfqpoint{0.010825in}{-0.018638in}}{\pgfqpoint{0.014731in}{-0.014731in}}%
\pgfpathcurveto{\pgfqpoint{0.018638in}{-0.010825in}}{\pgfqpoint{0.020833in}{-0.005525in}}{\pgfqpoint{0.020833in}{0.000000in}}%
\pgfpathcurveto{\pgfqpoint{0.020833in}{0.005525in}}{\pgfqpoint{0.018638in}{0.010825in}}{\pgfqpoint{0.014731in}{0.014731in}}%
\pgfpathcurveto{\pgfqpoint{0.010825in}{0.018638in}}{\pgfqpoint{0.005525in}{0.020833in}}{\pgfqpoint{0.000000in}{0.020833in}}%
\pgfpathcurveto{\pgfqpoint{-0.005525in}{0.020833in}}{\pgfqpoint{-0.010825in}{0.018638in}}{\pgfqpoint{-0.014731in}{0.014731in}}%
\pgfpathcurveto{\pgfqpoint{-0.018638in}{0.010825in}}{\pgfqpoint{-0.020833in}{0.005525in}}{\pgfqpoint{-0.020833in}{0.000000in}}%
\pgfpathcurveto{\pgfqpoint{-0.020833in}{-0.005525in}}{\pgfqpoint{-0.018638in}{-0.010825in}}{\pgfqpoint{-0.014731in}{-0.014731in}}%
\pgfpathcurveto{\pgfqpoint{-0.010825in}{-0.018638in}}{\pgfqpoint{-0.005525in}{-0.020833in}}{\pgfqpoint{0.000000in}{-0.020833in}}%
\pgfpathlineto{\pgfqpoint{0.000000in}{-0.020833in}}%
\pgfpathclose%
\pgfusepath{stroke,fill}%
}%
\begin{pgfscope}%
\pgfsys@transformshift{0.721023in}{0.536500in}%
\pgfsys@useobject{currentmarker}{}%
\end{pgfscope}%
\begin{pgfscope}%
\pgfsys@transformshift{0.728084in}{0.536594in}%
\pgfsys@useobject{currentmarker}{}%
\end{pgfscope}%
\begin{pgfscope}%
\pgfsys@transformshift{0.735145in}{0.536876in}%
\pgfsys@useobject{currentmarker}{}%
\end{pgfscope}%
\begin{pgfscope}%
\pgfsys@transformshift{0.742206in}{0.537346in}%
\pgfsys@useobject{currentmarker}{}%
\end{pgfscope}%
\begin{pgfscope}%
\pgfsys@transformshift{0.749267in}{0.538002in}%
\pgfsys@useobject{currentmarker}{}%
\end{pgfscope}%
\begin{pgfscope}%
\pgfsys@transformshift{0.756328in}{0.538845in}%
\pgfsys@useobject{currentmarker}{}%
\end{pgfscope}%
\begin{pgfscope}%
\pgfsys@transformshift{0.763390in}{0.539874in}%
\pgfsys@useobject{currentmarker}{}%
\end{pgfscope}%
\begin{pgfscope}%
\pgfsys@transformshift{0.770451in}{0.541088in}%
\pgfsys@useobject{currentmarker}{}%
\end{pgfscope}%
\begin{pgfscope}%
\pgfsys@transformshift{0.777512in}{0.542485in}%
\pgfsys@useobject{currentmarker}{}%
\end{pgfscope}%
\begin{pgfscope}%
\pgfsys@transformshift{0.784573in}{0.544064in}%
\pgfsys@useobject{currentmarker}{}%
\end{pgfscope}%
\begin{pgfscope}%
\pgfsys@transformshift{0.791634in}{0.545823in}%
\pgfsys@useobject{currentmarker}{}%
\end{pgfscope}%
\begin{pgfscope}%
\pgfsys@transformshift{0.798695in}{0.547761in}%
\pgfsys@useobject{currentmarker}{}%
\end{pgfscope}%
\begin{pgfscope}%
\pgfsys@transformshift{0.805756in}{0.549877in}%
\pgfsys@useobject{currentmarker}{}%
\end{pgfscope}%
\begin{pgfscope}%
\pgfsys@transformshift{0.812818in}{0.552167in}%
\pgfsys@useobject{currentmarker}{}%
\end{pgfscope}%
\begin{pgfscope}%
\pgfsys@transformshift{0.819879in}{0.554630in}%
\pgfsys@useobject{currentmarker}{}%
\end{pgfscope}%
\begin{pgfscope}%
\pgfsys@transformshift{0.826940in}{0.557264in}%
\pgfsys@useobject{currentmarker}{}%
\end{pgfscope}%
\begin{pgfscope}%
\pgfsys@transformshift{0.834001in}{0.560066in}%
\pgfsys@useobject{currentmarker}{}%
\end{pgfscope}%
\begin{pgfscope}%
\pgfsys@transformshift{0.841062in}{0.563034in}%
\pgfsys@useobject{currentmarker}{}%
\end{pgfscope}%
\begin{pgfscope}%
\pgfsys@transformshift{0.848123in}{0.566165in}%
\pgfsys@useobject{currentmarker}{}%
\end{pgfscope}%
\begin{pgfscope}%
\pgfsys@transformshift{0.855185in}{0.569456in}%
\pgfsys@useobject{currentmarker}{}%
\end{pgfscope}%
\begin{pgfscope}%
\pgfsys@transformshift{0.862246in}{0.572905in}%
\pgfsys@useobject{currentmarker}{}%
\end{pgfscope}%
\begin{pgfscope}%
\pgfsys@transformshift{0.869307in}{0.576508in}%
\pgfsys@useobject{currentmarker}{}%
\end{pgfscope}%
\begin{pgfscope}%
\pgfsys@transformshift{0.876368in}{0.580263in}%
\pgfsys@useobject{currentmarker}{}%
\end{pgfscope}%
\begin{pgfscope}%
\pgfsys@transformshift{0.883429in}{0.584166in}%
\pgfsys@useobject{currentmarker}{}%
\end{pgfscope}%
\begin{pgfscope}%
\pgfsys@transformshift{0.890490in}{0.588215in}%
\pgfsys@useobject{currentmarker}{}%
\end{pgfscope}%
\begin{pgfscope}%
\pgfsys@transformshift{0.897551in}{0.592405in}%
\pgfsys@useobject{currentmarker}{}%
\end{pgfscope}%
\begin{pgfscope}%
\pgfsys@transformshift{0.904613in}{0.596734in}%
\pgfsys@useobject{currentmarker}{}%
\end{pgfscope}%
\begin{pgfscope}%
\pgfsys@transformshift{0.911674in}{0.601199in}%
\pgfsys@useobject{currentmarker}{}%
\end{pgfscope}%
\begin{pgfscope}%
\pgfsys@transformshift{0.918735in}{0.605795in}%
\pgfsys@useobject{currentmarker}{}%
\end{pgfscope}%
\begin{pgfscope}%
\pgfsys@transformshift{0.925796in}{0.610520in}%
\pgfsys@useobject{currentmarker}{}%
\end{pgfscope}%
\begin{pgfscope}%
\pgfsys@transformshift{0.932857in}{0.615369in}%
\pgfsys@useobject{currentmarker}{}%
\end{pgfscope}%
\begin{pgfscope}%
\pgfsys@transformshift{0.939918in}{0.620340in}%
\pgfsys@useobject{currentmarker}{}%
\end{pgfscope}%
\begin{pgfscope}%
\pgfsys@transformshift{0.946979in}{0.625429in}%
\pgfsys@useobject{currentmarker}{}%
\end{pgfscope}%
\begin{pgfscope}%
\pgfsys@transformshift{0.954041in}{0.630631in}%
\pgfsys@useobject{currentmarker}{}%
\end{pgfscope}%
\begin{pgfscope}%
\pgfsys@transformshift{0.961102in}{0.635944in}%
\pgfsys@useobject{currentmarker}{}%
\end{pgfscope}%
\begin{pgfscope}%
\pgfsys@transformshift{0.968163in}{0.641365in}%
\pgfsys@useobject{currentmarker}{}%
\end{pgfscope}%
\begin{pgfscope}%
\pgfsys@transformshift{0.975224in}{0.646888in}%
\pgfsys@useobject{currentmarker}{}%
\end{pgfscope}%
\begin{pgfscope}%
\pgfsys@transformshift{0.982285in}{0.652511in}%
\pgfsys@useobject{currentmarker}{}%
\end{pgfscope}%
\begin{pgfscope}%
\pgfsys@transformshift{0.989346in}{0.658231in}%
\pgfsys@useobject{currentmarker}{}%
\end{pgfscope}%
\begin{pgfscope}%
\pgfsys@transformshift{0.996407in}{0.664042in}%
\pgfsys@useobject{currentmarker}{}%
\end{pgfscope}%
\begin{pgfscope}%
\pgfsys@transformshift{1.003469in}{0.669943in}%
\pgfsys@useobject{currentmarker}{}%
\end{pgfscope}%
\begin{pgfscope}%
\pgfsys@transformshift{1.010530in}{0.675930in}%
\pgfsys@useobject{currentmarker}{}%
\end{pgfscope}%
\begin{pgfscope}%
\pgfsys@transformshift{1.017591in}{0.681998in}%
\pgfsys@useobject{currentmarker}{}%
\end{pgfscope}%
\begin{pgfscope}%
\pgfsys@transformshift{1.024652in}{0.688145in}%
\pgfsys@useobject{currentmarker}{}%
\end{pgfscope}%
\begin{pgfscope}%
\pgfsys@transformshift{1.031713in}{0.694367in}%
\pgfsys@useobject{currentmarker}{}%
\end{pgfscope}%
\begin{pgfscope}%
\pgfsys@transformshift{1.038774in}{0.700661in}%
\pgfsys@useobject{currentmarker}{}%
\end{pgfscope}%
\begin{pgfscope}%
\pgfsys@transformshift{1.045835in}{0.707024in}%
\pgfsys@useobject{currentmarker}{}%
\end{pgfscope}%
\begin{pgfscope}%
\pgfsys@transformshift{1.052897in}{0.713451in}%
\pgfsys@useobject{currentmarker}{}%
\end{pgfscope}%
\begin{pgfscope}%
\pgfsys@transformshift{1.059958in}{0.719941in}%
\pgfsys@useobject{currentmarker}{}%
\end{pgfscope}%
\begin{pgfscope}%
\pgfsys@transformshift{1.067019in}{0.726489in}%
\pgfsys@useobject{currentmarker}{}%
\end{pgfscope}%
\begin{pgfscope}%
\pgfsys@transformshift{1.074080in}{0.733093in}%
\pgfsys@useobject{currentmarker}{}%
\end{pgfscope}%
\begin{pgfscope}%
\pgfsys@transformshift{1.081141in}{0.739749in}%
\pgfsys@useobject{currentmarker}{}%
\end{pgfscope}%
\begin{pgfscope}%
\pgfsys@transformshift{1.088202in}{0.746455in}%
\pgfsys@useobject{currentmarker}{}%
\end{pgfscope}%
\begin{pgfscope}%
\pgfsys@transformshift{1.095263in}{0.753208in}%
\pgfsys@useobject{currentmarker}{}%
\end{pgfscope}%
\begin{pgfscope}%
\pgfsys@transformshift{1.102325in}{0.760004in}%
\pgfsys@useobject{currentmarker}{}%
\end{pgfscope}%
\begin{pgfscope}%
\pgfsys@transformshift{1.109386in}{0.766842in}%
\pgfsys@useobject{currentmarker}{}%
\end{pgfscope}%
\begin{pgfscope}%
\pgfsys@transformshift{1.116447in}{0.773717in}%
\pgfsys@useobject{currentmarker}{}%
\end{pgfscope}%
\begin{pgfscope}%
\pgfsys@transformshift{1.123508in}{0.780628in}%
\pgfsys@useobject{currentmarker}{}%
\end{pgfscope}%
\begin{pgfscope}%
\pgfsys@transformshift{1.130569in}{0.787573in}%
\pgfsys@useobject{currentmarker}{}%
\end{pgfscope}%
\begin{pgfscope}%
\pgfsys@transformshift{1.137630in}{0.794547in}%
\pgfsys@useobject{currentmarker}{}%
\end{pgfscope}%
\begin{pgfscope}%
\pgfsys@transformshift{1.144691in}{0.801550in}%
\pgfsys@useobject{currentmarker}{}%
\end{pgfscope}%
\begin{pgfscope}%
\pgfsys@transformshift{1.151753in}{0.808578in}%
\pgfsys@useobject{currentmarker}{}%
\end{pgfscope}%
\begin{pgfscope}%
\pgfsys@transformshift{1.158814in}{0.815630in}%
\pgfsys@useobject{currentmarker}{}%
\end{pgfscope}%
\begin{pgfscope}%
\pgfsys@transformshift{1.165875in}{0.822703in}%
\pgfsys@useobject{currentmarker}{}%
\end{pgfscope}%
\begin{pgfscope}%
\pgfsys@transformshift{1.172936in}{0.829795in}%
\pgfsys@useobject{currentmarker}{}%
\end{pgfscope}%
\begin{pgfscope}%
\pgfsys@transformshift{1.179997in}{0.836904in}%
\pgfsys@useobject{currentmarker}{}%
\end{pgfscope}%
\begin{pgfscope}%
\pgfsys@transformshift{1.187058in}{0.844029in}%
\pgfsys@useobject{currentmarker}{}%
\end{pgfscope}%
\begin{pgfscope}%
\pgfsys@transformshift{1.194120in}{0.851167in}%
\pgfsys@useobject{currentmarker}{}%
\end{pgfscope}%
\begin{pgfscope}%
\pgfsys@transformshift{1.201181in}{0.858316in}%
\pgfsys@useobject{currentmarker}{}%
\end{pgfscope}%
\begin{pgfscope}%
\pgfsys@transformshift{1.208242in}{0.865475in}%
\pgfsys@useobject{currentmarker}{}%
\end{pgfscope}%
\begin{pgfscope}%
\pgfsys@transformshift{1.215303in}{0.872643in}%
\pgfsys@useobject{currentmarker}{}%
\end{pgfscope}%
\begin{pgfscope}%
\pgfsys@transformshift{1.222364in}{0.879817in}%
\pgfsys@useobject{currentmarker}{}%
\end{pgfscope}%
\begin{pgfscope}%
\pgfsys@transformshift{1.229425in}{0.886996in}%
\pgfsys@useobject{currentmarker}{}%
\end{pgfscope}%
\begin{pgfscope}%
\pgfsys@transformshift{1.236486in}{0.894179in}%
\pgfsys@useobject{currentmarker}{}%
\end{pgfscope}%
\begin{pgfscope}%
\pgfsys@transformshift{1.243548in}{0.901364in}%
\pgfsys@useobject{currentmarker}{}%
\end{pgfscope}%
\begin{pgfscope}%
\pgfsys@transformshift{1.250609in}{0.908550in}%
\pgfsys@useobject{currentmarker}{}%
\end{pgfscope}%
\begin{pgfscope}%
\pgfsys@transformshift{1.257670in}{0.915735in}%
\pgfsys@useobject{currentmarker}{}%
\end{pgfscope}%
\begin{pgfscope}%
\pgfsys@transformshift{1.264731in}{0.922919in}%
\pgfsys@useobject{currentmarker}{}%
\end{pgfscope}%
\begin{pgfscope}%
\pgfsys@transformshift{1.271792in}{0.930100in}%
\pgfsys@useobject{currentmarker}{}%
\end{pgfscope}%
\begin{pgfscope}%
\pgfsys@transformshift{1.278853in}{0.937278in}%
\pgfsys@useobject{currentmarker}{}%
\end{pgfscope}%
\begin{pgfscope}%
\pgfsys@transformshift{1.285914in}{0.944451in}%
\pgfsys@useobject{currentmarker}{}%
\end{pgfscope}%
\begin{pgfscope}%
\pgfsys@transformshift{1.292976in}{0.951618in}%
\pgfsys@useobject{currentmarker}{}%
\end{pgfscope}%
\begin{pgfscope}%
\pgfsys@transformshift{1.300037in}{0.958778in}%
\pgfsys@useobject{currentmarker}{}%
\end{pgfscope}%
\begin{pgfscope}%
\pgfsys@transformshift{1.307098in}{0.965931in}%
\pgfsys@useobject{currentmarker}{}%
\end{pgfscope}%
\begin{pgfscope}%
\pgfsys@transformshift{1.314159in}{0.973076in}%
\pgfsys@useobject{currentmarker}{}%
\end{pgfscope}%
\begin{pgfscope}%
\pgfsys@transformshift{1.321220in}{0.980212in}%
\pgfsys@useobject{currentmarker}{}%
\end{pgfscope}%
\begin{pgfscope}%
\pgfsys@transformshift{1.328281in}{0.987339in}%
\pgfsys@useobject{currentmarker}{}%
\end{pgfscope}%
\begin{pgfscope}%
\pgfsys@transformshift{1.335342in}{0.994455in}%
\pgfsys@useobject{currentmarker}{}%
\end{pgfscope}%
\begin{pgfscope}%
\pgfsys@transformshift{1.342404in}{1.001560in}%
\pgfsys@useobject{currentmarker}{}%
\end{pgfscope}%
\begin{pgfscope}%
\pgfsys@transformshift{1.349465in}{1.008655in}%
\pgfsys@useobject{currentmarker}{}%
\end{pgfscope}%
\begin{pgfscope}%
\pgfsys@transformshift{1.356526in}{1.015737in}%
\pgfsys@useobject{currentmarker}{}%
\end{pgfscope}%
\begin{pgfscope}%
\pgfsys@transformshift{1.363587in}{1.022808in}%
\pgfsys@useobject{currentmarker}{}%
\end{pgfscope}%
\begin{pgfscope}%
\pgfsys@transformshift{1.370648in}{1.029866in}%
\pgfsys@useobject{currentmarker}{}%
\end{pgfscope}%
\begin{pgfscope}%
\pgfsys@transformshift{1.377709in}{1.036910in}%
\pgfsys@useobject{currentmarker}{}%
\end{pgfscope}%
\begin{pgfscope}%
\pgfsys@transformshift{1.384770in}{1.043942in}%
\pgfsys@useobject{currentmarker}{}%
\end{pgfscope}%
\begin{pgfscope}%
\pgfsys@transformshift{1.391832in}{1.050960in}%
\pgfsys@useobject{currentmarker}{}%
\end{pgfscope}%
\begin{pgfscope}%
\pgfsys@transformshift{1.398893in}{1.057964in}%
\pgfsys@useobject{currentmarker}{}%
\end{pgfscope}%
\begin{pgfscope}%
\pgfsys@transformshift{1.405954in}{1.064954in}%
\pgfsys@useobject{currentmarker}{}%
\end{pgfscope}%
\begin{pgfscope}%
\pgfsys@transformshift{1.413015in}{1.071930in}%
\pgfsys@useobject{currentmarker}{}%
\end{pgfscope}%
\begin{pgfscope}%
\pgfsys@transformshift{1.420076in}{1.078891in}%
\pgfsys@useobject{currentmarker}{}%
\end{pgfscope}%
\begin{pgfscope}%
\pgfsys@transformshift{1.427137in}{1.085837in}%
\pgfsys@useobject{currentmarker}{}%
\end{pgfscope}%
\begin{pgfscope}%
\pgfsys@transformshift{1.434198in}{1.092768in}%
\pgfsys@useobject{currentmarker}{}%
\end{pgfscope}%
\begin{pgfscope}%
\pgfsys@transformshift{1.441260in}{1.099685in}%
\pgfsys@useobject{currentmarker}{}%
\end{pgfscope}%
\begin{pgfscope}%
\pgfsys@transformshift{1.448321in}{1.106586in}%
\pgfsys@useobject{currentmarker}{}%
\end{pgfscope}%
\begin{pgfscope}%
\pgfsys@transformshift{1.455382in}{1.113471in}%
\pgfsys@useobject{currentmarker}{}%
\end{pgfscope}%
\begin{pgfscope}%
\pgfsys@transformshift{1.462443in}{1.120341in}%
\pgfsys@useobject{currentmarker}{}%
\end{pgfscope}%
\begin{pgfscope}%
\pgfsys@transformshift{1.469504in}{1.127196in}%
\pgfsys@useobject{currentmarker}{}%
\end{pgfscope}%
\begin{pgfscope}%
\pgfsys@transformshift{1.476565in}{1.134035in}%
\pgfsys@useobject{currentmarker}{}%
\end{pgfscope}%
\begin{pgfscope}%
\pgfsys@transformshift{1.483626in}{1.140859in}%
\pgfsys@useobject{currentmarker}{}%
\end{pgfscope}%
\begin{pgfscope}%
\pgfsys@transformshift{1.490688in}{1.147667in}%
\pgfsys@useobject{currentmarker}{}%
\end{pgfscope}%
\begin{pgfscope}%
\pgfsys@transformshift{1.497749in}{1.154459in}%
\pgfsys@useobject{currentmarker}{}%
\end{pgfscope}%
\begin{pgfscope}%
\pgfsys@transformshift{1.504810in}{1.161237in}%
\pgfsys@useobject{currentmarker}{}%
\end{pgfscope}%
\begin{pgfscope}%
\pgfsys@transformshift{1.511871in}{1.167998in}%
\pgfsys@useobject{currentmarker}{}%
\end{pgfscope}%
\begin{pgfscope}%
\pgfsys@transformshift{1.518932in}{1.174745in}%
\pgfsys@useobject{currentmarker}{}%
\end{pgfscope}%
\begin{pgfscope}%
\pgfsys@transformshift{1.525993in}{1.181476in}%
\pgfsys@useobject{currentmarker}{}%
\end{pgfscope}%
\begin{pgfscope}%
\pgfsys@transformshift{1.533055in}{1.188193in}%
\pgfsys@useobject{currentmarker}{}%
\end{pgfscope}%
\begin{pgfscope}%
\pgfsys@transformshift{1.540116in}{1.194894in}%
\pgfsys@useobject{currentmarker}{}%
\end{pgfscope}%
\begin{pgfscope}%
\pgfsys@transformshift{1.547177in}{1.201581in}%
\pgfsys@useobject{currentmarker}{}%
\end{pgfscope}%
\begin{pgfscope}%
\pgfsys@transformshift{1.554238in}{1.208253in}%
\pgfsys@useobject{currentmarker}{}%
\end{pgfscope}%
\begin{pgfscope}%
\pgfsys@transformshift{1.561299in}{1.214911in}%
\pgfsys@useobject{currentmarker}{}%
\end{pgfscope}%
\begin{pgfscope}%
\pgfsys@transformshift{1.568360in}{1.221555in}%
\pgfsys@useobject{currentmarker}{}%
\end{pgfscope}%
\begin{pgfscope}%
\pgfsys@transformshift{1.575421in}{1.228185in}%
\pgfsys@useobject{currentmarker}{}%
\end{pgfscope}%
\begin{pgfscope}%
\pgfsys@transformshift{1.582483in}{1.234802in}%
\pgfsys@useobject{currentmarker}{}%
\end{pgfscope}%
\begin{pgfscope}%
\pgfsys@transformshift{1.589544in}{1.241405in}%
\pgfsys@useobject{currentmarker}{}%
\end{pgfscope}%
\begin{pgfscope}%
\pgfsys@transformshift{1.596605in}{1.247995in}%
\pgfsys@useobject{currentmarker}{}%
\end{pgfscope}%
\begin{pgfscope}%
\pgfsys@transformshift{1.603666in}{1.254572in}%
\pgfsys@useobject{currentmarker}{}%
\end{pgfscope}%
\begin{pgfscope}%
\pgfsys@transformshift{1.610727in}{1.261137in}%
\pgfsys@useobject{currentmarker}{}%
\end{pgfscope}%
\begin{pgfscope}%
\pgfsys@transformshift{1.617788in}{1.267689in}%
\pgfsys@useobject{currentmarker}{}%
\end{pgfscope}%
\begin{pgfscope}%
\pgfsys@transformshift{1.624849in}{1.274228in}%
\pgfsys@useobject{currentmarker}{}%
\end{pgfscope}%
\begin{pgfscope}%
\pgfsys@transformshift{1.631911in}{1.280756in}%
\pgfsys@useobject{currentmarker}{}%
\end{pgfscope}%
\begin{pgfscope}%
\pgfsys@transformshift{1.638972in}{1.287272in}%
\pgfsys@useobject{currentmarker}{}%
\end{pgfscope}%
\begin{pgfscope}%
\pgfsys@transformshift{1.646033in}{1.293776in}%
\pgfsys@useobject{currentmarker}{}%
\end{pgfscope}%
\begin{pgfscope}%
\pgfsys@transformshift{1.653094in}{1.300268in}%
\pgfsys@useobject{currentmarker}{}%
\end{pgfscope}%
\begin{pgfscope}%
\pgfsys@transformshift{1.660155in}{1.306750in}%
\pgfsys@useobject{currentmarker}{}%
\end{pgfscope}%
\begin{pgfscope}%
\pgfsys@transformshift{1.667216in}{1.313220in}%
\pgfsys@useobject{currentmarker}{}%
\end{pgfscope}%
\begin{pgfscope}%
\pgfsys@transformshift{1.674277in}{1.319679in}%
\pgfsys@useobject{currentmarker}{}%
\end{pgfscope}%
\begin{pgfscope}%
\pgfsys@transformshift{1.681339in}{1.326128in}%
\pgfsys@useobject{currentmarker}{}%
\end{pgfscope}%
\begin{pgfscope}%
\pgfsys@transformshift{1.688400in}{1.332565in}%
\pgfsys@useobject{currentmarker}{}%
\end{pgfscope}%
\begin{pgfscope}%
\pgfsys@transformshift{1.695461in}{1.338993in}%
\pgfsys@useobject{currentmarker}{}%
\end{pgfscope}%
\begin{pgfscope}%
\pgfsys@transformshift{1.702522in}{1.345410in}%
\pgfsys@useobject{currentmarker}{}%
\end{pgfscope}%
\begin{pgfscope}%
\pgfsys@transformshift{1.709583in}{1.351816in}%
\pgfsys@useobject{currentmarker}{}%
\end{pgfscope}%
\begin{pgfscope}%
\pgfsys@transformshift{1.716644in}{1.358213in}%
\pgfsys@useobject{currentmarker}{}%
\end{pgfscope}%
\begin{pgfscope}%
\pgfsys@transformshift{1.723705in}{1.364599in}%
\pgfsys@useobject{currentmarker}{}%
\end{pgfscope}%
\begin{pgfscope}%
\pgfsys@transformshift{1.730767in}{1.370976in}%
\pgfsys@useobject{currentmarker}{}%
\end{pgfscope}%
\begin{pgfscope}%
\pgfsys@transformshift{1.737828in}{1.377343in}%
\pgfsys@useobject{currentmarker}{}%
\end{pgfscope}%
\begin{pgfscope}%
\pgfsys@transformshift{1.744889in}{1.383701in}%
\pgfsys@useobject{currentmarker}{}%
\end{pgfscope}%
\begin{pgfscope}%
\pgfsys@transformshift{1.751950in}{1.390049in}%
\pgfsys@useobject{currentmarker}{}%
\end{pgfscope}%
\begin{pgfscope}%
\pgfsys@transformshift{1.759011in}{1.396388in}%
\pgfsys@useobject{currentmarker}{}%
\end{pgfscope}%
\begin{pgfscope}%
\pgfsys@transformshift{1.766072in}{1.402718in}%
\pgfsys@useobject{currentmarker}{}%
\end{pgfscope}%
\begin{pgfscope}%
\pgfsys@transformshift{1.773133in}{1.409039in}%
\pgfsys@useobject{currentmarker}{}%
\end{pgfscope}%
\begin{pgfscope}%
\pgfsys@transformshift{1.780195in}{1.415350in}%
\pgfsys@useobject{currentmarker}{}%
\end{pgfscope}%
\begin{pgfscope}%
\pgfsys@transformshift{1.787256in}{1.421653in}%
\pgfsys@useobject{currentmarker}{}%
\end{pgfscope}%
\begin{pgfscope}%
\pgfsys@transformshift{1.794317in}{1.427948in}%
\pgfsys@useobject{currentmarker}{}%
\end{pgfscope}%
\begin{pgfscope}%
\pgfsys@transformshift{1.801378in}{1.434233in}%
\pgfsys@useobject{currentmarker}{}%
\end{pgfscope}%
\begin{pgfscope}%
\pgfsys@transformshift{1.808439in}{1.440510in}%
\pgfsys@useobject{currentmarker}{}%
\end{pgfscope}%
\begin{pgfscope}%
\pgfsys@transformshift{1.815500in}{1.446779in}%
\pgfsys@useobject{currentmarker}{}%
\end{pgfscope}%
\begin{pgfscope}%
\pgfsys@transformshift{1.822562in}{1.453039in}%
\pgfsys@useobject{currentmarker}{}%
\end{pgfscope}%
\begin{pgfscope}%
\pgfsys@transformshift{1.829623in}{1.459291in}%
\pgfsys@useobject{currentmarker}{}%
\end{pgfscope}%
\begin{pgfscope}%
\pgfsys@transformshift{1.836684in}{1.465534in}%
\pgfsys@useobject{currentmarker}{}%
\end{pgfscope}%
\begin{pgfscope}%
\pgfsys@transformshift{1.843745in}{1.471770in}%
\pgfsys@useobject{currentmarker}{}%
\end{pgfscope}%
\begin{pgfscope}%
\pgfsys@transformshift{1.850806in}{1.477998in}%
\pgfsys@useobject{currentmarker}{}%
\end{pgfscope}%
\begin{pgfscope}%
\pgfsys@transformshift{1.857867in}{1.484218in}%
\pgfsys@useobject{currentmarker}{}%
\end{pgfscope}%
\begin{pgfscope}%
\pgfsys@transformshift{1.864928in}{1.490430in}%
\pgfsys@useobject{currentmarker}{}%
\end{pgfscope}%
\begin{pgfscope}%
\pgfsys@transformshift{1.871990in}{1.496635in}%
\pgfsys@useobject{currentmarker}{}%
\end{pgfscope}%
\begin{pgfscope}%
\pgfsys@transformshift{1.879051in}{1.502832in}%
\pgfsys@useobject{currentmarker}{}%
\end{pgfscope}%
\begin{pgfscope}%
\pgfsys@transformshift{1.886112in}{1.509022in}%
\pgfsys@useobject{currentmarker}{}%
\end{pgfscope}%
\begin{pgfscope}%
\pgfsys@transformshift{1.893173in}{1.515205in}%
\pgfsys@useobject{currentmarker}{}%
\end{pgfscope}%
\begin{pgfscope}%
\pgfsys@transformshift{1.900234in}{1.521380in}%
\pgfsys@useobject{currentmarker}{}%
\end{pgfscope}%
\begin{pgfscope}%
\pgfsys@transformshift{1.907295in}{1.527549in}%
\pgfsys@useobject{currentmarker}{}%
\end{pgfscope}%
\begin{pgfscope}%
\pgfsys@transformshift{1.914356in}{1.533710in}%
\pgfsys@useobject{currentmarker}{}%
\end{pgfscope}%
\begin{pgfscope}%
\pgfsys@transformshift{1.921418in}{1.539864in}%
\pgfsys@useobject{currentmarker}{}%
\end{pgfscope}%
\begin{pgfscope}%
\pgfsys@transformshift{1.928479in}{1.546011in}%
\pgfsys@useobject{currentmarker}{}%
\end{pgfscope}%
\begin{pgfscope}%
\pgfsys@transformshift{1.935540in}{1.552152in}%
\pgfsys@useobject{currentmarker}{}%
\end{pgfscope}%
\begin{pgfscope}%
\pgfsys@transformshift{1.942601in}{1.558285in}%
\pgfsys@useobject{currentmarker}{}%
\end{pgfscope}%
\begin{pgfscope}%
\pgfsys@transformshift{1.949662in}{1.564412in}%
\pgfsys@useobject{currentmarker}{}%
\end{pgfscope}%
\begin{pgfscope}%
\pgfsys@transformshift{1.956723in}{1.570532in}%
\pgfsys@useobject{currentmarker}{}%
\end{pgfscope}%
\begin{pgfscope}%
\pgfsys@transformshift{1.963784in}{1.576646in}%
\pgfsys@useobject{currentmarker}{}%
\end{pgfscope}%
\begin{pgfscope}%
\pgfsys@transformshift{1.970846in}{1.582753in}%
\pgfsys@useobject{currentmarker}{}%
\end{pgfscope}%
\begin{pgfscope}%
\pgfsys@transformshift{1.977907in}{1.588854in}%
\pgfsys@useobject{currentmarker}{}%
\end{pgfscope}%
\begin{pgfscope}%
\pgfsys@transformshift{1.984968in}{1.594948in}%
\pgfsys@useobject{currentmarker}{}%
\end{pgfscope}%
\begin{pgfscope}%
\pgfsys@transformshift{1.992029in}{1.601036in}%
\pgfsys@useobject{currentmarker}{}%
\end{pgfscope}%
\begin{pgfscope}%
\pgfsys@transformshift{1.999090in}{1.607118in}%
\pgfsys@useobject{currentmarker}{}%
\end{pgfscope}%
\begin{pgfscope}%
\pgfsys@transformshift{2.006151in}{1.613194in}%
\pgfsys@useobject{currentmarker}{}%
\end{pgfscope}%
\begin{pgfscope}%
\pgfsys@transformshift{2.013212in}{1.619264in}%
\pgfsys@useobject{currentmarker}{}%
\end{pgfscope}%
\begin{pgfscope}%
\pgfsys@transformshift{2.020274in}{1.625328in}%
\pgfsys@useobject{currentmarker}{}%
\end{pgfscope}%
\begin{pgfscope}%
\pgfsys@transformshift{2.027335in}{1.631387in}%
\pgfsys@useobject{currentmarker}{}%
\end{pgfscope}%
\begin{pgfscope}%
\pgfsys@transformshift{2.034396in}{1.637440in}%
\pgfsys@useobject{currentmarker}{}%
\end{pgfscope}%
\begin{pgfscope}%
\pgfsys@transformshift{2.041457in}{1.643487in}%
\pgfsys@useobject{currentmarker}{}%
\end{pgfscope}%
\begin{pgfscope}%
\pgfsys@transformshift{2.048518in}{1.649529in}%
\pgfsys@useobject{currentmarker}{}%
\end{pgfscope}%
\begin{pgfscope}%
\pgfsys@transformshift{2.055579in}{1.655566in}%
\pgfsys@useobject{currentmarker}{}%
\end{pgfscope}%
\begin{pgfscope}%
\pgfsys@transformshift{2.062640in}{1.661598in}%
\pgfsys@useobject{currentmarker}{}%
\end{pgfscope}%
\begin{pgfscope}%
\pgfsys@transformshift{2.069702in}{1.667624in}%
\pgfsys@useobject{currentmarker}{}%
\end{pgfscope}%
\begin{pgfscope}%
\pgfsys@transformshift{2.076763in}{1.673646in}%
\pgfsys@useobject{currentmarker}{}%
\end{pgfscope}%
\begin{pgfscope}%
\pgfsys@transformshift{2.083824in}{1.679663in}%
\pgfsys@useobject{currentmarker}{}%
\end{pgfscope}%
\begin{pgfscope}%
\pgfsys@transformshift{2.090885in}{1.685675in}%
\pgfsys@useobject{currentmarker}{}%
\end{pgfscope}%
\begin{pgfscope}%
\pgfsys@transformshift{2.097946in}{1.691682in}%
\pgfsys@useobject{currentmarker}{}%
\end{pgfscope}%
\begin{pgfscope}%
\pgfsys@transformshift{2.105007in}{1.697685in}%
\pgfsys@useobject{currentmarker}{}%
\end{pgfscope}%
\begin{pgfscope}%
\pgfsys@transformshift{2.112068in}{1.703683in}%
\pgfsys@useobject{currentmarker}{}%
\end{pgfscope}%
\begin{pgfscope}%
\pgfsys@transformshift{2.119130in}{1.709677in}%
\pgfsys@useobject{currentmarker}{}%
\end{pgfscope}%
\begin{pgfscope}%
\pgfsys@transformshift{2.126191in}{1.715666in}%
\pgfsys@useobject{currentmarker}{}%
\end{pgfscope}%
\begin{pgfscope}%
\pgfsys@transformshift{2.133252in}{1.721650in}%
\pgfsys@useobject{currentmarker}{}%
\end{pgfscope}%
\begin{pgfscope}%
\pgfsys@transformshift{2.140313in}{1.727630in}%
\pgfsys@useobject{currentmarker}{}%
\end{pgfscope}%
\begin{pgfscope}%
\pgfsys@transformshift{2.147374in}{1.733605in}%
\pgfsys@useobject{currentmarker}{}%
\end{pgfscope}%
\begin{pgfscope}%
\pgfsys@transformshift{2.154435in}{1.739576in}%
\pgfsys@useobject{currentmarker}{}%
\end{pgfscope}%
\begin{pgfscope}%
\pgfsys@transformshift{2.161497in}{1.745542in}%
\pgfsys@useobject{currentmarker}{}%
\end{pgfscope}%
\begin{pgfscope}%
\pgfsys@transformshift{2.168558in}{1.751504in}%
\pgfsys@useobject{currentmarker}{}%
\end{pgfscope}%
\begin{pgfscope}%
\pgfsys@transformshift{2.175619in}{1.757460in}%
\pgfsys@useobject{currentmarker}{}%
\end{pgfscope}%
\begin{pgfscope}%
\pgfsys@transformshift{2.182680in}{1.763412in}%
\pgfsys@useobject{currentmarker}{}%
\end{pgfscope}%
\begin{pgfscope}%
\pgfsys@transformshift{2.189741in}{1.769359in}%
\pgfsys@useobject{currentmarker}{}%
\end{pgfscope}%
\begin{pgfscope}%
\pgfsys@transformshift{2.196802in}{1.775300in}%
\pgfsys@useobject{currentmarker}{}%
\end{pgfscope}%
\begin{pgfscope}%
\pgfsys@transformshift{2.203863in}{1.781237in}%
\pgfsys@useobject{currentmarker}{}%
\end{pgfscope}%
\begin{pgfscope}%
\pgfsys@transformshift{2.210925in}{1.787169in}%
\pgfsys@useobject{currentmarker}{}%
\end{pgfscope}%
\begin{pgfscope}%
\pgfsys@transformshift{2.217986in}{1.793095in}%
\pgfsys@useobject{currentmarker}{}%
\end{pgfscope}%
\begin{pgfscope}%
\pgfsys@transformshift{2.225047in}{1.799017in}%
\pgfsys@useobject{currentmarker}{}%
\end{pgfscope}%
\begin{pgfscope}%
\pgfsys@transformshift{2.232108in}{1.804933in}%
\pgfsys@useobject{currentmarker}{}%
\end{pgfscope}%
\begin{pgfscope}%
\pgfsys@transformshift{2.239169in}{1.810844in}%
\pgfsys@useobject{currentmarker}{}%
\end{pgfscope}%
\begin{pgfscope}%
\pgfsys@transformshift{2.246230in}{1.816749in}%
\pgfsys@useobject{currentmarker}{}%
\end{pgfscope}%
\begin{pgfscope}%
\pgfsys@transformshift{2.253291in}{1.822649in}%
\pgfsys@useobject{currentmarker}{}%
\end{pgfscope}%
\begin{pgfscope}%
\pgfsys@transformshift{2.260353in}{1.828543in}%
\pgfsys@useobject{currentmarker}{}%
\end{pgfscope}%
\begin{pgfscope}%
\pgfsys@transformshift{2.267414in}{1.834433in}%
\pgfsys@useobject{currentmarker}{}%
\end{pgfscope}%
\begin{pgfscope}%
\pgfsys@transformshift{2.274475in}{1.840316in}%
\pgfsys@useobject{currentmarker}{}%
\end{pgfscope}%
\begin{pgfscope}%
\pgfsys@transformshift{2.281536in}{1.846195in}%
\pgfsys@useobject{currentmarker}{}%
\end{pgfscope}%
\begin{pgfscope}%
\pgfsys@transformshift{2.288597in}{1.852068in}%
\pgfsys@useobject{currentmarker}{}%
\end{pgfscope}%
\begin{pgfscope}%
\pgfsys@transformshift{2.295658in}{1.857935in}%
\pgfsys@useobject{currentmarker}{}%
\end{pgfscope}%
\begin{pgfscope}%
\pgfsys@transformshift{2.302719in}{1.863798in}%
\pgfsys@useobject{currentmarker}{}%
\end{pgfscope}%
\begin{pgfscope}%
\pgfsys@transformshift{2.309781in}{1.869654in}%
\pgfsys@useobject{currentmarker}{}%
\end{pgfscope}%
\begin{pgfscope}%
\pgfsys@transformshift{2.316842in}{1.875506in}%
\pgfsys@useobject{currentmarker}{}%
\end{pgfscope}%
\begin{pgfscope}%
\pgfsys@transformshift{2.323903in}{1.881352in}%
\pgfsys@useobject{currentmarker}{}%
\end{pgfscope}%
\begin{pgfscope}%
\pgfsys@transformshift{2.330964in}{1.887192in}%
\pgfsys@useobject{currentmarker}{}%
\end{pgfscope}%
\begin{pgfscope}%
\pgfsys@transformshift{2.338025in}{1.893028in}%
\pgfsys@useobject{currentmarker}{}%
\end{pgfscope}%
\begin{pgfscope}%
\pgfsys@transformshift{2.345086in}{1.898858in}%
\pgfsys@useobject{currentmarker}{}%
\end{pgfscope}%
\begin{pgfscope}%
\pgfsys@transformshift{2.352147in}{1.904683in}%
\pgfsys@useobject{currentmarker}{}%
\end{pgfscope}%
\begin{pgfscope}%
\pgfsys@transformshift{2.359209in}{1.910502in}%
\pgfsys@useobject{currentmarker}{}%
\end{pgfscope}%
\begin{pgfscope}%
\pgfsys@transformshift{2.366270in}{1.916317in}%
\pgfsys@useobject{currentmarker}{}%
\end{pgfscope}%
\begin{pgfscope}%
\pgfsys@transformshift{2.373331in}{1.922127in}%
\pgfsys@useobject{currentmarker}{}%
\end{pgfscope}%
\begin{pgfscope}%
\pgfsys@transformshift{2.380392in}{1.927932in}%
\pgfsys@useobject{currentmarker}{}%
\end{pgfscope}%
\begin{pgfscope}%
\pgfsys@transformshift{2.387453in}{1.933732in}%
\pgfsys@useobject{currentmarker}{}%
\end{pgfscope}%
\begin{pgfscope}%
\pgfsys@transformshift{2.394514in}{1.939527in}%
\pgfsys@useobject{currentmarker}{}%
\end{pgfscope}%
\begin{pgfscope}%
\pgfsys@transformshift{2.401575in}{1.945318in}%
\pgfsys@useobject{currentmarker}{}%
\end{pgfscope}%
\begin{pgfscope}%
\pgfsys@transformshift{2.408637in}{1.951105in}%
\pgfsys@useobject{currentmarker}{}%
\end{pgfscope}%
\begin{pgfscope}%
\pgfsys@transformshift{2.415698in}{1.956887in}%
\pgfsys@useobject{currentmarker}{}%
\end{pgfscope}%
\begin{pgfscope}%
\pgfsys@transformshift{2.422759in}{1.962665in}%
\pgfsys@useobject{currentmarker}{}%
\end{pgfscope}%
\begin{pgfscope}%
\pgfsys@transformshift{2.429820in}{1.968438in}%
\pgfsys@useobject{currentmarker}{}%
\end{pgfscope}%
\begin{pgfscope}%
\pgfsys@transformshift{2.436881in}{1.974208in}%
\pgfsys@useobject{currentmarker}{}%
\end{pgfscope}%
\begin{pgfscope}%
\pgfsys@transformshift{2.443942in}{1.979974in}%
\pgfsys@useobject{currentmarker}{}%
\end{pgfscope}%
\begin{pgfscope}%
\pgfsys@transformshift{2.451003in}{1.985735in}%
\pgfsys@useobject{currentmarker}{}%
\end{pgfscope}%
\begin{pgfscope}%
\pgfsys@transformshift{2.458065in}{1.991493in}%
\pgfsys@useobject{currentmarker}{}%
\end{pgfscope}%
\begin{pgfscope}%
\pgfsys@transformshift{2.465126in}{1.997247in}%
\pgfsys@useobject{currentmarker}{}%
\end{pgfscope}%
\begin{pgfscope}%
\pgfsys@transformshift{2.472187in}{2.002997in}%
\pgfsys@useobject{currentmarker}{}%
\end{pgfscope}%
\begin{pgfscope}%
\pgfsys@transformshift{2.479248in}{2.008743in}%
\pgfsys@useobject{currentmarker}{}%
\end{pgfscope}%
\begin{pgfscope}%
\pgfsys@transformshift{2.486309in}{2.014486in}%
\pgfsys@useobject{currentmarker}{}%
\end{pgfscope}%
\begin{pgfscope}%
\pgfsys@transformshift{2.493370in}{2.020225in}%
\pgfsys@useobject{currentmarker}{}%
\end{pgfscope}%
\begin{pgfscope}%
\pgfsys@transformshift{2.500432in}{2.025960in}%
\pgfsys@useobject{currentmarker}{}%
\end{pgfscope}%
\begin{pgfscope}%
\pgfsys@transformshift{2.507493in}{2.031692in}%
\pgfsys@useobject{currentmarker}{}%
\end{pgfscope}%
\begin{pgfscope}%
\pgfsys@transformshift{2.514554in}{2.037421in}%
\pgfsys@useobject{currentmarker}{}%
\end{pgfscope}%
\begin{pgfscope}%
\pgfsys@transformshift{2.521615in}{2.043147in}%
\pgfsys@useobject{currentmarker}{}%
\end{pgfscope}%
\begin{pgfscope}%
\pgfsys@transformshift{2.528676in}{2.048869in}%
\pgfsys@useobject{currentmarker}{}%
\end{pgfscope}%
\begin{pgfscope}%
\pgfsys@transformshift{2.535737in}{2.054588in}%
\pgfsys@useobject{currentmarker}{}%
\end{pgfscope}%
\begin{pgfscope}%
\pgfsys@transformshift{2.542798in}{2.060305in}%
\pgfsys@useobject{currentmarker}{}%
\end{pgfscope}%
\begin{pgfscope}%
\pgfsys@transformshift{2.549860in}{2.066018in}%
\pgfsys@useobject{currentmarker}{}%
\end{pgfscope}%
\begin{pgfscope}%
\pgfsys@transformshift{2.556921in}{2.071729in}%
\pgfsys@useobject{currentmarker}{}%
\end{pgfscope}%
\begin{pgfscope}%
\pgfsys@transformshift{2.563982in}{2.077438in}%
\pgfsys@useobject{currentmarker}{}%
\end{pgfscope}%
\begin{pgfscope}%
\pgfsys@transformshift{2.571043in}{2.083143in}%
\pgfsys@useobject{currentmarker}{}%
\end{pgfscope}%
\begin{pgfscope}%
\pgfsys@transformshift{2.578104in}{2.088847in}%
\pgfsys@useobject{currentmarker}{}%
\end{pgfscope}%
\begin{pgfscope}%
\pgfsys@transformshift{2.585165in}{2.094548in}%
\pgfsys@useobject{currentmarker}{}%
\end{pgfscope}%
\begin{pgfscope}%
\pgfsys@transformshift{2.592226in}{2.100247in}%
\pgfsys@useobject{currentmarker}{}%
\end{pgfscope}%
\begin{pgfscope}%
\pgfsys@transformshift{2.599288in}{2.105944in}%
\pgfsys@useobject{currentmarker}{}%
\end{pgfscope}%
\begin{pgfscope}%
\pgfsys@transformshift{2.606349in}{2.111639in}%
\pgfsys@useobject{currentmarker}{}%
\end{pgfscope}%
\begin{pgfscope}%
\pgfsys@transformshift{2.613410in}{2.117332in}%
\pgfsys@useobject{currentmarker}{}%
\end{pgfscope}%
\begin{pgfscope}%
\pgfsys@transformshift{2.620471in}{2.123024in}%
\pgfsys@useobject{currentmarker}{}%
\end{pgfscope}%
\begin{pgfscope}%
\pgfsys@transformshift{2.627532in}{2.128713in}%
\pgfsys@useobject{currentmarker}{}%
\end{pgfscope}%
\begin{pgfscope}%
\pgfsys@transformshift{2.634593in}{2.134401in}%
\pgfsys@useobject{currentmarker}{}%
\end{pgfscope}%
\begin{pgfscope}%
\pgfsys@transformshift{2.641654in}{2.140087in}%
\pgfsys@useobject{currentmarker}{}%
\end{pgfscope}%
\begin{pgfscope}%
\pgfsys@transformshift{2.648716in}{2.145771in}%
\pgfsys@useobject{currentmarker}{}%
\end{pgfscope}%
\begin{pgfscope}%
\pgfsys@transformshift{2.655777in}{2.151454in}%
\pgfsys@useobject{currentmarker}{}%
\end{pgfscope}%
\begin{pgfscope}%
\pgfsys@transformshift{2.662838in}{2.157136in}%
\pgfsys@useobject{currentmarker}{}%
\end{pgfscope}%
\begin{pgfscope}%
\pgfsys@transformshift{2.669899in}{2.162816in}%
\pgfsys@useobject{currentmarker}{}%
\end{pgfscope}%
\begin{pgfscope}%
\pgfsys@transformshift{2.676960in}{2.168494in}%
\pgfsys@useobject{currentmarker}{}%
\end{pgfscope}%
\begin{pgfscope}%
\pgfsys@transformshift{2.684021in}{2.174172in}%
\pgfsys@useobject{currentmarker}{}%
\end{pgfscope}%
\begin{pgfscope}%
\pgfsys@transformshift{2.691082in}{2.179848in}%
\pgfsys@useobject{currentmarker}{}%
\end{pgfscope}%
\begin{pgfscope}%
\pgfsys@transformshift{2.698144in}{2.185522in}%
\pgfsys@useobject{currentmarker}{}%
\end{pgfscope}%
\begin{pgfscope}%
\pgfsys@transformshift{2.705205in}{2.191196in}%
\pgfsys@useobject{currentmarker}{}%
\end{pgfscope}%
\begin{pgfscope}%
\pgfsys@transformshift{2.712266in}{2.196869in}%
\pgfsys@useobject{currentmarker}{}%
\end{pgfscope}%
\begin{pgfscope}%
\pgfsys@transformshift{2.719327in}{2.202540in}%
\pgfsys@useobject{currentmarker}{}%
\end{pgfscope}%
\begin{pgfscope}%
\pgfsys@transformshift{2.726388in}{2.208210in}%
\pgfsys@useobject{currentmarker}{}%
\end{pgfscope}%
\begin{pgfscope}%
\pgfsys@transformshift{2.733449in}{2.213879in}%
\pgfsys@useobject{currentmarker}{}%
\end{pgfscope}%
\begin{pgfscope}%
\pgfsys@transformshift{2.740510in}{2.219547in}%
\pgfsys@useobject{currentmarker}{}%
\end{pgfscope}%
\begin{pgfscope}%
\pgfsys@transformshift{2.747572in}{2.225214in}%
\pgfsys@useobject{currentmarker}{}%
\end{pgfscope}%
\begin{pgfscope}%
\pgfsys@transformshift{2.754633in}{2.230880in}%
\pgfsys@useobject{currentmarker}{}%
\end{pgfscope}%
\begin{pgfscope}%
\pgfsys@transformshift{2.761694in}{2.236545in}%
\pgfsys@useobject{currentmarker}{}%
\end{pgfscope}%
\begin{pgfscope}%
\pgfsys@transformshift{2.768755in}{2.242209in}%
\pgfsys@useobject{currentmarker}{}%
\end{pgfscope}%
\begin{pgfscope}%
\pgfsys@transformshift{2.775816in}{2.247871in}%
\pgfsys@useobject{currentmarker}{}%
\end{pgfscope}%
\begin{pgfscope}%
\pgfsys@transformshift{2.782877in}{2.253533in}%
\pgfsys@useobject{currentmarker}{}%
\end{pgfscope}%
\begin{pgfscope}%
\pgfsys@transformshift{2.789938in}{2.259193in}%
\pgfsys@useobject{currentmarker}{}%
\end{pgfscope}%
\begin{pgfscope}%
\pgfsys@transformshift{2.797000in}{2.264853in}%
\pgfsys@useobject{currentmarker}{}%
\end{pgfscope}%
\begin{pgfscope}%
\pgfsys@transformshift{2.804061in}{2.270511in}%
\pgfsys@useobject{currentmarker}{}%
\end{pgfscope}%
\begin{pgfscope}%
\pgfsys@transformshift{2.811122in}{2.276168in}%
\pgfsys@useobject{currentmarker}{}%
\end{pgfscope}%
\begin{pgfscope}%
\pgfsys@transformshift{2.818183in}{2.281824in}%
\pgfsys@useobject{currentmarker}{}%
\end{pgfscope}%
\begin{pgfscope}%
\pgfsys@transformshift{2.825244in}{2.287479in}%
\pgfsys@useobject{currentmarker}{}%
\end{pgfscope}%
\begin{pgfscope}%
\pgfsys@transformshift{2.832305in}{2.293133in}%
\pgfsys@useobject{currentmarker}{}%
\end{pgfscope}%
\begin{pgfscope}%
\pgfsys@transformshift{2.839367in}{2.298785in}%
\pgfsys@useobject{currentmarker}{}%
\end{pgfscope}%
\begin{pgfscope}%
\pgfsys@transformshift{2.846428in}{2.304436in}%
\pgfsys@useobject{currentmarker}{}%
\end{pgfscope}%
\begin{pgfscope}%
\pgfsys@transformshift{2.853489in}{2.310086in}%
\pgfsys@useobject{currentmarker}{}%
\end{pgfscope}%
\begin{pgfscope}%
\pgfsys@transformshift{2.860550in}{2.315734in}%
\pgfsys@useobject{currentmarker}{}%
\end{pgfscope}%
\begin{pgfscope}%
\pgfsys@transformshift{2.867611in}{2.321381in}%
\pgfsys@useobject{currentmarker}{}%
\end{pgfscope}%
\begin{pgfscope}%
\pgfsys@transformshift{2.874672in}{2.327026in}%
\pgfsys@useobject{currentmarker}{}%
\end{pgfscope}%
\begin{pgfscope}%
\pgfsys@transformshift{2.881733in}{2.332670in}%
\pgfsys@useobject{currentmarker}{}%
\end{pgfscope}%
\begin{pgfscope}%
\pgfsys@transformshift{2.888795in}{2.338312in}%
\pgfsys@useobject{currentmarker}{}%
\end{pgfscope}%
\begin{pgfscope}%
\pgfsys@transformshift{2.895856in}{2.343952in}%
\pgfsys@useobject{currentmarker}{}%
\end{pgfscope}%
\begin{pgfscope}%
\pgfsys@transformshift{2.902917in}{2.349590in}%
\pgfsys@useobject{currentmarker}{}%
\end{pgfscope}%
\begin{pgfscope}%
\pgfsys@transformshift{2.909978in}{2.355227in}%
\pgfsys@useobject{currentmarker}{}%
\end{pgfscope}%
\begin{pgfscope}%
\pgfsys@transformshift{2.917039in}{2.360861in}%
\pgfsys@useobject{currentmarker}{}%
\end{pgfscope}%
\begin{pgfscope}%
\pgfsys@transformshift{2.924100in}{2.366492in}%
\pgfsys@useobject{currentmarker}{}%
\end{pgfscope}%
\begin{pgfscope}%
\pgfsys@transformshift{2.931161in}{2.372121in}%
\pgfsys@useobject{currentmarker}{}%
\end{pgfscope}%
\begin{pgfscope}%
\pgfsys@transformshift{2.938223in}{2.377748in}%
\pgfsys@useobject{currentmarker}{}%
\end{pgfscope}%
\begin{pgfscope}%
\pgfsys@transformshift{2.945284in}{2.383371in}%
\pgfsys@useobject{currentmarker}{}%
\end{pgfscope}%
\begin{pgfscope}%
\pgfsys@transformshift{2.952345in}{2.388991in}%
\pgfsys@useobject{currentmarker}{}%
\end{pgfscope}%
\begin{pgfscope}%
\pgfsys@transformshift{2.959406in}{2.394608in}%
\pgfsys@useobject{currentmarker}{}%
\end{pgfscope}%
\begin{pgfscope}%
\pgfsys@transformshift{2.966467in}{2.400221in}%
\pgfsys@useobject{currentmarker}{}%
\end{pgfscope}%
\begin{pgfscope}%
\pgfsys@transformshift{2.973528in}{2.405831in}%
\pgfsys@useobject{currentmarker}{}%
\end{pgfscope}%
\begin{pgfscope}%
\pgfsys@transformshift{2.980589in}{2.411436in}%
\pgfsys@useobject{currentmarker}{}%
\end{pgfscope}%
\begin{pgfscope}%
\pgfsys@transformshift{2.987651in}{2.417038in}%
\pgfsys@useobject{currentmarker}{}%
\end{pgfscope}%
\begin{pgfscope}%
\pgfsys@transformshift{2.994712in}{2.422635in}%
\pgfsys@useobject{currentmarker}{}%
\end{pgfscope}%
\begin{pgfscope}%
\pgfsys@transformshift{3.001773in}{2.428228in}%
\pgfsys@useobject{currentmarker}{}%
\end{pgfscope}%
\begin{pgfscope}%
\pgfsys@transformshift{3.008834in}{2.433816in}%
\pgfsys@useobject{currentmarker}{}%
\end{pgfscope}%
\begin{pgfscope}%
\pgfsys@transformshift{3.015895in}{2.439400in}%
\pgfsys@useobject{currentmarker}{}%
\end{pgfscope}%
\begin{pgfscope}%
\pgfsys@transformshift{3.022956in}{2.444980in}%
\pgfsys@useobject{currentmarker}{}%
\end{pgfscope}%
\begin{pgfscope}%
\pgfsys@transformshift{3.030017in}{2.450555in}%
\pgfsys@useobject{currentmarker}{}%
\end{pgfscope}%
\begin{pgfscope}%
\pgfsys@transformshift{3.037079in}{2.456125in}%
\pgfsys@useobject{currentmarker}{}%
\end{pgfscope}%
\begin{pgfscope}%
\pgfsys@transformshift{3.044140in}{2.461692in}%
\pgfsys@useobject{currentmarker}{}%
\end{pgfscope}%
\begin{pgfscope}%
\pgfsys@transformshift{3.051201in}{2.467254in}%
\pgfsys@useobject{currentmarker}{}%
\end{pgfscope}%
\begin{pgfscope}%
\pgfsys@transformshift{3.058262in}{2.472812in}%
\pgfsys@useobject{currentmarker}{}%
\end{pgfscope}%
\begin{pgfscope}%
\pgfsys@transformshift{3.065323in}{2.478366in}%
\pgfsys@useobject{currentmarker}{}%
\end{pgfscope}%
\begin{pgfscope}%
\pgfsys@transformshift{3.072384in}{2.483917in}%
\pgfsys@useobject{currentmarker}{}%
\end{pgfscope}%
\begin{pgfscope}%
\pgfsys@transformshift{3.079445in}{2.489463in}%
\pgfsys@useobject{currentmarker}{}%
\end{pgfscope}%
\begin{pgfscope}%
\pgfsys@transformshift{3.086507in}{2.495007in}%
\pgfsys@useobject{currentmarker}{}%
\end{pgfscope}%
\begin{pgfscope}%
\pgfsys@transformshift{3.093568in}{2.500547in}%
\pgfsys@useobject{currentmarker}{}%
\end{pgfscope}%
\begin{pgfscope}%
\pgfsys@transformshift{3.100629in}{2.506084in}%
\pgfsys@useobject{currentmarker}{}%
\end{pgfscope}%
\begin{pgfscope}%
\pgfsys@transformshift{3.107690in}{2.511618in}%
\pgfsys@useobject{currentmarker}{}%
\end{pgfscope}%
\begin{pgfscope}%
\pgfsys@transformshift{3.114751in}{2.517149in}%
\pgfsys@useobject{currentmarker}{}%
\end{pgfscope}%
\begin{pgfscope}%
\pgfsys@transformshift{3.121812in}{2.522678in}%
\pgfsys@useobject{currentmarker}{}%
\end{pgfscope}%
\begin{pgfscope}%
\pgfsys@transformshift{3.128874in}{2.528204in}%
\pgfsys@useobject{currentmarker}{}%
\end{pgfscope}%
\begin{pgfscope}%
\pgfsys@transformshift{3.135935in}{2.533728in}%
\pgfsys@useobject{currentmarker}{}%
\end{pgfscope}%
\begin{pgfscope}%
\pgfsys@transformshift{3.142996in}{2.539250in}%
\pgfsys@useobject{currentmarker}{}%
\end{pgfscope}%
\begin{pgfscope}%
\pgfsys@transformshift{3.150057in}{2.544770in}%
\pgfsys@useobject{currentmarker}{}%
\end{pgfscope}%
\begin{pgfscope}%
\pgfsys@transformshift{3.157118in}{2.550287in}%
\pgfsys@useobject{currentmarker}{}%
\end{pgfscope}%
\begin{pgfscope}%
\pgfsys@transformshift{3.164179in}{2.555803in}%
\pgfsys@useobject{currentmarker}{}%
\end{pgfscope}%
\begin{pgfscope}%
\pgfsys@transformshift{3.171240in}{2.561316in}%
\pgfsys@useobject{currentmarker}{}%
\end{pgfscope}%
\begin{pgfscope}%
\pgfsys@transformshift{3.178302in}{2.566828in}%
\pgfsys@useobject{currentmarker}{}%
\end{pgfscope}%
\begin{pgfscope}%
\pgfsys@transformshift{3.185363in}{2.572338in}%
\pgfsys@useobject{currentmarker}{}%
\end{pgfscope}%
\begin{pgfscope}%
\pgfsys@transformshift{3.192424in}{2.577846in}%
\pgfsys@useobject{currentmarker}{}%
\end{pgfscope}%
\begin{pgfscope}%
\pgfsys@transformshift{3.199485in}{2.583351in}%
\pgfsys@useobject{currentmarker}{}%
\end{pgfscope}%
\begin{pgfscope}%
\pgfsys@transformshift{3.206546in}{2.588855in}%
\pgfsys@useobject{currentmarker}{}%
\end{pgfscope}%
\begin{pgfscope}%
\pgfsys@transformshift{3.213607in}{2.594357in}%
\pgfsys@useobject{currentmarker}{}%
\end{pgfscope}%
\begin{pgfscope}%
\pgfsys@transformshift{3.220668in}{2.599856in}%
\pgfsys@useobject{currentmarker}{}%
\end{pgfscope}%
\begin{pgfscope}%
\pgfsys@transformshift{3.227730in}{2.605354in}%
\pgfsys@useobject{currentmarker}{}%
\end{pgfscope}%
\begin{pgfscope}%
\pgfsys@transformshift{3.234791in}{2.610849in}%
\pgfsys@useobject{currentmarker}{}%
\end{pgfscope}%
\begin{pgfscope}%
\pgfsys@transformshift{3.241852in}{2.616342in}%
\pgfsys@useobject{currentmarker}{}%
\end{pgfscope}%
\begin{pgfscope}%
\pgfsys@transformshift{3.248913in}{2.621833in}%
\pgfsys@useobject{currentmarker}{}%
\end{pgfscope}%
\begin{pgfscope}%
\pgfsys@transformshift{3.255974in}{2.627322in}%
\pgfsys@useobject{currentmarker}{}%
\end{pgfscope}%
\begin{pgfscope}%
\pgfsys@transformshift{3.263035in}{2.632808in}%
\pgfsys@useobject{currentmarker}{}%
\end{pgfscope}%
\begin{pgfscope}%
\pgfsys@transformshift{3.270096in}{2.638293in}%
\pgfsys@useobject{currentmarker}{}%
\end{pgfscope}%
\begin{pgfscope}%
\pgfsys@transformshift{3.277158in}{2.643775in}%
\pgfsys@useobject{currentmarker}{}%
\end{pgfscope}%
\begin{pgfscope}%
\pgfsys@transformshift{3.284219in}{2.649255in}%
\pgfsys@useobject{currentmarker}{}%
\end{pgfscope}%
\begin{pgfscope}%
\pgfsys@transformshift{3.291280in}{2.654734in}%
\pgfsys@useobject{currentmarker}{}%
\end{pgfscope}%
\begin{pgfscope}%
\pgfsys@transformshift{3.298341in}{2.660210in}%
\pgfsys@useobject{currentmarker}{}%
\end{pgfscope}%
\begin{pgfscope}%
\pgfsys@transformshift{3.305402in}{2.665685in}%
\pgfsys@useobject{currentmarker}{}%
\end{pgfscope}%
\begin{pgfscope}%
\pgfsys@transformshift{3.312463in}{2.671158in}%
\pgfsys@useobject{currentmarker}{}%
\end{pgfscope}%
\begin{pgfscope}%
\pgfsys@transformshift{3.319524in}{2.676630in}%
\pgfsys@useobject{currentmarker}{}%
\end{pgfscope}%
\begin{pgfscope}%
\pgfsys@transformshift{3.326586in}{2.682100in}%
\pgfsys@useobject{currentmarker}{}%
\end{pgfscope}%
\begin{pgfscope}%
\pgfsys@transformshift{3.333647in}{2.687569in}%
\pgfsys@useobject{currentmarker}{}%
\end{pgfscope}%
\begin{pgfscope}%
\pgfsys@transformshift{3.340708in}{2.693037in}%
\pgfsys@useobject{currentmarker}{}%
\end{pgfscope}%
\begin{pgfscope}%
\pgfsys@transformshift{3.347769in}{2.698503in}%
\pgfsys@useobject{currentmarker}{}%
\end{pgfscope}%
\begin{pgfscope}%
\pgfsys@transformshift{3.354830in}{2.703968in}%
\pgfsys@useobject{currentmarker}{}%
\end{pgfscope}%
\begin{pgfscope}%
\pgfsys@transformshift{3.361891in}{2.709433in}%
\pgfsys@useobject{currentmarker}{}%
\end{pgfscope}%
\begin{pgfscope}%
\pgfsys@transformshift{3.368952in}{2.714896in}%
\pgfsys@useobject{currentmarker}{}%
\end{pgfscope}%
\begin{pgfscope}%
\pgfsys@transformshift{3.376014in}{2.720359in}%
\pgfsys@useobject{currentmarker}{}%
\end{pgfscope}%
\begin{pgfscope}%
\pgfsys@transformshift{3.383075in}{2.725820in}%
\pgfsys@useobject{currentmarker}{}%
\end{pgfscope}%
\begin{pgfscope}%
\pgfsys@transformshift{3.390136in}{2.731281in}%
\pgfsys@useobject{currentmarker}{}%
\end{pgfscope}%
\begin{pgfscope}%
\pgfsys@transformshift{3.397197in}{2.736741in}%
\pgfsys@useobject{currentmarker}{}%
\end{pgfscope}%
\begin{pgfscope}%
\pgfsys@transformshift{3.404258in}{2.742200in}%
\pgfsys@useobject{currentmarker}{}%
\end{pgfscope}%
\begin{pgfscope}%
\pgfsys@transformshift{3.411319in}{2.747658in}%
\pgfsys@useobject{currentmarker}{}%
\end{pgfscope}%
\begin{pgfscope}%
\pgfsys@transformshift{3.418380in}{2.753115in}%
\pgfsys@useobject{currentmarker}{}%
\end{pgfscope}%
\begin{pgfscope}%
\pgfsys@transformshift{3.425442in}{2.758571in}%
\pgfsys@useobject{currentmarker}{}%
\end{pgfscope}%
\begin{pgfscope}%
\pgfsys@transformshift{3.432503in}{2.764027in}%
\pgfsys@useobject{currentmarker}{}%
\end{pgfscope}%
\begin{pgfscope}%
\pgfsys@transformshift{3.439564in}{2.769481in}%
\pgfsys@useobject{currentmarker}{}%
\end{pgfscope}%
\begin{pgfscope}%
\pgfsys@transformshift{3.446625in}{2.774935in}%
\pgfsys@useobject{currentmarker}{}%
\end{pgfscope}%
\begin{pgfscope}%
\pgfsys@transformshift{3.453686in}{2.780388in}%
\pgfsys@useobject{currentmarker}{}%
\end{pgfscope}%
\begin{pgfscope}%
\pgfsys@transformshift{3.460747in}{2.785839in}%
\pgfsys@useobject{currentmarker}{}%
\end{pgfscope}%
\begin{pgfscope}%
\pgfsys@transformshift{3.467809in}{2.791290in}%
\pgfsys@useobject{currentmarker}{}%
\end{pgfscope}%
\begin{pgfscope}%
\pgfsys@transformshift{3.474870in}{2.796740in}%
\pgfsys@useobject{currentmarker}{}%
\end{pgfscope}%
\begin{pgfscope}%
\pgfsys@transformshift{3.481931in}{2.802189in}%
\pgfsys@useobject{currentmarker}{}%
\end{pgfscope}%
\begin{pgfscope}%
\pgfsys@transformshift{3.488992in}{2.807637in}%
\pgfsys@useobject{currentmarker}{}%
\end{pgfscope}%
\begin{pgfscope}%
\pgfsys@transformshift{3.496053in}{2.813084in}%
\pgfsys@useobject{currentmarker}{}%
\end{pgfscope}%
\begin{pgfscope}%
\pgfsys@transformshift{3.503114in}{2.818529in}%
\pgfsys@useobject{currentmarker}{}%
\end{pgfscope}%
\begin{pgfscope}%
\pgfsys@transformshift{3.510175in}{2.823974in}%
\pgfsys@useobject{currentmarker}{}%
\end{pgfscope}%
\begin{pgfscope}%
\pgfsys@transformshift{3.517237in}{2.829418in}%
\pgfsys@useobject{currentmarker}{}%
\end{pgfscope}%
\begin{pgfscope}%
\pgfsys@transformshift{3.524298in}{2.834862in}%
\pgfsys@useobject{currentmarker}{}%
\end{pgfscope}%
\begin{pgfscope}%
\pgfsys@transformshift{3.531359in}{2.840304in}%
\pgfsys@useobject{currentmarker}{}%
\end{pgfscope}%
\begin{pgfscope}%
\pgfsys@transformshift{3.538420in}{2.845745in}%
\pgfsys@useobject{currentmarker}{}%
\end{pgfscope}%
\begin{pgfscope}%
\pgfsys@transformshift{3.545481in}{2.851186in}%
\pgfsys@useobject{currentmarker}{}%
\end{pgfscope}%
\begin{pgfscope}%
\pgfsys@transformshift{3.552542in}{2.856625in}%
\pgfsys@useobject{currentmarker}{}%
\end{pgfscope}%
\begin{pgfscope}%
\pgfsys@transformshift{3.559603in}{2.862064in}%
\pgfsys@useobject{currentmarker}{}%
\end{pgfscope}%
\begin{pgfscope}%
\pgfsys@transformshift{3.566665in}{2.867502in}%
\pgfsys@useobject{currentmarker}{}%
\end{pgfscope}%
\begin{pgfscope}%
\pgfsys@transformshift{3.573726in}{2.872940in}%
\pgfsys@useobject{currentmarker}{}%
\end{pgfscope}%
\begin{pgfscope}%
\pgfsys@transformshift{3.580787in}{2.878377in}%
\pgfsys@useobject{currentmarker}{}%
\end{pgfscope}%
\begin{pgfscope}%
\pgfsys@transformshift{3.587848in}{2.883814in}%
\pgfsys@useobject{currentmarker}{}%
\end{pgfscope}%
\begin{pgfscope}%
\pgfsys@transformshift{3.594909in}{2.889250in}%
\pgfsys@useobject{currentmarker}{}%
\end{pgfscope}%
\begin{pgfscope}%
\pgfsys@transformshift{3.601970in}{2.894686in}%
\pgfsys@useobject{currentmarker}{}%
\end{pgfscope}%
\begin{pgfscope}%
\pgfsys@transformshift{3.609031in}{2.900122in}%
\pgfsys@useobject{currentmarker}{}%
\end{pgfscope}%
\begin{pgfscope}%
\pgfsys@transformshift{3.616093in}{2.905558in}%
\pgfsys@useobject{currentmarker}{}%
\end{pgfscope}%
\begin{pgfscope}%
\pgfsys@transformshift{3.623154in}{2.910994in}%
\pgfsys@useobject{currentmarker}{}%
\end{pgfscope}%
\begin{pgfscope}%
\pgfsys@transformshift{3.630215in}{2.916430in}%
\pgfsys@useobject{currentmarker}{}%
\end{pgfscope}%
\begin{pgfscope}%
\pgfsys@transformshift{3.637276in}{2.921866in}%
\pgfsys@useobject{currentmarker}{}%
\end{pgfscope}%
\begin{pgfscope}%
\pgfsys@transformshift{3.644337in}{2.927302in}%
\pgfsys@useobject{currentmarker}{}%
\end{pgfscope}%
\begin{pgfscope}%
\pgfsys@transformshift{3.651398in}{2.932738in}%
\pgfsys@useobject{currentmarker}{}%
\end{pgfscope}%
\begin{pgfscope}%
\pgfsys@transformshift{3.658459in}{2.938174in}%
\pgfsys@useobject{currentmarker}{}%
\end{pgfscope}%
\begin{pgfscope}%
\pgfsys@transformshift{3.665521in}{2.943611in}%
\pgfsys@useobject{currentmarker}{}%
\end{pgfscope}%
\begin{pgfscope}%
\pgfsys@transformshift{3.672582in}{2.949048in}%
\pgfsys@useobject{currentmarker}{}%
\end{pgfscope}%
\begin{pgfscope}%
\pgfsys@transformshift{3.679643in}{2.954485in}%
\pgfsys@useobject{currentmarker}{}%
\end{pgfscope}%
\begin{pgfscope}%
\pgfsys@transformshift{3.686704in}{2.959922in}%
\pgfsys@useobject{currentmarker}{}%
\end{pgfscope}%
\begin{pgfscope}%
\pgfsys@transformshift{3.693765in}{2.965360in}%
\pgfsys@useobject{currentmarker}{}%
\end{pgfscope}%
\begin{pgfscope}%
\pgfsys@transformshift{3.700826in}{2.970799in}%
\pgfsys@useobject{currentmarker}{}%
\end{pgfscope}%
\begin{pgfscope}%
\pgfsys@transformshift{3.707887in}{2.976237in}%
\pgfsys@useobject{currentmarker}{}%
\end{pgfscope}%
\begin{pgfscope}%
\pgfsys@transformshift{3.714949in}{2.981676in}%
\pgfsys@useobject{currentmarker}{}%
\end{pgfscope}%
\begin{pgfscope}%
\pgfsys@transformshift{3.722010in}{2.987116in}%
\pgfsys@useobject{currentmarker}{}%
\end{pgfscope}%
\begin{pgfscope}%
\pgfsys@transformshift{3.729071in}{2.992555in}%
\pgfsys@useobject{currentmarker}{}%
\end{pgfscope}%
\begin{pgfscope}%
\pgfsys@transformshift{3.736132in}{2.997996in}%
\pgfsys@useobject{currentmarker}{}%
\end{pgfscope}%
\begin{pgfscope}%
\pgfsys@transformshift{3.743193in}{3.003437in}%
\pgfsys@useobject{currentmarker}{}%
\end{pgfscope}%
\begin{pgfscope}%
\pgfsys@transformshift{3.750254in}{3.008878in}%
\pgfsys@useobject{currentmarker}{}%
\end{pgfscope}%
\begin{pgfscope}%
\pgfsys@transformshift{3.757315in}{3.014320in}%
\pgfsys@useobject{currentmarker}{}%
\end{pgfscope}%
\begin{pgfscope}%
\pgfsys@transformshift{3.764377in}{3.019763in}%
\pgfsys@useobject{currentmarker}{}%
\end{pgfscope}%
\begin{pgfscope}%
\pgfsys@transformshift{3.771438in}{3.025206in}%
\pgfsys@useobject{currentmarker}{}%
\end{pgfscope}%
\begin{pgfscope}%
\pgfsys@transformshift{3.778499in}{3.030649in}%
\pgfsys@useobject{currentmarker}{}%
\end{pgfscope}%
\begin{pgfscope}%
\pgfsys@transformshift{3.785560in}{3.036093in}%
\pgfsys@useobject{currentmarker}{}%
\end{pgfscope}%
\begin{pgfscope}%
\pgfsys@transformshift{3.792621in}{3.041538in}%
\pgfsys@useobject{currentmarker}{}%
\end{pgfscope}%
\begin{pgfscope}%
\pgfsys@transformshift{3.799682in}{3.046984in}%
\pgfsys@useobject{currentmarker}{}%
\end{pgfscope}%
\begin{pgfscope}%
\pgfsys@transformshift{3.806744in}{3.052430in}%
\pgfsys@useobject{currentmarker}{}%
\end{pgfscope}%
\begin{pgfscope}%
\pgfsys@transformshift{3.813805in}{3.057876in}%
\pgfsys@useobject{currentmarker}{}%
\end{pgfscope}%
\begin{pgfscope}%
\pgfsys@transformshift{3.820866in}{3.063323in}%
\pgfsys@useobject{currentmarker}{}%
\end{pgfscope}%
\begin{pgfscope}%
\pgfsys@transformshift{3.827927in}{3.068770in}%
\pgfsys@useobject{currentmarker}{}%
\end{pgfscope}%
\begin{pgfscope}%
\pgfsys@transformshift{3.834988in}{3.074218in}%
\pgfsys@useobject{currentmarker}{}%
\end{pgfscope}%
\begin{pgfscope}%
\pgfsys@transformshift{3.842049in}{3.079666in}%
\pgfsys@useobject{currentmarker}{}%
\end{pgfscope}%
\begin{pgfscope}%
\pgfsys@transformshift{3.849110in}{3.085115in}%
\pgfsys@useobject{currentmarker}{}%
\end{pgfscope}%
\begin{pgfscope}%
\pgfsys@transformshift{3.856172in}{3.090564in}%
\pgfsys@useobject{currentmarker}{}%
\end{pgfscope}%
\begin{pgfscope}%
\pgfsys@transformshift{3.863233in}{3.096013in}%
\pgfsys@useobject{currentmarker}{}%
\end{pgfscope}%
\begin{pgfscope}%
\pgfsys@transformshift{3.870294in}{3.101462in}%
\pgfsys@useobject{currentmarker}{}%
\end{pgfscope}%
\begin{pgfscope}%
\pgfsys@transformshift{3.877355in}{3.106911in}%
\pgfsys@useobject{currentmarker}{}%
\end{pgfscope}%
\begin{pgfscope}%
\pgfsys@transformshift{3.884416in}{3.112361in}%
\pgfsys@useobject{currentmarker}{}%
\end{pgfscope}%
\begin{pgfscope}%
\pgfsys@transformshift{3.891477in}{3.117811in}%
\pgfsys@useobject{currentmarker}{}%
\end{pgfscope}%
\end{pgfscope}%
\begin{pgfscope}%
\pgfpathrectangle{\pgfqpoint{0.562500in}{0.407000in}}{\pgfqpoint{3.487500in}{2.849000in}}%
\pgfusepath{clip}%
\pgfsetrectcap%
\pgfsetroundjoin%
\pgfsetlinewidth{1.505625pt}%
\definecolor{currentstroke}{rgb}{1.000000,0.000000,0.000000}%
\pgfsetstrokecolor{currentstroke}%
\pgfsetdash{}{0pt}%
\pgfpathmoveto{\pgfqpoint{0.721023in}{0.616798in}}%
\pgfpathlineto{\pgfqpoint{3.891477in}{3.126500in}}%
\pgfpathlineto{\pgfqpoint{3.891477in}{3.126500in}}%
\pgfusepath{stroke}%
\end{pgfscope}%
\begin{pgfscope}%
\pgfsetrectcap%
\pgfsetmiterjoin%
\pgfsetlinewidth{0.803000pt}%
\definecolor{currentstroke}{rgb}{0.000000,0.000000,0.000000}%
\pgfsetstrokecolor{currentstroke}%
\pgfsetdash{}{0pt}%
\pgfpathmoveto{\pgfqpoint{0.562500in}{0.407000in}}%
\pgfpathlineto{\pgfqpoint{0.562500in}{3.256000in}}%
\pgfusepath{stroke}%
\end{pgfscope}%
\begin{pgfscope}%
\pgfsetrectcap%
\pgfsetmiterjoin%
\pgfsetlinewidth{0.803000pt}%
\definecolor{currentstroke}{rgb}{0.000000,0.000000,0.000000}%
\pgfsetstrokecolor{currentstroke}%
\pgfsetdash{}{0pt}%
\pgfpathmoveto{\pgfqpoint{4.050000in}{0.407000in}}%
\pgfpathlineto{\pgfqpoint{4.050000in}{3.256000in}}%
\pgfusepath{stroke}%
\end{pgfscope}%
\begin{pgfscope}%
\pgfsetrectcap%
\pgfsetmiterjoin%
\pgfsetlinewidth{0.803000pt}%
\definecolor{currentstroke}{rgb}{0.000000,0.000000,0.000000}%
\pgfsetstrokecolor{currentstroke}%
\pgfsetdash{}{0pt}%
\pgfpathmoveto{\pgfqpoint{0.562500in}{0.407000in}}%
\pgfpathlineto{\pgfqpoint{4.050000in}{0.407000in}}%
\pgfusepath{stroke}%
\end{pgfscope}%
\begin{pgfscope}%
\pgfsetrectcap%
\pgfsetmiterjoin%
\pgfsetlinewidth{0.803000pt}%
\definecolor{currentstroke}{rgb}{0.000000,0.000000,0.000000}%
\pgfsetstrokecolor{currentstroke}%
\pgfsetdash{}{0pt}%
\pgfpathmoveto{\pgfqpoint{0.562500in}{3.256000in}}%
\pgfpathlineto{\pgfqpoint{4.050000in}{3.256000in}}%
\pgfusepath{stroke}%
\end{pgfscope}%
\begin{pgfscope}%
\pgfsetbuttcap%
\pgfsetmiterjoin%
\definecolor{currentfill}{rgb}{1.000000,1.000000,1.000000}%
\pgfsetfillcolor{currentfill}%
\pgfsetfillopacity{0.800000}%
\pgfsetlinewidth{1.003750pt}%
\definecolor{currentstroke}{rgb}{0.800000,0.800000,0.800000}%
\pgfsetstrokecolor{currentstroke}%
\pgfsetstrokeopacity{0.800000}%
\pgfsetdash{}{0pt}%
\pgfpathmoveto{\pgfqpoint{0.659722in}{2.757556in}}%
\pgfpathlineto{\pgfqpoint{1.933054in}{2.757556in}}%
\pgfpathquadraticcurveto{\pgfqpoint{1.960832in}{2.757556in}}{\pgfqpoint{1.960832in}{2.785334in}}%
\pgfpathlineto{\pgfqpoint{1.960832in}{3.158778in}}%
\pgfpathquadraticcurveto{\pgfqpoint{1.960832in}{3.186556in}}{\pgfqpoint{1.933054in}{3.186556in}}%
\pgfpathlineto{\pgfqpoint{0.659722in}{3.186556in}}%
\pgfpathquadraticcurveto{\pgfqpoint{0.631944in}{3.186556in}}{\pgfqpoint{0.631944in}{3.158778in}}%
\pgfpathlineto{\pgfqpoint{0.631944in}{2.785334in}}%
\pgfpathquadraticcurveto{\pgfqpoint{0.631944in}{2.757556in}}{\pgfqpoint{0.659722in}{2.757556in}}%
\pgfpathlineto{\pgfqpoint{0.659722in}{2.757556in}}%
\pgfpathclose%
\pgfusepath{stroke,fill}%
\end{pgfscope}%
\begin{pgfscope}%
\pgfsetbuttcap%
\pgfsetroundjoin%
\definecolor{currentfill}{rgb}{0.121569,0.466667,0.705882}%
\pgfsetfillcolor{currentfill}%
\pgfsetlinewidth{1.003750pt}%
\definecolor{currentstroke}{rgb}{0.121569,0.466667,0.705882}%
\pgfsetstrokecolor{currentstroke}%
\pgfsetdash{}{0pt}%
\pgfsys@defobject{currentmarker}{\pgfqpoint{-0.020833in}{-0.020833in}}{\pgfqpoint{0.020833in}{0.020833in}}{%
\pgfpathmoveto{\pgfqpoint{0.000000in}{-0.020833in}}%
\pgfpathcurveto{\pgfqpoint{0.005525in}{-0.020833in}}{\pgfqpoint{0.010825in}{-0.018638in}}{\pgfqpoint{0.014731in}{-0.014731in}}%
\pgfpathcurveto{\pgfqpoint{0.018638in}{-0.010825in}}{\pgfqpoint{0.020833in}{-0.005525in}}{\pgfqpoint{0.020833in}{0.000000in}}%
\pgfpathcurveto{\pgfqpoint{0.020833in}{0.005525in}}{\pgfqpoint{0.018638in}{0.010825in}}{\pgfqpoint{0.014731in}{0.014731in}}%
\pgfpathcurveto{\pgfqpoint{0.010825in}{0.018638in}}{\pgfqpoint{0.005525in}{0.020833in}}{\pgfqpoint{0.000000in}{0.020833in}}%
\pgfpathcurveto{\pgfqpoint{-0.005525in}{0.020833in}}{\pgfqpoint{-0.010825in}{0.018638in}}{\pgfqpoint{-0.014731in}{0.014731in}}%
\pgfpathcurveto{\pgfqpoint{-0.018638in}{0.010825in}}{\pgfqpoint{-0.020833in}{0.005525in}}{\pgfqpoint{-0.020833in}{0.000000in}}%
\pgfpathcurveto{\pgfqpoint{-0.020833in}{-0.005525in}}{\pgfqpoint{-0.018638in}{-0.010825in}}{\pgfqpoint{-0.014731in}{-0.014731in}}%
\pgfpathcurveto{\pgfqpoint{-0.010825in}{-0.018638in}}{\pgfqpoint{-0.005525in}{-0.020833in}}{\pgfqpoint{0.000000in}{-0.020833in}}%
\pgfpathlineto{\pgfqpoint{0.000000in}{-0.020833in}}%
\pgfpathclose%
\pgfusepath{stroke,fill}%
}%
\begin{pgfscope}%
\pgfsys@transformshift{0.826389in}{3.082389in}%
\pgfsys@useobject{currentmarker}{}%
\end{pgfscope}%
\end{pgfscope}%
\begin{pgfscope}%
\definecolor{textcolor}{rgb}{0.000000,0.000000,0.000000}%
\pgfsetstrokecolor{textcolor}%
\pgfsetfillcolor{textcolor}%
\pgftext[x=1.076389in,y=3.033778in,left,base]{\color{textcolor}\rmfamily\fontsize{10.000000}{12.000000}\selectfont MSD samples}%
\end{pgfscope}%
\begin{pgfscope}%
\pgfsetrectcap%
\pgfsetroundjoin%
\pgfsetlinewidth{1.505625pt}%
\definecolor{currentstroke}{rgb}{1.000000,0.000000,0.000000}%
\pgfsetstrokecolor{currentstroke}%
\pgfsetdash{}{0pt}%
\pgfpathmoveto{\pgfqpoint{0.687500in}{2.888723in}}%
\pgfpathlineto{\pgfqpoint{0.826389in}{2.888723in}}%
\pgfpathlineto{\pgfqpoint{0.965278in}{2.888723in}}%
\pgfusepath{stroke}%
\end{pgfscope}%
\begin{pgfscope}%
\definecolor{textcolor}{rgb}{0.000000,0.000000,0.000000}%
\pgfsetstrokecolor{textcolor}%
\pgfsetfillcolor{textcolor}%
\pgftext[x=1.076389in,y=2.840112in,left,base]{\color{textcolor}\rmfamily\fontsize{10.000000}{12.000000}\selectfont fit}%
\end{pgfscope}%
\end{pgfpicture}%
\makeatother%
\endgroup%

        }
        \caption{Mean square displacement (MSD).}
        \label{msd_aver}
    \end{subfigure}
    \begin{subfigure}{0.5\textwidth}
        \resizebox{\textwidth}{!}{
            %% Creator: Matplotlib, PGF backend
%%
%% To include the figure in your LaTeX document, write
%%   \input{<filename>.pgf}
%%
%% Make sure the required packages are loaded in your preamble
%%   \usepackage{pgf}
%%
%% Also ensure that all the required font packages are loaded; for instance,
%% the lmodern package is sometimes necessary when using math font.
%%   \usepackage{lmodern}
%%
%% Figures using additional raster images can only be included by \input if
%% they are in the same directory as the main LaTeX file. For loading figures
%% from other directories you can use the `import` package
%%   \usepackage{import}
%%
%% and then include the figures with
%%   \import{<path to file>}{<filename>.pgf}
%%
%% Matplotlib used the following preamble
%%   \usepackage[utf8]{inputenc}
%%   \usepackage[T1]{fontenc}
%%   \usepackage{siunitx}
%%
\begingroup%
\makeatletter%
\begin{pgfpicture}%
\pgfpathrectangle{\pgfpointorigin}{\pgfqpoint{4.500000in}{3.700000in}}%
\pgfusepath{use as bounding box, clip}%
\begin{pgfscope}%
\pgfsetbuttcap%
\pgfsetmiterjoin%
\definecolor{currentfill}{rgb}{1.000000,1.000000,1.000000}%
\pgfsetfillcolor{currentfill}%
\pgfsetlinewidth{0.000000pt}%
\definecolor{currentstroke}{rgb}{1.000000,1.000000,1.000000}%
\pgfsetstrokecolor{currentstroke}%
\pgfsetdash{}{0pt}%
\pgfpathmoveto{\pgfqpoint{0.000000in}{0.000000in}}%
\pgfpathlineto{\pgfqpoint{4.500000in}{0.000000in}}%
\pgfpathlineto{\pgfqpoint{4.500000in}{3.700000in}}%
\pgfpathlineto{\pgfqpoint{0.000000in}{3.700000in}}%
\pgfpathlineto{\pgfqpoint{0.000000in}{0.000000in}}%
\pgfpathclose%
\pgfusepath{fill}%
\end{pgfscope}%
\begin{pgfscope}%
\pgfsetbuttcap%
\pgfsetmiterjoin%
\definecolor{currentfill}{rgb}{1.000000,1.000000,1.000000}%
\pgfsetfillcolor{currentfill}%
\pgfsetlinewidth{0.000000pt}%
\definecolor{currentstroke}{rgb}{0.000000,0.000000,0.000000}%
\pgfsetstrokecolor{currentstroke}%
\pgfsetstrokeopacity{0.000000}%
\pgfsetdash{}{0pt}%
\pgfpathmoveto{\pgfqpoint{0.562500in}{0.407000in}}%
\pgfpathlineto{\pgfqpoint{4.050000in}{0.407000in}}%
\pgfpathlineto{\pgfqpoint{4.050000in}{3.256000in}}%
\pgfpathlineto{\pgfqpoint{0.562500in}{3.256000in}}%
\pgfpathlineto{\pgfqpoint{0.562500in}{0.407000in}}%
\pgfpathclose%
\pgfusepath{fill}%
\end{pgfscope}%
\begin{pgfscope}%
\pgfsetbuttcap%
\pgfsetroundjoin%
\definecolor{currentfill}{rgb}{0.000000,0.000000,0.000000}%
\pgfsetfillcolor{currentfill}%
\pgfsetlinewidth{0.803000pt}%
\definecolor{currentstroke}{rgb}{0.000000,0.000000,0.000000}%
\pgfsetstrokecolor{currentstroke}%
\pgfsetdash{}{0pt}%
\pgfsys@defobject{currentmarker}{\pgfqpoint{0.000000in}{-0.048611in}}{\pgfqpoint{0.000000in}{0.000000in}}{%
\pgfpathmoveto{\pgfqpoint{0.000000in}{0.000000in}}%
\pgfpathlineto{\pgfqpoint{0.000000in}{-0.048611in}}%
\pgfusepath{stroke,fill}%
}%
\begin{pgfscope}%
\pgfsys@transformshift{0.713962in}{0.407000in}%
\pgfsys@useobject{currentmarker}{}%
\end{pgfscope}%
\end{pgfscope}%
\begin{pgfscope}%
\definecolor{textcolor}{rgb}{0.000000,0.000000,0.000000}%
\pgfsetstrokecolor{textcolor}%
\pgfsetfillcolor{textcolor}%
\pgftext[x=0.713962in,y=0.309778in,,top]{\color{textcolor}\rmfamily\fontsize{10.000000}{12.000000}\selectfont \(\displaystyle {0.0}\)}%
\end{pgfscope}%
\begin{pgfscope}%
\pgfsetbuttcap%
\pgfsetroundjoin%
\definecolor{currentfill}{rgb}{0.000000,0.000000,0.000000}%
\pgfsetfillcolor{currentfill}%
\pgfsetlinewidth{0.803000pt}%
\definecolor{currentstroke}{rgb}{0.000000,0.000000,0.000000}%
\pgfsetstrokecolor{currentstroke}%
\pgfsetdash{}{0pt}%
\pgfsys@defobject{currentmarker}{\pgfqpoint{0.000000in}{-0.048611in}}{\pgfqpoint{0.000000in}{0.000000in}}{%
\pgfpathmoveto{\pgfqpoint{0.000000in}{0.000000in}}%
\pgfpathlineto{\pgfqpoint{0.000000in}{-0.048611in}}%
\pgfusepath{stroke,fill}%
}%
\begin{pgfscope}%
\pgfsys@transformshift{1.420076in}{0.407000in}%
\pgfsys@useobject{currentmarker}{}%
\end{pgfscope}%
\end{pgfscope}%
\begin{pgfscope}%
\definecolor{textcolor}{rgb}{0.000000,0.000000,0.000000}%
\pgfsetstrokecolor{textcolor}%
\pgfsetfillcolor{textcolor}%
\pgftext[x=1.420076in,y=0.309778in,,top]{\color{textcolor}\rmfamily\fontsize{10.000000}{12.000000}\selectfont \(\displaystyle {0.2}\)}%
\end{pgfscope}%
\begin{pgfscope}%
\pgfsetbuttcap%
\pgfsetroundjoin%
\definecolor{currentfill}{rgb}{0.000000,0.000000,0.000000}%
\pgfsetfillcolor{currentfill}%
\pgfsetlinewidth{0.803000pt}%
\definecolor{currentstroke}{rgb}{0.000000,0.000000,0.000000}%
\pgfsetstrokecolor{currentstroke}%
\pgfsetdash{}{0pt}%
\pgfsys@defobject{currentmarker}{\pgfqpoint{0.000000in}{-0.048611in}}{\pgfqpoint{0.000000in}{0.000000in}}{%
\pgfpathmoveto{\pgfqpoint{0.000000in}{0.000000in}}%
\pgfpathlineto{\pgfqpoint{0.000000in}{-0.048611in}}%
\pgfusepath{stroke,fill}%
}%
\begin{pgfscope}%
\pgfsys@transformshift{2.126191in}{0.407000in}%
\pgfsys@useobject{currentmarker}{}%
\end{pgfscope}%
\end{pgfscope}%
\begin{pgfscope}%
\definecolor{textcolor}{rgb}{0.000000,0.000000,0.000000}%
\pgfsetstrokecolor{textcolor}%
\pgfsetfillcolor{textcolor}%
\pgftext[x=2.126191in,y=0.309778in,,top]{\color{textcolor}\rmfamily\fontsize{10.000000}{12.000000}\selectfont \(\displaystyle {0.4}\)}%
\end{pgfscope}%
\begin{pgfscope}%
\pgfsetbuttcap%
\pgfsetroundjoin%
\definecolor{currentfill}{rgb}{0.000000,0.000000,0.000000}%
\pgfsetfillcolor{currentfill}%
\pgfsetlinewidth{0.803000pt}%
\definecolor{currentstroke}{rgb}{0.000000,0.000000,0.000000}%
\pgfsetstrokecolor{currentstroke}%
\pgfsetdash{}{0pt}%
\pgfsys@defobject{currentmarker}{\pgfqpoint{0.000000in}{-0.048611in}}{\pgfqpoint{0.000000in}{0.000000in}}{%
\pgfpathmoveto{\pgfqpoint{0.000000in}{0.000000in}}%
\pgfpathlineto{\pgfqpoint{0.000000in}{-0.048611in}}%
\pgfusepath{stroke,fill}%
}%
\begin{pgfscope}%
\pgfsys@transformshift{2.832305in}{0.407000in}%
\pgfsys@useobject{currentmarker}{}%
\end{pgfscope}%
\end{pgfscope}%
\begin{pgfscope}%
\definecolor{textcolor}{rgb}{0.000000,0.000000,0.000000}%
\pgfsetstrokecolor{textcolor}%
\pgfsetfillcolor{textcolor}%
\pgftext[x=2.832305in,y=0.309778in,,top]{\color{textcolor}\rmfamily\fontsize{10.000000}{12.000000}\selectfont \(\displaystyle {0.6}\)}%
\end{pgfscope}%
\begin{pgfscope}%
\pgfsetbuttcap%
\pgfsetroundjoin%
\definecolor{currentfill}{rgb}{0.000000,0.000000,0.000000}%
\pgfsetfillcolor{currentfill}%
\pgfsetlinewidth{0.803000pt}%
\definecolor{currentstroke}{rgb}{0.000000,0.000000,0.000000}%
\pgfsetstrokecolor{currentstroke}%
\pgfsetdash{}{0pt}%
\pgfsys@defobject{currentmarker}{\pgfqpoint{0.000000in}{-0.048611in}}{\pgfqpoint{0.000000in}{0.000000in}}{%
\pgfpathmoveto{\pgfqpoint{0.000000in}{0.000000in}}%
\pgfpathlineto{\pgfqpoint{0.000000in}{-0.048611in}}%
\pgfusepath{stroke,fill}%
}%
\begin{pgfscope}%
\pgfsys@transformshift{3.538420in}{0.407000in}%
\pgfsys@useobject{currentmarker}{}%
\end{pgfscope}%
\end{pgfscope}%
\begin{pgfscope}%
\definecolor{textcolor}{rgb}{0.000000,0.000000,0.000000}%
\pgfsetstrokecolor{textcolor}%
\pgfsetfillcolor{textcolor}%
\pgftext[x=3.538420in,y=0.309778in,,top]{\color{textcolor}\rmfamily\fontsize{10.000000}{12.000000}\selectfont \(\displaystyle {0.8}\)}%
\end{pgfscope}%
\begin{pgfscope}%
\definecolor{textcolor}{rgb}{0.000000,0.000000,0.000000}%
\pgfsetstrokecolor{textcolor}%
\pgfsetfillcolor{textcolor}%
\pgftext[x=2.306250in,y=0.131567in,,top]{\color{textcolor}\rmfamily\fontsize{10.000000}{12.000000}\selectfont \(\displaystyle t\)}%
\end{pgfscope}%
\begin{pgfscope}%
\pgfsetbuttcap%
\pgfsetroundjoin%
\definecolor{currentfill}{rgb}{0.000000,0.000000,0.000000}%
\pgfsetfillcolor{currentfill}%
\pgfsetlinewidth{0.803000pt}%
\definecolor{currentstroke}{rgb}{0.000000,0.000000,0.000000}%
\pgfsetstrokecolor{currentstroke}%
\pgfsetdash{}{0pt}%
\pgfsys@defobject{currentmarker}{\pgfqpoint{-0.048611in}{0.000000in}}{\pgfqpoint{-0.000000in}{0.000000in}}{%
\pgfpathmoveto{\pgfqpoint{-0.000000in}{0.000000in}}%
\pgfpathlineto{\pgfqpoint{-0.048611in}{0.000000in}}%
\pgfusepath{stroke,fill}%
}%
\begin{pgfscope}%
\pgfsys@transformshift{0.562500in}{0.427029in}%
\pgfsys@useobject{currentmarker}{}%
\end{pgfscope}%
\end{pgfscope}%
\begin{pgfscope}%
\definecolor{textcolor}{rgb}{0.000000,0.000000,0.000000}%
\pgfsetstrokecolor{textcolor}%
\pgfsetfillcolor{textcolor}%
\pgftext[x=0.179783in, y=0.379202in, left, base]{\color{textcolor}\rmfamily\fontsize{10.000000}{12.000000}\selectfont \(\displaystyle {\ensuremath{-}0.1}\)}%
\end{pgfscope}%
\begin{pgfscope}%
\pgfsetbuttcap%
\pgfsetroundjoin%
\definecolor{currentfill}{rgb}{0.000000,0.000000,0.000000}%
\pgfsetfillcolor{currentfill}%
\pgfsetlinewidth{0.803000pt}%
\definecolor{currentstroke}{rgb}{0.000000,0.000000,0.000000}%
\pgfsetstrokecolor{currentstroke}%
\pgfsetdash{}{0pt}%
\pgfsys@defobject{currentmarker}{\pgfqpoint{-0.048611in}{0.000000in}}{\pgfqpoint{-0.000000in}{0.000000in}}{%
\pgfpathmoveto{\pgfqpoint{-0.000000in}{0.000000in}}%
\pgfpathlineto{\pgfqpoint{-0.048611in}{0.000000in}}%
\pgfusepath{stroke,fill}%
}%
\begin{pgfscope}%
\pgfsys@transformshift{0.562500in}{0.743640in}%
\pgfsys@useobject{currentmarker}{}%
\end{pgfscope}%
\end{pgfscope}%
\begin{pgfscope}%
\definecolor{textcolor}{rgb}{0.000000,0.000000,0.000000}%
\pgfsetstrokecolor{textcolor}%
\pgfsetfillcolor{textcolor}%
\pgftext[x=0.287808in, y=0.695812in, left, base]{\color{textcolor}\rmfamily\fontsize{10.000000}{12.000000}\selectfont \(\displaystyle {0.0}\)}%
\end{pgfscope}%
\begin{pgfscope}%
\pgfsetbuttcap%
\pgfsetroundjoin%
\definecolor{currentfill}{rgb}{0.000000,0.000000,0.000000}%
\pgfsetfillcolor{currentfill}%
\pgfsetlinewidth{0.803000pt}%
\definecolor{currentstroke}{rgb}{0.000000,0.000000,0.000000}%
\pgfsetstrokecolor{currentstroke}%
\pgfsetdash{}{0pt}%
\pgfsys@defobject{currentmarker}{\pgfqpoint{-0.048611in}{0.000000in}}{\pgfqpoint{-0.000000in}{0.000000in}}{%
\pgfpathmoveto{\pgfqpoint{-0.000000in}{0.000000in}}%
\pgfpathlineto{\pgfqpoint{-0.048611in}{0.000000in}}%
\pgfusepath{stroke,fill}%
}%
\begin{pgfscope}%
\pgfsys@transformshift{0.562500in}{1.060251in}%
\pgfsys@useobject{currentmarker}{}%
\end{pgfscope}%
\end{pgfscope}%
\begin{pgfscope}%
\definecolor{textcolor}{rgb}{0.000000,0.000000,0.000000}%
\pgfsetstrokecolor{textcolor}%
\pgfsetfillcolor{textcolor}%
\pgftext[x=0.287808in, y=1.012423in, left, base]{\color{textcolor}\rmfamily\fontsize{10.000000}{12.000000}\selectfont \(\displaystyle {0.1}\)}%
\end{pgfscope}%
\begin{pgfscope}%
\pgfsetbuttcap%
\pgfsetroundjoin%
\definecolor{currentfill}{rgb}{0.000000,0.000000,0.000000}%
\pgfsetfillcolor{currentfill}%
\pgfsetlinewidth{0.803000pt}%
\definecolor{currentstroke}{rgb}{0.000000,0.000000,0.000000}%
\pgfsetstrokecolor{currentstroke}%
\pgfsetdash{}{0pt}%
\pgfsys@defobject{currentmarker}{\pgfqpoint{-0.048611in}{0.000000in}}{\pgfqpoint{-0.000000in}{0.000000in}}{%
\pgfpathmoveto{\pgfqpoint{-0.000000in}{0.000000in}}%
\pgfpathlineto{\pgfqpoint{-0.048611in}{0.000000in}}%
\pgfusepath{stroke,fill}%
}%
\begin{pgfscope}%
\pgfsys@transformshift{0.562500in}{1.376861in}%
\pgfsys@useobject{currentmarker}{}%
\end{pgfscope}%
\end{pgfscope}%
\begin{pgfscope}%
\definecolor{textcolor}{rgb}{0.000000,0.000000,0.000000}%
\pgfsetstrokecolor{textcolor}%
\pgfsetfillcolor{textcolor}%
\pgftext[x=0.287808in, y=1.329033in, left, base]{\color{textcolor}\rmfamily\fontsize{10.000000}{12.000000}\selectfont \(\displaystyle {0.2}\)}%
\end{pgfscope}%
\begin{pgfscope}%
\pgfsetbuttcap%
\pgfsetroundjoin%
\definecolor{currentfill}{rgb}{0.000000,0.000000,0.000000}%
\pgfsetfillcolor{currentfill}%
\pgfsetlinewidth{0.803000pt}%
\definecolor{currentstroke}{rgb}{0.000000,0.000000,0.000000}%
\pgfsetstrokecolor{currentstroke}%
\pgfsetdash{}{0pt}%
\pgfsys@defobject{currentmarker}{\pgfqpoint{-0.048611in}{0.000000in}}{\pgfqpoint{-0.000000in}{0.000000in}}{%
\pgfpathmoveto{\pgfqpoint{-0.000000in}{0.000000in}}%
\pgfpathlineto{\pgfqpoint{-0.048611in}{0.000000in}}%
\pgfusepath{stroke,fill}%
}%
\begin{pgfscope}%
\pgfsys@transformshift{0.562500in}{1.693472in}%
\pgfsys@useobject{currentmarker}{}%
\end{pgfscope}%
\end{pgfscope}%
\begin{pgfscope}%
\definecolor{textcolor}{rgb}{0.000000,0.000000,0.000000}%
\pgfsetstrokecolor{textcolor}%
\pgfsetfillcolor{textcolor}%
\pgftext[x=0.287808in, y=1.645644in, left, base]{\color{textcolor}\rmfamily\fontsize{10.000000}{12.000000}\selectfont \(\displaystyle {0.3}\)}%
\end{pgfscope}%
\begin{pgfscope}%
\pgfsetbuttcap%
\pgfsetroundjoin%
\definecolor{currentfill}{rgb}{0.000000,0.000000,0.000000}%
\pgfsetfillcolor{currentfill}%
\pgfsetlinewidth{0.803000pt}%
\definecolor{currentstroke}{rgb}{0.000000,0.000000,0.000000}%
\pgfsetstrokecolor{currentstroke}%
\pgfsetdash{}{0pt}%
\pgfsys@defobject{currentmarker}{\pgfqpoint{-0.048611in}{0.000000in}}{\pgfqpoint{-0.000000in}{0.000000in}}{%
\pgfpathmoveto{\pgfqpoint{-0.000000in}{0.000000in}}%
\pgfpathlineto{\pgfqpoint{-0.048611in}{0.000000in}}%
\pgfusepath{stroke,fill}%
}%
\begin{pgfscope}%
\pgfsys@transformshift{0.562500in}{2.010082in}%
\pgfsys@useobject{currentmarker}{}%
\end{pgfscope}%
\end{pgfscope}%
\begin{pgfscope}%
\definecolor{textcolor}{rgb}{0.000000,0.000000,0.000000}%
\pgfsetstrokecolor{textcolor}%
\pgfsetfillcolor{textcolor}%
\pgftext[x=0.287808in, y=1.962255in, left, base]{\color{textcolor}\rmfamily\fontsize{10.000000}{12.000000}\selectfont \(\displaystyle {0.4}\)}%
\end{pgfscope}%
\begin{pgfscope}%
\pgfsetbuttcap%
\pgfsetroundjoin%
\definecolor{currentfill}{rgb}{0.000000,0.000000,0.000000}%
\pgfsetfillcolor{currentfill}%
\pgfsetlinewidth{0.803000pt}%
\definecolor{currentstroke}{rgb}{0.000000,0.000000,0.000000}%
\pgfsetstrokecolor{currentstroke}%
\pgfsetdash{}{0pt}%
\pgfsys@defobject{currentmarker}{\pgfqpoint{-0.048611in}{0.000000in}}{\pgfqpoint{-0.000000in}{0.000000in}}{%
\pgfpathmoveto{\pgfqpoint{-0.000000in}{0.000000in}}%
\pgfpathlineto{\pgfqpoint{-0.048611in}{0.000000in}}%
\pgfusepath{stroke,fill}%
}%
\begin{pgfscope}%
\pgfsys@transformshift{0.562500in}{2.326693in}%
\pgfsys@useobject{currentmarker}{}%
\end{pgfscope}%
\end{pgfscope}%
\begin{pgfscope}%
\definecolor{textcolor}{rgb}{0.000000,0.000000,0.000000}%
\pgfsetstrokecolor{textcolor}%
\pgfsetfillcolor{textcolor}%
\pgftext[x=0.287808in, y=2.278865in, left, base]{\color{textcolor}\rmfamily\fontsize{10.000000}{12.000000}\selectfont \(\displaystyle {0.5}\)}%
\end{pgfscope}%
\begin{pgfscope}%
\pgfsetbuttcap%
\pgfsetroundjoin%
\definecolor{currentfill}{rgb}{0.000000,0.000000,0.000000}%
\pgfsetfillcolor{currentfill}%
\pgfsetlinewidth{0.803000pt}%
\definecolor{currentstroke}{rgb}{0.000000,0.000000,0.000000}%
\pgfsetstrokecolor{currentstroke}%
\pgfsetdash{}{0pt}%
\pgfsys@defobject{currentmarker}{\pgfqpoint{-0.048611in}{0.000000in}}{\pgfqpoint{-0.000000in}{0.000000in}}{%
\pgfpathmoveto{\pgfqpoint{-0.000000in}{0.000000in}}%
\pgfpathlineto{\pgfqpoint{-0.048611in}{0.000000in}}%
\pgfusepath{stroke,fill}%
}%
\begin{pgfscope}%
\pgfsys@transformshift{0.562500in}{2.643304in}%
\pgfsys@useobject{currentmarker}{}%
\end{pgfscope}%
\end{pgfscope}%
\begin{pgfscope}%
\definecolor{textcolor}{rgb}{0.000000,0.000000,0.000000}%
\pgfsetstrokecolor{textcolor}%
\pgfsetfillcolor{textcolor}%
\pgftext[x=0.287808in, y=2.595476in, left, base]{\color{textcolor}\rmfamily\fontsize{10.000000}{12.000000}\selectfont \(\displaystyle {0.6}\)}%
\end{pgfscope}%
\begin{pgfscope}%
\pgfsetbuttcap%
\pgfsetroundjoin%
\definecolor{currentfill}{rgb}{0.000000,0.000000,0.000000}%
\pgfsetfillcolor{currentfill}%
\pgfsetlinewidth{0.803000pt}%
\definecolor{currentstroke}{rgb}{0.000000,0.000000,0.000000}%
\pgfsetstrokecolor{currentstroke}%
\pgfsetdash{}{0pt}%
\pgfsys@defobject{currentmarker}{\pgfqpoint{-0.048611in}{0.000000in}}{\pgfqpoint{-0.000000in}{0.000000in}}{%
\pgfpathmoveto{\pgfqpoint{-0.000000in}{0.000000in}}%
\pgfpathlineto{\pgfqpoint{-0.048611in}{0.000000in}}%
\pgfusepath{stroke,fill}%
}%
\begin{pgfscope}%
\pgfsys@transformshift{0.562500in}{2.959914in}%
\pgfsys@useobject{currentmarker}{}%
\end{pgfscope}%
\end{pgfscope}%
\begin{pgfscope}%
\definecolor{textcolor}{rgb}{0.000000,0.000000,0.000000}%
\pgfsetstrokecolor{textcolor}%
\pgfsetfillcolor{textcolor}%
\pgftext[x=0.287808in, y=2.912086in, left, base]{\color{textcolor}\rmfamily\fontsize{10.000000}{12.000000}\selectfont \(\displaystyle {0.7}\)}%
\end{pgfscope}%
\begin{pgfscope}%
\definecolor{textcolor}{rgb}{0.000000,0.000000,0.000000}%
\pgfsetstrokecolor{textcolor}%
\pgfsetfillcolor{textcolor}%
\pgftext[x=0.124227in,y=1.831500in,,bottom,rotate=90.000000]{\color{textcolor}\rmfamily\fontsize{10.000000}{12.000000}\selectfont VACF}%
\end{pgfscope}%
\begin{pgfscope}%
\pgfpathrectangle{\pgfqpoint{0.562500in}{0.407000in}}{\pgfqpoint{3.487500in}{2.849000in}}%
\pgfusepath{clip}%
\pgfsetbuttcap%
\pgfsetroundjoin%
\pgfsetlinewidth{1.505625pt}%
\definecolor{currentstroke}{rgb}{0.000000,0.000000,0.000000}%
\pgfsetstrokecolor{currentstroke}%
\pgfsetdash{{5.550000pt}{2.400000pt}}{0.000000pt}%
\pgfpathmoveto{\pgfqpoint{0.562500in}{0.743640in}}%
\pgfpathlineto{\pgfqpoint{4.050000in}{0.743640in}}%
\pgfusepath{stroke}%
\end{pgfscope}%
\begin{pgfscope}%
\pgfpathrectangle{\pgfqpoint{0.562500in}{0.407000in}}{\pgfqpoint{3.487500in}{2.849000in}}%
\pgfusepath{clip}%
\pgfsetrectcap%
\pgfsetroundjoin%
\pgfsetlinewidth{1.505625pt}%
\definecolor{currentstroke}{rgb}{1.000000,0.647059,0.000000}%
\pgfsetstrokecolor{currentstroke}%
\pgfsetdash{}{0pt}%
\pgfpathmoveto{\pgfqpoint{0.721023in}{3.126500in}}%
\pgfpathlineto{\pgfqpoint{0.728084in}{3.125229in}}%
\pgfpathlineto{\pgfqpoint{0.735145in}{3.121601in}}%
\pgfpathlineto{\pgfqpoint{0.742206in}{3.115628in}}%
\pgfpathlineto{\pgfqpoint{0.749267in}{3.107330in}}%
\pgfpathlineto{\pgfqpoint{0.756328in}{3.096734in}}%
\pgfpathlineto{\pgfqpoint{0.770451in}{3.068792in}}%
\pgfpathlineto{\pgfqpoint{0.784573in}{3.032162in}}%
\pgfpathlineto{\pgfqpoint{0.798695in}{2.987303in}}%
\pgfpathlineto{\pgfqpoint{0.812818in}{2.934765in}}%
\pgfpathlineto{\pgfqpoint{0.826940in}{2.875168in}}%
\pgfpathlineto{\pgfqpoint{0.848123in}{2.774024in}}%
\pgfpathlineto{\pgfqpoint{0.869307in}{2.660942in}}%
\pgfpathlineto{\pgfqpoint{0.897551in}{2.495906in}}%
\pgfpathlineto{\pgfqpoint{0.932857in}{2.274825in}}%
\pgfpathlineto{\pgfqpoint{1.024652in}{1.691068in}}%
\pgfpathlineto{\pgfqpoint{1.059958in}{1.484524in}}%
\pgfpathlineto{\pgfqpoint{1.088202in}{1.332143in}}%
\pgfpathlineto{\pgfqpoint{1.116447in}{1.192940in}}%
\pgfpathlineto{\pgfqpoint{1.137630in}{1.097810in}}%
\pgfpathlineto{\pgfqpoint{1.158814in}{1.010908in}}%
\pgfpathlineto{\pgfqpoint{1.179997in}{0.932329in}}%
\pgfpathlineto{\pgfqpoint{1.201181in}{0.862030in}}%
\pgfpathlineto{\pgfqpoint{1.222364in}{0.799847in}}%
\pgfpathlineto{\pgfqpoint{1.243548in}{0.745510in}}%
\pgfpathlineto{\pgfqpoint{1.264731in}{0.698663in}}%
\pgfpathlineto{\pgfqpoint{1.285914in}{0.658890in}}%
\pgfpathlineto{\pgfqpoint{1.307098in}{0.625731in}}%
\pgfpathlineto{\pgfqpoint{1.321220in}{0.607060in}}%
\pgfpathlineto{\pgfqpoint{1.335342in}{0.590962in}}%
\pgfpathlineto{\pgfqpoint{1.349465in}{0.577283in}}%
\pgfpathlineto{\pgfqpoint{1.363587in}{0.565871in}}%
\pgfpathlineto{\pgfqpoint{1.377709in}{0.556570in}}%
\pgfpathlineto{\pgfqpoint{1.391832in}{0.549229in}}%
\pgfpathlineto{\pgfqpoint{1.405954in}{0.543698in}}%
\pgfpathlineto{\pgfqpoint{1.420076in}{0.539830in}}%
\pgfpathlineto{\pgfqpoint{1.441260in}{0.536834in}}%
\pgfpathlineto{\pgfqpoint{1.462443in}{0.536780in}}%
\pgfpathlineto{\pgfqpoint{1.483626in}{0.539207in}}%
\pgfpathlineto{\pgfqpoint{1.504810in}{0.543670in}}%
\pgfpathlineto{\pgfqpoint{1.533055in}{0.552081in}}%
\pgfpathlineto{\pgfqpoint{1.568360in}{0.565358in}}%
\pgfpathlineto{\pgfqpoint{1.624849in}{0.589732in}}%
\pgfpathlineto{\pgfqpoint{1.688400in}{0.616834in}}%
\pgfpathlineto{\pgfqpoint{1.730767in}{0.632647in}}%
\pgfpathlineto{\pgfqpoint{1.773133in}{0.645996in}}%
\pgfpathlineto{\pgfqpoint{1.815500in}{0.656898in}}%
\pgfpathlineto{\pgfqpoint{1.857867in}{0.665575in}}%
\pgfpathlineto{\pgfqpoint{1.900234in}{0.672197in}}%
\pgfpathlineto{\pgfqpoint{1.949662in}{0.677586in}}%
\pgfpathlineto{\pgfqpoint{2.006151in}{0.681270in}}%
\pgfpathlineto{\pgfqpoint{2.083824in}{0.683787in}}%
\pgfpathlineto{\pgfqpoint{2.288597in}{0.689680in}}%
\pgfpathlineto{\pgfqpoint{2.429820in}{0.696310in}}%
\pgfpathlineto{\pgfqpoint{2.599288in}{0.706432in}}%
\pgfpathlineto{\pgfqpoint{2.853489in}{0.723088in}}%
\pgfpathlineto{\pgfqpoint{2.938223in}{0.725803in}}%
\pgfpathlineto{\pgfqpoint{3.037079in}{0.726421in}}%
\pgfpathlineto{\pgfqpoint{3.164179in}{0.727766in}}%
\pgfpathlineto{\pgfqpoint{3.587848in}{0.735278in}}%
\pgfpathlineto{\pgfqpoint{3.785560in}{0.736334in}}%
\pgfpathlineto{\pgfqpoint{3.891477in}{0.737324in}}%
\pgfpathlineto{\pgfqpoint{3.891477in}{0.737324in}}%
\pgfusepath{stroke}%
\end{pgfscope}%
\begin{pgfscope}%
\pgfsetrectcap%
\pgfsetmiterjoin%
\pgfsetlinewidth{0.803000pt}%
\definecolor{currentstroke}{rgb}{0.000000,0.000000,0.000000}%
\pgfsetstrokecolor{currentstroke}%
\pgfsetdash{}{0pt}%
\pgfpathmoveto{\pgfqpoint{0.562500in}{0.407000in}}%
\pgfpathlineto{\pgfqpoint{0.562500in}{3.256000in}}%
\pgfusepath{stroke}%
\end{pgfscope}%
\begin{pgfscope}%
\pgfsetrectcap%
\pgfsetmiterjoin%
\pgfsetlinewidth{0.803000pt}%
\definecolor{currentstroke}{rgb}{0.000000,0.000000,0.000000}%
\pgfsetstrokecolor{currentstroke}%
\pgfsetdash{}{0pt}%
\pgfpathmoveto{\pgfqpoint{4.050000in}{0.407000in}}%
\pgfpathlineto{\pgfqpoint{4.050000in}{3.256000in}}%
\pgfusepath{stroke}%
\end{pgfscope}%
\begin{pgfscope}%
\pgfsetrectcap%
\pgfsetmiterjoin%
\pgfsetlinewidth{0.803000pt}%
\definecolor{currentstroke}{rgb}{0.000000,0.000000,0.000000}%
\pgfsetstrokecolor{currentstroke}%
\pgfsetdash{}{0pt}%
\pgfpathmoveto{\pgfqpoint{0.562500in}{0.407000in}}%
\pgfpathlineto{\pgfqpoint{4.050000in}{0.407000in}}%
\pgfusepath{stroke}%
\end{pgfscope}%
\begin{pgfscope}%
\pgfsetrectcap%
\pgfsetmiterjoin%
\pgfsetlinewidth{0.803000pt}%
\definecolor{currentstroke}{rgb}{0.000000,0.000000,0.000000}%
\pgfsetstrokecolor{currentstroke}%
\pgfsetdash{}{0pt}%
\pgfpathmoveto{\pgfqpoint{0.562500in}{3.256000in}}%
\pgfpathlineto{\pgfqpoint{4.050000in}{3.256000in}}%
\pgfusepath{stroke}%
\end{pgfscope}%
\end{pgfpicture}%
\makeatother%
\endgroup%

        }
        \caption{Velocity auto-correlation function (VAFC).}
        \label{vacf_aver}
    \end{subfigure}
    
    \begin{subfigure}{0.5\textwidth}
        \resizebox{\textwidth}{!}{
            \input{step3_hist_msd.pgf}
        }
        \caption{Distribution of $D_{MSD}$ obtained by MSD.}
        \label{msd_distrib}
    \end{subfigure}
    \begin{subfigure}{0.5\textwidth}
        \resizebox{\textwidth}{!}{
            %% Creator: Matplotlib, PGF backend
%%
%% To include the figure in your LaTeX document, write
%%   \input{<filename>.pgf}
%%
%% Make sure the required packages are loaded in your preamble
%%   \usepackage{pgf}
%%
%% Also ensure that all the required font packages are loaded; for instance,
%% the lmodern package is sometimes necessary when using math font.
%%   \usepackage{lmodern}
%%
%% Figures using additional raster images can only be included by \input if
%% they are in the same directory as the main LaTeX file. For loading figures
%% from other directories you can use the `import` package
%%   \usepackage{import}
%%
%% and then include the figures with
%%   \import{<path to file>}{<filename>.pgf}
%%
%% Matplotlib used the following preamble
%%   \usepackage[utf8]{inputenc}
%%   \usepackage[T1]{fontenc}
%%   \usepackage{siunitx}
%%
\begingroup%
\makeatletter%
\begin{pgfpicture}%
\pgfpathrectangle{\pgfpointorigin}{\pgfqpoint{4.500000in}{3.700000in}}%
\pgfusepath{use as bounding box, clip}%
\begin{pgfscope}%
\pgfsetbuttcap%
\pgfsetmiterjoin%
\definecolor{currentfill}{rgb}{1.000000,1.000000,1.000000}%
\pgfsetfillcolor{currentfill}%
\pgfsetlinewidth{0.000000pt}%
\definecolor{currentstroke}{rgb}{1.000000,1.000000,1.000000}%
\pgfsetstrokecolor{currentstroke}%
\pgfsetdash{}{0pt}%
\pgfpathmoveto{\pgfqpoint{0.000000in}{0.000000in}}%
\pgfpathlineto{\pgfqpoint{4.500000in}{0.000000in}}%
\pgfpathlineto{\pgfqpoint{4.500000in}{3.700000in}}%
\pgfpathlineto{\pgfqpoint{0.000000in}{3.700000in}}%
\pgfpathlineto{\pgfqpoint{0.000000in}{0.000000in}}%
\pgfpathclose%
\pgfusepath{fill}%
\end{pgfscope}%
\begin{pgfscope}%
\pgfsetbuttcap%
\pgfsetmiterjoin%
\definecolor{currentfill}{rgb}{1.000000,1.000000,1.000000}%
\pgfsetfillcolor{currentfill}%
\pgfsetlinewidth{0.000000pt}%
\definecolor{currentstroke}{rgb}{0.000000,0.000000,0.000000}%
\pgfsetstrokecolor{currentstroke}%
\pgfsetstrokeopacity{0.000000}%
\pgfsetdash{}{0pt}%
\pgfpathmoveto{\pgfqpoint{0.562500in}{0.407000in}}%
\pgfpathlineto{\pgfqpoint{4.050000in}{0.407000in}}%
\pgfpathlineto{\pgfqpoint{4.050000in}{3.256000in}}%
\pgfpathlineto{\pgfqpoint{0.562500in}{3.256000in}}%
\pgfpathlineto{\pgfqpoint{0.562500in}{0.407000in}}%
\pgfpathclose%
\pgfusepath{fill}%
\end{pgfscope}%
\begin{pgfscope}%
\pgfpathrectangle{\pgfqpoint{0.562500in}{0.407000in}}{\pgfqpoint{3.487500in}{2.849000in}}%
\pgfusepath{clip}%
\pgfsetbuttcap%
\pgfsetmiterjoin%
\definecolor{currentfill}{rgb}{0.121569,0.466667,0.705882}%
\pgfsetfillcolor{currentfill}%
\pgfsetlinewidth{0.000000pt}%
\definecolor{currentstroke}{rgb}{0.000000,0.000000,0.000000}%
\pgfsetstrokecolor{currentstroke}%
\pgfsetstrokeopacity{0.000000}%
\pgfsetdash{}{0pt}%
\pgfpathmoveto{\pgfqpoint{0.721023in}{0.407000in}}%
\pgfpathlineto{\pgfqpoint{0.784432in}{0.407000in}}%
\pgfpathlineto{\pgfqpoint{0.784432in}{0.615718in}}%
\pgfpathlineto{\pgfqpoint{0.721023in}{0.615718in}}%
\pgfpathlineto{\pgfqpoint{0.721023in}{0.407000in}}%
\pgfpathclose%
\pgfusepath{fill}%
\end{pgfscope}%
\begin{pgfscope}%
\pgfpathrectangle{\pgfqpoint{0.562500in}{0.407000in}}{\pgfqpoint{3.487500in}{2.849000in}}%
\pgfusepath{clip}%
\pgfsetbuttcap%
\pgfsetmiterjoin%
\definecolor{currentfill}{rgb}{0.121569,0.466667,0.705882}%
\pgfsetfillcolor{currentfill}%
\pgfsetlinewidth{0.000000pt}%
\definecolor{currentstroke}{rgb}{0.000000,0.000000,0.000000}%
\pgfsetstrokecolor{currentstroke}%
\pgfsetstrokeopacity{0.000000}%
\pgfsetdash{}{0pt}%
\pgfpathmoveto{\pgfqpoint{0.784432in}{0.407000in}}%
\pgfpathlineto{\pgfqpoint{0.847841in}{0.407000in}}%
\pgfpathlineto{\pgfqpoint{0.847841in}{0.407000in}}%
\pgfpathlineto{\pgfqpoint{0.784432in}{0.407000in}}%
\pgfpathlineto{\pgfqpoint{0.784432in}{0.407000in}}%
\pgfpathclose%
\pgfusepath{fill}%
\end{pgfscope}%
\begin{pgfscope}%
\pgfpathrectangle{\pgfqpoint{0.562500in}{0.407000in}}{\pgfqpoint{3.487500in}{2.849000in}}%
\pgfusepath{clip}%
\pgfsetbuttcap%
\pgfsetmiterjoin%
\definecolor{currentfill}{rgb}{0.121569,0.466667,0.705882}%
\pgfsetfillcolor{currentfill}%
\pgfsetlinewidth{0.000000pt}%
\definecolor{currentstroke}{rgb}{0.000000,0.000000,0.000000}%
\pgfsetstrokecolor{currentstroke}%
\pgfsetstrokeopacity{0.000000}%
\pgfsetdash{}{0pt}%
\pgfpathmoveto{\pgfqpoint{0.847841in}{0.407000in}}%
\pgfpathlineto{\pgfqpoint{0.911250in}{0.407000in}}%
\pgfpathlineto{\pgfqpoint{0.911250in}{0.407000in}}%
\pgfpathlineto{\pgfqpoint{0.847841in}{0.407000in}}%
\pgfpathlineto{\pgfqpoint{0.847841in}{0.407000in}}%
\pgfpathclose%
\pgfusepath{fill}%
\end{pgfscope}%
\begin{pgfscope}%
\pgfpathrectangle{\pgfqpoint{0.562500in}{0.407000in}}{\pgfqpoint{3.487500in}{2.849000in}}%
\pgfusepath{clip}%
\pgfsetbuttcap%
\pgfsetmiterjoin%
\definecolor{currentfill}{rgb}{0.121569,0.466667,0.705882}%
\pgfsetfillcolor{currentfill}%
\pgfsetlinewidth{0.000000pt}%
\definecolor{currentstroke}{rgb}{0.000000,0.000000,0.000000}%
\pgfsetstrokecolor{currentstroke}%
\pgfsetstrokeopacity{0.000000}%
\pgfsetdash{}{0pt}%
\pgfpathmoveto{\pgfqpoint{0.911250in}{0.407000in}}%
\pgfpathlineto{\pgfqpoint{0.974659in}{0.407000in}}%
\pgfpathlineto{\pgfqpoint{0.974659in}{0.407000in}}%
\pgfpathlineto{\pgfqpoint{0.911250in}{0.407000in}}%
\pgfpathlineto{\pgfqpoint{0.911250in}{0.407000in}}%
\pgfpathclose%
\pgfusepath{fill}%
\end{pgfscope}%
\begin{pgfscope}%
\pgfpathrectangle{\pgfqpoint{0.562500in}{0.407000in}}{\pgfqpoint{3.487500in}{2.849000in}}%
\pgfusepath{clip}%
\pgfsetbuttcap%
\pgfsetmiterjoin%
\definecolor{currentfill}{rgb}{0.121569,0.466667,0.705882}%
\pgfsetfillcolor{currentfill}%
\pgfsetlinewidth{0.000000pt}%
\definecolor{currentstroke}{rgb}{0.000000,0.000000,0.000000}%
\pgfsetstrokecolor{currentstroke}%
\pgfsetstrokeopacity{0.000000}%
\pgfsetdash{}{0pt}%
\pgfpathmoveto{\pgfqpoint{0.974659in}{0.407000in}}%
\pgfpathlineto{\pgfqpoint{1.038068in}{0.407000in}}%
\pgfpathlineto{\pgfqpoint{1.038068in}{0.407000in}}%
\pgfpathlineto{\pgfqpoint{0.974659in}{0.407000in}}%
\pgfpathlineto{\pgfqpoint{0.974659in}{0.407000in}}%
\pgfpathclose%
\pgfusepath{fill}%
\end{pgfscope}%
\begin{pgfscope}%
\pgfpathrectangle{\pgfqpoint{0.562500in}{0.407000in}}{\pgfqpoint{3.487500in}{2.849000in}}%
\pgfusepath{clip}%
\pgfsetbuttcap%
\pgfsetmiterjoin%
\definecolor{currentfill}{rgb}{0.121569,0.466667,0.705882}%
\pgfsetfillcolor{currentfill}%
\pgfsetlinewidth{0.000000pt}%
\definecolor{currentstroke}{rgb}{0.000000,0.000000,0.000000}%
\pgfsetstrokecolor{currentstroke}%
\pgfsetstrokeopacity{0.000000}%
\pgfsetdash{}{0pt}%
\pgfpathmoveto{\pgfqpoint{1.038068in}{0.407000in}}%
\pgfpathlineto{\pgfqpoint{1.101477in}{0.407000in}}%
\pgfpathlineto{\pgfqpoint{1.101477in}{0.407000in}}%
\pgfpathlineto{\pgfqpoint{1.038068in}{0.407000in}}%
\pgfpathlineto{\pgfqpoint{1.038068in}{0.407000in}}%
\pgfpathclose%
\pgfusepath{fill}%
\end{pgfscope}%
\begin{pgfscope}%
\pgfpathrectangle{\pgfqpoint{0.562500in}{0.407000in}}{\pgfqpoint{3.487500in}{2.849000in}}%
\pgfusepath{clip}%
\pgfsetbuttcap%
\pgfsetmiterjoin%
\definecolor{currentfill}{rgb}{0.121569,0.466667,0.705882}%
\pgfsetfillcolor{currentfill}%
\pgfsetlinewidth{0.000000pt}%
\definecolor{currentstroke}{rgb}{0.000000,0.000000,0.000000}%
\pgfsetstrokecolor{currentstroke}%
\pgfsetstrokeopacity{0.000000}%
\pgfsetdash{}{0pt}%
\pgfpathmoveto{\pgfqpoint{1.101477in}{0.407000in}}%
\pgfpathlineto{\pgfqpoint{1.164886in}{0.407000in}}%
\pgfpathlineto{\pgfqpoint{1.164886in}{0.824436in}}%
\pgfpathlineto{\pgfqpoint{1.101477in}{0.824436in}}%
\pgfpathlineto{\pgfqpoint{1.101477in}{0.407000in}}%
\pgfpathclose%
\pgfusepath{fill}%
\end{pgfscope}%
\begin{pgfscope}%
\pgfpathrectangle{\pgfqpoint{0.562500in}{0.407000in}}{\pgfqpoint{3.487500in}{2.849000in}}%
\pgfusepath{clip}%
\pgfsetbuttcap%
\pgfsetmiterjoin%
\definecolor{currentfill}{rgb}{0.121569,0.466667,0.705882}%
\pgfsetfillcolor{currentfill}%
\pgfsetlinewidth{0.000000pt}%
\definecolor{currentstroke}{rgb}{0.000000,0.000000,0.000000}%
\pgfsetstrokecolor{currentstroke}%
\pgfsetstrokeopacity{0.000000}%
\pgfsetdash{}{0pt}%
\pgfpathmoveto{\pgfqpoint{1.164886in}{0.407000in}}%
\pgfpathlineto{\pgfqpoint{1.228295in}{0.407000in}}%
\pgfpathlineto{\pgfqpoint{1.228295in}{0.407000in}}%
\pgfpathlineto{\pgfqpoint{1.164886in}{0.407000in}}%
\pgfpathlineto{\pgfqpoint{1.164886in}{0.407000in}}%
\pgfpathclose%
\pgfusepath{fill}%
\end{pgfscope}%
\begin{pgfscope}%
\pgfpathrectangle{\pgfqpoint{0.562500in}{0.407000in}}{\pgfqpoint{3.487500in}{2.849000in}}%
\pgfusepath{clip}%
\pgfsetbuttcap%
\pgfsetmiterjoin%
\definecolor{currentfill}{rgb}{0.121569,0.466667,0.705882}%
\pgfsetfillcolor{currentfill}%
\pgfsetlinewidth{0.000000pt}%
\definecolor{currentstroke}{rgb}{0.000000,0.000000,0.000000}%
\pgfsetstrokecolor{currentstroke}%
\pgfsetstrokeopacity{0.000000}%
\pgfsetdash{}{0pt}%
\pgfpathmoveto{\pgfqpoint{1.228295in}{0.407000in}}%
\pgfpathlineto{\pgfqpoint{1.291705in}{0.407000in}}%
\pgfpathlineto{\pgfqpoint{1.291705in}{0.615718in}}%
\pgfpathlineto{\pgfqpoint{1.228295in}{0.615718in}}%
\pgfpathlineto{\pgfqpoint{1.228295in}{0.407000in}}%
\pgfpathclose%
\pgfusepath{fill}%
\end{pgfscope}%
\begin{pgfscope}%
\pgfpathrectangle{\pgfqpoint{0.562500in}{0.407000in}}{\pgfqpoint{3.487500in}{2.849000in}}%
\pgfusepath{clip}%
\pgfsetbuttcap%
\pgfsetmiterjoin%
\definecolor{currentfill}{rgb}{0.121569,0.466667,0.705882}%
\pgfsetfillcolor{currentfill}%
\pgfsetlinewidth{0.000000pt}%
\definecolor{currentstroke}{rgb}{0.000000,0.000000,0.000000}%
\pgfsetstrokecolor{currentstroke}%
\pgfsetstrokeopacity{0.000000}%
\pgfsetdash{}{0pt}%
\pgfpathmoveto{\pgfqpoint{1.291705in}{0.407000in}}%
\pgfpathlineto{\pgfqpoint{1.355114in}{0.407000in}}%
\pgfpathlineto{\pgfqpoint{1.355114in}{1.033154in}}%
\pgfpathlineto{\pgfqpoint{1.291705in}{1.033154in}}%
\pgfpathlineto{\pgfqpoint{1.291705in}{0.407000in}}%
\pgfpathclose%
\pgfusepath{fill}%
\end{pgfscope}%
\begin{pgfscope}%
\pgfpathrectangle{\pgfqpoint{0.562500in}{0.407000in}}{\pgfqpoint{3.487500in}{2.849000in}}%
\pgfusepath{clip}%
\pgfsetbuttcap%
\pgfsetmiterjoin%
\definecolor{currentfill}{rgb}{0.121569,0.466667,0.705882}%
\pgfsetfillcolor{currentfill}%
\pgfsetlinewidth{0.000000pt}%
\definecolor{currentstroke}{rgb}{0.000000,0.000000,0.000000}%
\pgfsetstrokecolor{currentstroke}%
\pgfsetstrokeopacity{0.000000}%
\pgfsetdash{}{0pt}%
\pgfpathmoveto{\pgfqpoint{1.355114in}{0.407000in}}%
\pgfpathlineto{\pgfqpoint{1.418523in}{0.407000in}}%
\pgfpathlineto{\pgfqpoint{1.418523in}{0.615718in}}%
\pgfpathlineto{\pgfqpoint{1.355114in}{0.615718in}}%
\pgfpathlineto{\pgfqpoint{1.355114in}{0.407000in}}%
\pgfpathclose%
\pgfusepath{fill}%
\end{pgfscope}%
\begin{pgfscope}%
\pgfpathrectangle{\pgfqpoint{0.562500in}{0.407000in}}{\pgfqpoint{3.487500in}{2.849000in}}%
\pgfusepath{clip}%
\pgfsetbuttcap%
\pgfsetmiterjoin%
\definecolor{currentfill}{rgb}{0.121569,0.466667,0.705882}%
\pgfsetfillcolor{currentfill}%
\pgfsetlinewidth{0.000000pt}%
\definecolor{currentstroke}{rgb}{0.000000,0.000000,0.000000}%
\pgfsetstrokecolor{currentstroke}%
\pgfsetstrokeopacity{0.000000}%
\pgfsetdash{}{0pt}%
\pgfpathmoveto{\pgfqpoint{1.418523in}{0.407000in}}%
\pgfpathlineto{\pgfqpoint{1.481932in}{0.407000in}}%
\pgfpathlineto{\pgfqpoint{1.481932in}{0.928795in}}%
\pgfpathlineto{\pgfqpoint{1.418523in}{0.928795in}}%
\pgfpathlineto{\pgfqpoint{1.418523in}{0.407000in}}%
\pgfpathclose%
\pgfusepath{fill}%
\end{pgfscope}%
\begin{pgfscope}%
\pgfpathrectangle{\pgfqpoint{0.562500in}{0.407000in}}{\pgfqpoint{3.487500in}{2.849000in}}%
\pgfusepath{clip}%
\pgfsetbuttcap%
\pgfsetmiterjoin%
\definecolor{currentfill}{rgb}{0.121569,0.466667,0.705882}%
\pgfsetfillcolor{currentfill}%
\pgfsetlinewidth{0.000000pt}%
\definecolor{currentstroke}{rgb}{0.000000,0.000000,0.000000}%
\pgfsetstrokecolor{currentstroke}%
\pgfsetstrokeopacity{0.000000}%
\pgfsetdash{}{0pt}%
\pgfpathmoveto{\pgfqpoint{1.481932in}{0.407000in}}%
\pgfpathlineto{\pgfqpoint{1.545341in}{0.407000in}}%
\pgfpathlineto{\pgfqpoint{1.545341in}{0.824436in}}%
\pgfpathlineto{\pgfqpoint{1.481932in}{0.824436in}}%
\pgfpathlineto{\pgfqpoint{1.481932in}{0.407000in}}%
\pgfpathclose%
\pgfusepath{fill}%
\end{pgfscope}%
\begin{pgfscope}%
\pgfpathrectangle{\pgfqpoint{0.562500in}{0.407000in}}{\pgfqpoint{3.487500in}{2.849000in}}%
\pgfusepath{clip}%
\pgfsetbuttcap%
\pgfsetmiterjoin%
\definecolor{currentfill}{rgb}{0.121569,0.466667,0.705882}%
\pgfsetfillcolor{currentfill}%
\pgfsetlinewidth{0.000000pt}%
\definecolor{currentstroke}{rgb}{0.000000,0.000000,0.000000}%
\pgfsetstrokecolor{currentstroke}%
\pgfsetstrokeopacity{0.000000}%
\pgfsetdash{}{0pt}%
\pgfpathmoveto{\pgfqpoint{1.545341in}{0.407000in}}%
\pgfpathlineto{\pgfqpoint{1.608750in}{0.407000in}}%
\pgfpathlineto{\pgfqpoint{1.608750in}{1.241872in}}%
\pgfpathlineto{\pgfqpoint{1.545341in}{1.241872in}}%
\pgfpathlineto{\pgfqpoint{1.545341in}{0.407000in}}%
\pgfpathclose%
\pgfusepath{fill}%
\end{pgfscope}%
\begin{pgfscope}%
\pgfpathrectangle{\pgfqpoint{0.562500in}{0.407000in}}{\pgfqpoint{3.487500in}{2.849000in}}%
\pgfusepath{clip}%
\pgfsetbuttcap%
\pgfsetmiterjoin%
\definecolor{currentfill}{rgb}{0.121569,0.466667,0.705882}%
\pgfsetfillcolor{currentfill}%
\pgfsetlinewidth{0.000000pt}%
\definecolor{currentstroke}{rgb}{0.000000,0.000000,0.000000}%
\pgfsetstrokecolor{currentstroke}%
\pgfsetstrokeopacity{0.000000}%
\pgfsetdash{}{0pt}%
\pgfpathmoveto{\pgfqpoint{1.608750in}{0.407000in}}%
\pgfpathlineto{\pgfqpoint{1.672159in}{0.407000in}}%
\pgfpathlineto{\pgfqpoint{1.672159in}{1.033154in}}%
\pgfpathlineto{\pgfqpoint{1.608750in}{1.033154in}}%
\pgfpathlineto{\pgfqpoint{1.608750in}{0.407000in}}%
\pgfpathclose%
\pgfusepath{fill}%
\end{pgfscope}%
\begin{pgfscope}%
\pgfpathrectangle{\pgfqpoint{0.562500in}{0.407000in}}{\pgfqpoint{3.487500in}{2.849000in}}%
\pgfusepath{clip}%
\pgfsetbuttcap%
\pgfsetmiterjoin%
\definecolor{currentfill}{rgb}{0.121569,0.466667,0.705882}%
\pgfsetfillcolor{currentfill}%
\pgfsetlinewidth{0.000000pt}%
\definecolor{currentstroke}{rgb}{0.000000,0.000000,0.000000}%
\pgfsetstrokecolor{currentstroke}%
\pgfsetstrokeopacity{0.000000}%
\pgfsetdash{}{0pt}%
\pgfpathmoveto{\pgfqpoint{1.672159in}{0.407000in}}%
\pgfpathlineto{\pgfqpoint{1.735568in}{0.407000in}}%
\pgfpathlineto{\pgfqpoint{1.735568in}{1.554949in}}%
\pgfpathlineto{\pgfqpoint{1.672159in}{1.554949in}}%
\pgfpathlineto{\pgfqpoint{1.672159in}{0.407000in}}%
\pgfpathclose%
\pgfusepath{fill}%
\end{pgfscope}%
\begin{pgfscope}%
\pgfpathrectangle{\pgfqpoint{0.562500in}{0.407000in}}{\pgfqpoint{3.487500in}{2.849000in}}%
\pgfusepath{clip}%
\pgfsetbuttcap%
\pgfsetmiterjoin%
\definecolor{currentfill}{rgb}{0.121569,0.466667,0.705882}%
\pgfsetfillcolor{currentfill}%
\pgfsetlinewidth{0.000000pt}%
\definecolor{currentstroke}{rgb}{0.000000,0.000000,0.000000}%
\pgfsetstrokecolor{currentstroke}%
\pgfsetstrokeopacity{0.000000}%
\pgfsetdash{}{0pt}%
\pgfpathmoveto{\pgfqpoint{1.735568in}{0.407000in}}%
\pgfpathlineto{\pgfqpoint{1.798977in}{0.407000in}}%
\pgfpathlineto{\pgfqpoint{1.798977in}{1.554949in}}%
\pgfpathlineto{\pgfqpoint{1.735568in}{1.554949in}}%
\pgfpathlineto{\pgfqpoint{1.735568in}{0.407000in}}%
\pgfpathclose%
\pgfusepath{fill}%
\end{pgfscope}%
\begin{pgfscope}%
\pgfpathrectangle{\pgfqpoint{0.562500in}{0.407000in}}{\pgfqpoint{3.487500in}{2.849000in}}%
\pgfusepath{clip}%
\pgfsetbuttcap%
\pgfsetmiterjoin%
\definecolor{currentfill}{rgb}{0.121569,0.466667,0.705882}%
\pgfsetfillcolor{currentfill}%
\pgfsetlinewidth{0.000000pt}%
\definecolor{currentstroke}{rgb}{0.000000,0.000000,0.000000}%
\pgfsetstrokecolor{currentstroke}%
\pgfsetstrokeopacity{0.000000}%
\pgfsetdash{}{0pt}%
\pgfpathmoveto{\pgfqpoint{1.798977in}{0.407000in}}%
\pgfpathlineto{\pgfqpoint{1.862386in}{0.407000in}}%
\pgfpathlineto{\pgfqpoint{1.862386in}{2.389821in}}%
\pgfpathlineto{\pgfqpoint{1.798977in}{2.389821in}}%
\pgfpathlineto{\pgfqpoint{1.798977in}{0.407000in}}%
\pgfpathclose%
\pgfusepath{fill}%
\end{pgfscope}%
\begin{pgfscope}%
\pgfpathrectangle{\pgfqpoint{0.562500in}{0.407000in}}{\pgfqpoint{3.487500in}{2.849000in}}%
\pgfusepath{clip}%
\pgfsetbuttcap%
\pgfsetmiterjoin%
\definecolor{currentfill}{rgb}{0.121569,0.466667,0.705882}%
\pgfsetfillcolor{currentfill}%
\pgfsetlinewidth{0.000000pt}%
\definecolor{currentstroke}{rgb}{0.000000,0.000000,0.000000}%
\pgfsetstrokecolor{currentstroke}%
\pgfsetstrokeopacity{0.000000}%
\pgfsetdash{}{0pt}%
\pgfpathmoveto{\pgfqpoint{1.862386in}{0.407000in}}%
\pgfpathlineto{\pgfqpoint{1.925795in}{0.407000in}}%
\pgfpathlineto{\pgfqpoint{1.925795in}{1.137513in}}%
\pgfpathlineto{\pgfqpoint{1.862386in}{1.137513in}}%
\pgfpathlineto{\pgfqpoint{1.862386in}{0.407000in}}%
\pgfpathclose%
\pgfusepath{fill}%
\end{pgfscope}%
\begin{pgfscope}%
\pgfpathrectangle{\pgfqpoint{0.562500in}{0.407000in}}{\pgfqpoint{3.487500in}{2.849000in}}%
\pgfusepath{clip}%
\pgfsetbuttcap%
\pgfsetmiterjoin%
\definecolor{currentfill}{rgb}{0.121569,0.466667,0.705882}%
\pgfsetfillcolor{currentfill}%
\pgfsetlinewidth{0.000000pt}%
\definecolor{currentstroke}{rgb}{0.000000,0.000000,0.000000}%
\pgfsetstrokecolor{currentstroke}%
\pgfsetstrokeopacity{0.000000}%
\pgfsetdash{}{0pt}%
\pgfpathmoveto{\pgfqpoint{1.925795in}{0.407000in}}%
\pgfpathlineto{\pgfqpoint{1.989205in}{0.407000in}}%
\pgfpathlineto{\pgfqpoint{1.989205in}{2.076744in}}%
\pgfpathlineto{\pgfqpoint{1.925795in}{2.076744in}}%
\pgfpathlineto{\pgfqpoint{1.925795in}{0.407000in}}%
\pgfpathclose%
\pgfusepath{fill}%
\end{pgfscope}%
\begin{pgfscope}%
\pgfpathrectangle{\pgfqpoint{0.562500in}{0.407000in}}{\pgfqpoint{3.487500in}{2.849000in}}%
\pgfusepath{clip}%
\pgfsetbuttcap%
\pgfsetmiterjoin%
\definecolor{currentfill}{rgb}{0.121569,0.466667,0.705882}%
\pgfsetfillcolor{currentfill}%
\pgfsetlinewidth{0.000000pt}%
\definecolor{currentstroke}{rgb}{0.000000,0.000000,0.000000}%
\pgfsetstrokecolor{currentstroke}%
\pgfsetstrokeopacity{0.000000}%
\pgfsetdash{}{0pt}%
\pgfpathmoveto{\pgfqpoint{1.989205in}{0.407000in}}%
\pgfpathlineto{\pgfqpoint{2.052614in}{0.407000in}}%
\pgfpathlineto{\pgfqpoint{2.052614in}{1.972385in}}%
\pgfpathlineto{\pgfqpoint{1.989205in}{1.972385in}}%
\pgfpathlineto{\pgfqpoint{1.989205in}{0.407000in}}%
\pgfpathclose%
\pgfusepath{fill}%
\end{pgfscope}%
\begin{pgfscope}%
\pgfpathrectangle{\pgfqpoint{0.562500in}{0.407000in}}{\pgfqpoint{3.487500in}{2.849000in}}%
\pgfusepath{clip}%
\pgfsetbuttcap%
\pgfsetmiterjoin%
\definecolor{currentfill}{rgb}{0.121569,0.466667,0.705882}%
\pgfsetfillcolor{currentfill}%
\pgfsetlinewidth{0.000000pt}%
\definecolor{currentstroke}{rgb}{0.000000,0.000000,0.000000}%
\pgfsetstrokecolor{currentstroke}%
\pgfsetstrokeopacity{0.000000}%
\pgfsetdash{}{0pt}%
\pgfpathmoveto{\pgfqpoint{2.052614in}{0.407000in}}%
\pgfpathlineto{\pgfqpoint{2.116023in}{0.407000in}}%
\pgfpathlineto{\pgfqpoint{2.116023in}{1.137513in}}%
\pgfpathlineto{\pgfqpoint{2.052614in}{1.137513in}}%
\pgfpathlineto{\pgfqpoint{2.052614in}{0.407000in}}%
\pgfpathclose%
\pgfusepath{fill}%
\end{pgfscope}%
\begin{pgfscope}%
\pgfpathrectangle{\pgfqpoint{0.562500in}{0.407000in}}{\pgfqpoint{3.487500in}{2.849000in}}%
\pgfusepath{clip}%
\pgfsetbuttcap%
\pgfsetmiterjoin%
\definecolor{currentfill}{rgb}{0.121569,0.466667,0.705882}%
\pgfsetfillcolor{currentfill}%
\pgfsetlinewidth{0.000000pt}%
\definecolor{currentstroke}{rgb}{0.000000,0.000000,0.000000}%
\pgfsetstrokecolor{currentstroke}%
\pgfsetstrokeopacity{0.000000}%
\pgfsetdash{}{0pt}%
\pgfpathmoveto{\pgfqpoint{2.116023in}{0.407000in}}%
\pgfpathlineto{\pgfqpoint{2.179432in}{0.407000in}}%
\pgfpathlineto{\pgfqpoint{2.179432in}{2.807256in}}%
\pgfpathlineto{\pgfqpoint{2.116023in}{2.807256in}}%
\pgfpathlineto{\pgfqpoint{2.116023in}{0.407000in}}%
\pgfpathclose%
\pgfusepath{fill}%
\end{pgfscope}%
\begin{pgfscope}%
\pgfpathrectangle{\pgfqpoint{0.562500in}{0.407000in}}{\pgfqpoint{3.487500in}{2.849000in}}%
\pgfusepath{clip}%
\pgfsetbuttcap%
\pgfsetmiterjoin%
\definecolor{currentfill}{rgb}{0.121569,0.466667,0.705882}%
\pgfsetfillcolor{currentfill}%
\pgfsetlinewidth{0.000000pt}%
\definecolor{currentstroke}{rgb}{0.000000,0.000000,0.000000}%
\pgfsetstrokecolor{currentstroke}%
\pgfsetstrokeopacity{0.000000}%
\pgfsetdash{}{0pt}%
\pgfpathmoveto{\pgfqpoint{2.179432in}{0.407000in}}%
\pgfpathlineto{\pgfqpoint{2.242841in}{0.407000in}}%
\pgfpathlineto{\pgfqpoint{2.242841in}{3.120333in}}%
\pgfpathlineto{\pgfqpoint{2.179432in}{3.120333in}}%
\pgfpathlineto{\pgfqpoint{2.179432in}{0.407000in}}%
\pgfpathclose%
\pgfusepath{fill}%
\end{pgfscope}%
\begin{pgfscope}%
\pgfpathrectangle{\pgfqpoint{0.562500in}{0.407000in}}{\pgfqpoint{3.487500in}{2.849000in}}%
\pgfusepath{clip}%
\pgfsetbuttcap%
\pgfsetmiterjoin%
\definecolor{currentfill}{rgb}{0.121569,0.466667,0.705882}%
\pgfsetfillcolor{currentfill}%
\pgfsetlinewidth{0.000000pt}%
\definecolor{currentstroke}{rgb}{0.000000,0.000000,0.000000}%
\pgfsetstrokecolor{currentstroke}%
\pgfsetstrokeopacity{0.000000}%
\pgfsetdash{}{0pt}%
\pgfpathmoveto{\pgfqpoint{2.242841in}{0.407000in}}%
\pgfpathlineto{\pgfqpoint{2.306250in}{0.407000in}}%
\pgfpathlineto{\pgfqpoint{2.306250in}{1.868026in}}%
\pgfpathlineto{\pgfqpoint{2.242841in}{1.868026in}}%
\pgfpathlineto{\pgfqpoint{2.242841in}{0.407000in}}%
\pgfpathclose%
\pgfusepath{fill}%
\end{pgfscope}%
\begin{pgfscope}%
\pgfpathrectangle{\pgfqpoint{0.562500in}{0.407000in}}{\pgfqpoint{3.487500in}{2.849000in}}%
\pgfusepath{clip}%
\pgfsetbuttcap%
\pgfsetmiterjoin%
\definecolor{currentfill}{rgb}{0.121569,0.466667,0.705882}%
\pgfsetfillcolor{currentfill}%
\pgfsetlinewidth{0.000000pt}%
\definecolor{currentstroke}{rgb}{0.000000,0.000000,0.000000}%
\pgfsetstrokecolor{currentstroke}%
\pgfsetstrokeopacity{0.000000}%
\pgfsetdash{}{0pt}%
\pgfpathmoveto{\pgfqpoint{2.306250in}{0.407000in}}%
\pgfpathlineto{\pgfqpoint{2.369659in}{0.407000in}}%
\pgfpathlineto{\pgfqpoint{2.369659in}{2.494179in}}%
\pgfpathlineto{\pgfqpoint{2.306250in}{2.494179in}}%
\pgfpathlineto{\pgfqpoint{2.306250in}{0.407000in}}%
\pgfpathclose%
\pgfusepath{fill}%
\end{pgfscope}%
\begin{pgfscope}%
\pgfpathrectangle{\pgfqpoint{0.562500in}{0.407000in}}{\pgfqpoint{3.487500in}{2.849000in}}%
\pgfusepath{clip}%
\pgfsetbuttcap%
\pgfsetmiterjoin%
\definecolor{currentfill}{rgb}{0.121569,0.466667,0.705882}%
\pgfsetfillcolor{currentfill}%
\pgfsetlinewidth{0.000000pt}%
\definecolor{currentstroke}{rgb}{0.000000,0.000000,0.000000}%
\pgfsetstrokecolor{currentstroke}%
\pgfsetstrokeopacity{0.000000}%
\pgfsetdash{}{0pt}%
\pgfpathmoveto{\pgfqpoint{2.369659in}{0.407000in}}%
\pgfpathlineto{\pgfqpoint{2.433068in}{0.407000in}}%
\pgfpathlineto{\pgfqpoint{2.433068in}{1.450590in}}%
\pgfpathlineto{\pgfqpoint{2.369659in}{1.450590in}}%
\pgfpathlineto{\pgfqpoint{2.369659in}{0.407000in}}%
\pgfpathclose%
\pgfusepath{fill}%
\end{pgfscope}%
\begin{pgfscope}%
\pgfpathrectangle{\pgfqpoint{0.562500in}{0.407000in}}{\pgfqpoint{3.487500in}{2.849000in}}%
\pgfusepath{clip}%
\pgfsetbuttcap%
\pgfsetmiterjoin%
\definecolor{currentfill}{rgb}{0.121569,0.466667,0.705882}%
\pgfsetfillcolor{currentfill}%
\pgfsetlinewidth{0.000000pt}%
\definecolor{currentstroke}{rgb}{0.000000,0.000000,0.000000}%
\pgfsetstrokecolor{currentstroke}%
\pgfsetstrokeopacity{0.000000}%
\pgfsetdash{}{0pt}%
\pgfpathmoveto{\pgfqpoint{2.433068in}{0.407000in}}%
\pgfpathlineto{\pgfqpoint{2.496477in}{0.407000in}}%
\pgfpathlineto{\pgfqpoint{2.496477in}{1.450590in}}%
\pgfpathlineto{\pgfqpoint{2.433068in}{1.450590in}}%
\pgfpathlineto{\pgfqpoint{2.433068in}{0.407000in}}%
\pgfpathclose%
\pgfusepath{fill}%
\end{pgfscope}%
\begin{pgfscope}%
\pgfpathrectangle{\pgfqpoint{0.562500in}{0.407000in}}{\pgfqpoint{3.487500in}{2.849000in}}%
\pgfusepath{clip}%
\pgfsetbuttcap%
\pgfsetmiterjoin%
\definecolor{currentfill}{rgb}{0.121569,0.466667,0.705882}%
\pgfsetfillcolor{currentfill}%
\pgfsetlinewidth{0.000000pt}%
\definecolor{currentstroke}{rgb}{0.000000,0.000000,0.000000}%
\pgfsetstrokecolor{currentstroke}%
\pgfsetstrokeopacity{0.000000}%
\pgfsetdash{}{0pt}%
\pgfpathmoveto{\pgfqpoint{2.496477in}{0.407000in}}%
\pgfpathlineto{\pgfqpoint{2.559886in}{0.407000in}}%
\pgfpathlineto{\pgfqpoint{2.559886in}{2.076744in}}%
\pgfpathlineto{\pgfqpoint{2.496477in}{2.076744in}}%
\pgfpathlineto{\pgfqpoint{2.496477in}{0.407000in}}%
\pgfpathclose%
\pgfusepath{fill}%
\end{pgfscope}%
\begin{pgfscope}%
\pgfpathrectangle{\pgfqpoint{0.562500in}{0.407000in}}{\pgfqpoint{3.487500in}{2.849000in}}%
\pgfusepath{clip}%
\pgfsetbuttcap%
\pgfsetmiterjoin%
\definecolor{currentfill}{rgb}{0.121569,0.466667,0.705882}%
\pgfsetfillcolor{currentfill}%
\pgfsetlinewidth{0.000000pt}%
\definecolor{currentstroke}{rgb}{0.000000,0.000000,0.000000}%
\pgfsetstrokecolor{currentstroke}%
\pgfsetstrokeopacity{0.000000}%
\pgfsetdash{}{0pt}%
\pgfpathmoveto{\pgfqpoint{2.559886in}{0.407000in}}%
\pgfpathlineto{\pgfqpoint{2.623295in}{0.407000in}}%
\pgfpathlineto{\pgfqpoint{2.623295in}{2.702897in}}%
\pgfpathlineto{\pgfqpoint{2.559886in}{2.702897in}}%
\pgfpathlineto{\pgfqpoint{2.559886in}{0.407000in}}%
\pgfpathclose%
\pgfusepath{fill}%
\end{pgfscope}%
\begin{pgfscope}%
\pgfpathrectangle{\pgfqpoint{0.562500in}{0.407000in}}{\pgfqpoint{3.487500in}{2.849000in}}%
\pgfusepath{clip}%
\pgfsetbuttcap%
\pgfsetmiterjoin%
\definecolor{currentfill}{rgb}{0.121569,0.466667,0.705882}%
\pgfsetfillcolor{currentfill}%
\pgfsetlinewidth{0.000000pt}%
\definecolor{currentstroke}{rgb}{0.000000,0.000000,0.000000}%
\pgfsetstrokecolor{currentstroke}%
\pgfsetstrokeopacity{0.000000}%
\pgfsetdash{}{0pt}%
\pgfpathmoveto{\pgfqpoint{2.623295in}{0.407000in}}%
\pgfpathlineto{\pgfqpoint{2.686705in}{0.407000in}}%
\pgfpathlineto{\pgfqpoint{2.686705in}{1.868026in}}%
\pgfpathlineto{\pgfqpoint{2.623295in}{1.868026in}}%
\pgfpathlineto{\pgfqpoint{2.623295in}{0.407000in}}%
\pgfpathclose%
\pgfusepath{fill}%
\end{pgfscope}%
\begin{pgfscope}%
\pgfpathrectangle{\pgfqpoint{0.562500in}{0.407000in}}{\pgfqpoint{3.487500in}{2.849000in}}%
\pgfusepath{clip}%
\pgfsetbuttcap%
\pgfsetmiterjoin%
\definecolor{currentfill}{rgb}{0.121569,0.466667,0.705882}%
\pgfsetfillcolor{currentfill}%
\pgfsetlinewidth{0.000000pt}%
\definecolor{currentstroke}{rgb}{0.000000,0.000000,0.000000}%
\pgfsetstrokecolor{currentstroke}%
\pgfsetstrokeopacity{0.000000}%
\pgfsetdash{}{0pt}%
\pgfpathmoveto{\pgfqpoint{2.686705in}{0.407000in}}%
\pgfpathlineto{\pgfqpoint{2.750114in}{0.407000in}}%
\pgfpathlineto{\pgfqpoint{2.750114in}{1.346231in}}%
\pgfpathlineto{\pgfqpoint{2.686705in}{1.346231in}}%
\pgfpathlineto{\pgfqpoint{2.686705in}{0.407000in}}%
\pgfpathclose%
\pgfusepath{fill}%
\end{pgfscope}%
\begin{pgfscope}%
\pgfpathrectangle{\pgfqpoint{0.562500in}{0.407000in}}{\pgfqpoint{3.487500in}{2.849000in}}%
\pgfusepath{clip}%
\pgfsetbuttcap%
\pgfsetmiterjoin%
\definecolor{currentfill}{rgb}{0.121569,0.466667,0.705882}%
\pgfsetfillcolor{currentfill}%
\pgfsetlinewidth{0.000000pt}%
\definecolor{currentstroke}{rgb}{0.000000,0.000000,0.000000}%
\pgfsetstrokecolor{currentstroke}%
\pgfsetstrokeopacity{0.000000}%
\pgfsetdash{}{0pt}%
\pgfpathmoveto{\pgfqpoint{2.750114in}{0.407000in}}%
\pgfpathlineto{\pgfqpoint{2.813523in}{0.407000in}}%
\pgfpathlineto{\pgfqpoint{2.813523in}{2.702897in}}%
\pgfpathlineto{\pgfqpoint{2.750114in}{2.702897in}}%
\pgfpathlineto{\pgfqpoint{2.750114in}{0.407000in}}%
\pgfpathclose%
\pgfusepath{fill}%
\end{pgfscope}%
\begin{pgfscope}%
\pgfpathrectangle{\pgfqpoint{0.562500in}{0.407000in}}{\pgfqpoint{3.487500in}{2.849000in}}%
\pgfusepath{clip}%
\pgfsetbuttcap%
\pgfsetmiterjoin%
\definecolor{currentfill}{rgb}{0.121569,0.466667,0.705882}%
\pgfsetfillcolor{currentfill}%
\pgfsetlinewidth{0.000000pt}%
\definecolor{currentstroke}{rgb}{0.000000,0.000000,0.000000}%
\pgfsetstrokecolor{currentstroke}%
\pgfsetstrokeopacity{0.000000}%
\pgfsetdash{}{0pt}%
\pgfpathmoveto{\pgfqpoint{2.813523in}{0.407000in}}%
\pgfpathlineto{\pgfqpoint{2.876932in}{0.407000in}}%
\pgfpathlineto{\pgfqpoint{2.876932in}{1.450590in}}%
\pgfpathlineto{\pgfqpoint{2.813523in}{1.450590in}}%
\pgfpathlineto{\pgfqpoint{2.813523in}{0.407000in}}%
\pgfpathclose%
\pgfusepath{fill}%
\end{pgfscope}%
\begin{pgfscope}%
\pgfpathrectangle{\pgfqpoint{0.562500in}{0.407000in}}{\pgfqpoint{3.487500in}{2.849000in}}%
\pgfusepath{clip}%
\pgfsetbuttcap%
\pgfsetmiterjoin%
\definecolor{currentfill}{rgb}{0.121569,0.466667,0.705882}%
\pgfsetfillcolor{currentfill}%
\pgfsetlinewidth{0.000000pt}%
\definecolor{currentstroke}{rgb}{0.000000,0.000000,0.000000}%
\pgfsetstrokecolor{currentstroke}%
\pgfsetstrokeopacity{0.000000}%
\pgfsetdash{}{0pt}%
\pgfpathmoveto{\pgfqpoint{2.876932in}{0.407000in}}%
\pgfpathlineto{\pgfqpoint{2.940341in}{0.407000in}}%
\pgfpathlineto{\pgfqpoint{2.940341in}{1.868026in}}%
\pgfpathlineto{\pgfqpoint{2.876932in}{1.868026in}}%
\pgfpathlineto{\pgfqpoint{2.876932in}{0.407000in}}%
\pgfpathclose%
\pgfusepath{fill}%
\end{pgfscope}%
\begin{pgfscope}%
\pgfpathrectangle{\pgfqpoint{0.562500in}{0.407000in}}{\pgfqpoint{3.487500in}{2.849000in}}%
\pgfusepath{clip}%
\pgfsetbuttcap%
\pgfsetmiterjoin%
\definecolor{currentfill}{rgb}{0.121569,0.466667,0.705882}%
\pgfsetfillcolor{currentfill}%
\pgfsetlinewidth{0.000000pt}%
\definecolor{currentstroke}{rgb}{0.000000,0.000000,0.000000}%
\pgfsetstrokecolor{currentstroke}%
\pgfsetstrokeopacity{0.000000}%
\pgfsetdash{}{0pt}%
\pgfpathmoveto{\pgfqpoint{2.940341in}{0.407000in}}%
\pgfpathlineto{\pgfqpoint{3.003750in}{0.407000in}}%
\pgfpathlineto{\pgfqpoint{3.003750in}{1.763667in}}%
\pgfpathlineto{\pgfqpoint{2.940341in}{1.763667in}}%
\pgfpathlineto{\pgfqpoint{2.940341in}{0.407000in}}%
\pgfpathclose%
\pgfusepath{fill}%
\end{pgfscope}%
\begin{pgfscope}%
\pgfpathrectangle{\pgfqpoint{0.562500in}{0.407000in}}{\pgfqpoint{3.487500in}{2.849000in}}%
\pgfusepath{clip}%
\pgfsetbuttcap%
\pgfsetmiterjoin%
\definecolor{currentfill}{rgb}{0.121569,0.466667,0.705882}%
\pgfsetfillcolor{currentfill}%
\pgfsetlinewidth{0.000000pt}%
\definecolor{currentstroke}{rgb}{0.000000,0.000000,0.000000}%
\pgfsetstrokecolor{currentstroke}%
\pgfsetstrokeopacity{0.000000}%
\pgfsetdash{}{0pt}%
\pgfpathmoveto{\pgfqpoint{3.003750in}{0.407000in}}%
\pgfpathlineto{\pgfqpoint{3.067159in}{0.407000in}}%
\pgfpathlineto{\pgfqpoint{3.067159in}{1.659308in}}%
\pgfpathlineto{\pgfqpoint{3.003750in}{1.659308in}}%
\pgfpathlineto{\pgfqpoint{3.003750in}{0.407000in}}%
\pgfpathclose%
\pgfusepath{fill}%
\end{pgfscope}%
\begin{pgfscope}%
\pgfpathrectangle{\pgfqpoint{0.562500in}{0.407000in}}{\pgfqpoint{3.487500in}{2.849000in}}%
\pgfusepath{clip}%
\pgfsetbuttcap%
\pgfsetmiterjoin%
\definecolor{currentfill}{rgb}{0.121569,0.466667,0.705882}%
\pgfsetfillcolor{currentfill}%
\pgfsetlinewidth{0.000000pt}%
\definecolor{currentstroke}{rgb}{0.000000,0.000000,0.000000}%
\pgfsetstrokecolor{currentstroke}%
\pgfsetstrokeopacity{0.000000}%
\pgfsetdash{}{0pt}%
\pgfpathmoveto{\pgfqpoint{3.067159in}{0.407000in}}%
\pgfpathlineto{\pgfqpoint{3.130568in}{0.407000in}}%
\pgfpathlineto{\pgfqpoint{3.130568in}{0.824436in}}%
\pgfpathlineto{\pgfqpoint{3.067159in}{0.824436in}}%
\pgfpathlineto{\pgfqpoint{3.067159in}{0.407000in}}%
\pgfpathclose%
\pgfusepath{fill}%
\end{pgfscope}%
\begin{pgfscope}%
\pgfpathrectangle{\pgfqpoint{0.562500in}{0.407000in}}{\pgfqpoint{3.487500in}{2.849000in}}%
\pgfusepath{clip}%
\pgfsetbuttcap%
\pgfsetmiterjoin%
\definecolor{currentfill}{rgb}{0.121569,0.466667,0.705882}%
\pgfsetfillcolor{currentfill}%
\pgfsetlinewidth{0.000000pt}%
\definecolor{currentstroke}{rgb}{0.000000,0.000000,0.000000}%
\pgfsetstrokecolor{currentstroke}%
\pgfsetstrokeopacity{0.000000}%
\pgfsetdash{}{0pt}%
\pgfpathmoveto{\pgfqpoint{3.130568in}{0.407000in}}%
\pgfpathlineto{\pgfqpoint{3.193977in}{0.407000in}}%
\pgfpathlineto{\pgfqpoint{3.193977in}{2.076744in}}%
\pgfpathlineto{\pgfqpoint{3.130568in}{2.076744in}}%
\pgfpathlineto{\pgfqpoint{3.130568in}{0.407000in}}%
\pgfpathclose%
\pgfusepath{fill}%
\end{pgfscope}%
\begin{pgfscope}%
\pgfpathrectangle{\pgfqpoint{0.562500in}{0.407000in}}{\pgfqpoint{3.487500in}{2.849000in}}%
\pgfusepath{clip}%
\pgfsetbuttcap%
\pgfsetmiterjoin%
\definecolor{currentfill}{rgb}{0.121569,0.466667,0.705882}%
\pgfsetfillcolor{currentfill}%
\pgfsetlinewidth{0.000000pt}%
\definecolor{currentstroke}{rgb}{0.000000,0.000000,0.000000}%
\pgfsetstrokecolor{currentstroke}%
\pgfsetstrokeopacity{0.000000}%
\pgfsetdash{}{0pt}%
\pgfpathmoveto{\pgfqpoint{3.193977in}{0.407000in}}%
\pgfpathlineto{\pgfqpoint{3.257386in}{0.407000in}}%
\pgfpathlineto{\pgfqpoint{3.257386in}{0.407000in}}%
\pgfpathlineto{\pgfqpoint{3.193977in}{0.407000in}}%
\pgfpathlineto{\pgfqpoint{3.193977in}{0.407000in}}%
\pgfpathclose%
\pgfusepath{fill}%
\end{pgfscope}%
\begin{pgfscope}%
\pgfpathrectangle{\pgfqpoint{0.562500in}{0.407000in}}{\pgfqpoint{3.487500in}{2.849000in}}%
\pgfusepath{clip}%
\pgfsetbuttcap%
\pgfsetmiterjoin%
\definecolor{currentfill}{rgb}{0.121569,0.466667,0.705882}%
\pgfsetfillcolor{currentfill}%
\pgfsetlinewidth{0.000000pt}%
\definecolor{currentstroke}{rgb}{0.000000,0.000000,0.000000}%
\pgfsetstrokecolor{currentstroke}%
\pgfsetstrokeopacity{0.000000}%
\pgfsetdash{}{0pt}%
\pgfpathmoveto{\pgfqpoint{3.257386in}{0.407000in}}%
\pgfpathlineto{\pgfqpoint{3.320795in}{0.407000in}}%
\pgfpathlineto{\pgfqpoint{3.320795in}{0.824436in}}%
\pgfpathlineto{\pgfqpoint{3.257386in}{0.824436in}}%
\pgfpathlineto{\pgfqpoint{3.257386in}{0.407000in}}%
\pgfpathclose%
\pgfusepath{fill}%
\end{pgfscope}%
\begin{pgfscope}%
\pgfpathrectangle{\pgfqpoint{0.562500in}{0.407000in}}{\pgfqpoint{3.487500in}{2.849000in}}%
\pgfusepath{clip}%
\pgfsetbuttcap%
\pgfsetmiterjoin%
\definecolor{currentfill}{rgb}{0.121569,0.466667,0.705882}%
\pgfsetfillcolor{currentfill}%
\pgfsetlinewidth{0.000000pt}%
\definecolor{currentstroke}{rgb}{0.000000,0.000000,0.000000}%
\pgfsetstrokecolor{currentstroke}%
\pgfsetstrokeopacity{0.000000}%
\pgfsetdash{}{0pt}%
\pgfpathmoveto{\pgfqpoint{3.320795in}{0.407000in}}%
\pgfpathlineto{\pgfqpoint{3.384205in}{0.407000in}}%
\pgfpathlineto{\pgfqpoint{3.384205in}{0.615718in}}%
\pgfpathlineto{\pgfqpoint{3.320795in}{0.615718in}}%
\pgfpathlineto{\pgfqpoint{3.320795in}{0.407000in}}%
\pgfpathclose%
\pgfusepath{fill}%
\end{pgfscope}%
\begin{pgfscope}%
\pgfpathrectangle{\pgfqpoint{0.562500in}{0.407000in}}{\pgfqpoint{3.487500in}{2.849000in}}%
\pgfusepath{clip}%
\pgfsetbuttcap%
\pgfsetmiterjoin%
\definecolor{currentfill}{rgb}{0.121569,0.466667,0.705882}%
\pgfsetfillcolor{currentfill}%
\pgfsetlinewidth{0.000000pt}%
\definecolor{currentstroke}{rgb}{0.000000,0.000000,0.000000}%
\pgfsetstrokecolor{currentstroke}%
\pgfsetstrokeopacity{0.000000}%
\pgfsetdash{}{0pt}%
\pgfpathmoveto{\pgfqpoint{3.384205in}{0.407000in}}%
\pgfpathlineto{\pgfqpoint{3.447614in}{0.407000in}}%
\pgfpathlineto{\pgfqpoint{3.447614in}{0.615718in}}%
\pgfpathlineto{\pgfqpoint{3.384205in}{0.615718in}}%
\pgfpathlineto{\pgfqpoint{3.384205in}{0.407000in}}%
\pgfpathclose%
\pgfusepath{fill}%
\end{pgfscope}%
\begin{pgfscope}%
\pgfpathrectangle{\pgfqpoint{0.562500in}{0.407000in}}{\pgfqpoint{3.487500in}{2.849000in}}%
\pgfusepath{clip}%
\pgfsetbuttcap%
\pgfsetmiterjoin%
\definecolor{currentfill}{rgb}{0.121569,0.466667,0.705882}%
\pgfsetfillcolor{currentfill}%
\pgfsetlinewidth{0.000000pt}%
\definecolor{currentstroke}{rgb}{0.000000,0.000000,0.000000}%
\pgfsetstrokecolor{currentstroke}%
\pgfsetstrokeopacity{0.000000}%
\pgfsetdash{}{0pt}%
\pgfpathmoveto{\pgfqpoint{3.447614in}{0.407000in}}%
\pgfpathlineto{\pgfqpoint{3.511023in}{0.407000in}}%
\pgfpathlineto{\pgfqpoint{3.511023in}{0.615718in}}%
\pgfpathlineto{\pgfqpoint{3.447614in}{0.615718in}}%
\pgfpathlineto{\pgfqpoint{3.447614in}{0.407000in}}%
\pgfpathclose%
\pgfusepath{fill}%
\end{pgfscope}%
\begin{pgfscope}%
\pgfpathrectangle{\pgfqpoint{0.562500in}{0.407000in}}{\pgfqpoint{3.487500in}{2.849000in}}%
\pgfusepath{clip}%
\pgfsetbuttcap%
\pgfsetmiterjoin%
\definecolor{currentfill}{rgb}{0.121569,0.466667,0.705882}%
\pgfsetfillcolor{currentfill}%
\pgfsetlinewidth{0.000000pt}%
\definecolor{currentstroke}{rgb}{0.000000,0.000000,0.000000}%
\pgfsetstrokecolor{currentstroke}%
\pgfsetstrokeopacity{0.000000}%
\pgfsetdash{}{0pt}%
\pgfpathmoveto{\pgfqpoint{3.511023in}{0.407000in}}%
\pgfpathlineto{\pgfqpoint{3.574432in}{0.407000in}}%
\pgfpathlineto{\pgfqpoint{3.574432in}{0.407000in}}%
\pgfpathlineto{\pgfqpoint{3.511023in}{0.407000in}}%
\pgfpathlineto{\pgfqpoint{3.511023in}{0.407000in}}%
\pgfpathclose%
\pgfusepath{fill}%
\end{pgfscope}%
\begin{pgfscope}%
\pgfpathrectangle{\pgfqpoint{0.562500in}{0.407000in}}{\pgfqpoint{3.487500in}{2.849000in}}%
\pgfusepath{clip}%
\pgfsetbuttcap%
\pgfsetmiterjoin%
\definecolor{currentfill}{rgb}{0.121569,0.466667,0.705882}%
\pgfsetfillcolor{currentfill}%
\pgfsetlinewidth{0.000000pt}%
\definecolor{currentstroke}{rgb}{0.000000,0.000000,0.000000}%
\pgfsetstrokecolor{currentstroke}%
\pgfsetstrokeopacity{0.000000}%
\pgfsetdash{}{0pt}%
\pgfpathmoveto{\pgfqpoint{3.574432in}{0.407000in}}%
\pgfpathlineto{\pgfqpoint{3.637841in}{0.407000in}}%
\pgfpathlineto{\pgfqpoint{3.637841in}{0.407000in}}%
\pgfpathlineto{\pgfqpoint{3.574432in}{0.407000in}}%
\pgfpathlineto{\pgfqpoint{3.574432in}{0.407000in}}%
\pgfpathclose%
\pgfusepath{fill}%
\end{pgfscope}%
\begin{pgfscope}%
\pgfpathrectangle{\pgfqpoint{0.562500in}{0.407000in}}{\pgfqpoint{3.487500in}{2.849000in}}%
\pgfusepath{clip}%
\pgfsetbuttcap%
\pgfsetmiterjoin%
\definecolor{currentfill}{rgb}{0.121569,0.466667,0.705882}%
\pgfsetfillcolor{currentfill}%
\pgfsetlinewidth{0.000000pt}%
\definecolor{currentstroke}{rgb}{0.000000,0.000000,0.000000}%
\pgfsetstrokecolor{currentstroke}%
\pgfsetstrokeopacity{0.000000}%
\pgfsetdash{}{0pt}%
\pgfpathmoveto{\pgfqpoint{3.637841in}{0.407000in}}%
\pgfpathlineto{\pgfqpoint{3.701250in}{0.407000in}}%
\pgfpathlineto{\pgfqpoint{3.701250in}{0.824436in}}%
\pgfpathlineto{\pgfqpoint{3.637841in}{0.824436in}}%
\pgfpathlineto{\pgfqpoint{3.637841in}{0.407000in}}%
\pgfpathclose%
\pgfusepath{fill}%
\end{pgfscope}%
\begin{pgfscope}%
\pgfpathrectangle{\pgfqpoint{0.562500in}{0.407000in}}{\pgfqpoint{3.487500in}{2.849000in}}%
\pgfusepath{clip}%
\pgfsetbuttcap%
\pgfsetmiterjoin%
\definecolor{currentfill}{rgb}{0.121569,0.466667,0.705882}%
\pgfsetfillcolor{currentfill}%
\pgfsetlinewidth{0.000000pt}%
\definecolor{currentstroke}{rgb}{0.000000,0.000000,0.000000}%
\pgfsetstrokecolor{currentstroke}%
\pgfsetstrokeopacity{0.000000}%
\pgfsetdash{}{0pt}%
\pgfpathmoveto{\pgfqpoint{3.701250in}{0.407000in}}%
\pgfpathlineto{\pgfqpoint{3.764659in}{0.407000in}}%
\pgfpathlineto{\pgfqpoint{3.764659in}{0.407000in}}%
\pgfpathlineto{\pgfqpoint{3.701250in}{0.407000in}}%
\pgfpathlineto{\pgfqpoint{3.701250in}{0.407000in}}%
\pgfpathclose%
\pgfusepath{fill}%
\end{pgfscope}%
\begin{pgfscope}%
\pgfpathrectangle{\pgfqpoint{0.562500in}{0.407000in}}{\pgfqpoint{3.487500in}{2.849000in}}%
\pgfusepath{clip}%
\pgfsetbuttcap%
\pgfsetmiterjoin%
\definecolor{currentfill}{rgb}{0.121569,0.466667,0.705882}%
\pgfsetfillcolor{currentfill}%
\pgfsetlinewidth{0.000000pt}%
\definecolor{currentstroke}{rgb}{0.000000,0.000000,0.000000}%
\pgfsetstrokecolor{currentstroke}%
\pgfsetstrokeopacity{0.000000}%
\pgfsetdash{}{0pt}%
\pgfpathmoveto{\pgfqpoint{3.764659in}{0.407000in}}%
\pgfpathlineto{\pgfqpoint{3.828068in}{0.407000in}}%
\pgfpathlineto{\pgfqpoint{3.828068in}{0.824436in}}%
\pgfpathlineto{\pgfqpoint{3.764659in}{0.824436in}}%
\pgfpathlineto{\pgfqpoint{3.764659in}{0.407000in}}%
\pgfpathclose%
\pgfusepath{fill}%
\end{pgfscope}%
\begin{pgfscope}%
\pgfpathrectangle{\pgfqpoint{0.562500in}{0.407000in}}{\pgfqpoint{3.487500in}{2.849000in}}%
\pgfusepath{clip}%
\pgfsetbuttcap%
\pgfsetmiterjoin%
\definecolor{currentfill}{rgb}{0.121569,0.466667,0.705882}%
\pgfsetfillcolor{currentfill}%
\pgfsetlinewidth{0.000000pt}%
\definecolor{currentstroke}{rgb}{0.000000,0.000000,0.000000}%
\pgfsetstrokecolor{currentstroke}%
\pgfsetstrokeopacity{0.000000}%
\pgfsetdash{}{0pt}%
\pgfpathmoveto{\pgfqpoint{3.828068in}{0.407000in}}%
\pgfpathlineto{\pgfqpoint{3.891477in}{0.407000in}}%
\pgfpathlineto{\pgfqpoint{3.891477in}{0.615718in}}%
\pgfpathlineto{\pgfqpoint{3.828068in}{0.615718in}}%
\pgfpathlineto{\pgfqpoint{3.828068in}{0.407000in}}%
\pgfpathclose%
\pgfusepath{fill}%
\end{pgfscope}%
\begin{pgfscope}%
\pgfpathrectangle{\pgfqpoint{0.562500in}{0.407000in}}{\pgfqpoint{3.487500in}{2.849000in}}%
\pgfusepath{clip}%
\pgfsetrectcap%
\pgfsetroundjoin%
\pgfsetlinewidth{0.803000pt}%
\definecolor{currentstroke}{rgb}{0.690196,0.690196,0.690196}%
\pgfsetstrokecolor{currentstroke}%
\pgfsetdash{}{0pt}%
\pgfpathmoveto{\pgfqpoint{0.827756in}{0.407000in}}%
\pgfpathlineto{\pgfqpoint{0.827756in}{3.256000in}}%
\pgfusepath{stroke}%
\end{pgfscope}%
\begin{pgfscope}%
\pgfsetbuttcap%
\pgfsetroundjoin%
\definecolor{currentfill}{rgb}{0.000000,0.000000,0.000000}%
\pgfsetfillcolor{currentfill}%
\pgfsetlinewidth{0.803000pt}%
\definecolor{currentstroke}{rgb}{0.000000,0.000000,0.000000}%
\pgfsetstrokecolor{currentstroke}%
\pgfsetdash{}{0pt}%
\pgfsys@defobject{currentmarker}{\pgfqpoint{0.000000in}{-0.048611in}}{\pgfqpoint{0.000000in}{0.000000in}}{%
\pgfpathmoveto{\pgfqpoint{0.000000in}{0.000000in}}%
\pgfpathlineto{\pgfqpoint{0.000000in}{-0.048611in}}%
\pgfusepath{stroke,fill}%
}%
\begin{pgfscope}%
\pgfsys@transformshift{0.827756in}{0.407000in}%
\pgfsys@useobject{currentmarker}{}%
\end{pgfscope}%
\end{pgfscope}%
\begin{pgfscope}%
\definecolor{textcolor}{rgb}{0.000000,0.000000,0.000000}%
\pgfsetstrokecolor{textcolor}%
\pgfsetfillcolor{textcolor}%
\pgftext[x=0.827756in,y=0.309778in,,top]{\color{textcolor}\rmfamily\fontsize{10.000000}{12.000000}\selectfont \(\displaystyle {1.5}\)}%
\end{pgfscope}%
\begin{pgfscope}%
\pgfpathrectangle{\pgfqpoint{0.562500in}{0.407000in}}{\pgfqpoint{3.487500in}{2.849000in}}%
\pgfusepath{clip}%
\pgfsetrectcap%
\pgfsetroundjoin%
\pgfsetlinewidth{0.803000pt}%
\definecolor{currentstroke}{rgb}{0.690196,0.690196,0.690196}%
\pgfsetstrokecolor{currentstroke}%
\pgfsetdash{}{0pt}%
\pgfpathmoveto{\pgfqpoint{1.688355in}{0.407000in}}%
\pgfpathlineto{\pgfqpoint{1.688355in}{3.256000in}}%
\pgfusepath{stroke}%
\end{pgfscope}%
\begin{pgfscope}%
\pgfsetbuttcap%
\pgfsetroundjoin%
\definecolor{currentfill}{rgb}{0.000000,0.000000,0.000000}%
\pgfsetfillcolor{currentfill}%
\pgfsetlinewidth{0.803000pt}%
\definecolor{currentstroke}{rgb}{0.000000,0.000000,0.000000}%
\pgfsetstrokecolor{currentstroke}%
\pgfsetdash{}{0pt}%
\pgfsys@defobject{currentmarker}{\pgfqpoint{0.000000in}{-0.048611in}}{\pgfqpoint{0.000000in}{0.000000in}}{%
\pgfpathmoveto{\pgfqpoint{0.000000in}{0.000000in}}%
\pgfpathlineto{\pgfqpoint{0.000000in}{-0.048611in}}%
\pgfusepath{stroke,fill}%
}%
\begin{pgfscope}%
\pgfsys@transformshift{1.688355in}{0.407000in}%
\pgfsys@useobject{currentmarker}{}%
\end{pgfscope}%
\end{pgfscope}%
\begin{pgfscope}%
\definecolor{textcolor}{rgb}{0.000000,0.000000,0.000000}%
\pgfsetstrokecolor{textcolor}%
\pgfsetfillcolor{textcolor}%
\pgftext[x=1.688355in,y=0.309778in,,top]{\color{textcolor}\rmfamily\fontsize{10.000000}{12.000000}\selectfont \(\displaystyle {2.0}\)}%
\end{pgfscope}%
\begin{pgfscope}%
\pgfpathrectangle{\pgfqpoint{0.562500in}{0.407000in}}{\pgfqpoint{3.487500in}{2.849000in}}%
\pgfusepath{clip}%
\pgfsetrectcap%
\pgfsetroundjoin%
\pgfsetlinewidth{0.803000pt}%
\definecolor{currentstroke}{rgb}{0.690196,0.690196,0.690196}%
\pgfsetstrokecolor{currentstroke}%
\pgfsetdash{}{0pt}%
\pgfpathmoveto{\pgfqpoint{2.548955in}{0.407000in}}%
\pgfpathlineto{\pgfqpoint{2.548955in}{3.256000in}}%
\pgfusepath{stroke}%
\end{pgfscope}%
\begin{pgfscope}%
\pgfsetbuttcap%
\pgfsetroundjoin%
\definecolor{currentfill}{rgb}{0.000000,0.000000,0.000000}%
\pgfsetfillcolor{currentfill}%
\pgfsetlinewidth{0.803000pt}%
\definecolor{currentstroke}{rgb}{0.000000,0.000000,0.000000}%
\pgfsetstrokecolor{currentstroke}%
\pgfsetdash{}{0pt}%
\pgfsys@defobject{currentmarker}{\pgfqpoint{0.000000in}{-0.048611in}}{\pgfqpoint{0.000000in}{0.000000in}}{%
\pgfpathmoveto{\pgfqpoint{0.000000in}{0.000000in}}%
\pgfpathlineto{\pgfqpoint{0.000000in}{-0.048611in}}%
\pgfusepath{stroke,fill}%
}%
\begin{pgfscope}%
\pgfsys@transformshift{2.548955in}{0.407000in}%
\pgfsys@useobject{currentmarker}{}%
\end{pgfscope}%
\end{pgfscope}%
\begin{pgfscope}%
\definecolor{textcolor}{rgb}{0.000000,0.000000,0.000000}%
\pgfsetstrokecolor{textcolor}%
\pgfsetfillcolor{textcolor}%
\pgftext[x=2.548955in,y=0.309778in,,top]{\color{textcolor}\rmfamily\fontsize{10.000000}{12.000000}\selectfont \(\displaystyle {2.5}\)}%
\end{pgfscope}%
\begin{pgfscope}%
\pgfpathrectangle{\pgfqpoint{0.562500in}{0.407000in}}{\pgfqpoint{3.487500in}{2.849000in}}%
\pgfusepath{clip}%
\pgfsetrectcap%
\pgfsetroundjoin%
\pgfsetlinewidth{0.803000pt}%
\definecolor{currentstroke}{rgb}{0.690196,0.690196,0.690196}%
\pgfsetstrokecolor{currentstroke}%
\pgfsetdash{}{0pt}%
\pgfpathmoveto{\pgfqpoint{3.409554in}{0.407000in}}%
\pgfpathlineto{\pgfqpoint{3.409554in}{3.256000in}}%
\pgfusepath{stroke}%
\end{pgfscope}%
\begin{pgfscope}%
\pgfsetbuttcap%
\pgfsetroundjoin%
\definecolor{currentfill}{rgb}{0.000000,0.000000,0.000000}%
\pgfsetfillcolor{currentfill}%
\pgfsetlinewidth{0.803000pt}%
\definecolor{currentstroke}{rgb}{0.000000,0.000000,0.000000}%
\pgfsetstrokecolor{currentstroke}%
\pgfsetdash{}{0pt}%
\pgfsys@defobject{currentmarker}{\pgfqpoint{0.000000in}{-0.048611in}}{\pgfqpoint{0.000000in}{0.000000in}}{%
\pgfpathmoveto{\pgfqpoint{0.000000in}{0.000000in}}%
\pgfpathlineto{\pgfqpoint{0.000000in}{-0.048611in}}%
\pgfusepath{stroke,fill}%
}%
\begin{pgfscope}%
\pgfsys@transformshift{3.409554in}{0.407000in}%
\pgfsys@useobject{currentmarker}{}%
\end{pgfscope}%
\end{pgfscope}%
\begin{pgfscope}%
\definecolor{textcolor}{rgb}{0.000000,0.000000,0.000000}%
\pgfsetstrokecolor{textcolor}%
\pgfsetfillcolor{textcolor}%
\pgftext[x=3.409554in,y=0.309778in,,top]{\color{textcolor}\rmfamily\fontsize{10.000000}{12.000000}\selectfont \(\displaystyle {3.0}\)}%
\end{pgfscope}%
\begin{pgfscope}%
\definecolor{textcolor}{rgb}{0.000000,0.000000,0.000000}%
\pgfsetstrokecolor{textcolor}%
\pgfsetfillcolor{textcolor}%
\pgftext[x=2.306250in,y=0.131567in,,top]{\color{textcolor}\rmfamily\fontsize{10.000000}{12.000000}\selectfont \(\displaystyle D\) [\(\displaystyle \si{cm^2/s}\)]}%
\end{pgfscope}%
\begin{pgfscope}%
\definecolor{textcolor}{rgb}{0.000000,0.000000,0.000000}%
\pgfsetstrokecolor{textcolor}%
\pgfsetfillcolor{textcolor}%
\pgftext[x=4.050000in,y=0.145456in,right,top]{\color{textcolor}\rmfamily\fontsize{10.000000}{12.000000}\selectfont \(\displaystyle \times{10^{\ensuremath{-}5}}{}\)}%
\end{pgfscope}%
\begin{pgfscope}%
\definecolor{textcolor}{rgb}{0.000000,0.000000,0.000000}%
\pgfsetstrokecolor{textcolor}%
\pgfsetfillcolor{textcolor}%
\pgftext[x=0.506944in,y=1.831500in,,bottom,rotate=90.000000]{\color{textcolor}\rmfamily\fontsize{10.000000}{12.000000}\selectfont Distribution}%
\end{pgfscope}%
\begin{pgfscope}%
\pgfsetrectcap%
\pgfsetmiterjoin%
\pgfsetlinewidth{0.803000pt}%
\definecolor{currentstroke}{rgb}{0.000000,0.000000,0.000000}%
\pgfsetstrokecolor{currentstroke}%
\pgfsetdash{}{0pt}%
\pgfpathmoveto{\pgfqpoint{0.562500in}{0.407000in}}%
\pgfpathlineto{\pgfqpoint{0.562500in}{3.256000in}}%
\pgfusepath{stroke}%
\end{pgfscope}%
\begin{pgfscope}%
\pgfsetrectcap%
\pgfsetmiterjoin%
\pgfsetlinewidth{0.803000pt}%
\definecolor{currentstroke}{rgb}{0.000000,0.000000,0.000000}%
\pgfsetstrokecolor{currentstroke}%
\pgfsetdash{}{0pt}%
\pgfpathmoveto{\pgfqpoint{4.050000in}{0.407000in}}%
\pgfpathlineto{\pgfqpoint{4.050000in}{3.256000in}}%
\pgfusepath{stroke}%
\end{pgfscope}%
\begin{pgfscope}%
\pgfsetrectcap%
\pgfsetmiterjoin%
\pgfsetlinewidth{0.803000pt}%
\definecolor{currentstroke}{rgb}{0.000000,0.000000,0.000000}%
\pgfsetstrokecolor{currentstroke}%
\pgfsetdash{}{0pt}%
\pgfpathmoveto{\pgfqpoint{0.562500in}{0.407000in}}%
\pgfpathlineto{\pgfqpoint{4.050000in}{0.407000in}}%
\pgfusepath{stroke}%
\end{pgfscope}%
\begin{pgfscope}%
\pgfsetrectcap%
\pgfsetmiterjoin%
\pgfsetlinewidth{0.803000pt}%
\definecolor{currentstroke}{rgb}{0.000000,0.000000,0.000000}%
\pgfsetstrokecolor{currentstroke}%
\pgfsetdash{}{0pt}%
\pgfpathmoveto{\pgfqpoint{0.562500in}{3.256000in}}%
\pgfpathlineto{\pgfqpoint{4.050000in}{3.256000in}}%
\pgfusepath{stroke}%
\end{pgfscope}%
\begin{pgfscope}%
\definecolor{textcolor}{rgb}{0.000000,0.000000,0.000000}%
\pgfsetstrokecolor{textcolor}%
\pgfsetfillcolor{textcolor}%
\pgftext[x=2.306250in,y=3.339333in,,base]{\color{textcolor}\rmfamily\fontsize{12.000000}{14.400000}\selectfont D\_vacf}%
\end{pgfscope}%
\end{pgfpicture}%
\makeatother%
\endgroup%

        }
        \caption{Distribution of $D_{VACF}$ obtained by VAFC.}
        \label{vacf_distrib}
    \end{subfigure}
    \caption{MSD, VACF and diffusion coefficient $D$ statistics obtained over $N_{run} = 400$ simulations for different initial pseudo-random positions and velocities belonging to the same equilibrium state $T = 94.4 K$. Each simulation has been run until $t=1.0$ and, exploiting the stability of the system, the initial positions and velocities of the next simulation were obtained by the final state of the preceding one. The resulting means of the distributions of $D$ are $\langle D_{MSD} \rangle = 2.394 \cdot 10^{-5} \si{cm^2/s}$, $\langle D_{VACF} \rangle =  2.397 \cdot 10^{-5} \si{cm^2/s}$ and the related standard deviations are $\sigma(D_{MSD}) = 1.782 \cdot 10^{-6} \si{cm^2/s}$ and $\sigma(D_{AVCF}) = 3.271 \cdot 10^{-6} \si{cm^2/s}$.
    Time and MSD are expressed in [LJ] units (see \ref{app:ljunits}).}
\end{figure}

\paragraph{Mean square displacement estimation}
The diffusion coefficient can be obtained by noticing that the mean square displacement is directly proportional to time after $t=0.2$. Then fitting a linear curve $MSD(t) = D + Ct$ (as it's shown in figure \ref{msd_aver}) for some constant $C$ and $D$ on the samples of $t \in [0.8, 1.0]$ should give an enough accurate result by taking $D_{MSD} = C$ from the fitted parameters.

\paragraph{Velocity auto-correlation function}
This value is obtained by estimating a finite difference integration of all collected values of the velocity auto-correlation function (figure \ref{vacf_aver}). Because the time of the simulation is in the $[0, 1]$ interval, then it suffices to average all obtained samples of the VACF.

Applying those method for a large enough number of simulation and counting the occurrences of the estimates it's possible to obtain the distribution of both $D_{MSD}$ and $D_{VACF}$ random variables. That's what it's shown in figures (\ref{msd_distrib}) and (\ref{vacf_distrib}).
From those distributions it's sufficient to take the mean and the standard deviation on the mean as error estimate in order to give the definitive values of the diffusion coefficient. Finally, 

\begin{align}
    &D_{MSD} = (2.394 \pm 0.009) \cdot 10^{-5} \si{cm^2/s} 
    &D_{VACF} = (2.397 \pm 0.018) \cdot 10^{-5} \si{cm^2/s}
\end{align}

The distance between the values above is so small that they can be considered as equal. This confirms the equivalence of the two methods and the ensures the correctness of the estimated diffusion coefficient. Both values are also really close to the one estimated by Rahman \cite{Rahman} which is $\hat{D}_{MSD} = 2.43 \cdot 10^{-5}$.

%%%%%%%%%%%%%%%%%%%%%%
%%%%%%%%%%%%%%%%%%%%%%
\section{Conclusion}
\label{conc}
Molecular dynamics simulation performed with Runge-Kutta time integration turns out to be a good method for certain conditions involving a small number of particles and a small time step for stability. However, the biggest limitation of this method is the scalability: having more particles implies more computations in the order of $\mathcal{O}(N^3)$, which is not clearly feasible. On the other hand the obtained results for the radial pair correlation function, the structure factor and also the diffusion coefficient are very satisfying and they are very similar to the results of Rahman \cite{Rahman}. 

%%%%%%%%%%%%%%%%%%%%%%
%%%%%%%%%%%%%%%%%%%%%%
\def\refname{D\MakeLowercase{ocumentation and sources}}
\begin{thebibliography}{99}

\bibitem{ljunits}
\url{https://cms.sjtu.edu.cn/doc/courseware/2019/LJ.pdf}

\bibitem{Rahman}
Correlations in the Motion of Atoms in Liquid Argon, A. Rahman, Phys. Rev. 136, A405 – Published 19 October 1964

\bibitem{course05}
Course-05.pdf slides on Moodle

\bibitem{course031}
Course-031.pdf slides on Moodle

\bibitem{Chandler}
Chandler, D. (1987). "7.3". Introduction to Modern Statistical Mechanics. Oxford University Press.
%Physics Today 41, 12, 114 (1988); \url{https://doi.org/10.1063/1.2811680}

\end{thebibliography}


%%%%%%%%%%%%%%%%%%%%%%
%%%%%%%%%%%%%%%%%%%%%%
\appendix
\label{appendix}

\section{Lennard Jones units} \label{app:ljunits}

\begin{table}[H]
\begin{tabular}{c|c}
quantity & LJ units \\
\hline
distance & $\sigma$ \\
energy & $\varepsilon$ \\
velocity & $(\varepsilon / M)^{1/2}$ \\
temperature & $\varepsilon / k_B$ \\
time & $\sigma (\varepsilon / M)^{-1/2}$
\end{tabular}
\end{table}

where $M = 6.69 \cdot 10^{-26}$ \si{Kg} is the mass of Argon.

\end{document}

